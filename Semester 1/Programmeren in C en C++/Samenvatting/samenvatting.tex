\documentclass[12pt]{report} 

% PACKAGES 
\usepackage[dutch]{babel}
\usepackage[utf8]{inputenc}
\usepackage{color}
\usepackage{amsmath} % Matrices
\usepackage{booktabs}
\usepackage{xcolor}
\usepackage{sectsty}
\usepackage{lipsum}
\usepackage{listings}
\usepackage[normalem]{ulem}
\usepackage{pgfplots}
\pgfplotsset{compat=1.13}


\partfont{\color{brown}}
\chapterfont{\color{teal}}
\sectionfont{\color{cyan}}

% DOCUMENT INFORMATION
\title{Programmeren in C en C++}
\author{Bert De Saffel}
\date{2017-2018}


% CUSTOM COMMANDS
\newcommand{\todo}[1] {
\color{red}\textunderscore{\textit{TODO: #1}}
}

\newcommand{\important}[1] {\textbf{\color{orange}#1}}

\lstset{
  frame = single,
  keepspaces = false
}

% DOCUMENT
\begin{document}
\maketitle
\tableofcontents

\part{C}
\chapter{Functies}
\section{rand()}
Een getal tussen een minimum en maximum krijgen
\newline
\begin{lstlisting}
 int g = (int)rand() * (max - min + 1) + min
\end{lstlisting}




\section{printf()}
Conversiekarakters: \%[-][w][.p]c
\begin{itemize}
 \item \important{-}: Links aligneren, default rechts
 \item \important{w}: minimum breedte
 \item \important{.p}: Aantal kommagetallen OF karakters
 \item \important{c} conversiekarakter

\end{itemize}

\section{scanf()}
scanf \% c, \&i

\chapter{Pointers}

\begin{lstlisting}
int g = 6;
int *pg = &g;
\end{lstlisting}

\begin{tikzpicture}
 \draw (0,2) rectangle (1,3);
 \draw (0, 0) rectangle(1,1);
   
 \node[text] at (0.5, 2.5) {6};
 \node[text] at (-0.5, 2.5) {g};
 \node at (0.5,0.5) {\textbullet};
 \node[text] at (-0.5, 0.5) {pg};
 \draw[thick, ->](0.5,0.5) -- (0.5, 2);
\end{tikzpicture}


\begin{lstlisting}
 void *v;
 double d = *(double *)v;
\end{lstlisting}

\begin{lstlisting}
 int n        = 3;
 const int *q = &n;
 int *p       = &n;
\end{lstlisting}

\begin{tikzpicture}
  \coordinate (Ps) at (-3,-3);
  \coordinate (Pe) at (-2,-2);
  \coordinate (Pm) at (-2.5 ,-2.5);
  
  \coordinate (Qs) at (1,-3);
  \coordinate (Qe) at (2,-2);
  \coordinate (Qm) at (1.5 ,-2.5);
  
  \coordinate (Ns) at (-1,-1);
  \coordinate (Ne) at (0,0);
  \coordinate (Nm) at (-0.5 ,-0.5);
 
  \draw (Ps) rectangle (Pe);
  \draw (Ns) rectangle (Ne);
  \draw (Qs) rectangle (Qe);
  
  \node[text] at (-3.5, -2.5) {p};
  \node at (Pm) {\textbullet};
  \draw[thick, ->] (Pm) -- (-0.75, -1);
  
  \node[text] at (0.5, -2.5) {q};
  \node at (Qm) {\textbullet};
  \draw[thick, ->] (Qm) -- (-0.25, -1);
  \draw (0.25,-1.5) circle (0.5cm);
  
  \node[text] at (Nm) {3};
  \node[text] at (-1.5, -0.5) {n};
\end{tikzpicture}

\begin{lstlisting}[escapechar=\%]
 q = p; 
 %\sout{p = q}%;
\end{lstlisting}

\section{Functiepointers}
functie-pointers\\
voidpointers + functiepointers

\begin{lstlisting}
 void *base => pointer naar eerste element
 size_t nitems => aantal elementen
 size_t size => grotte van 1 element
\end{lstlisting}
char *h = (char*) base

\chapter{C-strings}
Nulkarakter op het einde\\
\begin{lstlisting}
 char *s1    = "tekst";
 char s2[80] = "tekst";
 char s3[]   = "tekst";
\end{lstlisting}

\begin{tikzpicture}
  

 \node[text] at (0,0) {$s_1$};
 \draw (0.5,0) rectangle (0.75,0.25);
 \draw[thick, ->] (0.625, 0.125) -- (1, 0.125);
 \draw(1,0) rectangle(2,0.5);
 \node[text] at (1.5, 0.25) {t};
 \draw(2,0) rectangle(3,0.5);
 \node[text] at (2.5, 0.25) {e};
 \draw(3,0) rectangle(4,0.5);
 \node[text] at (3.5, 0.25) {k};
 \draw(4,0) rectangle(5,0.5);
 \node[text] at (4.5, 0.25) {s};
 \draw(5,0) rectangle(6,0.5);
 \node[text] at (5.5, 0.25) {t};
 \draw(6,0) rectangle(7,0.5);
 \node[text] at (6.5, 0.25) {$\phi$};
 
 
 
 \node[text] at (0,-1) {$s_2$};
 \node[text] at (0,-2) {$s_3$};
 \node[text] at (0,-3) {$s_4$};
\end{tikzpicture}

$S2 \Rightarrow [t][e][k][s][t][0][][][][][]$
$S3 \Rightarrow [t][e][k][s][t][0]$

enkel $s1=''nieuw''$



\end{document}
