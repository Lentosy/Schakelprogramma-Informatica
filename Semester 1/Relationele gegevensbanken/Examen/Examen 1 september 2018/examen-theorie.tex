\documentclass{article}

\title{Examen relationele gegevensbanken 1 september 2018 - Theorie}
\date{}
\author{}


\begin{document}

\maketitle
\begin{enumerate}
 \item Wat is het nut van \textit{refererende-actieregels}? Bij welk type constraint worden deze gebruikt en wat zijn de verschillende waarden dat deze regels kunnen aannemen?
 \item Leg het verschil uit tussen een \textit{CTE} en een \textit{view} alsook in het gebruik ervan.
 \item Leg het indexeringsmechanisme bij gedimensioneerde gegevensmodellering uit.
 \item Wat betekent \textit{granulariteit} in de context van \textit{locking}. Wat zijn de waarden en op welke plaatsen kan het geconfigureerd worden.
 \item Op welke voorwaarden kan men \textit{NoSQL} overwegen. Geef 2 voorbeelden.
 \item Leg precies uit wat een index join is.
 \item Leg precies uit wat een join index is.
 \item Bespreek de voor -en nadelen van een \textit{Heap} bestand.
 \item Waarom genieten \textit{Stored Procedures} de voorkeur over traditionele \textit{3GL} scripts.
 \item Welke indices gebruikt men best bij een \textit{index-sequentieël} bestand? Wat is de naam van zo een index?
 \item Leg het \textit{isolation level} uit bij transacties. Wat zijn de verschillende waarden
 en wat zijn hun corresponderende nadelen.
 \item Leg \textit{selectiviteit} uit bij indexeringsmechanismes. Wanneer is deze hoog of laag? Welke indextechniek kan men gebruiken voor deze extreme waarden?
 \item In welke omstandigheden zijn cursors nog enigzins bruikbaar?
 \item Leg uit waarom triggers aan het \textit{ECA-model} voldoen.
 \item Geef de verschillende \textit{DML}-triggers en hun toepassingsgebied.
 \item Wat voor soort predikaten werken bevorderend indien een \textit{hash} bestand gebruikt wordt.
 \item Leg \textit{covered index} uit. Welke functionaliteit kan je ermee benaderen.
 \item Onder welke voorwaarden is een CTE of een view \textit{wijzigbaar}?
 \item Wat is het nut van de \textit{with check option} bij een view?
\end{enumerate}

\end{document}
