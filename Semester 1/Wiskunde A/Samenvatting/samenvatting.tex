\documentclass[12pt]{report} 

% PACKAGES 
\usepackage[dutch]{babel}
\usepackage[utf8]{inputenc}
\usepackage{color}
\usepackage{amsmath} % Matrices
\usepackage{amsfonts}
\usepackage{enumitem}
\usepackage{booktabs}
\usepackage{xcolor}
\usepackage{sectsty}
\usepackage{graphicx}
\usepackage{lipsum}
\usepackage{pgfplots}
\pgfplotsset{compat=1.13}




\partfont{\color{brown}}
\chapterfont{\color{teal}}
\sectionfont{\color{cyan}}

% DOCUMENT INFORMATION
\title{Wiskunde A}
\author{Bert De Saffel}
\date{2017-2018}


% CUSTOM COMMANDS
\newcommand{\todo}[1] {
\color{red}\textunderscore{\textit{TODO: #1}}
}

\newcommand{\important}[1] {\textbf{\color{orange}#1}}
\newcommand{\mathimportant}[2] {\textbf{\color{#2}$#1$}}

% DOCUMENT
\begin{document}
\maketitle
\tableofcontents

\part{Theorie}
\chapter{Complexe Getallen}
\important{Inleiding}
\begin{itemize}
 \item $\mathbb{N}$ = Natuurlijke getallen: \{0, 1, 2, 3, ...\}
 \item $\mathbb{Z}$ = Gehele getallen: \{..., -2, -1, 0, 1, 2, ...\}
 \item $\mathbb{Q}$ = Rationale getallen: \{$\frac{1}{3}$, $-\frac{1}{4}$, $\frac{7}{2}$, ... \}
 \item $\mathbb{R}$ = Reële getallen: \{ $\sqrt{2}$ , $\pi$ \}
 \item $\mathbb{C}$ = Complexe getallen: $j^2 = -1$, $j = $ imaginaire eenheid
\end{itemize}
\important{Definitie}
$z = \mathimportant{a}{orange} + \mathimportant{b}{orange}j$ met $z \in \mathbb{C}$, $a \in \mathbb{R}, b \in \mathbb{R}$ en $j = \sqrt{-1}$ met
\begin{itemize}
 \item $Re(z) = a$
 \item $Im(z) = b$
\end{itemize}
\important{3 Vormen}
\begin{itemize}
 \item Cartesische vorm: $z = a + bj$
 \item Goniometrische vorm: $z = r[cos(\theta) + jsin(\theta)]$
 \item Exponentiële vorm: $re^{j\theta}$
\end{itemize}
\important{a en b}
\begin{itemize}
 \item $a = rcos(\theta)$
 \item $b = rsin(\theta)$
\end{itemize}
\important{r en $\theta$}
\begin{itemize}
 \item $r \geq 0$
 \item $r = \sqrt{a^2 + b^2}$
 
 \item $\theta \in [0, 2\pi]$
 \item $\theta \in ]-\pi, \pi[$
 \item $tg(\theta) = \frac{b}{a} (+ \pi)$
\end{itemize}
\important{Complex toegevoegde}
\begin{itemize}
 \item Cartesische vorm: $\overline{z} = a - bj$ 
 \item Exponentiële vorm: $\overline{z} = re^{-j\theta}$
\end{itemize}
\important{Bewerkingen}
\begin{itemize}
 \item $z_1 + z_2$
 \item $z_1 . z_2 = (r_1 . r_2)e^{j(\theta_1 + \theta_2)}$
 \item $\frac{z_1}{z_2} = \frac{r_1}{r_2}e^{j(\theta_1 - \theta_2)}$
 \item $z^{n} = r^{n}e^{jn\theta}$
 \item $\sqrt[n]{z} = r^{\frac{1}{n}}e^{j\frac{1}{n}}(\theta + 2k\pi)$
\end{itemize}

\chapter{Limieten}
\important{Limiet naderen}
\begin{itemize}
 \item $\lim_{x\to\infty} f(x) = \lim_{x\to-\infty} en \lim_{x\to+\infty}$
 \item $\lim_{x\to a} f(x) = \lim_{x\to a^-} en \lim_{x\to a^+}$ met $a \in \mathbb{R}$
\end{itemize}
\important{Bijzondere limieten}
\begin{itemize}
 \item $\lim_{x\to\infty} 
 \frac{a_nx^n + a_{n - 1}x^{n - 1} + ... + a_1x + a_0}{b_kx^k + b_{k - 1}x^{k - 1} + ... + b_1x + b_0}
 = \frac{a_nx^n}{b_kx^k}$
 \item $\lim_{x\to0} \frac{sin(x)}{x} = 1$
 \item $\lim_{x\to0} \frac{tg(x)}{x} = 1$
 \item $\lim_{x\to\infty} (1 + x)^{\frac{1}{x}} = \lim_{y\to\infty}(1 + \frac{1}{y})^y = e$
\end{itemize}
\important{Onbepaaldheden}
\begin{itemize}
 \item $\frac{0}{0}$
 \item $\frac{\infty}{\infty}$
 \item $+ \infty - \infty$
 \item $0 . \infty$
 \item $0^0$
 \item $\infty^0$
 \item $1^\infty$
\end{itemize}
\important{Wegwerken onbepaaldheden}
\begin{itemize}
 \item Gemeenschappelijke factor van teller en noemer vinden
 \item Toegevoegde waarde van teller, noemer of beiden
 \item $f(x) * g(x) = \frac{f(x)}{1/g(x)}$
 \item $f(x)^{g(x)} = e^{ln(f(x)^{g(x)})}$
 \begin{enumerate}[label={}]
     \item Geldig voor:
     \item $0^0$
     \item $\infty^0$
  \end{enumerate}
 \item $\lim_{x\to...} (1 + g(x))^{\frac{1}{g(x)}}$
  \begin{enumerate}[label={}]
       \item Geldig voor:
    \item $1^\infty$
  \end{enumerate}

\end{itemize}

\chapter{Afgeleiden}
\important{Berekenen van afgeleiden}
\begin{itemize}
 \item $(f . g)' = f'. g + f.g'$
 \item $(\dfrac{f}{g})' = \dfrac{f'.g - f.g'}{g^2}$
 \item $(f^{n})' = nf^{n-1} . f'$
 \item $(ln f)' = \dfrac{f'}{f}$
 \item $(e^f)' = e^f .f'$
 \item $(bgtg\;x)' = \dfrac{1}{1 + x^2}$
 \item $(bgsin\;x)' = \dfrac{1}{\sqrt{1 + x^2}}$
 \item $(bgcos\;x)' = \dfrac{-1}{\sqrt{1 + x^2}}$
\end{itemize}
\important{Regel van L'Hopital}
 \begin{itemize}
  \item Als $\lim_{x\to...} \dfrac{f}{g} = \dfrac{0}{0}\;OF\;\dfrac{\infty}{\infty}$
  \item Dan $\lim_{x\to...} \dfrac{f}{g} = \lim_{x\to...} \dfrac{f'}{g'}$
 \end{itemize}
\important{Raaklijn en normaal}
\begin{itemize}
 \item Raaklijn : $y-y_p = y_p'(x - x_p)$
 \item Normaal  : $y-y_p = \dfrac{1}{y_p'}(x - x_p)$ 
\end{itemize}
\important{Foutentheorie}
\begin{itemize}
 \item $AF(y) \approx |f'(x)| \pm AF(x)$
 \item $f(x + AF(x)) \approx f(x) + |f'(x)| \pm AF(x)$
\end{itemize}
\important{Extremum}
\begin{itemize}
 \item f heeft een extremum in x = a als
 \begin{enumerate}
  \item $f'(a) = 0$
  \item $f'$ verandert van teken in x = a
 \end{enumerate}
 $\Rightarrow$ via tekentabel
\end{itemize}
\chapter{Parameterkrommen}
\important{Notatie}
\\
\\
$ K: 
 \begin{cases}
  x(t) \\
  y(t)
 \end{cases}
$
\\
\\
\important{Afleiden}
\begin{itemize}
 \item $y' = \dfrac{dy}{dx} = \dfrac{dy/dt}{dx/dt}$
 \item $y'' = \dfrac{dy'}{dx} = \dfrac{dy'/dt}{dx/dt}$
\end{itemize}
\important{Raaklijn en normaal}
\begin{itemize}
 \item Raaklijn : $y-y_p = y_p'(x - x_p)$
 \item Normaal  : $y-y_p = \dfrac{1}{y_p'}(x - x_p)$ 
\end{itemize}
\chapter{Poolcoördinaten}
\important{Voorstelling}
\begin{itemize}
 \item \important{Cartesische coördinaten:}
\begin{tikzpicture}

\coordinate (O) at (0,0);
\coordinate (A) at (2,2);
\coordinate (X) at (2,0);
\coordinate (Y) at (0,2);

\draw[thick, ->] (O) -- (3, 0) node[anchor=north west] {x};
\draw[thick, ->] (O) -- (0, 3) node[anchor=south east] {y};
\draw[thick] (-1, 0) -- (O);
\draw[thick] (0, -1) -- (O);
\node at (O) {\textbullet};
\node[text] at (-0.25,-0.25) {0};


\node at (2,2) {\textbullet};
\node[text] at (2.3,2.3) {p(x,y)};
\end{tikzpicture}
$$
\begin{cases}
 x = r\cos\theta \\
 y = r\sin\theta
\end{cases}
$$

\item \important{Poolcoördinaten:}
\begin{tikzpicture}
\coordinate (O) at (0,0);
\coordinate (A) at (2,2);
\coordinate (X) at (2,0);


\draw[thick, ->] (O) -- (3, 0) node[anchor=north west] {po};
\draw[thick] (O) -- (A) node[anchor=south east] {r};
\node at (A) {\textbullet};
\node at (O) {\textbullet};
\node[text] at (-0.25,-0.25) {0};
\node[text] at (3, 2.3) {p(r, $\theta$)};

\draw (1,1) arc(45:0:1.4);
\node[] at (30:1.8) {$\theta$};
\end{tikzpicture}
$$
\begin{cases}
 r = \sqrt{x^2 + y^2} \\
 \theta = bgtg\dfrac{y}{x}
\end{cases}
$$
\end{itemize}
\important{Rechte}
\begin{itemize}
 \item $r\cos(\theta-\theta_0) = d_0 \;\;\;\; (d_0 > 0)$
 \begin{tikzpicture}
  \coordinate (O) at (0,0);
  \draw[thick, ->] (O) -- (3,0);
  \draw (O) -- (1,2);
  \node at (O) {\textbullet};
  \node[text] at (-0.25,-0.25) {0};
 \end{tikzpicture}
\end{itemize}
\important{Halfrechte}
\\
\important{Verloop functieonderzoek}
\begin{enumerate}
 \item \begin{enumerate}
        \item domein en beeld
        \item Periode
        \item Symmetrie
        \begin{itemize}
         \item Symmetrie t.o.v. poolas als $r(-\theta) = r(\theta)$
         \item Symmetrie t.o.v. pool als $r(\pi + \theta) = r(\theta)$
         \item Symmetrie t.o.v. $\theta \pm \dfrac{\pi}{2}$ als $r(\pi - \theta) = r(\theta)$
        \end{itemize}

        \item Snijpunten met poolas
        \item Gedrag in pool
        \item Raaklijnen pool
       \end{enumerate}
 \item Afleiden
 \begin{itemize}
  \item $r' > 0 \Rightarrow r $ stijgt enkel als $\theta$ stijgt
  \item $r' < 0 \Rightarrow r $ daalt enkel als $\theta$ stijgt
  \item $r' = 0 \Rightarrow $ raaklijn $\perp$ voerstraal. r moet $\neq$ 0 
  \item $r' = \infty \Rightarrow $ raaklijn = voerstraal. r moet $\neq$ 0
 \end{itemize}

 \item Tabel en schets

\end{enumerate}


\chapter{Dubbelintegralen}

\hline
Bereken $\iint_G x^{2}\:dx\:dy$ met G het gebied in het eerste kwadrant tussen $xy=16$; $y=x$; 
$y = 0$; $x = 8$.



  
\end{document}
