\documentclass{article}

\usepackage{amsmath}
\usepackage{mathtools}
\usepackage[utf8]{inputenc}
\title{Test Wiskunde A 27 november 2017}
\author{}
\date{}


\begin{document}
\maketitle
\begin{enumerate}
 \item Bereken de integraal: 
      \begin{align*}
       \int \cos^{2}(t) - \sin^{3}\cos^{6}dt &= \int \cos^{2}(t)dt - \int \sin^{3}\cos^{6}dt
      \end{align*}
      \begin{align*}
       \int \cos^{2}(t)dt &= \int \frac{1 + \cos(2t)}{2} dt \\
       &= \frac{1}{2} \biggl(\int dt + \int \cos(2t) dt\biggl) \\
       &= \frac{1}{2} \biggl(t + \frac{\cos(2t)}{2} \biggl) \\
       &= \frac{t}{2} + \frac{\cos(2t)}{4}
      \end{align*}
      \begin{align*}
       \int \sin^{3}\cos^{6}dt &= \int \sin(t)\sin^{2}(t)cos^{6}(t)dt \\
       &= \int \sin(t)\biggl(1 - \cos^{2}(t)\biggl)cos^{6}(t)dt \\
       &= -\int (1 - u^{2}) u^{6} du \\
       &= -\biggl(\int u^{6} du - \int u^{8} du \biggl)\\
       &= -\int u^{6} du + \int u^{8} du \\
       &= -\frac{u^{7}}{7} + \frac{u^{9}}{9} \\
       &= -\frac{\cos^{7}(t)}{7} + \frac{\cos^{9}(t)}{9} 
      \end{align*}
      \begin{align*}
	\int \cos^{2}(t)dt - \int \sin^{3}\cos^{6}dt &= \frac{t}{2} + \frac{\cos(2t)}{4} - (-\frac{\cos^{7}(t)}{7} + \frac{\cos^{9}(t)}{9} )
      \\&= \frac{t}{2} + \frac{\cos(2t)}{4} + \frac{\cos^{7}(t)}{7} - \frac{\cos^{9}(t)}{9}
     \end{align*}
 \newpage
 \item Bepaal de 2de orde afgeleide van: 
 \newpage
 \item Geef de oplossingen van $z$ in $ a + bj $ vorm: $z^{3} = -8\biggl(\frac{\sqrt{3}}{2} + \frac{1}{2}j\biggl)^{30}$
 \begin{align*}
    &r = \sqrt{\biggl(\frac{\sqrt{3}}{2}\biggl)^{2} + \biggl(\frac{1}{2}\biggl)^{2}} = 1 \\
    &\theta = \arctan{\biggl(\frac{\frac{1}{2}}{\frac{\sqrt{3}}{2}}\biggl)} = \arctan{\biggl(\frac{1}{\sqrt{3}}}\biggl) = \frac{\pi}{6} \\
    &\frac{\sqrt{3}}{2} + \frac{1}{2}j = e^{j\frac{\pi}{6}} \\
    &\Rightarrow -8(e^{j\frac{\pi}{6}})^{30} \\
    &= -8e^{j\frac{30\pi}{6}} \\ 
    &= -8e^{j5\pi} \\ 
    &= -8e^{j\pi} 
 \end{align*}
 \begin{align*}
  &z^{3} = -8e^{j\pi}  \\
  &z=\sqrt[3]{-8e^{j\pi}} \\
  &z = -2e^{j(\frac{\pi}{3} + 2k\pi)}\;\text{met k} \in \{0, 1, 2\}
 \end{align*}
 
 \begin{itemize}
  \item $z_1 = -2e^{j\frac{\pi}{3}} = -\sqrt{3} - 1j$
  \item $z_2 = -2e^{j\frac{7\pi}{3}} = -\sqrt{3} - 1j$
  \item $z_3 = -2e^{j\frac{14\pi}{3}} = -\sqrt{3} - 1j$
 \end{itemize}

\newpage
 \item Teken de rechte in poolcoördinaten: $r\sin(\theta + \frac{\pi}{3}) = 3$
 \begin{align*}
  &     r\sin(\theta + \frac{\pi}{3}) = 3  \\
  &<=> r\cos\biggl(\frac{\pi}{2} - (\theta + \frac{\pi}{3})\biggl) = 3 \\
  &<=> r\cos\biggl(\frac{\pi}{2} - \theta - \frac{\pi}{3}\biggl) = 3 \\
  &<=> r\cos\biggl(\frac{\pi}{6} - \theta\biggl) = 3 \\
  &<=> r\cos\biggl( - \theta + \frac{\pi}{6}\biggl) = 3 \\
  &<=> r\cos\biggl(\theta - \frac{\pi}{6}\biggl) = 3 
 \end{align*}


  \newpage
 \item Bereken de limiet:
      \begin{align*}
       \lim_{x\to\frac{\pi}{2}}(\sin(x))^{\frac{1}{\pi - 2x}} &= (\sin(\frac{\pi}{2}))^\frac{1}{\pi - 2\frac{\pi}{2}} = 1^{\infty} \\
       &= \lim_{x\to\frac{\pi}{2}} e^{\frac{\ln(\sin(x))}{\pi - 2x}} 
      \\ &= e^{\frac{\lim_{x\to\frac{\pi}{2}} \ln(\sin(x))}{\lim_{x\to\frac{\pi}{2}} \pi - 2x}} = e^{\frac{0}{0}}
      \\ &= \lim_{x\to\frac{\pi}{2}} e^{\frac{\frac{d[\ln(\sin(x))]}{dx}}{\frac{d[\pi - 2x]}{dx}}}
      \\ &= \lim_{x\to\frac{\pi}{2}} e^{\frac{\cot(x)}{2}} 
      \\ &=  e^{\frac{\lim_{x\to\frac{\pi}{2}}\cot(x)}{\lim_{x\to\frac{\pi}{2}}2}}
      \\ &= e^{\frac{0}{2}}
      \\ &= e^{0}
      \\ &= 1
      \end{align*} 

\end{enumerate}
 
 
\end{document}
