\documentclass{article}
\usepackage[utf8]{inputenc}
\usepackage[english]{babel}
\usepackage{color}

\def\warning#1{\color{red} #1 \color{black}}
\def\note#1{\color{cyan} #1 \color{black}}

\begin{document}

\title{Examen Wiskunde A 8 januari 2018}
\date{}
\author{}
\maketitle

\section*{Open Vragen}
	\begin{enumerate}
		\item {Beschouw de kromme $r^2(\theta) = a^2sin(3\theta)$ met $a > 0$ en $0 \leq \theta \leq \pi/3$ die het gebied G omringd.
		    \begin{enumerate}
		      	\item {Maak een schets van deze kromme en arceer het gebied G}
		      	\item {Stel de bepaalde integraal op voor het gebied G en bereken}
		    \end{enumerate}}

		\item {Gegeven de volgende matrix:
					$$A=\left(
							\matrix{-7 & -4 & 8\cr
									 4 & -1 & 2\cr
								     5 & 2 & a \cr}
						\right).$$

			\begin{enumerate}
				\item {Toon aan dat $a$ gelijk moet zijn aan 6 zodat het spectrum $\{-2, -1, 1\}$ is.}
				\item {Veronderstel $a = 6$.
					\begin{enumerate}
						\item {Bepaal de karakteristieke vergelijking, het spoor en de determinant van A}
						\item {Bepaal de eigenvector voor de hoogste waarde uit het spectrum.}
					\end{enumerate}}
			\end{enumerate}}       
	\end{enumerate}
\section*{Meerkeuzevragen}
\warning{Alle verschillende antwoorden ken ik niet meer en zullen hier dus ook niet op staan. Meestal komt het erop neer dat je de oefening maakt zoals normaal en dat je dan jouw antwoord vergelijkt met de antwoorden op het blad.}
\begin{enumerate}
	\item {Bereken de volgende integraal door de integratievolgorde om te wisselen:
		$$\int_{0}^{\pi/2} \int_{x}^{\pi/2} \frac{\sin y}{y} dydx$$
	}
	\item {Bereken rot (grad f) met f = \warning{***}.}
	\item {Geef het functievoorschrift van de raaklijn voor de functie $y = x \ln y - 2x = 0$ in het punt $y = 1$.}
	\item {Bepaal de termen indien $\frac{afx^2 + bgx + ch}{(x^4 + 2x^2 + 1)}$ opgesplitst wordt in partieëlbreuken.}
	\item {Geef de transformatiematrix indien een vector gespiegeld wordt over de eerste bissectrice.}
	\item {Geef een benadering van volgende uitdrukking door een gepaste lineaire benadering uit te voeren:
		$$ln \left({\frac{\root 3 \of 26.5}{3}}\right)$$
	}
	\item {Gegeven $\vec{r} = e_x\hat{i} + e_y\hat{j} + e_z\hat{k}.$ \note{Er worden 5 stellingen gegeven en je moet aanduiden welke dat niet waar is. De stelling zijn in de vorm van gradiënten / divergenten en rotoren berekenen. Analoog aan oefening 14 van hoofdstuk 10. Je moet dus heel snel gradiënten, divergenten en rotoren berekenen.}}
	\item {\warning{Iets met ruimtemeetkunde.}}
	\item {Bepaal het aantal gemeenschappelijke punten van $r = 4\sin t$ en $r = -\cos t$. Zijn dit raakpunten of snijpunten?}
	\item {Bepaal de punten waar de raaklijn van de kromme K evenwijdig is met de X-as.
			$$K: \cases{\ln 2t - 1\cr
						 t^2 +t\cr}.$$
			\note{Let op het domein van de eerste functie}
	}
	\item {Geef de richting waar de functie $f$ het meest verandert met $f = 3y^2\cos x - zy^4$}
	\item {Geef de oplossingen van $z^3 = -j$. \note{Bij elk antwoord werden de 3 oplossingen in een andere vorm gegeven. Je moet dus heel goed in staat zijn om snel complexe getallen om te vormen.}}
	\item {Gegeven $\frac{x^2}{2}+\frac{y^2}{16}=1$, bereken de volgende integraal door gebruik te maken van substitutie $ x = \sqrt{2}r \cos t$ en $y = 4r \sin t$ $$\int\int_G xy\,dxdy$$}
	\item {Gegeven de coördinatentransformatie $\cases{u = x^2\cr v = yu\cr}$. Wat is de waarde van het oppervlakte-element? \note{Jacobiaan berekenen}}
	\item {Bereken de volgende limiet: $$\lim_{x \to +\infty} \bigg(1 + \tan \frac{\pi}{x}\bigg)^{\pi(x - 1)}$$}
	\item {Bereken $d^2f$ met $f = e^x(y-z^2)$}
	\item {\warning{***}}
	\item {Geef de vergelijking van het vlak dat loodrecht staat op het YZ-vlak, loodrecht op y = 2 en door het punt p(1, 1, 1) gaat. \note{Één van de antwoorden was dat zo een vlak niet bestaat.}}
	\item {Bereken de volgende lijnintegraal met $\vec{F} =  \{\warning{***}, -3x\}$ en L het vierkant met punten $(0, 0), (0, 1)(1, 1)(1, 0)$ $$\oint_L \vec{F}\cdot d\vec{r}$$}
	\item {Wat is het domein van volgende functie: $f(x, y) = \sqrt{\ln(x + 2y^2)}$}

\end{enumerate}
\end{document}


% green 147 BLUE 240