\documentclass[12pt]{report} 

% PACKAGES 
\usepackage[dutch]{babel}
\usepackage[utf8]{inputenc}
\usepackage{color}
\usepackage{amsmath} % Matrices
\usepackage{booktabs}
\usepackage{xcolor}
\usepackage{sectsty}
\usepackage{lipsum}
\usepackage{enumitem}
\usepackage{pgfplots}
\pgfplotsset{compat=1.13}


\partfont{\color{brown}}
\chapterfont{\color{teal}}
\sectionfont{\color{cyan}}

% DOCUMENT INFORMATION
\title{Discrete Wiskunde}
\author{Bert De Saffel}
\date{2017-2018}


% CUSTOM COMMANDS
\newcommand{\todo}[1] {
\color{red}\textunderscore{\textit{TODO: #1}}
\color{black}
}

\newcommand{\important}[1] {\textbf{\color{orange}#1}}


% DOCUMENT
\begin{document}
\maketitle
\tableofcontents

\begin{abstract}
	Deze tekst vat de theorie van Discrete Wiskunde samen zoals die gegeven werd in het academiejaar 2017-2018. 
		 
\end{abstract}


\part{Discrete Wiskunde}
\chapter{Eindige Velden}

\section{Proloog}
\todo{veel korter schrijven}
Vooraleer velden kunnen uitgelegd worden moet eerst de inleiding van hoofdstuk \ref{ch:groepen} (Groepen) gegeven worden. 
\begin{itemize}
\item {\textbf{Groep} (symbool = \important{G}): Een verzameling elementen die elk met elkaar onderling interageren.}
\item {\textbf{Groepstabel}: Een matrix dat interacties voorstelt.}
\end{itemize}

Ter opmerking, het symbool $\oplus$ stelt het additieve voor en $\otimes$ stelt het multiplicatieve voor. Dit is een hulpmiddel
voor ons zodat we kunnen vergelijken met de + en . operator uit de wiskunde.
Er zijn 4 eigenschappen nodig om een geldige groepstabel te hebben.
\begin{itemize}
	\item {\textbf{Inwendigheid}: x $\oplus$ y = element van G}
	\item {\textbf{Associativiteit}: (a $\oplus$ b) $\oplus$ c = a $\oplus$ (b $\oplus$ c)}
	\item {\textbf{Neutraal Element (\important{n})}: x $\oplus$ n = x}
	\item {\textbf{Invers element}: x $\oplus$ $\overline{x}$ = {\important{n}}}
\end{itemize}
Een extra, maar niet verplichte eigenschap is \textbf{Commutativiteit}. x $\oplus$ y = y $\oplus$ x.

Enkele begrippen met betrekking tot groepstabel.
\begin{itemize}
	\item {\important{Latijns Vierkant}: Een groepstabel waarvan elk element exact één keer voorkomt in elke rij en kolom (denk aan Sudoku).}
	\item {\important{Isomorfe groepen}: Verschillende groepen die niets met elkaar te maken hebben kunnen isomorf zijn. Dit betekent dat ze identiek zijn na eventuele herlabeling
	of permutaties van kolommen of rijen.}
	\item {\important{Discrete groepen}: Dit zijn groepen met een eindig aantal elementen. Modulo 12 heeft zo 12 elementen.}
	\item {\important{De orde}: Enerzijds is dit getal het aantal elementen van een groep. Anderzijds is dit het 
		aantal keer dat je een element met zichzelf moet laten interageren om het neutraal element te bekomen. De orde is dus voor
	elk element verschillend.}
\end{itemize}


Als je de verzameling 'Modulo 12' bekijkt heb je 12 elementen. 0, 1, 2, 3, 4, 5, 6, 7, 8, 9, 10 en 11.
\begin{table}[]
	\centering
	\caption{Een groepstabel voor de interactie `optellen' in modulo 12}
	\begin{tabular}{|l|llllllllllll}
		\hline
		+  & \multicolumn{1}{l|}{0} & \multicolumn{1}{l|}{1} & \multicolumn{1}{l|}{2} & \multicolumn{1}{l|}{3} & \multicolumn{1}{l|}{4} & \multicolumn{1}{l|}{5} & \multicolumn{1}{l|}{6} & \multicolumn{1}{l|}{7} & \multicolumn{1}{l|}{8} & \multicolumn{1}{l|}{9} & \multicolumn{1}{l|}{10} & \multicolumn{1}{l|}{11} \\ \hline
		0  & 0                      & 1                      & 2                      & 3                      & 4                      & 5                      & 6                      & 7                      & 8                      & 9  & 10 & 11 \\ \cline{1-1}
		1  & 1                      & 2                      & 3                      & 4                      & 5                      & 6                      & 7                      & 8                      & 9                      & 10 & 11 & 0  \\ \cline{1-1}
		2  & 2                      & 3                      & 4                      & 5                      & 6                      & 7                      & 8                      & 9                      & 10                     & 11 & 0  & 1  \\ \cline{1-1}
		3  & 3                      & 4                      & 5                      & 6                      & 7                      & 8                      & 9                      & 10                     & 11                     & 0  & 1  & 2  \\ \cline{1-1}
		4  & 4                      & 5                      & 6                      & 7                      & 8                      & 9                      & 10                     & 11                     & 0                      & 1  & 2  & 3  \\ \cline{1-1}
		5  & 5                      & 6                      & 7                      & 8                      & 9                      & 10                     & 11                     & 0                      & 1                      & 2  & 3  & 4  \\ \cline{1-1}
		6  & 6                      & 7                      & 8                      & 9                      & 10                     & 11                     & 0                      & 1                      & 2                      & 3  & 4  & 5  \\ \cline{1-1}
		7  & 7                      & 8                      & 9                      & 10                     & 11                     & 0                      & 1                      & 2                      & 3                      & 4  & 5  & 6  \\ \cline{1-1}
		8  & 8                      & 9                      & 10                     & 11                     & 0                      & 1                      & 2                      & 3                      & 4                      & 5  & 6  & 7  \\ \cline{1-1}
		9  & 9                      & 10                     & 11                     & 0                      & 1                      & 2                      & 3                      & 4                      & 5                      & 6  & 7  & 8  \\ \cline{1-1}
		10 & 10                     & 11                     & 0                      & 1                      & 2                      & 3                      & 4                      & 5                      & 6                      & 7  & 8  & 9  \\ \cline{1-1}
		11 & 11                     & 0                      & 1                      & 2                      & 3                      & 4                      & 5                      & 6                      & 7                      & 8  & 9  & 10 \\ \cline{1-1}
	\end{tabular}
	\label{table:groepstabel}
\end{table}
\begin{itemize}[label={*}]
	\item Hoeveel keer moet je 1 met zichzelf optellen om 0 te bekomen? (1, 2, 3, 4, 5, 6, 7, 8, 9, 10, 11, 0)$\rightarrow$ 12
	\item Hoeveel keer moet je 2 met zichzelf optellen om 0 te bekomen? (2, 4, 6, 8, 10, 0) $\rightarrow$ 6
	\item Hoeveel keer moet je 5 met zichzelf optellen om 0 te bekomen? (5, 10, 3, 8, 1, 6, 11, 4, 9, 2, 7, 0) $\rightarrow$ 12
\end{itemize}
De orde van 1 is dus 12, die van 2 is 6 en die van 5 is ook 12.

\begin{itemize}
\item {\important{Generator}: Een element of een combinatie van elementen die, als je die met elkaar laat interageren, alle andere elementen van de groep
  tegenkomt. In de vorige voorbeelden kan je zien dat zowel 1 als 5 een generator zijn aangezien ze elk element tegenkomen.
Je kan ook elementen combineren om een generator te vormen. Zo is $<2, 3>$ ook een generator want (2 + 2 + 3 + 3 + 3) $\mod$ 12 = 1.}
\end{itemize}


Tot nu toe hebben we enkel additieve groeptabellen gezien. In tabel \ref{table:groepstabel} kan je zien dat de groep Modulo 12 
voor de bewerking $\oplus$ een discrete groep is

Multiplicatie is geen groepstabel tenzij we de 0 uitsluiten en als de groep $n$ aantal element bevat 
waarbij $n$ een priemgetal is. Dit wordt duidelijk gemaakt in het onderdeel Priemvelden

Een cyclische groep is een groep dat slechts 1 element heeft als generator. 
$i = (29w^i) \% 36$
\newline
29 = invers element via algemeen algoritme van euclides

\section{Priemvelden}
Priemvelden is een eerste methode om een cyclische groep te vinden van een bepaalde orde.
\begin{itemize}
	\item \important{Veld}: Verzameling F van elementen \{a, b, c, ...\} die onderling interageren via een interactie $\oplus$ en via 
	      een interactie $\otimes$.
	      	      
	\item \important{Veldaxioma's}
	      \begin{enumerate}
	      	\item (F, $\oplus$) $\rightarrow$ Een additieve groep met neutraal element \important{0} (\important{nul}element)
	      	\item (F$\backslash$0 , $\otimes$) $\rightarrow$ Multiplicatieve groep met neutraal element \important{1} 
	      	      (\important{eenheids}element)
	      	\item Distributieve eigenschap: a $\otimes$ (b $\oplus$ c) = (a $\otimes$ b) $\oplus$ (a $\otimes$ c)
	      \end{enumerate}
	\item \important{Eindig veld}: Veld met eindig aantal element n
	\item Voor elke orde n is er maximum één enkel eindig veld met orde n
	\item De multiplicatieve groep (F$\backslash$0 , $\otimes$) van een eindig veld (F, $\oplus$ , $\otimes$) is cyclisch
\end{itemize}
Hoe kan het veld met orde n geconstrueerd worden met groepstabellen (F, $\oplus$) en (F, $\otimes$)? $\rightarrow$ Alle eindige velden
kunnen met behulp van één van twee methodes geconstrueerd worden.

(Z$_p$, +, .) vormt een veld met nulelement \important{0} en eenheidselement \important{1} op voorwaarde dat p een priemgetal is. Dit heet een
\important{priemveld}. Hieruit volgt dat (Z$_p\backslash$0, .) een cyclische groep is indien p een priemgetal is.


\subsection{Priemontbinding}
\todo{todo}



\subsection{Algoritme van Euclides}
\begin{itemize}
	\item Numerieke uitwerking van het \important{algoritme van Euclides}: \newline
	      	      
	      \begin{tabular}{l l l}
	      	99  &                            &     \\
	      	-84 & \important{1} (84 kan \important{1} keer in 99 99 - 84 = 15)  & 84  \\
	      	15  & \important{5} (15 kan \important{5} keer in 84) & -75 \\
	      	-9  & 1  (1 keer in 15 = 1 * 9)  & 9   \\
	      	6   & 1   (1 keer in 9 = 1 * 6)  & -6  \\
	      	-6  & 2                          & 3   \\
	      	0   &                            &     
	      \end{tabular}
	      	      
	      	      
\end{itemize}


 pi = 3 + 1/(7 + 1/(15))...
 
\subsection{Multiplicatieve functies}
\subsubsection{Euler $\phi$} 
\important{Definitie}: Hoeveel getallen kleiner dan x zijn priemgetallen?
\newline
Voorbeeld: 90 = $2 * 3^2 * 5$
\newline $\phi(90) = 90 * (1/2) * (2/3) * (4/5) = $ \important{24}

\subsection{Möbius $\mu$}
Heeft 3 uitkomsten: -1, 0 of 1.
\begin{itemize}
 \item \important{0}: Minstens één van de priemfactoren komt meer dan 1 keer voor.
 \item \important{-1}: Oneven aantal priemfactoren
 \item \important{1}: Even aantal priemfactoren
 \item $\mu(35) = 1 $, want $ 5 * 7 = even aantal priemfactoren$
\end{itemize}
 
\important{Möbius Inversie}:
\begin{itemize}
 \item Wordt gebruikt om $\phi(x)$ te berekenen.
 \item $\phi(x)$ = $\mu(1) * (90/1)$ 
  \newline +  $\mu(2) * (90/2)$
  \newline +  $\mu(3) * (90/3)$
  \newline +  $\mu(5) * (90/5)$
  \newline +  $\mu(10) * (90/10)$
  \newline +  $\mu(15) * (90/15)$
  \newline +  $\mu(30) * (90/30)$
\end{itemize}

\subsection{Primitieve wortels}
Interacties met zichzelf om uiteindelijk 1 uit te komen. 
Bij de primitieve wortel moet je de $orde$ aantal keer interageren
met zichzelf om 1 uit te komen
\newline
stel $p = 601$
600 = aantal elementen 
600 = $2^3 * 3 * 5^2$
\newline
rooster opstellen \todo{rooster}
$8 * 3 * 25 = 600$ is het enige dat als uitkomst 1 mag hebben,
zonder computer kan je best enkel de laatste aansluitingen controleren

\subsection{Discrete logaritmen}
Bepaal index $i$ van het getal $x$ van verzameling $p$ met primitieve wortel $w$
\newline
VB: $p = 401, w=3, i=13$
\subsection{Naïve methode}

\subsection{Baby-step Giant-step}

\section{Galoisvelden}
\chapter{Grafen}
\section{Terminologie}
\begin{itemize}
 \item \important{Graad} van een knoop: aantal buren van een knoop.
 \item Aantal \important{bogen}: de helft van de som van de graden van alle knopen.
 \item $\delta$: De kleinste graad in een graaf.
 \item $\Delta$: De grootste graad in een graaf.
 \item Een graaf is \important{regulier} als $\delta = \Delta$.
 \item Een graaf is \important{compleet} als elke knoop met elke andere knoop is verbonden.
 \item \important{Cykelgraaf}: Een graaf waarbij elke knoop 2 buren heeft.
 \item \important{Wandelen}: Van knoop naar knoop gaan.
 \item \important{Pad}: Elk knooppunt mag maar één keer in een wandeling voorkomen.
 \item \important{Brug}: Een boog dat 2 grafen verbindt.
 \item \important{Gerichte graaf}: Graaf waar de bogen een richting hebben.
 \item \important{Gewogen graaf}: Graaf waar elke boog een kost heeft.
 \item \important{Subgraaf}: Een graaf G'' waarvan de knopen - en bogenverzameling een deelverzameling is van een graaf G'.
 \item \important{Isomorfe graaf}: Een graaf G'' die, als je knopen en bogen verlegt, overeen komt met graaf G'.
 \item \important{Bipartiete graaf}: De graaf is gepartitioneerd in partities. Enkel bogen tussen partities zijn mogelijk.
 \item \important{Complete Bipartiete graaf}: Elke knoop van één partitie is verbonden met elke knoop uit de andere partitie.
 \item \important{Kubus}: Een bipartiete graaf met $2^n$ knopen, binair geïdentificeerd, met bogen tussen knopen waarvan de identificatie 1 bit verschilt. 
\end{itemize}
\begin{tikzpicture}
 \node[text] at (-0.25,-0.25) {0};
 \node[text] at (2.25,-0.25) {1};
 \draw[thick] (0,0) -- (2,0);
 
 \node[text] at (3.75,-0.25) {00};
 \node[text] at (6.25,-0.25) {01};
 \node[text] at (3.75, 2.25) {10};
 \node[text] at (6.25, 2.25) {11};
 \draw[thick] (4,0) -- (6,0); % links onder naar rechts onder
 \draw[thick] (4,0) -- (4,2); % links onder naar links boven
 \draw[thick] (4,2) -- (6,2); % links boven naar rechts boven
 \draw[thick] (6,2) -- (6,0); % rechts boven naar rechts onder
 
 \node[text] at (7.75,-0.25) {000};
 \node[text] at (10.25,-0.25) {001};
 \node[text] at (7.75, 2.25) {100};
 \node[text] at (10.25, 2.25) {101};
 \node[text] at (8.25,0.5) {\tiny{101}};
 \node[text] at (8.25, 1.5) {\tiny{110}};
 \node[text] at (9.75, 0.5) {\tiny{011}};
 \node[text] at (9.75, 1.5) {\tiny{111}};
 \draw[thick] (8,0) -- (10,0); % links onder naar rechts onder
 \draw[thick] (8,0) -- (8,2); % links onder naar links boven
 \draw[thick] (8,2) -- (10,2); % links boven naar rechts boven
 \draw[thick] (10,2) -- (10,0); % rechts boven naar rechts onder
 \draw[thick] (8,0) -- (8.5,0.5);
 \draw[thick] (8,2) -- (8.5, 1.5);
 \draw[thick] (10,2) -- (9.5, 1.5);
 \draw[thick] (10,0) -- (9.5, 0.5);
 
 \draw[thick] (8.5,0.5) -- (8.5, 1.5);
 \draw[thick] (8.5, 1.5) -- (9.5, 1.5);
 \draw[thick] (9.5, 1.5) -- (9.5, 0.5);
 \draw[thick] (9.5, 0.5) -- (8.5, 0.5);
\end{tikzpicture}

\section{Connectiviteit}
\subsection{Eulercircuit - \important{Niet kennen}}
Elke boog mag maar één keer in een wandeling voorkomen.
\subsection{Hamiltoncircuit}
Elk knooppunt mag maar één keer in een wandeling voorkomen. Hier gelden twee voorwaarden (\important{Niet te kennen})

De \important{Gray Code} is een Hamiltoncircuit in een kubus. 
Zie slide 30 van 2a.pptx. Op het examen komt dit in de vorm : \important{Geef gray code van een dimensie 3}.

\begin{tikzpicture}
 \node[text] at (-0.25,-0.25) {0000};
 \node[text] at (2.25,-0.25) {0001};
 \draw[thick, ->, red] (0,0) -- (2,0);
 
 \node[text] at (3.75,-0.25) {0000};
 \node[text] at (6.25,-0.25) {0001};
 \node[text] at (3.75, 2.25) {0011};
 \node[text] at (6.25, 2.25) {0010};
 \draw[thick, ->,red] (4,0) -- (6,0); % links onder naar rechts onder
 \draw[thick] (4,0) -- (4,2); % links onder naar links boven
 \draw[thick, ->,red] (6,2) -- (4,2); % links boven naar rechts boven
 \draw[thick, ->,red] (6,0) -- (6,2); 
 
 \node[text] at (7.75,-0.25) {000};
 \node[text] at (10.25,-0.25) {001};
 \node[text] at (7.75, 2.25) {100};
 \node[text] at (10.25, 2.25) {101};
 \node[text] at (8.25,0.5) {\tiny{101}};
 \node[text] at (8.25, 1.5) {\tiny{110}};
 \node[text] at (9.75, 0.5) {\tiny{011}};
 \node[text] at (9.75, 1.5) {\tiny{111}};
 \draw[thick] (8,0) -- (10,0); % links onder naar rechts onder
 \draw[thick] (8,0) -- (8,2); % links onder naar links boven
 \draw[thick] (8,2) -- (10,2); % links boven naar rechts boven
 \draw[thick] (10,2) -- (10,0); % rechts boven naar rechts onder
 \draw[thick] (8,0) -- (8.5,0.5);
 \draw[thick] (8,2) -- (8.5, 1.5);
 \draw[thick] (10,2) -- (9.5, 1.5);
 \draw[thick] (10,0) -- (9.5, 0.5);
 
 \draw[thick] (8.5,0.5) -- (8.5, 1.5);
 \draw[thick] (8.5, 1.5) -- (9.5, 1.5);
 \draw[thick] (9.5, 1.5) -- (9.5, 0.5);
 \draw[thick] (9.5, 0.5) -- (8.5, 0.5);
\end{tikzpicture}
\\
In excel\todo{...}: \begin{tabular}{l}
           000\important{0} \\
           000\important{1} \\
           \hline
           001\important{1} \\
           001\important{0}\\
           \hline
           0110\\
           0111\\
           0101\\
           0100\\
              \hline
           1100\\
           1101\\
           1111\\
           1110\\
           1010\\
           1011\\
           1001\\
           1000
        
          \end{tabular}


\subsection{Boogconnectiviteit \important{Niet kennen}}
\subsection{Knoopconnectiviteit}
De knoopconnectiviteit is het aantal elementen in de kleinste knoopsnede van een graaf. Een knoopsnede is een soort brug dat bestaat uit meerdere knopen.
Als deze knoopsnede verwijderdt wordt uit de graaf is de graaf niet meer samenhangend, maar bestaat dan uit 2 grafen.
Voor een willekeurige samenhangende graaf : $K \leq \lambda \leq \delta$. zie slides pls

\section{Matrixvoorstellingen}
\subsection{Adjacant (\important{''Waardeloos'' - Moreau})}
\subsection{Incidentiematrix}
\section{Bomen}
\subsection{Examenvraag 1: \important{Identificeer een boom volgens Prüfers methode}}
De knoop met het kleinste label en slechts één buurt heeft verwijderen en het label opschrijven aan welke knoop deze knoop vasthangde.
Het eindresultaat moeten 2 knopen zijn.

\begin{tikzpicture}
 \coordinate (2) at (0, 2);
 \coordinate (T2) at (-0.25, 1.75);
 
 \coordinate (4) at (0, 0);
 \coordinate (T4) at (-0.25, -0.25);
 
 \coordinate (3) at (2, 0);
 \coordinate (T3) at (1.75, -0.25);
 
 \coordinate (6) at (2, -2);
 \coordinate (T6) at (1.75, -2.25);
 
 \coordinate (1) at (4, 0);
 \coordinate (T1) at (3.75, -0.25);
 
 \coordinate (5) at (5, 1);
 \coordinate (T5) at (4.75, 1);
 
 \coordinate (7) at (5, -1);
 \coordinate (T7) at (4.75, -1.25);
 
 \draw (2) -- (4);
 \draw (4) -- (3);
 \draw (3) -- (6);
 \draw (3) -- (1);
 \draw (1) -- (5);
 \draw (1) -- (7);
 
 \node[text] at (1) {\textbullet};
 \node[text] at (2) {\textbullet};
 \node[text] at (3) {\textbullet};
 \node[text] at (4) {\textbullet};
 \node[text] at (5) {\textbullet};
 \node[text] at (6) {\textbullet};
 \node[text] at (7) {\textbullet};
 
 \node[text] at (T1) {1};
 \node[text] at (T2) {2};
 \node[text] at (T3) {3};
 \node[text] at (T4) {4};
 \node[text] at (T5) {5};
 \node[text] at (T6) {6};   
 \node[text] at (T7) {7};   
\end{tikzpicture}
\\
De knoop met het kleinste label is \important{2} en hangt vast aan \important{4}.
$\sigma$ = 4

\begin{tikzpicture}
 
 \coordinate (4) at (0, 0);
 \coordinate (T4) at (-0.25, -0.25);
 
 \coordinate (3) at (2, 0);
 \coordinate (T3) at (1.75, -0.25);
 
 \coordinate (6) at (2, -2);
 \coordinate (T6) at (1.75, -2.25);
 
 \coordinate (1) at (4, 0);
 \coordinate (T1) at (3.75, -0.25);
 
 \coordinate (5) at (5, 1);
 \coordinate (T5) at (4.75, 1);
 
 \coordinate (7) at (5, -1);
 \coordinate (T7) at (4.75, -1.25);
 
 \draw (4) -- (3);
 \draw (3) -- (6);
 \draw (3) -- (1);
 \draw (1) -- (5);
 \draw (1) -- (7);
 
 \node[text] at (1) {\textbullet};
 \node[text] at (3) {\textbullet};
 \node[text] at (4) {\textbullet};
 \node[text] at (5) {\textbullet};
 \node[text] at (6) {\textbullet};
 \node[text] at (7) {\textbullet};
 
 \node[text] at (T1) {1};
 \node[text] at (T3) {3};
 \node[text] at (T4) {4};
 \node[text] at (T5) {5};
 \node[text] at (T6) {6};   
 \node[text] at (T7) {7};   
\end{tikzpicture}
\\
De knoop met het kleinste label is \important{4} en hangt vast aan \important{3}.
$\sigma$ = 4, 3

\begin{tikzpicture}
 
 \coordinate (3) at (2, 0);
 \coordinate (T3) at (1.75, -0.25);
 
 \coordinate (6) at (2, -2);
 \coordinate (T6) at (1.75, -2.25);
 
 \coordinate (1) at (4, 0);
 \coordinate (T1) at (3.75, -0.25);
 
 \coordinate (5) at (5, 1);
 \coordinate (T5) at (4.75, 1);
 
 \coordinate (7) at (5, -1);
 \coordinate (T7) at (4.75, -1.25);
 
 \draw (3) -- (6);
 \draw (3) -- (1);
 \draw (1) -- (5);
 \draw (1) -- (7);
 
 \node[text] at (1) {\textbullet};
 \node[text] at (3) {\textbullet};
 \node[text] at (5) {\textbullet};
 \node[text] at (6) {\textbullet};
 \node[text] at (7) {\textbullet};
 
 \node[text] at (T1) {1};
 \node[text] at (T3) {3};
 \node[text] at (T5) {5};
 \node[text] at (T6) {6};   
 \node[text] at (T7) {7};   
\end{tikzpicture}
\\
De knoop met het kleinste label is \important{5} en hangt vast aan \important{1}.
$\sigma$ = 4, 3, 1

\begin{tikzpicture}
 
 \coordinate (3) at (2, 0);
 \coordinate (T3) at (1.75, -0.25);
 
 \coordinate (6) at (2, -2);
 \coordinate (T6) at (1.75, -2.25);
 
 \coordinate (1) at (4, 0);
 \coordinate (T1) at (3.75, -0.25);

 \coordinate (7) at (5, -1);
 \coordinate (T7) at (4.75, -1.25);
 
 \draw (3) -- (6);
 \draw (3) -- (1);
 \draw (1) -- (7);
 
 \node[text] at (1) {\textbullet};
 \node[text] at (3) {\textbullet};
 \node[text] at (6) {\textbullet};
 \node[text] at (7) {\textbullet};
 
 \node[text] at (T1) {1};
 \node[text] at (T3) {3};
 \node[text] at (T6) {6};   
 \node[text] at (T7) {7};   
\end{tikzpicture}
\\
De knoop met het kleinste label is \important{6} en hangt vast aan \important{3}.
$\sigma$ = 4, 3, 1, 3

\begin{tikzpicture}
 
 \coordinate (3) at (2, 0);
 \coordinate (T3) at (1.75, -0.25);
 
 \coordinate (1) at (4, 0);
 \coordinate (T1) at (3.75, -0.25);

 \coordinate (7) at (5, -1);
 \coordinate (T7) at (4.75, -1.25);
 
 \draw (3) -- (1);
 \draw (1) -- (7);
 
 \node[text] at (1) {\textbullet};
 \node[text] at (3) {\textbullet};
 \node[text] at (7) {\textbullet};
 
 \node[text] at (T1) {1};
 \node[text] at (T3) {3};  
 \node[text] at (T7) {7};   
\end{tikzpicture}
\\
De knoop met het kleinste label is \important{3} en hangt vast aan \important{1}.
$\sigma$ = 4, 3, 1, 3, 1

\begin{tikzpicture}
 \coordinate (1) at (4, 0);
 \coordinate (T1) at (3.75, -0.25);

 \coordinate (7) at (5, -1);
 \coordinate (T7) at (4.75, -1.25);
 
 \draw (1) -- (7);
 
 \node[text] at (1) {\textbullet};
 \node[text] at (7) {\textbullet};
 
 \node[text] at (T1) {1};
 \node[text] at (T7) {7};   
\end{tikzpicture}
\\
2 knopen resterend. Eindresultaat: $\sigma$ = 4, 3, 1, 3, 1
\subsection{Examenvraag 2: \important{Teken een boom volgens Prüfers methode}}
$\sigma$ = 4, 3, 1, 3, 1
\\
$S$ = {1, 2, 3, 4, 5, 6, 7}

\todo{uitwerken}
2 komt niet voor in $\sigma$, 4$\rightarrow$ 2
\\
4 komt niet voor in $\sigma$ 3$\rightarrow$4

\subsection{Binaire boom}
Een hiërarchische boom waar op elk niveau maximaal 2 aftakkingen mogelijk zijn. Het \important{Catalan} getal zegt hoeveel bomen er 
zijn met $n$ aantal knopen. 
$$ C_n = \frac{1}{n + 1}\binom{2n}{n}$$


\label{ch:groepen}
\end{document}
