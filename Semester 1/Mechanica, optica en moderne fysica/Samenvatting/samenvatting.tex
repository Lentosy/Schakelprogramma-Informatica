\documentclass[12pt]{report} 

% PACKAGES 
\usepackage[dutch]{babel}
\usepackage[utf8]{inputenc}
\usepackage{color}
\usepackage{amsmath} % Matrices
\usepackage{booktabs}
\usepackage{xcolor}
\usepackage{sectsty}
\usepackage{lipsum}



\partfont{\color{brown}}
\chapterfont{\color{teal}}
\sectionfont{\color{cyan}}

% DOCUMENT INFORMATION
\title{Fysica: mechanica, optica en moderne fysica}
\author{Bert De Saffel}
\date{2017-2018}


% CUSTOM COMMANDS
\newcommand{\todo}[1] {
\color{red}\textunderscore{\textit{TODO: #1}}
}

\newcommand{\important}[1] {\textbf{\color{orange}#1}}

% DOCUMENT
\begin{document}
\maketitle
\tableofcontents

\chapter{Foutentheorie}
\section{Vorm}
\important{Gemeten waarde} $\pm$ \important{absolute fout(AF)}
\begin{itemize}
 \item 6,458 $\pm$ 0,027 mV
 \item 8,67  $\pm$ 0.05 . 10$^3$m
\end{itemize}
\important{Relative Fout(RF)} = $\frac{AF}{Gemeten\;waarde}$
\begin{itemize}
 \item RF = $\frac{0,027}{6,458} = 0.04 = 0.4\%$
 \item RF = $\frac{0,05\;.\;10^3}{8,67} = 5.77 = 577\%$
\end{itemize}
\section{Soorten}
\begin{enumerate}
 \item Fout op meting
 \item Statistische fout
 \item Fout op berekening
\end{enumerate}

\subsection{Fout op meting}
\begin{itemize}
 \item Is afhankelijk van de nauwkeurigheid van het meettoestel
 \item Op een meetlat: $\pm 1mm$
 \item Op een chronometer: $\pm 0.01s$
\end{itemize}
Meten van de lengte van een tafel met een meetlat: $5 \pm 1.10^{-3} m$ 

\subsection{Statistische fout}

Dezelfde lengten van een tafel 5 keer meten met een meetlat = 
Het gemiddelde nemen van de gemeten waarden en het gemiddelde van de absolute
fouten


\subsection{Fout op berekening}
Voor de voorbeelden worden volgende X en Y gebruikt: \newline
$X = 16,5 \pm 0.5 $
\newline
$Y = 237,1 \pm 0.9 $
\begin{itemize}
 \item \important{Som/Verschil}: $AF(R) = \sqrt{AF(X)^2 + AF(Y)^2}$
 \begin{enumerate}
  \item $X + Y = ?$
  \item $AF(R) = \sqrt{0,5^2 + 0,9^2}$
  \item $AF(R) = \sqrt{1,06}$
  \item $AF(R) = 1,03$
  \item $16,5 + 237,1 \pm 1,03$
  \item \important{$253,6 \pm 1,0$}
 \end{enumerate}
 \begin{enumerate}
  \item $X - Y = ?$
  \item $AF(R)_{X-Y} = AF(R)_{X+Y}$ 
  \item \important{$220,6 \pm 1,0$}
 \end{enumerate}
 \item \important{Product/Deling}: $RF(R) = \sqrt{RF(X)^2 + (RF(Y)^2}$
  \begin{enumerate}
   \item $X * Y = ?$
   \item $RF(R) = \sqrt{(\frac{0,5}{16,5})^2 + (\frac{0,9}{237,1})^2}$
   \item $RF(R) = 0,03$
   \item $16,5 * 237,1 = 3912,15$
   \item $AF(R) = 3912,15 * RF(R)$
   \item $AF(R) = 3912,15 * 0,03$
   \item $AF(R) = 117,38$
   \item \important{$3912,2 \pm 117,4$}
  \end{enumerate}
  \begin{enumerate}
   \item $\frac{X}{Y} = ?$
   \item $RF(R)_{\frac{X}{Y}} = RF(R)_{X * Y}$
   \item $RF(R) = 0,03$
   \item $\frac{16,5}{237,1} = 0,0696$
   \item $AF(R) = 0,0696 * 0.03$
   \item $AF(R) = 0,0021$
   \item \important{$0,0696 \pm 0,0021$}
  \end{enumerate}
  \item {\important{Macht/Wortel}: $RF(R) = nRF(X)$}
  \begin{enumerate}
   \item $x^3 = ?$
   \item $RF(R) = 3RF(X)$
   \item $RF(R) = 0,09$
   \item $(16,5)^3 \pm 0,09$
   \item \important{$4492,13 \pm 0,09$}
  \end{enumerate}
  \begin{enumerate}
   \item $\sqrt[3]{x} = ?$ 
   \item $x^{\frac{1}{3}}$
   \item $RF(R) = \frac{1}{3}RF(X)$
   \item $RF(R)  = 0,1$
   \item $\sqrt[3]{16,5} \pm 0,1$
   \item \important{$2,5 \pm 0,1$}
  \end{enumerate}
  
  \item{\important{Functies}}
  \begin{enumerate}
   \item $tg(45\;45' \pm 3') = ?$
   \item $3' = \frac{3}{60} graden = \frac{\pi}{3600}rad$
   \item $AF(tg(X)) = \frac{1}{cos^2x}.AF(X)$
   \item $AF(tg(X)) = \frac{1}{cos^2x}*\frac{\pi}{3600}$
   \item $AF(tg(X)) = 0,0018$
   \item $tg(45\;45') \pm 0,0018$
   \item \important{$1,0265 \pm 0,0018$}
  \end{enumerate}



\end{itemize}




\part{Mechanica}
\chapter{Beweging in 2 en 3 dimensies}
\section{Begrippen}
\begin{itemize}
\item {\important{Vector}: Een eenheid dat zowel een richting als een hoeveelheid heeft. Dit wordt voorgesteld door een lijnstuk met een pijl dat de richting aangeeft en de lengte die de grootte aangeeft.}
\item {\important{Eenheidsvector}: Een vector met als grootte 1.}
\item {\important{Netto verplaatsing}: De totale afstand van punt a tot punt b in vogelvlucht.}
\item {\important{Afgelegde afstand}: De totale afstand die afgelegd werd om van punt a tot punt b te bekomen.}
\end{itemize}

\section{Formules}
\begin{itemize}
  \item {\important{Snelheid}: $v = \frac{s}{t}$}
  \item {\important{Versnelling}: $a = \frac{v}{t}$}
  \item {\important{Projectile motion}: $y = x.tan(\theta_0) - \frac{g}{2v^{2}_0cos^2\theta_0}*x^2$}
  \item {\important{Uniforme circulaire beweging}: $a = \frac{v^2}{r}$}
\end{itemize}

\section{Conceptvraag: Helikopter}
\begin{itemize}
\item {\textbf{Vraag}: Een helikopter vliegt horizontaal en laat in positie A een kist met hulpgoederen vallen. Welke baan volgt die kist (wrijving met de lucht verwaarlozen)?}
\item {\textbf{Antwoord}: \begin{enumerate}
    \item {Baan (a) kan niet, de kist zal nooit achteruit gaan.}
    \item {Baan (b) kan niet, dit zou enkel gebeuren als de helikopter stil staat.}
    \item {Baan (c) kan niet, de zwaartekracht blijkt geen impact te hebben op de kist.}
    \item {Baan(d) is correct. De kist zal eerst de snelheid van het vliegtuig overnemen, en dan zo alsmaar sneller
        naar beneden vallen door de zwaartekracht.}
    \item {Baan(e) kan niet, de kist blijft te lang op dezelfde hoogte.}
    \end{enumerate}}
\end{itemize}


\section{Conceptvraag: Snelheid en versnelling}
\begin{itemize}
\item {\textbf{Vraag}: Gegeven zijn de snelheid en versnelling van een bewegende persoon. In welk geval vertraagt de persoon en wijkt af naar rechts(vanuit het standpunt van de persoon)}
\item {\textbf{Antwoord}: \begin{enumerate}
    \item {Bij (A) blijft de snelheid constant.}
    \item {Bij (B) vertraagt de snelheid en wijkt af naar rechts.}
    \item {Bij (C) versnelt de snelheid en wijkt af naar links.}
    \item {Bij (D) versnelt de snelheid en wijkt af naar rechts.}
      \item {Bij (E) vertraagt de snelheid.}
    \end{enumerate}}
\end{itemize}



\chapter{Kracht en beweging}
\section{Begrippen}
\begin{itemize}
\item {\important{Eerste wet van Newton}: Een voorwerp in uniforme beweging blijft in uniforme beweging. Een voorwerp in 
rust blijft in rust.}
\item {\important{Tweede wet van Newton}: De verandering in snelheid is gelijk aan de netto kracht die uitgeoefend wordt op het voorwerp}
\item {\important{Derde wet van Newton}: Als voorwerp A een kracht uitoefend op voorwerp B, dan zal B een tegengestelde kracht uitoefenen op A}

\end{itemize}

\section{Formules}
\begin{itemize}
  \item {\important{Tweede wet van Newton}: 
  $\underline{\overrightarrow{F}_{net} = m\overrightarrow{a}}$ met $\overrightarrow{F}_{net}$ de som van de vectoren van alle krachten die worden uitgeoefend 
  op het voorwerp, en $ma$ het product van de massa van het voorwerp en zijn versnelling.}
  \item {\important{Gewicht}: $\underline{\overrightarrow{w} = m\overrightarrow{g}}$ met $m$ de massa van het voorwerp en $g$ de gravitatieconstante}
  \item {\important{Wet van Hooke (veren)}: $\underline{F_{s} = -kx}$ met $k$ de krachtconstante van de veer en $x$ de afstand}
  \item {\important{Lineair momentum}: 
  $\underline{\overrightarrow{p} = m\overrightarrow{v}}$ 
  met $\overrightarrow{p}$ de impuls, $m$ de massa en $\overrightarrow{v}$ de snelheid.
  }
\end{itemize}
\chapter{Toepassen wetten van Newton}
\section{Begrippen}
\begin{itemize}
\item {\important{Wrijving}:}
\end{itemize}

\section{Formules}
\begin{itemize}

\item formule
\end{itemize}


\chapter{Arbeid, energie en vermogen}

\chapter{Behoud van energie}

\chapter{Systemen van deeltjes}

\chapter{Rotatiebewegingen}

\chapter{Rotatie vektoren en impulsmoment}

\chapter{Statisch evenwicht}

\chapter{Trillingen}

\chapter{Golven}

\chapter{Electromagnetische golven}

\part{Optica}

\chapter{Breking en terugkaatsing}

\chapter{Beelden en optische instrumenten}

\chapter{Interferentie en diffractie}

\chapter{Deeltjes en golven}

\part{Kernfysica}
\chapter{Kernfysica}



\end{document}
