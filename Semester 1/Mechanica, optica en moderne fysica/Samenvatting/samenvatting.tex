\documentclass[12pt]{report} 

% PACKAGES 
\usepackage[dutch]{babel}
\usepackage[utf8]{inputenc}
\usepackage{color}
\usepackage{amsmath} % Matrices
\usepackage{booktabs}
\usepackage{xcolor}
\usepackage{sectsty}
\usepackage{lipsum}


\partfont{\color{brown}}
\chapterfont{\color{teal}}
\sectionfont{\color{cyan}}

% DOCUMENT INFORMATION
\title{Fysica: mechanica, optica en moderne fysica}
\author{Bert De Saffel}
\date{2017-2018}


% CUSTOM COMMANDS
\newcommand{\todo}[1] {
\color{red}\textunderscore{\textit{TODO: #1}}
}

\newcommand{\important}[1] {\textbf{\color{orange}#1}}

% DOCUMENT
\begin{document}
\maketitle
\tableofcontents

\part{Mechanica}
\chapter{Beweging in 2 en 3 dimensies}
\section{Begrippen}
\begin{itemize}
\item {\important{Vector}: Een eenheid dat zowel een richting als een hoeveelheid heeft. Dit wordt voorgesteld door een lijnstuk met een pijl dat de richting aangeeft en de lengte die de grootte aangeeft.}
\item {\important{Eenheidsvector}: Een vector met als grootte 1.}
\item {\important{Netto verplaatsing}: De totale afstand van punt a tot punt b in vogelvlucht.}
\item {\important{Afgelegde afstand}: De totale afstand die afgelegd werd om van punt a tot punt b te bekomen.}
\end{itemize}

\section{Formules}
\begin{itemize}
  \item {\important{Snelheid}: }
\end{itemize}

\section{Conceptvraag: Helikopter}
\begin{itemize}
\item {\textbf{Vraag}: Een helikopter vliegt horizontaal en laat in positie A een kist met hulpgoederen vallen. Welke baan volgt die kist (wrijving met de lucht verwaarlozen)?}
\item {\textbf{Antwoord}: \begin{enumerate}
    \item {Baan (a) kan niet, de kist zal nooit achteruit gaan.}
    \item {Baan (b) kan niet, dit zou enkel gebeuren als de helikopter stil staat.}
    \item {Baan (c) kan niet, de zwaartekracht blijkt geen impact te hebben op de kist.}
    \item {Baan(d) is correct. De kist zal eerst de snelheid van het vliegtuig overnemen, en dan zo alsmaar sneller
        naar beneden vallen door de zwaartekracht.}
    \item {Baan(e) kan niet, de kist blijft te lang op dezelfde hoogte.}
    \end{enumerate}}
\end{itemize}


\section{Conceptvraag: Snelheid en versnelling}
\begin{itemize}
\item {\textbf{Vraag}: Gegeven zijn de snelheid en versnelling van een bewegende persoon. In welk geval vertraagt de persoon en wijkt af naar rechts(vanuit het standpunt van de persoon)}
\item {\textbf{Antwoord}: \begin{enumerate}
    \item {Bij (A) blijft de snelheid constant.}
    \item {Bij (B) vertraagt de snelheid en wijkt af naar rechts.}
    \item {Bij (C) versnelt de snelheid en wijkt af naar links.}
    \item {Bij (D) versnelt de snelheid en wijkt af naar rechts.}
      \item {Bij (E) vertraagt de snelheid.}
    \end{enumerate}}
\end{itemize}



\chapter{Kracht en beweging}

\chapter{Toepassen wetten van Newton}

\chapter{Arbeid, energie en vermogen}

\chapter{Behoud van energie}

\chapter{Systemen van deeltjes}

\chapter{Rotatiebewegingen}

\chapter{Rotatie vektoren en impulsmoment}

\chapter{Statisch evenwicht}

\chapter{Trillingen}

\chapter{Golven}

\chapter{Electromagnetische golven}

\part{Optica}

\chapter{Breking en terugkaatsing}

\chapter{Beelden en optische instrumenten}

\chapter{Interferentie en diffractie}

\chapter{Deeltjes en golven}

\part{Kernfysica}
\chapter{Kernfysica}



\end{document}
