
\documentclass{article}
\usepackage[utf8]{inputenc}
\begin{document}
\pagenumbering{gobble}
\title{Examen Fysica 19 januari 2018}
\date{}
\maketitle

\begin{enumerate}
        \item {Geef een uitdrukking voor een hoek die een hellende bocht met straal {\bf r} en ontwerpsnelheid
        {\bf v} moet maken zodat er geen wrijving is tussen deze bocht en een voorwerp. Bereken deze hoek voor r = 100m en v = 60 km/h}
        \item {Leg volgende twee uitdrukkingen bondig uit. (tekst, figuur, voorbeelden)
              $$I = \int r^{2} dm$$
              $$K_{rot}=\frac{1}{2}I\omega^{2}$$}
        \item {Een ladder (lengte = L, massa = 20 kg) leunt tegen een verticale muur zonder wrijving. De onderkant van de ladder rust op een horizontaal oppervlak waarmee het een hoek van 67 graden maakt. De wrijvingscoefficient tussen de ladder en de vloer is 0.3. Een persoon klimt omhoog totdat deze persoon $3/4$ van de ladder opgeklommen is. Wat is de maximale massa van deze persoon opdat de ladder niet zal vallen.}     
        \item {Bij een PET scan worden isotopen uitgezonden. Een isotoop die uitgezonden wordt is zuurstof-15 en heeft een vervaltijd van 2.1 minuten
        \begin{enumerate}
                \item {Indien de initiële activiteit gelijk is aan $1.0 *10^9 Bq$, hoelang duurt het totdat de activiteit gedaald is tot $8 kBq$.}
                \item {Welke kern wordt gevormd indien de isotoop een positron uitzend.}
                \item {Wat gebeurd er met deze positron en welke soort straling gaat hieraan gepaard.}
        \end{enumerate}}
        \item {Bewijs de pulsduurverlenging bij een multimode stepped index optic fiber. Bepaal ook de frequentie ...}
        \item {Een lenzenstelsel bevat een bolle en een holle lens. De onderlinge afstand tussen deze twee lenzen bedraagt 50 cm. Het brandpuntafstand van de holle lens bedraagt 30 cm. Er wordt een voorwerp 25 cm voor de bolle lens gezet. Bepaal het brandpuntafstand van de bolle lens. Bepaal ook de sterkte die dit lenzenstelsel heeft.}

\end{enumerate}
\end{document}