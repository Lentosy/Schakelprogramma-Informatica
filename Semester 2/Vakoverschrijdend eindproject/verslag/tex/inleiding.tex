\chapter{Inleiding}
\label{ch:inleiding}
\section{Context}
Binnen de UGent loopt een onderwijsinnovatie project waarin video annotatie wordt ingezet om communicatievaardigheden, die centraal staan in diverse opleidingsonderdelen in verschillende opleidingen, te trainen en op een authentieke manier te toetsen. Hiervoor wordt ingezet op peer-assessment, waarbij studenten elkaar feedback geven op gefilmde situaties waarin de student communicatievaardigheden toepast.
\section{Probleemstelling}
Er bestaat al commerci\"{e}le video-annotatiesoftware, maar deze hebben echter een duur prijskaartje en zijn bovendien complex om te gebruiken. Ondertussen heeft dit project wel een gratis alternatief gevonden, CommentBubble \footnote{\url{https://commentbubble.com/}}. CommentBubble maakt gebruik van YouTube om video's te hosten. Door het feit dat het videomateriaal enerzijds vertrouwelijke informatie bevat, en anderzijds om voldoende privacy te garanderen, gaat de voorkeur uit naar een annotatietool waarbij alle data intern opgeslagen word binnen de UGent.
\section{Doelstelling}
De video-annotatietool die ontworpen moet worden zal gebruikt worden door zowel docenten als studenten. Om installaties van de software te vermijden wordt er gekozen om deze tool te implementeren als een webapplicatie. 
\section{Overzicht van dit document}
In hoofdstuk \ref{ch:inleiding} werd de opdracht al toegelicht. In hoofdstuk \ref{ch:gebruikersaspecten} worden de noodzakelijke features besproken. In hoofdstuk \ref{ch:systeemarchitectuur} wordt de onderliggende structuur van het project besproken. In hoofdstuk \ref{ch:testplan} wordt besproken hoe het project uitgetest zal worden. In hoofdstuk \ref{ch:installatiehandleiding} wordt een overzicht gegeven hoe de software te installeren op een toestel.
