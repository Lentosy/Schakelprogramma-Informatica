

\documentclass{article}
\usepackage{amsmath}
\usepackage[margin=1in]{geometry}
\title{Examen Statistiek 15 juni 2018}
\author{}
\date{}

\begin{document}
\maketitle
 \begin{enumerate}
  \item Op een stof wordt er analyse uitgevoerd om het aantal gram actieve deeltjes te controleren. Het is geweten dat dit aantal normaal verdeeld is met een standaardafwijking van 0.6 gram. Hoeveel monsters moeten er genomen worden zodat het gemiddelde zeker kleiner is dan 0.5 gram met een betrouwbaarheid van 95\%.
  \item Een doos bevat 3 dobbelstenen waarvan er 1 een onvervalste dobbelsteen is en de andere 2 vervalste dobbelstenen zijn. Bij een vervalste dobbelsteen is de kans om een even aantal ogen te gooien het dubbele van de kans om een oneven aantal ogen te gooien. Er wordt willekeurig een dobbelsteen uit de doos genoemen en er wordt hiermee gerold.
    \begin{enumerate}
        \item Wat is de kans om een 2 te gooien.
        \item Indien een 4 werd gegooid, wat is dan de kans dat het bij een onvervalste dobbelsteen is.
    \end{enumerate}
  
  \item Stel $\mu: \chi^2(27\; d.f.)$ en $\nu: \chi^2(6\;d.f.)$. Bepaal $\alpha$ zodat $P(4\mu > 9\alpha\nu) = 0.9$.
  
  \item Een computer genereert getallen volgens een normale verdeling $N(1, 2)$. 
    \begin{enumerate}
        \item Bepaal de kans dat de absolute waarde van een gegenereerd getal kleiner is dan 3.
        \item Bepaal de kans dat er 1 negatief getal is bij 4 generaties.
    \end{enumerate}
  
  \item Een steekproef met 10 waarden is normaal verdeeld. Stel het 95$\%$ betrouwbaarheidsinterval op voor het gemiddelde. Bewijs ook de verdeling die wordt gebruikt tijdens de opbouw.
  
  \item Elias wil controleren of dat het alcoholgehalte van bepaalde biersoorten wel degelijk het alcoholgehalte is dat op het etiket staat. Hij doet metingen bij vier biersoorten en bekomt volgende resultaten:
  
  \begin{tabular}{l | l | l | l | l}
  biersoort  & bier 1 & bier 2 & bier 3 & bier 4 \\
  \hline
  opgegeven \% & 6.5 & 8.5 & 7 & 6  \\

  gemeten \%  & 6.5 & 9 & 5.8 & 7.5
   
  \end{tabular}
  
  Kan Elias met 95\% betrouwbaarheid zeggen dat het echte gemiddelde van de gemeten waarden kleiner is dan 3\% van het echte gemiddelde van de opgegeven waarden?

  
  \item Gegeven volgende functie:   $$f(x) = \begin{cases}
                                            a(1 - x)   & -1 \leq x \le 1 \\
                                            a(x - 1)/2 & 1 \leq x \leq 2 \\
                                            0          & x > 2
                                            
                                           \end{cases}
                                    $$
        Geef het gemiddelde, de cumulatieve distributiefunctie, de mediaan en de modus.
 \end{enumerate}

\end{document}
