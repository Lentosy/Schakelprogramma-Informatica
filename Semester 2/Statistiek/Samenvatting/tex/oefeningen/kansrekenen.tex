\chapter{Kansrekenen}
  
\begin{itemize}[label={}, leftmargin=*]
	\item {\exercise{2. Een geldstuk is vervalst zodat kop dubbel zoveel kan voorkomen als munt. Als het geldstuk drie keer geworpen wordt, wat is de kans om juist 2 keer munt te hebben?}{
		Er zijn twee evenementen te definieëren. Kop gooien \textbf{K} en munt gooien \textbf{M}. Kop gooien kan twee keer zoveel voorkomen als munt gooien.
		$$P(K) = 2P(M)$$
		We kunnen gebruik maken van het feit dat de som van alle kansen gelijk is aan 1.
		
		
		\begin{equation*}
			\begin{split}
				&  P(K) + P(M) = 1    \\
				\Leftrightarrow & 2P(M) + P(M) = 1    \\
				\Leftrightarrow & 3P(M) = 1           \\
				\Leftrightarrow & P(M) = \frac{1}{3}  \\
			\end{split}
		\end{equation*}
		dus
		$$P(K) = \frac{2}{3}$$
		Aangezien de gebeurtenis munt twee keer moet voorkomen moet kop dus slechts één maal voorkomen.
		$$P(2M \cap K)$$
		Er moet rekening gehouden worden met de verschillende combinaties:
		\begin{equation*}
			\begin{split}
				& P((M \cap M \cap K) \cup (M \cap K \cap M) \cup (K \cap M \cap M))  \\
				= & P(M \cap M \cap K) \cup P(M \cap K \cap M) \cup (K \cap M \cap M)) \\
				= & P(M)P(M)P(K) + P(M)P(K)P(M) + P(K)P(M)P(M) \\
				= & \frac{1}{3} \cdot \frac{1}{3} \cdot \frac{2}{3} + \frac{1}{3} \cdot \frac{2}{3} \cdot \frac{1}{3} + \frac{2}{3} \cdot \frac{1}{3} \cdot \frac{1}{3} \\
				= & 3(\frac{1}{3} \cdot \frac{1}{3} \cdot \frac{2}{3}) \\
				= & 3(\frac{2}{27}) = \frac{6}{27} = \frac{2}{9} 
			\end{split}
		\end{equation*}
	}}
	    
	    
	\item { \exercise{3. Een dobbelsteen is vervalst zodat de kans dat een gegeven aantal ogen geworpen wordt evenredig is met het aantal ogen. Is A de gebeurtenis een even aantal te gooien, B de gebeurtenis een priemgetal te gooien en C de gebeurtenis een oneven getal te gooien,
		\begin{enumerate}
			\item Bepaal P(A), P(B) en P(C)
			\item Bereken de kans dat men een even getal of een priemgetal gooit.
			\item Bereken de kans dat men een even getal gooit dat geen priemgetal is.
			\item Bereken de kans dat men een oneven getal of een priemgetal gooit.
		\end{enumerate}}
		{
			De kans kan als formule worden voorgesteld.
			$$P(i) = ip$$
			waarbij p een willekeurig getal is.
			We weten dat de som van alle kansen gelijk is aan 1.
			$$\sum_{i = 1}^{6} P(i) = 1p + 2p + 3p + 4p + 5p + 6p = 1$$
			Los op naar p
			$$21p = 1$$
			$$p = \frac{1}{21}$$
			De deeloplossingen:
			\begin{enumerate}
				\item   \begin{equation*}
				      \begin{split}
				      	P(A) = & \frac{2}{21} + \frac{4}{21} + \frac{6}{21} = \frac{12}{21} = \frac{4}{7}\\
				      	P(B) = &\frac{2}{21} + \frac{3}{21} + \frac{5}{21} = \frac{10}{21}\\
				      	P(C) = & 1 - P(A) = 1 - \frac{4}{7} = \frac{3}{7}
				      \end{split}
				\end{equation*}
				\item $$P(A\cup B) = P(A) + P(B) - P(A \cap B) = \frac{4}{7} +\frac{10}{21} - \frac{2}{21} = \frac{20}{21}$$
				\item $$P(A \cap \overline{B}) = P(A) - P(A \cap B) = \frac{4}{7} - \frac{2}{21} = \frac{10}{21}$$
				\item $$P(B \cup C) = P(B) + P(C) - P(B \cap C) = \frac{10}{21} + \frac{3}{7} - (\frac{3}{21} + \frac{5}{21}) = \frac{11}{21}$$
			\end{enumerate}}}
	  
	  
	\item {\exercise{4. A en B zijn verschijnselen met P(A) = 0.1, P(B) = 0.5. Bepaal $P(A \cup B)$, $P(\overline{A})$, $P(\overline{A} \cap B)$ indien a) de verschijnselen elkaar uitsluiten en b) ze onafhankelijk zijn.}
		{
			\begin{enumerate}[label=(\alph*)]
				\item \begin{align*}
				      P(A \cup B) =  & P(A) + P(B) - P(A \cap B) = \frac{1}{10} + \frac{1}{2} - 0 = \frac{3}{5} \\
				      P(\overline{A}) =  & 1 - P(A) = 1 - \frac{1}{10} = \frac{9}{10} \\
				      P(\overline{A} \cap B) = & P(B) - P(A \cap B) = \frac{1}{2} - 0 = \frac{1}{2} \\
				\end{align*}
				\item \begin{align*}
				      P(A \cup B) = & P(A) + P(B)  - P(A \cap B) = \frac{1}{10} + \frac{1}{2} - \frac{1}{20} = \frac{11}{20} \\
				      P(\overline{A}) =  &1 - P(A) = 1 - \frac{1}{10} = \frac{9}{10} \\
				      P(\overline{A} \cap B) = & P(B) - P(A \cap B) = \frac{1}{2} - \frac{1}{20} = \frac{9}{20}
				\end{align*}
			\end{enumerate}}}
	
	\item \exercise{6. Bereken voor een familie van 3 kinderen de kans op a) 3 jongens en  b) 2 jongens en 1 meisje}{
	      $$P(J) \hbox{: Kans op een jongen} = \frac{1}{2}$$
	      $$P(M) \hbox{: Kans op een meisje} = \frac{1}{2}$$
	      \begin{enumerate}[label=(\alph*)]
	      	\item \begin{equation*}
	      	      \begin{split}
	      	      	P(J \cap J \cap J) & = P(J)P(J)P(J)\\
	      	      	& = (P(J))^{3} \\
	      	      	& = \bigg(\frac{1}{2}\bigg)^{3} \\
	      	      	& = \frac{1}{8}
	      	      \end{split}
	      	\end{equation*}
	      	\item \begin{equation*}
	      	      \begin{split}
	      	      	& P(J \cap J \cap M) + P(J \cap M \cap J) + P(M \cap J \cap J)  \\
	      	      	= & 3P(J \cap J \cap M)\\
	      	      	= & 3(P(J))^3 \;\;\;\;\; \hbox{(aangezien P(J) = P(M))} \\
	      	      	= & 3 \bigg(\frac{1}{2}\bigg)^{3} = \frac{3}{8}
	      	      \end{split}
	      	\end{equation*}
	      \end{enumerate}}
	      
	      
	      
	\item{\exercise{7. Een paar onvervalste dobbelstenen worden geworpen. Wat is de kans dat de som van de ogen een totaal van minstens 8 vertoont.}{
	      \begin{equation*}
	      	\begin{split}
	      		A =  &\;\hbox{som van de ogen} \geq 8\\
	      		P(A) =  &\{2, 6\} \cup \{3, 5\} \cup \{3, 6\} \cup ... \cup \{6,6\} \\
	      		P(A) = & \frac{15}{36} = \frac{5}{12}
	      	\end{split}
	      \end{equation*}}}
	      
	\item{\exercise{11. Gegeven 3 kasten A, B en C. Elke kast heeft een aantal laden die ofwel een goudstuk(G), ofwel een zilverstuk(Z) ofwel niets bevatten(N) en dit als volgt:
	\begin{itemize}[label={}]
	 \item A: [G|G|G|Z]
	 \item B: [G|Z|Z]
	 \item C: [G|G|N|Z|Z]
	\end{itemize}
	Men kiest willekeurig één van de kasten, opent daarvan at random één lade, en grijpt het muntstuk(indien mogelijk). Wat is de kans dat men kast A heeft uigekozen indien men een goudstuk heeft genomen?
	}{
	De kans om eender welke lade te pakken:
	$$P(A) = P(B) = P(C) = \frac{1}{3}$$
	De kans om een goudstuk te pakken afhankelijk van de lade:
	\begin{gather*}
	 P(G|A) = \frac{3}{4}\\
	 P(G|B) = \frac{1}{3}\\
	 P(G|C) = \frac{2}{5}
	\end{gather*}
	De kans dat lade A gekozen werd indien het een goudstuk genomen is:
	\begin{equation*}
	 \begin{split}
	  P(A|G) & = \frac{P(A)P(G|A)}{P(A)P(G|A)+P(B)P(G|B)+P(C)P(G|C)} \\
	         & = \frac{\frac{1}{3}\cdot\frac{3}{4}}{\frac{1}{3}\cdot\frac{3}{4}+\frac{1}{3}\cdot\frac{1}{3}+\frac{1}{3}\cdot\frac{2}{5}} \\
	         & = \frac{\frac{1}{4}}{\frac{1}{4} + \frac{1}{9} + \frac{2}{15}} \\
	         & = \frac{45}{89}
	 \end{split}
	\end{equation*}

	}}
	
	\item{\exercise{16. Wanneer men het waarheidsserum toedient aan een schuldig persoon is het voor 90\% betrouwbaar en aan een onschuldig persoon is het voor 99\% betrouwbaar. Als een verdachte gekozen wordt uit een groep, waarvan 5\% reeds een misdrijf begaan hebben, wat is de kans dat die persoon niet schuldig is als het waarheidsserum schuldig aanwijst?}{
	2 Gebeurtenissen:
	\begin{enumerate}
	 \item PS: De persoon is schuldig
	 \item WS: Het waarheidsserum wijst schuldig aan
	\end{enumerate}
	De kans dat de persoon niet schuldig is als het waarheidsserum schuldig aanwijst:
	\begin{equation*}
	 \begin{split}
	 P(\overline{PS}|WS) & = \frac{P(\overline{PS})P(WS|\overline{PS})}{P(\overline{PS})P(WS|\overline{PS})+P(PS)P(WS|PS)} \\
	  & = \frac{\frac{19}{20}\cdot\frac{1}{100}}{\frac{19}{20}\cdot\frac{1}{100} + \frac{1}{20}\cdot\frac{9}{10}} \\
	  & = \frac{\frac{19}{2000}}{\frac{19}{2000} + \frac{9}{200}} \\
	  & = \frac{\frac{19}{2000}}{\frac{19}{2000} + \frac{90}{2000}} \\
	  & = \frac{\frac{19}{2000}}{\frac{109}{2000}} \\
	  & = \frac{19}{109}
	 \end{split}
	\end{equation*}

	}
	}
	       
	\item \exercise{17. Uit een spel van 52 kaarten trekt men willekeurig maar tezelfdertijd vijf kaarten. Bereken de kans dat 
	      \begin{enumerate}
	      	\item het vijf zwarte kaarten zijn,
	      	\item het drie heren en twee vrouwen zijn,
	      	\item er tenminsten één aas bij is,
	      	\item er ten hoogste één harten bij is.
	      \end{enumerate}}{
	      \begin{enumerate}
	      	\item {
	      		\begin{equation*}
	      			\begin{split}
	      				P(5Z) & = \frac{C_{26}^{5}}{C_{52}^{5}}                     \\
	      				& = \frac{\frac{26!}{5!(26 - 5)!}}{\frac{52!}{5!(52 - 5)!}}\;\;\; \hbox{(vereenvoudigen voor rekenmachine)}\\
	      				& = \frac{26!}{21!} \cdot \frac{47!}{52!}                   \\
	      				& = \frac{26 \cdot 25 \cdot 24 \cdot 23 \cdot 22}{52 \cdot 51 \cdot 50 \cdot 49 \cdot 48} \\
	      				& = 0.025
	      			\end{split}
	      		\end{equation*}
	      	}
	      	\item {
	      		$$P(3H \cap 2V) = \frac{C_{4}^{3} \cdot C_{4}^{2}}{C_{52}^{5}}$$
	      		Meestal zullen ze vragen 'geef de correcte uitdrukking' om geen tijd te verliezen aan banaal rekenwerk.
	      	}
	      	\item {
	      		$$P(\hbox{minstens 1 aas}) = 1 - P(\hbox{geen aas}) = 1 - \frac{C_{4}^{0} \cdot C_{48}^{5}}{C_{52}^{5}}$$
	      	}
	      	\item {
	      		\begin{equation*}
	      			\begin{split}
	      				P(\hbox{ten hoogste 1 hart}) & = P(\hbox{0 hart} \cap \overline{\hbox{5 hart}}) \cup P(\hbox{1 hart} \cap \overline{\hbox{4 hart}})  \\
	      				& =  \frac{C_{39}^{5}}{C_{52}^{5}} + \frac{C_{13}^{1} \cdot C_{39}^{4}}{C_{52}^{5}}
	      			\end{split}
	      		\end{equation*}
	      	}
	      	   
	      \end{enumerate} }
	  
	\item{\exercise{20. Bepaal de kans om minstens één maal zes te gooien bij 4 worpen met een dobbelsteen. Bepaal de kans om minstens één maal dubbel zes te gooien bij 24 worpen met 2 dobbelstenen.}{
	        
	      \begin{equation*}
	      	\begin{split}
	      		P(\hbox{minstens één 6 bij 4 worpen}) & = 1 - P(\hbox{geen 6 bij 4 worpen})\\
	      		& = 1 - \bigg(\frac{5}{6}\bigg)^4
	      	\end{split}
	      \end{equation*}
	      \begin{equation*}
	      	\begin{split}
	      		P(\hbox{minstens één keer dubbel 6 bij 24 worpen}) & = 1 - P(\hbox{geen dubbel 6 bij 24 worpten})\\
	      		& = 1 - \bigg(\frac{35}{36}\bigg)^{24}
	      	\end{split}
	      \end{equation*}
	        
	}}
	  
	\item{\exercise{21. Bepaal de kans om met de belgische lotto a)drie cijfers b)vier cijfers en c)zes cijfers goed te hebben.}{  $$P(x) = \frac{C_6^x \cdot C_{42 - x}^{6 - x}}{C_{42}^{6}}$$
	x in te vullen met 3, 4 of 6.}}
	
	\item {\exercise{23. Een urne bevat 10 ballen waarvan \textit{n} rode. De kans om twee rode ballen te trekken op drie trekkingen met terugleggen is 1.08 keer de kans om twee rode ballen te trekken op drie trekkingen zonder terugleggen. Bepaal \textit{n} en sluit de triviale gevallen uit.}{
	Twee gebeurtenissen:
	\begin{enumerate}
	 \item A: 2 rode ballen trekken op 3 trekkingen met teruglegging
	 \item B: 2 rode ballen trekken op 3 trekkingen zonder teruglegging
	\end{enumerate}
	Verder geldt dat $P(A) = 1.08P(B)$ en $2 \leq n \leq 10$
	
	De kans op B:
	\begin{equation*}
	 \begin{split}
	  P(B) & = \frac{C_{n}^{2}C_{10 - n}^{1}}{C_{10}^{3}} \\
	       & = \frac{\frac{n!}{2!(n-2)!}\frac{(10-n)!}{1!(9-n)!}}{\frac{10!}{3!7!}} \\
	       & = \frac{n!}{2!(n-2)!}\frac{(10-n)!}{(9-n)!}\frac{3!7!}{10!} \\
	       & = \frac{3n(n-1)(10-n)}{10\cdot9\cdot8}
	 \end{split}
	\end{equation*}
	De kans op geen rode bal:
	\begin{equation*}
	 \begin{split}
	  & P(R) = \frac{n}{10} \\
	  \Rightarrow&  P(\overline{R}) = 1 - \frac{n}{10} = \frac{10 - n}{n} 
	 \end{split}
	\end{equation*}
	De kans op A:
	\begin{equation*}
	 \begin{split}
	  P(A) & = C_3^2P(R)^2P(\overline{R}) \\
	  & = 3\bigg(\frac{n}{10}\bigg)^2\bigg(\frac{10 - n}{10}\bigg) \\
	  & = \frac{3n^2(10 - n)}{10^3}
	 \end{split}
	\end{equation*}

	Berekening van \textit{n}
	\begin{equation*}
	 \begin{split}
	  & P(A) = 1.08P(B) \\
	  \Leftrightarrow & \frac{3n^2(10 - n)}{10^3} = \frac{3n(n-1)(10-n)}{10\cdot9\cdot8} \\
	  \Leftrightarrow & \frac{n}{10^2} = 1.08\frac{n - 1}{9\cdot 8} \\
	  \Leftrightarrow & \frac{n}{100} = 1.08\frac{n - 1}{72} \\
	  \Leftrightarrow & 72n = 108(n-1)\\
	  \Leftrightarrow & 72n = 108n - 108 \\
	  \Leftrightarrow & 108n - 72n = 108 \\
	  \Leftrightarrow & 36n = 108 \\
	  \Leftrightarrow & n = 3
	 \end{split}
	\end{equation*}

	  }
	
	
	}
\end{itemize}
