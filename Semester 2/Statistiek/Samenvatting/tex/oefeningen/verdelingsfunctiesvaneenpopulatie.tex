\chapter{Verdelingsfunctie van een populatie}
\begin{itemize}
 \item{\exercise{3. Ga na of de volgende functi F(x) een cumulatieve distributiefunctie kan zijn. Indien ja, bepaal de corresponderende dichtheidsfunctie f(x)
 $$F(x) = \begin{cases}
           0 & x \leq 0 \\
           1 - e^{-x^{2}} & x > 0
          \end{cases}
$$}{
Voer de drie controles uit:
\begin{enumerate}
 \item De functie is nooit negatief.
 \item De functie is nooit dalend.
 \item De limiet naar $+\infty$ is 1.
\end{enumerate}
\begin{enumerate}
\item Controle functie nooit negatief:
\begin{equation*}
 \begin{split}
  & 1-e^{-x^{2}} > 0 \\
  \Leftrightarrow & -e^{-x^{2}} > - 1 \\
  \Leftrightarrow & e^{-x^{2}} < 1 \\
  \Leftrightarrow & e^{-x^{2}} > e^0 \\
  \Leftrightarrow & -x^{2} > 0 \\
 \end{split}
\end{equation*}
\item Controle functie nooit dalend:
\begin{equation*}
 \begin{split}
  F'(x) = 2xe^{-x^2}
 \end{split}
\end{equation*}
De afgeleide is altijd positief, dus daalt de functie nooit.
\item Controle limiet naar $+\infty$ is gelijk aan 1.
\begin{equation*}
 \begin{split}
  \lim_{t\to+\infty}F(t) & = \lim_{t\to+\infty}1 - e^{-x^{2}} \\
                         & =  \lim_{t\to+\infty}1 - \frac{1}{e^{x^{2}}} \\
                         & = 1 - 0 = 1
 \end{split}
\end{equation*}
De kansfunctie hebben we al berekent in stap 2. $f(x) = 2xe^{-x^{2}}$
\end{enumerate}


}}
 \item{\exercise{5. Bepaal C zodat de volgende functie een dichtheidsfunctie is. Bepaal de corresponderende cumulatieve distributiefunctie
 $$f(x) = \begin{cases}
           C(4x - 2x^2) & 0 < x < 2 \\
           0 & \hbox{elders}
          \end{cases}
$$}{
Bepalen C:
\begin{equation*}
 \begin{split}
  & \int_{-\infty}^{+\infty}C(4x-2x^2)dx = 1 \\
  \Leftrightarrow & 2C\int_{0}^{2}2x-x^2dx = 1 \\
  \Leftrightarrow & 2C\bigg[\frac{2x^2}{2} - \frac{x^3}{3}\bigg]_0^2 = 1 \\
  \Leftrightarrow & 2C\bigg[x^2 - \frac{x^3}{3}\bigg]_0^2 = 1 \\
  \Leftrightarrow & 2C\bigg[\bigg(4 - \frac{8}{3}\bigg) - \bigg(0 - \frac{0}{3}\bigg)\bigg]= 1 \\
  \Leftrightarrow & 2C\bigg(\frac{4}{3}\bigg)= 1 \\
  \Leftrightarrow & C = \frac{3}{8}
 \end{split}
\end{equation*}
Dus:
 $$f(x) = \begin{cases}
           \frac{3}{8}(4x - 2x^2) & 0 < x < 2 \\
           0 & \hbox{elders}
          \end{cases}
$$
De cumulatieve distributiefunctie:
$$F(x) = \begin{cases}
          0 & x \leq 2 \\ 
          0 + \frac{3}{8}\int_0^x (4t - t^2)dt = \frac{3}{8}\bigg(\frac{4x^2}{2} - \frac{2x^3}{3}\bigg) & 0 < x < 2 \\
          1 & x \geq 2
         \end{cases}
$$
}}
 \item{\exercise{6. Een vaas bevat 4 rode en 6 witte ballen. Men neemt 3 ballen zonder teruglegging. Als x het aantal getrokken rode ballen voorstelt bepaal dan:
    \begin{enumerate}
     \item de dichtheidsfunctie van x
     \item de grafiek van de dichtheidsfunctie en de corresponderende verdelingsfunctie 
     \item de gemiddelde waarde en de modus
     \item de variantie
     \item $P(x \geq 1)$
    \end{enumerate}
}{
    \begin{enumerate}
     \item  $x = \# $ getrokken rode ballen.
     
            De kans om x = 0, 1, 2 of 3 rode ballen te trekken kan voorgesteld worden als :
            $$f(x) = \frac{C_4^xC_6^{3 - x}}{C_{10}^3}$$
            Bijgevolg zijn de kansen:
            \begin{tabular}{c | c c c c}
                x & 0 & 1 & 2 & 3 \\
                f(x) & $\frac{1}{6}$ & $\frac{1}{2}$ & $\frac{3}{10}$ & $\frac{1}{30}$
            \end{tabular}
    \item \todo{grafiek}
        $$F(x) = \begin{cases}
                    0 & x < 0 \\
                    \frac{1}{6} & 0 \leq x \leq 1 \\
                    \frac{1}{6} + \frac{1}{2} = \frac{2}{3} & 1 \leq x \leq 2 \\
                    \frac{2}{3} + \frac{3}{10} = \frac{29}{30} & 2 \leq x \leq 3 \\
                    \frac{29}{30} + \frac{1}{30}= 1 & x \geq 3 \\
                    
                 \end{cases}$$
    \item Het gemiddelde:
        \begin{equation*}
         \begin{split}
          \mu = \sum_{i = 0}^{3}x_i f(x_i) \\
            & = 0f(0) + 1f(1) + 2f(2) + 3f(3) \\
            & = 0 + \frac{1}{6} + 2\frac{2}{3} + 3\frac{29}{30} \\
            & = 1.2
         \end{split}
        \end{equation*}
        De modus is 1.
    \item De variantie:
        \begin{equation*}
         \begin{split}
          \sigma^2 & = \bigg[\sum x^2_i \big(f(x_i)\big)\bigg] - \mu^2 \\
                   & = 0^2f(0) + 1^2f(1) + 2^2f(2) + 3^2f(3) - \mu^2 \\
                   & = 0.56
         \end{split}
        \end{equation*}
    \item 
        $$P(x \geq 1) = 1 - P(x < 1) = 1 - F(0) = 1 - \frac{1}{6} = \frac{5}{6}$$
    \end{enumerate}
}}
\item {
    \exercise{
        10. Voor een gokspel met drie onvervalste dobbelstenen bedraagt de inzet steeds 5 euro. Indien iemand juist één 6 werpt krijgt hij zijn inzet terug, indien juist twee stenen een 6 vertonen krijgt hij 10 euro terug en indien de drie dobbelstenen een 6 vertonen krijgt hij 15 euro. Wat is de gemiddelde winst (of verlies)?
    }{
        De verwachte waarde van een functie:
        $$E[g(x)] = \sum g(x_i) f(x_i)$$
        met $g(x) = $ de winst met een inzet van 5 euro
        
        en $f(x) = $ de kans om x aantal zessen te hebben.
        
        $x = \#6 $ bij het werpen van een dobbelsteen 
        De functie $f$ kan geschreven worden als:
        $$f(x) = C_{3}^{x}\bigg(\frac{1}{6}\bigg)^x \bigg(\frac{5}{6}\bigg)^{3-x}$$
        Dus:
        \begin{tabular}{c | c c c c}
            x & 0 & 1& 2& 3 \\
            f(x) & $\big(\frac{5}{6}\big)^3$ & $3\frac{5^2}{6^3}$ & $\frac{15}{6^3}$ & $\frac{1}{6^3}$ \\
            g(x) & -5 & 0 & 5 & 10
        \end{tabular}
        Hieruit volgt:
        \begin{equation*}
         \begin{split}
          E[g(x)] & = f(0)g(0) + f(1)g(1) + f(2)g(2) + f(3)g(3) \\
                  & = -2.5
         \end{split}
        \end{equation*}
        Er is dus gemiddeld een verlies van -2,5 euro
    }
}
\item {
    \exercise{
        13. Een toevalsveranderlijke x heeft een gemiddelde $\mu = 12$, een dispersie $\sigma = 3$ en zijn dichtheidsfunctie is niet gekend.
        
        Bepaal een ondergrens voor: $P(6 < x < 18)$ en voor $P(3 < x < 21)$
    }{
        \begin{equation*}
         \begin{split}
          P(6 < x < 18) & = P(6 - \mu < x - \mu < 18 - \mu) \\
                        & = P(-6 < x - \mu < 6) \\
                        & = P(-2\sigma < x - \mu < 2\sigma) \\
                        & = P(|x - \mu| < 2\sigma) \\
                        & = 1 - \frac{1}{2^2} \\
                        & = 1 - \frac{1}{4} \\
                        & = \frac{3}{4}
         \end{split}
        \end{equation*}
        \begin{equation*}
         \begin{split}
          P(3 < x < 21) & = P(3 - \mu < x - \mu < 21 - \mu) \\
                        & = P(-9 < x - \mu < 9) \\
                        & = P(-3\sigma < x - \mu < 3\sigma) \\
                        & = P(|x - \mu| < 3\sigma) \\
                        & = 1 - \frac{1}{3^2} \\
                        & = 1 - \frac{1}{9} \\
                        & = \frac{8}{9}
         \end{split}
        \end{equation*}

    }
}
\item {
    \exercise{
        16. Onderstel dat het aantal producten in een fabriek, aangemaakt gedurende één week, een stochastische veranderlijke is met gemiddelde $\mu = 50$. Deze veranderlijke heeft een symmetrische verdeling t.o.v. $\mu$
        \begin{enumerate}
         \item Bepaal een bovengrens voor de kans dat de productie van een bepaalde week minstens 75 zal dragen.
         \item Indien bovendien de variantie gekend is ($\sigma^2 = 25$) wat is dan een ondergrens voor de waarschijnlijkheid dat de productie van een bepaalde week strikt tussen 40 en 60 zal liggen.
        \end{enumerate}
    }{
        \begin{enumerate}
         \item \begin{equation*}
                \begin{split}
                P(x \leq \mu - k\sigma) = P(x \geq \mu + k\sigma) \leq \frac{1}{2k^2} \\
                \Rightarrow p(x \geq 75) & = P(x -\mu \geq 25) = \frac{1}{2k^2} \\
                                        & = \frac{1}{2(\frac{25}{\sigma})^2} \\
                                        & = \frac{\sigma^2}{1250}
                \end{split}
                \end{equation*}
        \item 
            $$\sigma = \sqrt{\sigma^2} = \sqrt{5^2} = 5$$
             \begin{equation*}
               \begin{split}
                P(40 < x < 60) & = P(-10 < x - \mu < 10) \\
                               & = P(-2\sigma < x - \mu < 2\sigma) \\
                               & = 1 - \frac{1}{2^2} \\
                               & = \frac{3}{4}
               \end{split}
              \end{equation*}

        \end{enumerate}


    }
}
\end{itemize}

