\chapter{Continue verdelingenen}
\exercise{
    11. Een Geigerteller levert gemiddeld 30 tellen per minuut in de omgeving van een radio-actief materiaal. Stel dat het aantal tellen per minuut Poisson verdeeld is. Bepaal de kans dat 
    \begin{enumerate}
     \item er juist 32 tellen zijn,
     \item er tussen 23 en 35 tellen zijn, grenzen niet inbegrepen.
    \end{enumerate}
}{
	Stel: \begin{itemize}
			\item $x: \#$ tellen
			\item x is Poisson verdeeld met $\lambda = 30$
			\item $f(i) = \frac{e^{-30}(30)i}{i!}$
		  \end{itemize}
	\begin{enumerate}
		\item $P(i = 32) = \frac{e^{-30}(30)^32}{32!} = ?$ (teveel werk met rekenmachine)

		We lossen door gebruik te maken van de limietstelling:
		$$\lambda = 30 \geq 15 \Rightarrow x:N(30, \sqrt{30})$$
		\begin{equation*}
			\begin{split}
				P(x = 32) & \approx P(31,5 < x < 32,5) \\
						  & = P(\frac{31,5 - 30}{\sqrt{30}} < z < \frac{32,5 - 30}{\sqrt{30}}) \\
						  & = P(0,27 < z < 0,46) \\
						  & = P(0 < z < 0,46) - P(0 < z < 0,27) \\
						  & = 0,1772 - 0,1064 \\
						  & = 0,0708
			\end{split}
		\end{equation*}
		\item 
		\begin{equation*}
			\begin{split}
				P(23 < x < 35) & \approx P(23,5 < x < 34,5) \\
							   & = P(\frac{23,5 - 30}{\sqrt{30}} < z < \frac{34,5 - 30}{\sqrt{30}}) \\
							   & = P(-1,19 < z < 0,82) \\
							   & = P(0 < z < 0,82) - P(0 < z < 1,19) \\
							   & = 0,2939 + 0,3830 \\
							   & = 0,6769
			\end{split}
		\end{equation*}
	\end{enumerate}
}

\exercise{
	Extra: Een speler gooit met een dobbelsteen. Bepaal de kans dat hij bij 90 worpen tenminste 39 keer een 5 of een 6 gooit. Geef de exacte formule en bereken met de limiet-stelling.
}{
	Stel:
		\begin{itemize}
			\item $x : \#$ keer een vijf of zes.
			\item kans op x = p = $1/3$
			\item x is binomiaal verdeeld
			\item $f(i) = C_n^i p^i(1 - p)^{n - i}$ 
		\end{itemize}
	De exacte formule wordt:
	$$P(x \geq 39) = 1 - P(x \leq 38) = 1 - \sum_{i = 0}^{38} C_{90}^{i} \bigg(\frac{1}{3}\bigg)^i\bigg(\frac{2}{3}\bigg)^{n - i}$$
	We lossen dit op met de limietstelling 2 van het formularium. Bewijs:
	\begin{itemize}
		\item $np = 90\frac{1}{3} = 30 \geq 5$
		\item $n(1 - p) = 90\frac{2}{3} = 60 \geq 5$
	\end{itemize}
	De variabele x benadert een normale verdeling.
	$$x : N(np, \sqrt{np(1 - p)}) = N(30, \sqrt{20})$$
	\begin{equation*}
		\begin{split}
			P(x \geq 39) & \approx P(x \geq 38,5) \\
						 & = P(z \geq \frac{38,5 - 30}{\sqrt{20}}) \\
						 & = P(z \geq 1,9) \\
						 & = 0.5 - 0.4713 \\
						 & = 0.0287
		\end{split}
	\end{equation*}
}
