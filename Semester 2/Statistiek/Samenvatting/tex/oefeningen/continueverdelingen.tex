\chapter{Continue verdelingenen}
\exercise{
    11. Een Geigerteller levert gemiddeld 30 tellen per minuut in de omgeving van een radio-actief materiaal. Stel dat het aantal tellen per minuut Poisson verdeeld is. Bepaal de kans dat 
    \begin{enumerate}
     \item er juist 32 tellen zijn,
     \item er tussen 23 en 35 tellen zijn, grenzen niet inbegrepen.
    \end{enumerate}
}{
	Stel: \begin{itemize}
			\item $x: \#$ tellen
			\item x is Poisson verdeeld met $\lambda = 30$
			\item $f(i) = \frac{e^{-30}(30)i}{i!}$
		  \end{itemize}
	\begin{enumerate}
		\item $P(i = 32) = \frac{e^{-30}(30)^32}{32!} = ?$ (teveel werk met rekenmachine)

		We lossen door gebruik te maken van de limietstelling:
		$$\lambda = 30 \geq 15 \Rightarrow x:N(30, \sqrt{30})$$
		\begin{equation*}
			\begin{split}
				P(x = 32) & \approx P(31,5 < x < 32,5) \\
						  & = P(\frac{31,5 - 30}{\sqrt{30}} < z < \frac{32,5 - 30}{\sqrt{30}}) \\
						  & = P(0,27 < z < 0,46) \\
						  & = P(0 < z < 0,46) - P(0 < z < 0,27) \\
						  & = 0,1772 - 0,1064 \\
						  & = 0,0708
			\end{split}
		\end{equation*}
		\item 
		\begin{equation*}
			\begin{split}
				P(23 < x < 35) & \approx P(23,5 < x < 34,5) \\
							   & = P(\frac{23,5 - 30}{\sqrt{30}} < z < \frac{34,5 - 30}{\sqrt{30}}) \\
							   & = P(-1,19 < z < 0,82) \\
							   & = P(0 < z < 0,82) - P(0 < z < 1,19) \\
							   & = 0,2939 + 0,3830 \\
							   & = 0,6769
			\end{split}
		\end{equation*}
	\end{enumerate}
}

\exercise{
	Extra: Een speler gooit met een dobbelsteen. Bepaal de kans dat hij bij 90 worpen tenminste 39 keer een 5 of een 6 gooit. Geef de exacte formule en bereken met de limiet-stelling.
}{
	Stel:
		\begin{itemize}
			\item $x : \#$ keer een vijf of zes.
			\item kans op x = p = $1/3$
			\item x is binomiaal verdeeld
			\item $f(i) = C_n^i p^i(1 - p)^{n - i}$ 
		\end{itemize}
	De exacte formule wordt:
	$$P(x \geq 39) = 1 - P(x \leq 38) = 1 - \sum_{i = 0}^{38} C_{90}^{i} \bigg(\frac{1}{3}\bigg)^i\bigg(\frac{2}{3}\bigg)^{n - i}$$
	We lossen dit op met de limietstelling 2 van het formularium. Bewijs:
	\begin{itemize}
		\item $np = 90\frac{1}{3} = 30 \geq 5$
		\item $n(1 - p) = 90\frac{2}{3} = 60 \geq 5$
	\end{itemize}
	De variabele x benadert een normale verdeling.
	$$x : N(np, \sqrt{np(1 - p)}) = N(30, \sqrt{20})$$
	\begin{equation*}
		\begin{split}
			P(x \geq 39) & \approx P(x \geq 38,5) \\
						 & = P(z \geq \frac{38,5 - 30}{\sqrt{20}}) \\
						 & = P(z \geq 1,9) \\
						 & = 0.5 - 0.4713 \\
						 & = 0.0287
		\end{split}
	\end{equation*}
}

\exercise{
	13. Stel dat $x : t(20 df)$ verdeeld is. Bepaal
		\begin{enumerate}
			\item $a$ zodat $P(x > a) = 0.3$
			\item $b$ zodat $P(|x| > b) = 0.2$
		\end{enumerate}
}{
	\begin{enumerate}
		\item $1 - 0.3 = 0.7$
			$$a = t_{0.7}(20 df) = 0.533$$
		\item $|x| > b \rightarrow x < -b \quad \hbox{en} \quad x > b$
			Dus $P(x < -b) + P(x > b) = 0.2$ maar wegens symmetrie beschouwen we $P(x > b) = 0.1$
			Hieruit volgt: $b = t_{0.9}(20 df) = 1.32$
	\end{enumerate}
}

\exercise{
	14. Stel $x: N(0, 3)$ en $y: \chi^2(16 df)$ verdeeld met $x$ en $y$ onafhankelijk. Bepaal $a < 0$ zodat $P(ax \geq \sqrt{y}) = 0.3$. 
}{
	We weten dat de $t$-verdeling kan omschreven worden als $t(v df) = \frac{z}{\sqrt{y/v}}$ met $z: N(0, 1)$ en $y : \chi^2(v df)$. In ons geval herschrijven we de kans eerst als $P(ax \geq \sqrt{y}) = P\bigg(\frac{x}{\sqrt{y}} \leq \frac{1}{a}\bigg)$ zodat we de definitie van de $t$-verdeling kunnen toepassen. Eerst moeten we $x$ normeren $\rightarrow \frac{x - 0}{3}$ is $N(0, 1)$. Hieruit volgt dat 
	$$\frac{x/3}{\sqrt{y/16}}$$
	t(16 df) verdeeld is. De kans wordt:
	$$P\bigg(\frac{4}{3}\frac{x}{\sqrt{y}} \leq \frac{4}{3}\frac{1}{a}\bigg)$$
	Via het formularium kan afgleidt worden dat $-\frac{4}{3}\frac{1}{a} = t_{0.7}(16df) = 0.535 \Rightarrow a \approx -2.4922$
}

\exercise{
	15. Stel dat $x : F(10, 15 df)$ verdeeld is. Bepaal $a$ en $b$ zodat $P(x < a) = 0.95$ en $P(x > b) = 0.95$
}{
	$a = F_{0.05}(10, 15 df) = 2.54$ en $b = F_{0.95}(10 ,15 df) = \frac{1}{F_{0.05}(15, 10 df)} = \frac{1}{2.85} \approx 0.3509$
}

\exercise{
	18. Stel dat $x : \chi^2(350 df)$ verdeeld is, bepaald $P(x > 390)$
}{
	Een stelling dat NIET op het formularium staat zegt dat als $v \geq 30$, dan $z = \frac{x - v}{\sqrt{2v}}$ is $N(0, 1)$
	Hieruit volgt direct dat $\frac{x - 350}{\sqrt{700}} N(0, 1)$ verdeeld is. 
	De kans wordt:
	$$P(x > 390) = P\bigg(z > \frac{390 - 350}{\sqrt{700}}\bigg) = P(z > 1.51) = 0.5 - 0.4345 = 0.0655$$
}