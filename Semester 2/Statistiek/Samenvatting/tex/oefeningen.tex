\part{Oefeningen}
\chapter{Kansrekenen}
  
\begin{itemize}[label={}, leftmargin=*]
	\item {\exercise{2. Een geldstuk is vervalst zodat kop dubbel zoveel kan voorkomen als munt. Als het geldstuk drie keer geworpen wordt, wat is de kans om juist 2 keer munt te hebben?}{
		Er zijn twee evenementen te definieëren. Kop gooien \textbf{K} en munt gooien \textbf{M}. Kop gooien kan twee keer zoveel voorkomen als munt gooien.
		$$P(K) = 2P(M)$$
		We kunnen gebruik maken van het feit dat de som van alle kansen gelijk is aan 1.
		
		
		\begin{equation*}
			\begin{split}
				&  P(K) + P(M) = 1    \\
				\Leftrightarrow & 2P(M) + P(M) = 1    \\
				\Leftrightarrow & 3P(M) = 1           \\
				\Leftrightarrow & P(M) = \frac{1}{3}  \\
			\end{split}
		\end{equation*}
		dus
		$$P(K) = \frac{2}{3}$$
		Aangezien de gebeurtenis munt twee keer moet voorkomen moet kop dus slechts één maal voorkomen.
		$$P(2M \cap K)$$
		Er moet rekening gehouden worden met de verschillende combinaties:
		\begin{equation*}
			\begin{split}
				& P((M \cap M \cap K) \cup (M \cap K \cap M) \cup (K \cap M \cap M))  \\
				= & P(M \cap M \cap K) \cup P(M \cap K \cap M) \cup (K \cap M \cap M)) \\
				= & P(M)P(M)P(K) + P(M)P(K)P(M) + P(K)P(M)P(M) \\
				= & \frac{1}{3} \cdot \frac{1}{3} \cdot \frac{2}{3} + \frac{1}{3} \cdot \frac{2}{3} \cdot \frac{1}{3} + \frac{2}{3} \cdot \frac{1}{3} \cdot \frac{1}{3} \\
				= & 3(\frac{1}{3} \cdot \frac{1}{3} \cdot \frac{2}{3}) \\
				= & 3(\frac{2}{27}) = \frac{6}{27} = \frac{2}{9} 
			\end{split}
		\end{equation*}
	}}
	    
	    
	\item { \exercise{3. Een dobbelsteen is vervalst zodat de kans dat een gegeven aantal ogen geworpen wordt evenredig is met het aantal ogen. Is A de gebeurtenis een even aantal te gooien, B de gebeurtenis een priemgetal te gooien en C de gebeurtenis een oneven getal te gooien,
		\begin{enumerate}
			\item Bepaal P(A), P(B) en P(C)
			\item Bereken de kans dat men een even getal of een priemgetal gooit.
			\item Bereken de kans dat men een even getal gooit dat geen priemgetal is.
			\item Bereken de kans dat men een oneven getal of een priemgetal gooit.
		\end{enumerate}}
		{
			De kans kan als formule worden voorgesteld.
			$$P(i) = ip$$
			waarbij p een willekeurig getal is.
			We weten dat de som van alle kansen gelijk is aan 1.
			$$\sum_{i = 1}^{6} P(i) = 1p + 2p + 3p + 4p + 5p + 6p = 1$$
			Los op naar p
			$$21p = 1$$
			$$p = \frac{1}{21}$$
			De deeloplossingen:
			\begin{enumerate}
				\item   \begin{equation*}
				      \begin{split}
				      	P(A) = & \frac{2}{21} + \frac{4}{21} + \frac{6}{21} = \frac{12}{21} = \frac{4}{7}\\
				      	P(B) = &\frac{2}{21} + \frac{3}{21} + \frac{5}{21} = \frac{10}{21}\\
				      	P(C) = & 1 - P(A) = 1 - \frac{4}{7} = \frac{3}{7}
				      \end{split}
				\end{equation*}
				\item $$P(A\cup B) = P(A) + P(B) - P(A \cap B) = \frac{4}{7} +\frac{10}{21} - \frac{2}{21} = \frac{20}{21}$$
				\item $$P(A \cap \overline{B}) = P(A) - P(A \cap B) = \frac{4}{7} - \frac{2}{21} = \frac{10}{21}$$
				\item $$P(B \cup C) = P(B) + P(C) - P(B \cap C) = \frac{10}{21} + \frac{3}{7} - (\frac{3}{21} + \frac{5}{21}) = \frac{11}{21}$$
			\end{enumerate}}}
	  
	  
	\item {\exercise{4. A en B zijn verschijnselen met P(A) = 0.1, P(B) = 0.5. Bepaal $P(A \cup B)$, $P(\overline{A})$, $P(\overline{A} \cap B)$ indien a) de verschijnselen elkaar uitsluiten en b) ze onafhankelijk zijn.}
		{
			\begin{enumerate}[label=(\alph*)]
				\item \begin{align*}
				      P(A \cup B) =  & P(A) + P(B) - P(A \cap B) = \frac{1}{10} + \frac{1}{2} - 0 = \frac{3}{5} \\
				      P(\overline{A}) =  & 1 - P(A) = 1 - \frac{1}{10} = \frac{9}{10} \\
				      P(\overline{A} \cap B) = & P(B) - P(A \cap B) = \frac{1}{2} - 0 = \frac{1}{2} \\
				\end{align*}
				\item \begin{align*}
				      P(A \cup B) = & P(A) + P(B)  - P(A \cap B) = \frac{1}{10} + \frac{1}{2} - \frac{1}{20} = \frac{11}{20} \\
				      P(\overline{A}) =  &1 - P(A) = 1 - \frac{1}{10} = \frac{9}{10} \\
				      P(\overline{A} \cap B) = & P(B) - P(A \cap B) = \frac{1}{2} - \frac{1}{20} = \frac{9}{20}
				\end{align*}
			\end{enumerate}}}
	
	\item \exercise{6. Bereken voor een familie van 3 kinderen de kans op a) 3 jongens en  b) 2 jongens en 1 meisje}{
	      $$P(J) \hbox{: Kans op een jongen} = \frac{1}{2}$$
	      $$P(M) \hbox{: Kans op een meisje} = \frac{1}{2}$$
	      \begin{enumerate}[label=(\alph*)]
	      	\item \begin{equation*}
	      	      \begin{split}
	      	      	P(J \cap J \cap J) & = P(J)P(J)P(J)\\
	      	      	& = (P(J))^{3} \\
	      	      	& = \bigg(\frac{1}{2}\bigg)^{3} \\
	      	      	& = \frac{1}{8}
	      	      \end{split}
	      	\end{equation*}
	      	\item \begin{equation*}
	      	      \begin{split}
	      	      	& P(J \cap J \cap M) + P(J \cap M \cap J) + P(M \cap J \cap J)  \\
	      	      	= & 3P(J \cap J \cap M)\\
	      	      	= & 3(P(J))^3 \;\;\;\;\; \hbox{(aangezien P(J) = P(M))} \\
	      	      	= & 3 \bigg(\frac{1}{2}\bigg)^{3} = \frac{3}{8}
	      	      \end{split}
	      	\end{equation*}
	      \end{enumerate}}
	      
	      
	      
	\item{\exercise{7. Een paar onvervalste dobbelstenen worden geworpen. Wat is de kans dat de som van de ogen een totaal van minstens 8 vertoont.}{
	      \begin{equation*}
	      	\begin{split}
	      		A =  &\;\hbox{som van de ogen} \geq 8\\
	      		P(A) =  &\{2, 6\} \cup \{3, 5\} \cup \{3, 6\} \cup ... \cup \{6,6\} \\
	      		P(A) = & \frac{15}{36} = \frac{5}{12}
	      	\end{split}
	      \end{equation*}}}
	      
	\item{\exercise{11. Gegeven 3 kasten A, B en C. Elke kast heeft een aantal laden die ofwel een goudstuk(G), ofwel een zilverstuk(Z) ofwel niets bevatten(N) en dit als volgt:
	\begin{itemize}[label={}]
	 \item A: [G|G|G|Z]
	 \item B: [G|Z|Z]
	 \item C: [G|G|N|Z|Z]
	\end{itemize}
	Men kiest willekeurig één van de kasten, opent daarvan at random één lade, en grijpt het muntstuk(indien mogelijk). Wat is de kans dat men kast A heeft uigekozen indien men een goudstuk heeft genomen?
	}{
	De kans om eender welke lade te pakken:
	$$P(A) = P(B) = P(C) = \frac{1}{3}$$
	De kans om een goudstuk te pakken afhankelijk van de lade:
	\begin{gather*}
	 P(G|A) = \frac{3}{4}\\
	 P(G|B) = \frac{1}{3}\\
	 P(G|C) = \frac{2}{5}
	\end{gather*}
	De kans dat lade A gekozen werd indien het een goudstuk genomen is:
	\begin{equation*}
	 \begin{split}
	  P(A|G) & = \frac{P(A)P(G|A)}{P(A)P(G|A)+P(B)P(G|B)+P(C)P(G|C)} \\
	         & = \frac{\frac{1}{3}\cdot\frac{3}{4}}{\frac{1}{3}\cdot\frac{3}{4}+\frac{1}{3}\cdot\frac{1}{3}+\frac{1}{3}\cdot\frac{2}{5}} \\
	         & = \frac{\frac{1}{4}}{\frac{1}{4} + \frac{1}{9} + \frac{2}{15}} \\
	         & = \frac{45}{89}
	 \end{split}
	\end{equation*}

	}}
	
	\item{\exercise{16. Wanneer men het waarheidsserum toedient aan een schuldig persoon is het voor 90\% betrouwbaar en aan een onschuldig persoon is het voor 99\% betrouwbaar. Als een verdachte gekozen wordt uit een groep, waarvan 5\% reeds een misdrijf begaan hebben, wat is de kans dat die persoon niet schuldig is als het waarheidsserum schuldig aanwijst?}{
	2 Gebeurtenissen:
	\begin{enumerate}
	 \item PS: De persoon is schuldig
	 \item WS: Het waarheidsserum wijst schuldig aan
	\end{enumerate}
	De kans dat de persoon niet schuldig is als het waarheidsserum schuldig aanwijst:
	\begin{equation*}
	 \begin{split}
	 P(\overline{PS}|WS) & = \frac{P(\overline{PS})P(WS|\overline{PS})}{P(\overline{PS})P(WS|\overline{PS})+P(PS)P(WS|PS)} \\
	  & = \frac{\frac{19}{20}\cdot\frac{1}{100}}{\frac{19}{20}\cdot\frac{1}{100} + \frac{1}{20}\cdot\frac{9}{10}} \\
	  & = \frac{\frac{19}{2000}}{\frac{19}{2000} + \frac{9}{200}} \\
	  & = \frac{\frac{19}{2000}}{\frac{19}{2000} + \frac{90}{2000}} \\
	  & = \frac{\frac{19}{2000}}{\frac{109}{2000}} \\
	  & = \frac{19}{109}
	 \end{split}
	\end{equation*}

	}
	}
	       
	\item \exercise{17. Uit een spel van 52 kaarten trekt men willekeurig maar tezelfdertijd vijf kaarten. Bereken de kans dat 
	      \begin{enumerate}
	      	\item het vijf zwarte kaarten zijn,
	      	\item het drie heren en twee vrouwen zijn,
	      	\item er tenminsten één aas bij is,
	      	\item er ten hoogste één harten bij is.
	      \end{enumerate}}{
	      \begin{enumerate}
	      	\item {
	      		\begin{equation*}
	      			\begin{split}
	      				P(5Z) & = \frac{C_{26}^{5}}{C_{52}^{5}}                     \\
	      				& = \frac{\frac{26!}{5!(26 - 5)!}}{\frac{52!}{5!(52 - 5)!}}\;\;\; \hbox{(vereenvoudigen voor rekenmachine)}\\
	      				& = \frac{26!}{21!} \cdot \frac{47!}{52!}                   \\
	      				& = \frac{26 \cdot 25 \cdot 24 \cdot 23 \cdot 22}{52 \cdot 51 \cdot 50 \cdot 49 \cdot 48} \\
	      				& = 0.025
	      			\end{split}
	      		\end{equation*}
	      	}
	      	\item {
	      		$$P(3H \cap 2V) = \frac{C_{4}^{3} \cdot C_{4}^{2}}{C_{52}^{5}}$$
	      		Meestal zullen ze vragen 'geef de correcte uitdrukking' om geen tijd te verliezen aan banaal rekenwerk.
	      	}
	      	\item {
	      		$$P(\hbox{minstens 1 aas}) = 1 - P(\hbox{geen aas}) = 1 - \frac{C_{4}^{0} \cdot C_{48}^{5}}{C_{52}^{5}}$$
	      	}
	      	\item {
	      		\begin{equation*}
	      			\begin{split}
	      				P(\hbox{ten hoogste 1 hart}) & = P(\hbox{0 hart} \cap \overline{\hbox{5 hart}}) \cup P(\hbox{1 hart} \cap \overline{\hbox{4 hart}})  \\
	      				& =  \frac{C_{39}^{5}}{C_{52}^{5}} + \frac{C_{13}^{1} \cdot C_{39}^{4}}{C_{52}^{5}}
	      			\end{split}
	      		\end{equation*}
	      	}
	      	   
	      \end{enumerate} }
	  
	\item{\exercise{20. Bepaal de kans om minstens één maal zes te gooien bij 4 worpen met een dobbelsteen. Bepaal de kans om minstens één maal dubbel zes te gooien bij 24 worpen met 2 dobbelstenen.}{
	        
	      \begin{equation*}
	      	\begin{split}
	      		P(\hbox{minstens één 6 bij 4 worpen}) & = 1 - P(\hbox{geen 6 bij 4 worpen})\\
	      		& = 1 - \bigg(\frac{5}{6}\bigg)^4
	      	\end{split}
	      \end{equation*}
	      \begin{equation*}
	      	\begin{split}
	      		P(\hbox{minstens één keer dubbel 6 bij 24 worpen}) & = 1 - P(\hbox{geen dubbel 6 bij 24 worpten})\\
	      		& = 1 - \bigg(\frac{35}{36}\bigg)^{24}
	      	\end{split}
	      \end{equation*}
	        
	}}
	  
	\item{\exercise{21. Bepaal de kans om met de belgische lotto a)drie cijfers b)vier cijfers en c)zes cijfers goed te hebben.}{  $$P(x) = \frac{C_6^x \cdot C_{42 - x}^{6 - x}}{C_{42}^{6}}$$
	x in te vullen met 3, 4 of 6.}}
	
	\item {\exercise{23. Een urne bevat 10 ballen waarvan \textit{n} rode. De kans om twee rode ballen te trekken op drie trekkingen met terugleggen is 1.08 keer de kans om twee rode ballen te trekken op drie trekkingen zonder terugleggen. Bepaal \textit{n} en sluit de triviale gevallen uit.}{
	Twee gebeurtenissen:
	\begin{enumerate}
	 \item A: 2 rode ballen trekken op 3 trekkingen met teruglegging
	 \item B: 2 rode ballen trekken op 3 trekkingen zonder teruglegging
	\end{enumerate}
	Verder geldt dat $P(A) = 1.08P(B)$ en $2 \leq n \leq 10$
	
	De kans op B:
	\begin{equation*}
	 \begin{split}
	  P(B) & = \frac{C_{n}^{2}C_{10 - n}^{1}}{C_{10}^{3}} \\
	       & = \frac{\frac{n!}{2!(n-2)!}\frac{(10-n)!}{1!(9-n)!}}{\frac{10!}{3!7!}} \\
	       & = \frac{n!}{2!(n-2)!}\frac{(10-n)!}{(9-n)!}\frac{3!7!}{10!} \\
	       & = \frac{3n(n-1)(10-n)}{10\cdot9\cdot8}
	 \end{split}
	\end{equation*}
	De kans op geen rode bal:
	\begin{equation*}
	 \begin{split}
	  & P(R) = \frac{n}{10} \\
	  \Rightarrow&  P(\overline{R}) = 1 - \frac{n}{10} = \frac{10 - n}{n} 
	 \end{split}
	\end{equation*}
	De kans op A:
	\begin{equation*}
	 \begin{split}
	  P(A) & = C_3^2P(R)^2P(\overline{R}) \\
	  & = 3\bigg(\frac{n}{10}\bigg)^2\bigg(\frac{10 - n}{10}\bigg) \\
	  & = \frac{3n^2(10 - n)}{10^3}
	 \end{split}
	\end{equation*}

	Berekening van \textit{n}
	\begin{equation*}
	 \begin{split}
	  & P(A) = 1.08P(B) \\
	  \Leftrightarrow & \frac{3n^2(10 - n)}{10^3} = \frac{3n(n-1)(10-n)}{10\cdot9\cdot8} \\
	  \Leftrightarrow & \frac{n}{10^2} = 1.08\frac{n - 1}{9\cdot 8} \\
	  \Leftrightarrow & \frac{n}{100} = 1.08\frac{n - 1}{72} \\
	  \Leftrightarrow & 72n = 108(n-1)\\
	  \Leftrightarrow & 72n = 108n - 108 \\
	  \Leftrightarrow & 108n - 72n = 108 \\
	  \Leftrightarrow & 36n = 108 \\
	  \Leftrightarrow & n = 3
	 \end{split}
	\end{equation*}

	  }
	
	
	}
\end{itemize}
\chapter{Verdelingsfunctie van een populatie}
\begin{itemize}
 \item{\exercise{3. Ga na of de volgende functi F(x) een cumulatieve distributiefunctie kan zijn. Indien ja, bepaal de corresponderende dichtheidsfunctie f(x)
 $$F(x) = \begin{cases}
           0 & x \leq 0 \\
           1 - e^{-x^{2}} & x > 0
          \end{cases}
$$}{
Voer de drie controles uit:
\begin{enumerate}
 \item De functie is nooit negatief.
 \item De functie is nooit dalend.
 \item De limiet naar $+\infty$ is 1.
\end{enumerate}
\begin{enumerate}
\item Controle functie nooit negatief:
\begin{equation*}
 \begin{split}
  & 1-e^{-x^{2}} > 0 \\
  \Leftrightarrow & -e^{-x^{2}} > - 1 \\
  \Leftrightarrow & e^{-x^{2}} < 1 \\
  \Leftrightarrow & e^{-x^{2}} > e^0 \\
  \Leftrightarrow & -x^{2} > 0 \\
 \end{split}
\end{equation*}
\item Controle functie nooit dalend:
\begin{equation*}
 \begin{split}
  F'(x) = 2xe^{-x^2}
 \end{split}
\end{equation*}
De afgeleide is altijd positief, dus daalt de functie nooit.
\item Controle limiet naar $+\infty$ is gelijk aan 1.
\begin{equation*}
 \begin{split}
  \lim_{t\to+\infty}F(t) & = \lim_{t\to+\infty}1 - e^{-x^{2}} \\
                         & =  \lim_{t\to+\infty}1 - \frac{1}{e^{x^{2}}} \\
                         & = 1 - 0 = 1
 \end{split}
\end{equation*}
De kansfunctie hebben we al berekent in stap 2. $f(x) = 2xe^{-x^{2}}$
\end{enumerate}


}}
 \item{\exercise{5. Bepaal C zodat de volgende functie een dichtheidsfunctie is. Bepaal de corresponderende cumulatieve distributiefunctie
 $$f(x) = \begin{cases}
           C(4x - 2x^2) & 0 < x < 2 \\
           0 & \hbox{elders}
          \end{cases}
$$}{
Bepalen C:
\begin{equation*}
 \begin{split}
  & \int_{-\infty}^{+\infty}C(4x-2x^2)dx = 1 \\
  \Leftrightarrow & 2C\int_{0}^{2}2x-x^2dx = 1 \\
  \Leftrightarrow & 2C\bigg[\frac{2x^2}{2} - \frac{x^3}{3}\bigg]_0^2 = 1 \\
  \Leftrightarrow & 2C\bigg[x^2 - \frac{x^3}{3}\bigg]_0^2 = 1 \\
  \Leftrightarrow & 2C\bigg[\bigg(4 - \frac{8}{3}\bigg) - \bigg(0 - \frac{0}{3}\bigg)\bigg]= 1 \\
  \Leftrightarrow & 2C\bigg(\frac{4}{3}\bigg)= 1 \\
  \Leftrightarrow & C = \frac{3}{8}
 \end{split}
\end{equation*}
Dus:
 $$f(x) = \begin{cases}
           \frac{3}{8}(4x - 2x^2) & 0 < x < 2 \\
           0 & \hbox{elders}
          \end{cases}
$$
De cumulatieve distributiefunctie:
$$F(x) = \begin{cases}
          0 & x \leq 2 \\ 
          0 + \frac{3}{8}\int_0^x (4t - t^2)dt = \frac{3}{8}\bigg(\frac{4x^2}{2} - \frac{2x^3}{3}\bigg) & 0 < x < 2 \\
          1 & x \geq 2
         \end{cases}
$$
}}
 \item{\exercise{6. Een vaas bevat 4 rode en 6 witte ballen. Men neemt 3 ballen zonder teruglegging. Als x het aantal getrokken rode ballen voorstelt bepaal dan:
    \begin{enumerate}
     \item de dichtheidsfunctie van x
     \item de grafiek van de dichtheidsfunctie en de corresponderende verdelingsfunctie 
     \item de gemiddelde waarde en de modus
     \item de variantie
     \item $P(x \geq 1)$
    \end{enumerate}
}{
    \begin{enumerate}
     \item  $x = \# $ getrokken rode ballen.
     
            De kans om x = 0, 1, 2 of 3 rode ballen te trekken kan voorgesteld worden als :
            $$f(x) = \frac{C_4^xC_6^{3 - x}}{C_{10}^3}$$
            Bijgevolg zijn de kansen:
            \begin{tabular}{c | c c c c}
                x & 0 & 1 & 2 & 3 \\
                f(x) & $\frac{1}{6}$ & $\frac{1}{2}$ & $\frac{3}{10}$ & $\frac{1}{30}$
            \end{tabular}
    \item \todo{grafiek}
        $$F(x) = \begin{cases}
                    0 & x < 0 \\
                    \frac{1}{6} & 0 \leq x \leq 1 \\
                    \frac{1}{6} + \frac{1}{2} = \frac{2}{3} & 1 \leq x \leq 2 \\
                    \frac{2}{3} + \frac{3}{10} = \frac{29}{30} & 2 \leq x \leq 3 \\
                    \frac{29}{30} + \frac{1}{30}= 1 & x \geq 3 \\
                    
                 \end{cases}$$
    \item Het gemiddelde:
        \begin{equation*}
         \begin{split}
          \mu = \sum_{i = 0}^{3}x_i f(x_i) \\
            & = 0f(0) + 1f(1) + 2f(2) + 3f(3) \\
            & = 0 + \frac{1}{6} + 2\frac{2}{3} + 3\frac{29}{30} \\
            & = 1.2
         \end{split}
        \end{equation*}
        De modus is 1.
    \item De variantie:
        \begin{equation*}
         \begin{split}
          \sigma^2 & = \bigg[\sum x^2_i \big(f(x_i)\big)\bigg] - \mu^2 \\
                   & = 0^2f(0) + 1^2f(1) + 2^2f(2) + 3^2f(3) - \mu^2 \\
                   & = 0.56
         \end{split}
        \end{equation*}
    \item 
        $$P(x \geq 1) = 1 - P(x < 1) = 1 - F(0) = 1 - \frac{1}{6} = \frac{5}{6}$$
    \end{enumerate}
}}
\item {
    \exercise{
        10. Voor een gokspel met drie onvervalste dobbelstenen bedraagt de inzet steeds 5 euro. Indien iemand juist één 6 werpt krijgt hij zijn inzet terug, indien juist twee stenen een 6 vertonen krijgt hij 10 euro terug en indien de drie dobbelstenen een 6 vertonen krijgt hij 15 euro. Wat is de gemiddelde winst (of verlies)?
    }{
        De verwachte waarde van een functie:
        $$E[g(x)] = \sum g(x_i) f(x_i)$$
        met $g(x) = $ de winst met een inzet van 5 euro
        
        en $f(x) = $ de kans om x aantal zessen te hebben.
        
        $x = \#6 $ bij het werpen van een dobbelsteen 
        De functie $f$ kan geschreven worden als:
        $$f(x) = C_{3}^{x}\bigg(\frac{1}{6}\bigg)^x \bigg(\frac{5}{6}\bigg)^{3-x}$$
        Dus:
        \begin{tabular}{c | c c c c}
            x & 0 & 1& 2& 3 \\
            f(x) & $\big(\frac{5}{6}\big)^3$ & $3\frac{5^2}{6^3}$ & $\frac{15}{6^3}$ & $\frac{1}{6^3}$ \\
            g(x) & -5 & 0 & 5 & 10
        \end{tabular}
        Hieruit volgt:
        \begin{equation*}
         \begin{split}
          E[g(x)] & = f(0)g(0) + f(1)g(1) + f(2)g(2) + f(3)g(3) \\
                  & = -2.5
         \end{split}
        \end{equation*}
        Er is dus gemiddeld een verlies van -2,5 euro
    }
}
\item {
    \exercise{
        13. Een toevalsveranderlijke x heeft een gemiddelde $\mu = 12$, een dispersie $\sigma = 3$ en zijn dichtheidsfunctie is niet gekend.
        
        Bepaal een ondergrens voor: $P(6 < x < 18)$ en voor $P(3 < x < 21)$
    }{
        \begin{equation*}
         \begin{split}
          P(6 < x < 18) & = P(6 - \mu < x - \mu < 18 - \mu) \\
                        & = P(-6 < x - \mu < 6) \\
                        & = P(-2\sigma < x - \mu < 2\sigma) \\
                        & = P(|x - \mu| < 2\sigma) \\
                        & = 1 - \frac{1}{2^2} \\
                        & = 1 - \frac{1}{4} \\
                        & = \frac{3}{4}
         \end{split}
        \end{equation*}
        \begin{equation*}
         \begin{split}
          P(3 < x < 21) & = P(3 - \mu < x - \mu < 21 - \mu) \\
                        & = P(-9 < x - \mu < 9) \\
                        & = P(-3\sigma < x - \mu < 3\sigma) \\
                        & = P(|x - \mu| < 3\sigma) \\
                        & = 1 - \frac{1}{3^2} \\
                        & = 1 - \frac{1}{9} \\
                        & = \frac{8}{9}
         \end{split}
        \end{equation*}

    }
}
\item {
    \exercise{
        16. Onderstel dat het aantal producten in een fabriek, aangemaakt gedurende één week, een stochastische veranderlijke is met gemiddelde $\mu = 50$. Deze veranderlijke heeft een symmetrische verdeling t.o.v. $\mu$
        \begin{enumerate}
         \item Bepaal een bovengrens voor de kans dat de productie van een bepaalde week minstens 75 zal dragen.
         \item Indien bovendien de variantie gekend is ($\sigma^2 = 25$) wat is dan een ondergrens voor de waarschijnlijkheid dat de productie van een bepaalde week strikt tussen 40 en 60 zal liggen.
        \end{enumerate}
    }{
        \begin{enumerate}
         \item \begin{equation*}
                \begin{split}
                P(x \leq \mu - k\sigma) = P(x \geq \mu + k\sigma) \leq \frac{1}{2k^2} \\
                \Rightarrow p(x \geq 75) & = P(x -\mu \geq 25) = \frac{1}{2k^2} \\
                                        & = \frac{1}{2(\frac{25}{\sigma})^2} \\
                                        & = \frac{\sigma^2}{1250}
                \end{split}
                \end{equation*}
        \item 
            $$\sigma = \sqrt{\sigma^2} = \sqrt{5^2} = 5$$
             \begin{equation*}
               \begin{split}
                P(40 < x < 60) & = P(-10 < x - \mu < 10) \\
                               & = P(-2\sigma < x - \mu < 2\sigma) \\
                               & = 1 - \frac{1}{2^2} \\
                               & = \frac{3}{4}
               \end{split}
              \end{equation*}

        \end{enumerate}


    }
}
\end{itemize}

\chapter{Discrete verdelingen}
\begin{itemize}[label={}]
 \item {\exercise{
    8. Het zelfmoordpercentage in een Amerikaanse staat bedraagt per maand 1 per 100 000 inwoners.
    \begin{enumerate}
     \item Wat is de kans dat van 400 000 inwoners in deze staat er 8 of meer zelfmoorden plaatsgrijpen in een bepaalde maand?
     \item Wat is de kans dat er tenminste 2 maanden in het jaar zullen zijn met minstens 8 zelfmoorden?
    \end{enumerate}
 }{
    \begin{enumerate}
     \item  Gem \# zelfmoorden bij 100 000 inw/maand = 1\\
            Gem \# zelfmoorden bij 400 000 inw/maand = 4 \\
            Dit is poisson verdeeld met $\lambda = 4$ dus $f(i) = e^{-4}\frac{4^i}{i!}$
            \begin{equation*}
            \begin{split}
            P(x \geq 8) & = 1 - p(x < 8) \\
                        & = 1 - \sum_{i = 0}^{7}e^{-4}\frac{4^i}{i!} \\
                        & = 1 - e^{-4}\sum_{i = 0}^{7}\frac{4^i}{i!} \\
                        & \approx 0.0511
            \end{split}
            \end{equation*}
    \item \begin{itemize}
            \item {n = 2}
            \item {i = \# maanden met $>$ 8 zelfmoorden}
            \item {p = 0.0511}
          \end{itemize}
          \begin{equation*}
            \begin{split}
                P(i \geq 2) &= 1 - P(i < 2) \\
                            &= 1 - \bigg(C_{12}^{0}p^{0}(1 - p)^{12} + C_{12}^{1}p^{1}(1 - p)^{11}\bigg) \\
                            &\approx 0.1227
            \end{split}
          \end{equation*}  
    \end{enumerate}
 }}
 \item {
    \exercise{
        Extra oefening.         
        Het aantal olietankers dat per dag een bepaalde haven binnenvaart is poisson verdeeld met gemiddelde 2. De haven kan ten hoogste 3 olietankers per dag verwerken. Als er 2 olietankers per dag binnenvaren dan worden ze doorgestuurd.
            \begin{enumerate}
             \item Bepaal de kans dat er op een dag olietankers worden bediend.
             \item Bepaal het gemiddelde aantal olietankers dat per dag bediend wordt.
            \end{enumerate}
    }{
        \begin{enumerate}
         \item \begin{itemize}
                    \item $x_i$ : \# olietankers die binnenvaren 
                    \item $x$   : poisson verdeeld
                    \item $\lambda = 2 \Rightarrow f(i) = e^{-2}\frac{2^i}{i!}$
                \end{itemize}
                \begin{equation*}
                 \begin{split}
                  P(x > 3- & = 1 - P(x < 3) \\
                            & = 1 - \sum_{i = 0}^{3}e^{-2}\frac{2^i}{i!} \\
                            & = 1 - e^{-2}(1 + 2 + 2 + \frac{8}{6}) \\
                            & \approx 0.1429
                 \end{split}
                \end{equation*}
        \item y: \# olietankers die bediend worden $\Rightarrow$ E[y] $\Rightarrow \sum x_if(x_i)$
            \begin{equation*}
             \begin{split}
              & 0f(0) + 1f(1) + 2f(2) + 3f(3) + 3P(x > 3) \\
              & = 0 + 1\cdot 2e^{-2} + 2\cdot 2e^{-2}+ 3\cdot \frac{8}{6}e^{-2} + 3\cdot 0.1429\\
              & \approx 1.7821
             \end{split}
            \end{equation*}
        \end{enumerate}
    }
 }
 \item {
    \exercise{
        13. Een machine produceert bouten waarvan er $2\%$ defect zijn. Wat is de kans dat bij 50 bouten ten hoogste 2 defect zijn?
    }{
        \begin{itemize}
            \item n = 50
            \item p = 0.02
            \item i : \# bouten dat defect zijn
            \item i: binomiaal verdeeld
        \end{itemize}
        \begin{equation*}
         \begin{split}
          p(i \leq 2) & = \sum_{i = 0}^{2} C_{50}^ip^i(1 - p)^{50 - i} \\
                      & = 0.9216
         \end{split}
        \end{equation*}
        Kan ook opgelost worden met de benadering van poisson want $n = 50 \geq 50$ en $p = 0.02 \leq 0.1$ dus $\lambda = np = 1$ en $f(i) = e^{-1}\frac{1}{i!}$
        \begin{equation*}
         \begin{split}
          P(i \leq 2 ) & = \sum_{i = 0}^{2}e^{-1}\frac{1}{i!} \\
                       & \approx 0.9197
         \end{split}
        \end{equation*}
    }
 }
 \item {
    \exercise{ 
        17. De kans dat een geïnfecteerd persoon sterft aan een ademhalingsinfectie is 0.002. Bepaal de kans dat er minder dan 5 personen van een groep van 2000 geïnfecteerde personen zullen sterven. Gebruik de ongelijkheid van Chebychev en interpreteer het interval $]\mu - 2\sigma, \mu + 2\sigma[$
    }{
        De verdeling is binomiaal verdeeld maar we gebruiken de poisson benadering (+ bewijs).
        \begin{itemize}
            \item {$n = 2000 \geq 50$}
            \item {$p = 0.002 \leq 0.1$}
            \item {$\lambda = np = 2000\cdot(0.002) = 4$}
            \item {$f(i) = e^{-4}\frac{2^i}{i!}$}
        \end{itemize}
        \begin{enumerate}
         \item 
                \begin{equation*}
                    \begin{split}
                     P(i < 5) & = \sum_{i = 0}^{4}e^{-4}\frac{2^i}{i!}\\
                              & \approx 0.6228
                    \end{split}
                \end{equation*}
         \item
            Het interval $]\mu - 2\sigma, \mu + 2\sigma[$:
            \begin{equation*}
             \begin{split}
              & P(|x -\mu| < k\sigma) \geq 1 - \frac{1}{k^2} \\
               \Leftrightarrow & P(-k\sigma + \mu < x < k\sigma + \mu) \geq 1 - \frac{1}{k^2} \\
               & \mu = \lambda = 2 \rightarrow \sigma = \sqrt{\lambda} = 2 \\
               \hbox{dus} & ]4 - 2\cdot 2, 4 + 2\cdot 2[ \Rightarrow ]0, 8[ \\
               \Leftrightarrow & P(0 < x < 8) \geq 1 - \frac{1}{k^2} \\
                              & = 0.75\\
             \end{split}
            \end{equation*}

        \end{enumerate}

    }
 }
 \item {
    \exercise{ 
        18. Uit een kaartspel met 52 kaarten wordt 1 kaart getrokken en teruggelegd.
        \begin{enumerate}
         \item Wat is de kans det men bij de vierde beurt voor het eerst een aas trekt
         \item Wat is de kans dat men minstens zes beurten nodig heeft om een aas te trekken
        \end{enumerate}
    }{
        Geometrische verdeling gebruiken
        \begin{enumerate}
         \item $x = 4$ en $p = \frac{1}{13}$
            \begin{equation*}
            \begin{split}
            f(4) & = \bigg(\big(1 - \frac{1}{13}\big)^{4 - 1}\cdot\frac{1}{13}\bigg) \\
                    & = \frac{\big(1 - \frac{1}{13}\big)^{3}}{13} \\
                    & \approx 0.0605
            \end{split}
            \end{equation*}
         \item 
            \begin{equation*}
             \begin{split}
              P(\hbox{minstens 6 beurten nodig om een aas te trekken}) & = \big(\frac{12}{13}\big)^{5} \\
              & \approx 0.6702
             \end{split}
            \end{equation*}
        \end{enumerate}
    }
 }
 \item{
    \exercise{
        Extra oefening. Het aantal fouten dat een bepaalde secretaresse op een blad typt is poisson verdeeld met gemiddelde 1. De secretaresse typt 50 bladzijden. Wat is de kans dat de eerste fout op bladzijde 3 staat.
    }{
        \begin{itemize}
            \item x: eerste blad met typfout $\rightarrow P(x = 3) = (1 - p)^2p$
            \item y: poisson met $\lambda = 1 \rightarrow f(i) = e^{-1}\frac{1^i}{i!}$
        \end{itemize}
        \begin{equation*}
         \begin{split}
          & P(y = 1) = 1 - f(0) = 1 - \frac{1}{e} \\
          & P(x = 3) = \big(\frac{1}{e}\big)^{2}\big(1 -\frac{1}{e}\big) \\
          & \approx 0.0855
         \end{split}
        \end{equation*}

    }
 }
\end{itemize}

\chapter{Continue verdelingenen}
\exercise{
    11. Een Geigerteller levert gemiddeld 30 tellen per minuut in de omgeving van een radio-actief materiaal. Stel dat het aantal tellen per minuut Poisson verdeeld is. Bepaal de kans dat 
    \begin{enumerate}
     \item er juist 32 tellen zijn,
     \item er tussen 23 en 35 tellen zijn, grenzen niet inbegrepen.
    \end{enumerate}
}{
	Stel: \begin{itemize}
			\item $x: \#$ tellen
			\item x is Poisson verdeeld met $\lambda = 30$
			\item $f(i) = \frac{e^{-30}(30)i}{i!}$
		  \end{itemize}
	\begin{enumerate}
		\item $P(i = 32) = \frac{e^{-30}(30)^32}{32!} = ?$ (teveel werk met rekenmachine)

		We lossen door gebruik te maken van de limietstelling:
		$$\lambda = 30 \geq 15 \Rightarrow x:N(30, \sqrt{30})$$
		\begin{equation*}
			\begin{split}
				P(x = 32) & \approx P(31,5 < x < 32,5) \\
						  & = P(\frac{31,5 - 30}{\sqrt{30}} < z < \frac{32,5 - 30}{\sqrt{30}}) \\
						  & = P(0,27 < z < 0,46) \\
						  & = P(0 < z < 0,46) - P(0 < z < 0,27) \\
						  & = 0,1772 - 0,1064 \\
						  & = 0,0708
			\end{split}
		\end{equation*}
		\item 
		\begin{equation*}
			\begin{split}
				P(23 < x < 35) & \approx P(23,5 < x < 34,5) \\
							   & = P(\frac{23,5 - 30}{\sqrt{30}} < z < \frac{34,5 - 30}{\sqrt{30}}) \\
							   & = P(-1,19 < z < 0,82) \\
							   & = P(0 < z < 0,82) - P(0 < z < 1,19) \\
							   & = 0,2939 + 0,3830 \\
							   & = 0,6769
			\end{split}
		\end{equation*}
	\end{enumerate}
}
\todo{verder oefeningen 17\/04}

\chapter{Herhalingsoefeningen}
\begin{itemize}[label={}]
 \item {\exercise{1.Een zak bevat 10 rode, 6 groene en 4 blauwe bollen. Men trekt gelijktijdig 7 bollen uit de zak. Wat is de kans
 dat er minsten 1 groene en minstens 1 blauwe bal getrokken wordt?}{
 Twee gebeurtenissen: \begin{enumerate}
                       \item M1G: Minstens 1 groene bal getrokken
                       \item M1B: Minstens 1 blauwe bal getrokken
                      \end{enumerate}
 De kans om minstens 1 groene en 1 blauwe bal te trekken: 
 \begin{equation*}
  \begin{split}
   P(M1G \cap M1B) & = P(\overline{\overline{M1G} \cup \overline{M1B}}) \\
                   & = 1 - P(\overline{M1G} \cup \overline{M1B})        \\
                   & = 1 - P(0G \cup 0B) \\
                   & = 1 - [P(0G)+P(0B)-P(0G \cap 0B)] \\
                   & = 1 - \bigg[\frac{C_{4}^{0}C_{16}^{7}}{C_{20}^{7}} + \frac{C_{6}^{0}C_{14}^{7}}{C_{20}^{7}} - \frac{C_{4}^{0}C_{6}^{0}C_{10}^{7}}{C_{20}^{7}} \bigg] \\
                   & = 1 - \bigg[\frac{C_{16}^{7}}{C_{20}^{7}} + \frac{C_{14}^{7}}{C_{20}^{7}} - \frac{C_{10}^{7}}{C_{20}^{7}} \bigg] \\
                   & = 1 - \frac{1}{C_{20}^{7}}\bigg(C_{16}^{7} + C_{14}^{7} - C_{10}^{7}\bigg)
  \end{split}
 \end{equation*}

 }}
 
 \item{\exercise{16. Om te achterhalen of een persoon een bepaalde ziekte heeft, wordt een bloedtest genomen. Voor de personen die inderdaad ziek zijn, detecteert de bloedtest in 99\% van de gevallen de ziekte; echter voor de personen die niet zijk zijn blijk te bloedtest in 3\% van de gevallen wel ten onrechte de ziekte te detecteren. Als je weet dat 1\% van de bevolking de ziekte heeft, bepaal dan de kans dat een persoon de ziekte heeft in geval de bloedtest dit aangeeft.}{
 Twee gebeurtenissen:\begin{enumerate}
                      \item PZ: De persoon is ziek
                      \item BZ: De bloedtest toont ziek
                     \end{enumerate}
  De kans dat een persoon ziek is indien de bloedtest dit aangeeft:
  \begin{equation*}
   \begin{split}
    P(PZ|BZ) & = \frac{P(PZ)P(BZ|PZ)}{P(PZ)P(BZ|PZ) + P(PZ)P(\overline{BZ}|PZ)} \\
             & = \frac{\frac{1}{100}\cdot\frac{99}{100}}{\frac{1}{100}\cdot\frac{99}{100} + \frac{99}{100}\cdot\frac{3}{100}} \\
             & = \frac{\frac{99}{100^2}}{\frac{99}{100^2} + \frac{297}{100^2}} \\
             & = \frac{99}{396}
   \end{split}
  \end{equation*}


 
 }}
\end{itemize}
