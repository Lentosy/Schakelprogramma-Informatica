\documentclass{article}
\usepackage[utf8]{inputenc}
\usepackage[english]{babel}
\usepackage{color}
\usepackage{listings}

\def\warning#1{\color{red} #1 \color{black}}
\def\note#1{\color{cyan} #1 \color{black}}

\begin{document}
\pagenumbering{gobble}
\title{Examen Algoritmen I 8 juni 2018}
\date{}
\author{}
\maketitle

\section{Vraag 1}
    Gegeven de klasse \textbf{Openhash} in het bestand \textit{openhash.h}. Deze klasse maakt gebruik van open adressering om elementen van type T te hashen. Veronderstel dat T een functie \texttt{int T::hash(int grootte, int index) const} heeft. Deze functie geeft een correcte hashwaarde terug voor een element T. 
    
    Schrijf de functie \texttt{int verwijder(const T\& sleutel)} die alle kopie\"en die een bepaalde sleutel hebben verwijderd. De functie geeft het aantal verwijderde sleutels terug. Vul ook de enum \textbf{staat} aan.

\section{Vraag 2}
    Gegeven de klasse \textbf{GGraaf} in het bestand \textit{ggraaf.h}. Deze klasse stelt een gerichte lusloze graaf voor.
    
    Schrijf de functie \texttt{vector<int> GGraaf::sorteerTopologisch() const} die de graaf topologisch gesorteerd teruggeeft als een vector. Dat de graaf geen lussen heeft is een preconditie en moet dus niet gecontroleerd worden.

\end{document}
