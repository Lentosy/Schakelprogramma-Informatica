
\documentclass{report}
\usepackage[dutch]{babel}
\usepackage[left=3cm,right=3cm,top=3cm,bottom=3cm]{geometry}
\usepackage[utf8]{inputenc}
\usepackage{listings}
\usepackage{amsmath}

\setlength{\parindent}{0mm}


\title{Samenvatting Algoritmen I}
\author{}

\begin{document}
\maketitle
\tableofcontents

\part{Rangschikken}
\chapter{Inleiding}
Rangschikken is het sorteren van gegevens van klein naar groot.
    
De performantie van een algoritme hangt af van
\begin{itemize}
    \item het aantal gegevens (= n)
    \item de aard van die gegevens
    \item de mate waarin de gegevens reeds geordend zijn
    \item de plaats van de gegevens (\textit{inwendig} en \textit{uitwendig})
\end{itemize}
Naast de performantie is ook belangrijk:
\begin{itemize}
    \item Het geheugengebruik.
    \item De stabiliteit.
\end{itemize}
\chapter{Eenvoudige methodes}
\section{Rangschikken door tussenvoegen}
\subsection{Insertion Sort}
\begin{lstlisting}
void insertion_sort(vector<T>& v){
    for(int i = 1; i < v.size(); i++){
        T h = move(v[i]);
        int j = i - 1;
        while(j >= 0 && h < v[j]){
            v[j + 1] = move(v[j]);
            j--;
        }
        v[j + 1] = move(h);
    }
}
\end{lstlisting}
\subsubsection{Stappen}
\begin{enumerate}
 \item Overloop heel de tabel startend vanaf $i = 1$.
 \item Sla de waarde van het element op index $i$ tijdelijk op in $h$.
 \item Controleer of $h$ kleiner is dan de waarde op index $j = i - 1$.
 \item Indien ja, verplaats de waarde van index $j$ naar de index $j + 1$.
 \item Blijf dit doen totdat $j = 0$ of tot dat er een element gevonden is dat kleiner is als $h$.
 \item De waarde $h$ moet geplaatst worden op index $j + 1$.
\end{enumerate}
\subsubsection{Analyse}
Insertion Sort werkt door elementen één per één op te schuiven. Per schuifoperatie is er dus hoogsten één sleutelvergelijking. Helemaal op het einde als de tabel gesorteerd is, wordt er nog één keer gecontroleerd of alles op zijn plaats staat, dit is een extra term $n$. Een paar elementen dat niet in de juiste volgorde staan zijn een inversie.

 Voorbeeld met n = 4.
 
    $$|4|3|2|1|$$
    \begin{itemize}
        \item 4 vormt een inversie met 3, 2 en 1
        \item 3 vormt een inversie met 2 en 1
        \item 2 vormt een inversie met 1
    \end{itemize}
    Het aantal inversies is $\frac{4\cdot3}{2} = 6$. Er zullen 10 sleutelvergelijkingen en 6 schuifoperaties zijn. 

\begin{itemize}
 \item \textit{slechtste} geval | $\Theta(n^2)$: Aangezien er $n(n - 1)/2$ inversies zijn , zijn er $n(n - 1)/2$ schuifoperaties en $n(n - 1)/2 + n$ sleutelvergelijkingen.
 

 \item \textit{gemiddelde} geval | $\Theta(n^2)$: Elke permutatie van de tabelelementen zijn even waarschijnlijk. Gemiddeld is het aantal inversies gelijk aan $\frac{n(n - 1)}{4} (= \frac{n(n - 1)}{2}\cdot \frac{1}{2})$
 \item \textit{beste} geval | $O(n)$: De tabel is al gesorteerd en de while lus zal dus nooit uitgevoerd worden. Er is ook een speciaal geval als de tabel \textit{bijna} in juiste volgorde staat zolang het aantal inversies $O(n)$ is.
\end{itemize}
\subsubsection{Eigenschappen}
\begin{itemize}
 \item Sorteert ter plaatse
 \item Sorteert niet stabiel. Insertion Sort kan stabiel sorteren indien de lus achterstevoren uitgevoerd wordt.
\end{itemize}

\subsection{Shell Sort}
\begin{lstlisting}
void shell_sort(vector<T>& v){
    int k = ... // initieel increment
    while(k >= 1){
        for(int i = k; i < v.size(); i++){
            T h = move(v[i]);
            int j = i - k;
            while(j >=0 && h < v[j]){
                v[j + k] = move(v[j]);
                j -= k;
            }
            v[j + k] = move(h);
        }
        k = ... // volgend increment
    }
}
\end{lstlisting}
\subsubsection{Stappen}
\begin{enumerate}
 \item Stel $k$ gelijk aan de eerste waarde van een bepaalde reeks (zie analyse voor meer info over de reeks)
 \item Overloop de tabel startend vanaf $i = k$.
 \item Voer Insertion Sort uit met maar vervang de sprongen door $k$.
 \item kies de volgende waarde van de reeks en ga terug naar stap 2 zolang $k >= 0$.
\end{enumerate}
\subsubsection{Analyse}
Insertion Sort heeft in het slechtste geval een kwadratische uitvoeringstijd. Dit komt omdat elementen één per één opgeschoven worden, of het wegnemen van één inversie per schuifoperatie. Shell Sort lost dit op door grotere sprongen te nemen en dus meerdere inversies per sprong weg te nemen. Shell Sort gaat er voor zorgen dat elementen die $k$ posities van elkaar liggen gesorteerd worden. De reeksen die gebruikt worden bestaan meestal uit $\log n$ elementen. Belangrijke reeksen zijn:
\begin{itemize}
 \item Reeks van Shell(1959): $${\frac{n}{2},\frac{n}{4},\frac{n}{8},..., 1}$$
 \item Reeks van Sedgewick(1986): $$\sum_{k = 0}^{\log n} \begin{cases}
                                    9(2^k - 2^{k/2}) + 1 & \hbox{voor} \quad k \quad \hbox{even}\\
                                    8\cdot2^k - 6\cdot2^{(k+1)/2}) + 1 & \hbox{voor} \quad k \quad \hbox{oneven}
                                   \end{cases} (= 1, 5, 19, 41, 109, ...)$$ 
 \item Reeks van Tokuda(1992): $$\sum_{k = 0}^{\log n} \frac{1}{5}\bigg(9\cdot\bigg(\frac{9}{4}\bigg)^{k - 1} - 4\bigg) (= 1, 4, 9, 20, 46, 103, ...)$$
 \item Reeks van Ciura(2001): Experimenteel gevonden $1, 4, 10, 23, 57, 132, 301, 701, ...$
\end{itemize}
Belangrijk is dat de reeks van groot naar klein moet gaan, anders sorteer je eerst de tabel met Insertion Sort ($k = 1$) en dan hebben de grotere sprongen geen zin meer want de tabel is dan helemaal gesorteerd. De reeksen van Sedgewick,Tokuda en Ciura zijn gedefinieerd van klein naar groot maar moeten dus omgedraaid worden.

De performantie hangt af van de reeks. Voor de reeks van Shell kan het slechtste geval nog altijd $\Theta(n^2)$ zijn. Voor de reeks van Sedgewick kan aangetoond worden (via complexe analyse) dat de gemiddelde uitvoeringstijd slechts $O(n^{7/6})$. 
\subsubsection{Eigenschappen}
\begin{itemize}
 \item Sorteert ter plaatse
 \item Sorteert niet stabiel aangezien verschillende deelreeksen de volgorde van gelijke elementen kan wijzigen.
\end{itemize}

\section{Rangschikken door eenvoudig selecteren}
\subsection{Selection Sort}
\begin{lstlisting}
void selection_sort(vector<T>& v){
    for(int i = v.size() - 1; i > 0; i--){
        int imax = i;
        for(int j = 0; j < i; j++){
            if(v[j] > v[imax]){
                imax = j;
            }
        }
        swap(v[i], v[imax]);
    }
}
\end{lstlisting}
\subsubsection{Stappen}
\begin{itemize}
 \item Overloop de tabel in dalende volgorde startend vanaf $i = n - 1$. De index $i$ is de index waar we $k$-de grootste getal willen zetten ($k = 1, 2, ..., n$).
 \item De variabele $imax$ zal bijhouden waar het $k$-de grootste getal zit.
 \item Overloop de tabel nu startend vanaf $j = 0$ en zoek het grootste element tot en met $j = i -1$.
 \item Aangezien $imax$ de index bevat van het getal met de grootste waarde die op index $i$ moet komen, voeren we de swap uit.
\end{itemize}
\subsubsection{Analyse}
Het principe hier is: zoek het grootste element en zet het op de laatste index, zoek het tweede grootste element en zet het op de voorlaatste index, enz. Het aantal sleutelvergelijkingen is hier onafhankelijk van de oorspronkelijke volgorde (elk element moet altijd met elk ander element vergeleken worden). Het aantal sleutelvergelijkingen is dus 
$$\frac{n(n - 1)}{2}$$
Het aantal verwisselingen is wel slechts $n - 1$. Dit algoritme wordt gedomineerd door de binnenste for lus en is dus $\Theta(n^2)$ voor zowel het slechtste, beste als gemiddelde geval (aangezien die altijd uitgevoerd wordt onafhankelijk van de oorspronkelijke staat van de tabel).
\subsubsection{Eigenschappen}
\begin{itemize}
 \item Sorteert ter plaatse
 \item Sorteert niet stabiel
\end{itemize}

    
\chapter{Effici\"ente methodes}
\section{Rangschikken door effici\"ent selecteren}
\subsection{Heaps}
Een heap is een complete binaire boom en heeft dus volgende eigenschappen:
\begin{itemize}
 \item Elk niveau is volledig opgevuld. Indien een niveau niet volledig opgevuld is liggen alle knopen zoveel mogelijk links. Er kan dus hoogstens één knoop zijn met één kind.
 \item Elke knoop kan hoogstens 2 kinderen hebben. Een heap wordt vaak voorgesteld door een één dimensionale tabel. De wortel komt op index 0, zijn kinderen komen op index 1 en 2, de kinderen van 1 op plaatsen 3 en 4, enz. Dit kan vertaald worden naar eenvoudige formules met $i$ de index van een willekeurige knoop.
 \begin{enumerate}
  \item Het linkerkind van een knoop staat op positie $2i + 1$.
  \item Het rechterkind van een knoop staat op positie $2i + 2$.
  \item De ouder van een knoop staat op positie $(i - 1)/2$.
 \end{enumerate}
 \item De hoogte $h$ van de heap is gelijk aan $\left \lfloor \log_2 n \right \rfloor$.
 \item De heap moet voldoen aan de \textit{heapvoorwaarde}. 
 \begin{itemize}
  \item Bij een stijgende heap (maxheap) moet de sleutel van de ouder groter zijn dan de sleutels van zijn kinderen.
  \item Bij een dalende heap (minheap) moet de sleutel van de ouder kleiner zijn dan de sleutels van zijn kinderen.
 \end{itemize}
\end{itemize}
\subsection{Bewerkingen op heaps}
Een heap ondersteunt volgende bewerkingen:
\begin{itemize}
 \item \underline{Toevoegen van een element}. Voeg het element toe op index $n + 1$ en zorg ervoor dat de heapvoorwaarde terug voldaan is. Controleer dus of de ouder kleiner is dan het nieuwe element, en vervang desnoods. Doe dit zolang de volgende ouders ook kleiner zijn. De langste weg dat afgelegd moet worden is de hoogte $h$ van de heap. Toevoegen is $O(\log_2 n)$
 \item \underline{Het wortelelement vervangen}. Vervang de waarde van de wortel met een nieuwe waarde $g$. Controleer of de heapvoorwaarde nog altijd geldig is. In dit geval moeten de waarden van beide kinderen gecontroleerd worden. Indien de wortel kleiner is dan beide kinderen, moet het grootste kind naar de wortel gebracht worden. Doe dit zolang er kinderen zijn die groter zijn dan de waarde $g$.
 
 Ook hier is de langste weg dat afgelegd moet worden de hoogte $h$ van de heap. Deze operatie is $O(\log_2 n)$
 \item \underline{Het wortelelement verwijderen}. In dit geval verwissel je de wortel met het element op index $n - 1$, laat je de tabel krimpen zodat het laatste element (nu de wortel) weg is, en voer je de procedure van het wortelelement vervangen uit. Deze operatie is dan ook $O(\log_2 n)$
 \item \underline{Het element van een willekeurige knoop vervangen}. Hier zijn er twee mogelijkheden. Als het nieuwe element groter is dan het vorige, kan enkel de heapvoorwaarde naar de wortel verstoord worden. Dit komt neer op het toevoegen van een element startend bij de aangepaste knoop. Als de nieuwe knoop kleiner is dan het vorige, kan enkel de heapvoorwaarde van de deelheap waarvan die knoop wortel is, verstoord geraken. Dit komt neer op het uitvoeren van het vervangen van een wortelelement op die deelheap. Beide operaties zijn dus $O(\log_2 n)$
\end{itemize}

\subsection{Constructie van een heap}
Een heap kan op twee manier geconstrueerd worden.
\begin{enumerate}
 \item \underline{Door toevoegen}. Elk element gaat één per één toegevoegd worden. Na elke toevoegoperatie moet de heapvoorwaarde voldaan zijn. Aangezien er slechts $n - 1$ toevoegingen zijn, en dat elk element hoogstens helemaal naar boven moet stijgen komt de uitvoeringstijd neer op:
 $$T(n) = 2\cdot 1 + 4\cdot + 8\cdot 3 + ... + 2^{h - 1}(h - 1) + 2^h h$$
 wat neerkomt op $O(n \log_2 n)$
 \item \underline{Door samenvoegen van deelheaps}. TODO lol ik snap dit niet $O(n)$
\end{enumerate}

\subsection{Heap sort}
\begin{lstlisting}
void heap_sort(vector<T>& v){
    maak_heap(v);
    int i = v.size() - 1;
    while(i > 0){
        swap(v[i], v[0]);
        i -= 1;
        herstel_heap(v, 0, i);
    }
}
\end{lstlisting}
\subsubsection{Stappen}
\begin{enumerate}
 \item Maak van de tabel een heap. 
 \item Stel $i$ gelijk aan het laatse element van de heap.
 \item Verwissel het laatste element met de wortel (index = 0, bevat de hoogste waarde van de tabel). Op dit moment staat de hoogste waarde helemaal achteraan
 \item Verminder $i$ met 1 om zo de volgende index te bepalen.
 \item Herstel de heap want aangezien het laatste element nu op de wortel staat kan de heapvoorwaarde onvoldaan zijn
 \item Ga terug naar stap 3 zolang $i >= 0$.
\end{enumerate}
\subsubsection{Analyse}
Aangezien er gebruik gemaakt wordt van een maxheap, zal de grootste waarde van de tabel altijd in de wortel zitten. We swappen de wortel met het laatste element zodat de laatste index nu het grootste element bevat. De functie \texttt{herstel\_heap(v, 0, i)} herstelt de heap $v$ vanaf index 0 tot en met index i. Dit zorgt ervoor dat het grootste element ( die nu helemaal vanachter staat ) niet meegerekend wordt in het herconstrueren.


De efficiënte heapconstructie is $O(n)$. Het rangschikken vewijdert $n - 1$ keer de wortel en dit is in het slechtste geval $O(n \log_2 n)$ (zelfde redenering 1ste methode constructie heap). In het slechtste geval is heapsort dus $O(n \log_2 n)$. Voor het beste en gemiddelde geval is dit ook zo. Heapsort is dus $\Theta(n \log_2 n)$
\subsubsection{Eigenschappen}
\begin{itemize}
 \item Sorteert ter plaatse
 \item Sorteert niet stabiel omdat de verplaatsing van elementen tijdens het herstellen van de heap de volgorde kan wijzigen.
\end{itemize}
\section{Rangschikken door samenvoegen}
\subsection{Merge Sort}
\begin{lstlisting}
void merge(vector<T>& v, int l, int m, int r, vector<T>& h){
    int i = l, j = m;
    for(k = l; k < r; k++){
        if(i < m && (j >= r || v[i] <= v[j])) {
            h[k] = v[i];
            i++;
        } else{
            h[k] = v[j];
            j++;
        }
    }
}

void merge_sort(vector<T>& v, int l, int r, vector<T>& h){
    if(l < r - 1){
        int m = l + (r + l)/2;
        merge_sort(v, l, m, h);
        merge_sort(v, m, r, h);
        merge(v, l, m, r, h);
    }
}

void merge_sort(vector<T>& v){
    vector<T> h(v.size() / 2);
    merge_sort(v, 0, v.size(), h);
}
\end{lstlisting}
\subsubsection{Stappen}
kill me
\subsubsection{Analyse}

\subsubsection{Eigenschappen}


\chapter{Speciale methodes}
\chapter{De selectie-operatie}
\chapter{Uitwendig rangschikken}
\end{document}
