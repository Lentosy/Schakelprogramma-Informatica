\documentclass{report}

\usepackage[utf8]{inputenc}
\usepackage[dutch]{babel}
\usepackage{listings}


\setlength{\parindent}{0pt}

\title{Besturingssystemen 2}
\author{Bert De Saffel}



\begin{document}
\maketitle
\tableofcontents

\part{Unix Theorie}
\chapter{Basic commando's en direcory hiërarchie}
\section{Inleiding}
Er bestaan verschillende Unix varianten die elk hun eigen manier hanteren om de user te laten communiceren met de kernel en wederom. Het enige dat ze zeker gemeenschappelijk hebben is het feit dat ze de POSIX standaard volgen. Hierdoor ontstaat er compabiliteit tussen de verschillende varianten. Tegenwoordig worden de verschillende Unix varianten geleverd als een hele distributie. Deze distributies bevatten typisch:
\begin{itemize}
	\item De kernel (systeemaanroepen en POSIX)
	\item Services (servers, daemons)
	\item Utilities (GNU)
	\item Programmeeromgeving
	\item GUI omgeving
\end{itemize}

\section{De shell}
De shell is een C(ommand) L(ine) I(nterface) waardoor er gebruik kan gemaakt worden van de verschillende functies die het besturingssysteem beschikbaar stelt. Het is een command interpreter en kan dus shell scripts uitvoeren. Er zijn verschillende soorten Unix shells beschikbaar zoals: Bourne (.sh), Korn (.ksh), Joy (.csh) en Falstad (.zsh). Om te weten welke shell er op een bepaald systeem staat wordt het commando \textit{echo \$SHELL}. In de curus wordt de Bourne shell(bash) gebruikt.

\section{Shell variabelen}
Het is mogelijk om globale variabelen in te stellen die kunnen gebruikt worden in het huidige proces en eventueel onderliggende kindprocessen. Om een lijst op te vragen van alle variabelen wordt er gebruik gemaakt van het commando \textit{declare -p}. Om een variabele in te stellen wordt het commando \textit{declare x = 5874} gebruikt. Deze variabele zal alleen kunnen gebruikt worden in het huidige proces. Om een variabele globaal te definieëren zodat het ook door kindprocessen gebruikt kan worden, wordt de optie \textit{-x} meegegeven aan het \textit{declare} commando.

\section{Shell instellingen}
Er zijn een aantal configuratiemogelijkheden beschikbaar voor de shell als voor bash. Voor configuratie in de shell wordt er gebruik gemaakt van het commando \textit{set}. Hier wordt er altijd een optie \textit{$\pm$o} meegegeven. + zal een optie aanzetten terwijl - een optie zal uitschakelen. Verder zijn er ook configuratiemogelijkheden voor bash. Deze worden bewerkt via het commando \textit{shopt}. Voor een configuratie aan te zetten wordt de optie \textit{-s} gebruikt en voor het uit te schakelen wordt de optie -u gebruikt.

Belangrijke \textit{shopt} opties zijn \textit{extglob} en \textit{globstar}. Belangrijke \textit{set} opties zijn \textit{posix}, \textit{verbose}, \textit{xtrace}, \textit{noclobber}, \textit{errexit}, \textit{nounsout}.


\section{Interne en externe functies}
Een commando is een interne (builtin) of een externe functie. Om dit te acterhalen

\section{I/O}

\section{Help en manpages}
Unix bevat veel soorten informatie zoals \textit{man} voor externe functies en \textit{help} voor interne functies. Manpages worden gebruikt door het \textit{man} commando met daarna de naam van het commando waarover meer informatie is gewenst zoals \textit{man kill}. In het geval van \textit{kill} is dit ook een systeemaanroep. Om alle verschillende versies van een commando te achterhalen wordt de -f optie gebruikt. \textit{man -f kill} zal alle manpages van kill tonen. Om een specifieke manpage te openen wordt gebruik gemaakt van het commando \textit{man x kill} waar x de sectienummer is. \textit{man -f kill} is het equivalent van \textit{whatis kill}. \textit{man -k kill} is het equivalent van \textit{apropos kill}. apropos zal niet letterlijk alle kill commando's tonen, maar ook commando's waar kill in voorkomt zoals \textit{pkill}. Een ander alternatief is het \textit{info} commando. Dit is een vernieuwende versie van \textit{man} en toont ook vaker voorbeelden. Tot slot hebben sommige functies ook een --help parameter dat het gebruik van dit commando uitlegt. Een andere mogelijkheid is het \textit{info} commando. Dit is een vernieuwende versie van man en biedt vaak ook voorbeelden.

Het \textit{man} commando wordt dus gebruikt voor informatie van externe functies. Voor interne functies moet het commando \textit{help} gebruikt worden. Voorbeelden hiervan zijn \textit{help shopt} en \textit{help for}.

\section{Directories}
Om een directorystructuur weer te geven kan het commando \textit{tree} gebruikt worden. \textit{tree /etc} zal alle (sub)directories en bestanden die zich in /etc bevinden uitprinten. De optie -d zal alleen de directories tonen, de optie -i zal de indentatie weglaten en de optie -f zal altijd het volledige pad uitprinten.

Soms komt het voor dat je tijdelijk naar een andere directory wil gaan en dan terug naar de oude directory wil gaan. Dit kan met het \textit{pushd} en \textit{popd} commando. \textit{pushd} zal een directory op de stack plaatsen, alsook het als huidige locatie instellen in de shell. Om een lijst van directories op deze stack te zien wordt het commando \textit{dirs} gebruikt. Om een directory van deze stack te halen wordt het commando \textit{popd} gebruikt.
\chapter{Devices}
\section{Devices Files}
Devices worden in Unix op twee manieren gerepresenteerd. Enerzijd in de \textbf{/dev} folder als een \textit{device file}. Dit wordt voornamelijk gebruikt om programmatisch deze apparatuur te bevragen. Anderzijds in het \textbf{/sys} virtueel bestandssysteem. Dit wordt dan gebruikt om detailinformatie per device te verzamelen.

Bij het uitvoeren van \textit{ls -ladH} in de \textbf{/dev} folder krijg je output gelijkaardig aan ls -l. De H optie zorgt ervoor dat er bij elk bestand extra informatie staat met betrekking tot het device file type (\textbf{b}lock, \textbf{c}haracter, \textbf{s}ocket of \textbf{p}ipe). Er is ook een major en een minor getal. Het major getal zegt welke driver er gebruikt wordt voor deze file. Het minor getal is een instantie per driver om onderscheid te kunnen maken tussen de verschillende devices. Device files kunnen bewerkt worden zoals eender welke file (bv \textit{cat} en \textit{echo}).

\subsection{Block devices}
Typisch aan een block device is dat ze dienen voor massa opslag. Daarom kan het ook verschillende buffergroottes aannemen. Voorbeelden van block devices zijn harde schijven en read-only drivers voor CD's. Een write driver is geen block device maar een character device. Loop devices zijn "valse" devices die zich gedragen als een block device. Een voorbeeld van een loop device is een .iso bestand.

\subsection{Character devices}
Character devices hebben in tegenstelling tot block devices een vaste buffergrootte. Dit is de grootste restrictie van een character device. Voorbeelden van character devices zijn terminals en pseudo-devices. Pseudo-devices worden in de volgende paragrafen toegelicht. Terminals worden besproken in sectie \ref{sec:Terminals}

\subsubsection{/dev/null}
Er zijn een aantal nuttige pseudo-devices die kunnen gebruikt worden tijdens het maken van shell-scripts. De eerste is \textbf{/dev/null}. /dev/null kan worden gebruikt om bepaalde uitvoer te vernietigen zoals in het volgende voorbeeld:
\begin{lstlisting}
ls -l {01..09} 2>/dev/null | wc -l
\end{lstlisting}
In dit geval bestaan bestanden 08 en 09 niet, wat een foutmelding zal geven, maar aangezien het standaard foutenkanaal(2) wordt omgeleid naar /dev/null zal deze niet op de terminal komen maar direct verwijdert worden. Als we nu geïnteresseerd zijn om enkel de fouten te tonenn kan het volgende commando gebruikt worden:
\begin{lstlisting}
ls -l {01..09} 2>1 1>/dev/null | wc -l 
\end{lstlisting}
Hier wordt het foutkanaal omgeleid naar het standaard invoerkanaal. Het standaard invoerkanaal wordt omgeleid naar /dev/null. Op het eerste zicht lijkt dit raar, maar de argumenten worden in deze instantie in omgekeerde volgorde uitgevoerd.
\newline
Beschouw het volgende commando:
\begin{lstlisting}
grep a $(ls 0?)
\end{lstlisting}
Op normale wijze zal dit commando elk bestand overlopen die begint met een 0 en daarna nog een karakter heeft en deze bestanden zoeken op de letter 'a'. In het geval dat er geen bestanden gevonden zijn zal grep blijven hangen en verwacht hij input. Dan moet je zelf invoer geven wat zeker niet gewesnt is. Dit kan opgelost worden door /dev/null als parameter mee te geven.
\begin{lstlisting}
grep a $(ls 0? 2/dev/null) /dev/null
\end{lstlisting}

Het laatste voorbeeld zal gebruik maken van het \textit{xargs} commando. De bedoeling is om een aantal bestanden te overlopen en dan als uitvoer geven welk bestand er ergens een letter 'a' heeft.
\begin{lstlisting}
grep ls -l 0? | xargs -n3 grep a
\end{lstlisting}
Dit geeft het volgende als uitvoer:
\begin{lstlisting}
01 plya
fzaj
\end{lstlisting}
In bestand 01 zit er dus een letter a, maar we weten niet in welk bestand de andere letter a zit. Als we de -t optie toevoegen aan het \textit{xargs} commando krijgen we dit als output:
\begin{lstlisting}
grep 01 02 03
01 plya
grep 04 05 06
grep 07
fzaj
\end{lstlisting}
Aangezien we groepjes van drie hebben en maar zeven bestanden hebben zal de laatste groepering maar één element hebben. Het \textit{grep} commando zet er dan niet meer bij van welk bestand dit komt. Dit kan ook weer opgelost worden met /dev/null op de volgende manier:
\begin{lstlisting}
ls -l 0? | xargs -n3 -t grep a /dev/null
\end{lstlisting}
Dit geeft als uitvoer:
\begin{lstlisting}
01 plya
07 fzaj
\end{lstlisting}
\subsubsection{/dev/zero}
/dev/zero is een device dat null characters (0x00 of whitespace) zal genereren. Bekijk het volgende voorbeeld:
\begin{lstlisting}
head -c 60 < /dev/zero | tr '\0' '-' > file
\end{lstlisting}
Het \textit{head} commando zal 60 whitespace characters uit /dev/zero halen. Het \textit{tr} commando zal deze whitespaces omvormen tot een streepje en uiteindelijk zal deze uitvoer in het bestand \textit{file} worden geplaatst. Als je dit bestand opent zal je zien dat de inhuid 60 streepjes bevat.

\subsubsection{/dev/random en /dev/urandom}
/dev/random en /dev/urandom zijn beide randomgenerators. Het verschil ligt erin dat /dev/random een beperkte pool heeft. Deze pool is snel leeg en duurt minuten om terug te hervullen, wat de randomgenerator tijdelijk onbeschikbaar maakt. /dev/urandom heeft dit probleem niet, maar de randomgenerator van /dev/urandom is niet zo sterk als die van /dev/random.
Het volgende voorbeeld zal random IPv4 adressen genereren:
\begin{lstlisting}
head -c 80 < /dev/urandom | od -An -w4 -t u1
		      	  | tr -s ' ' '.' 
		 	  | cut -c 2-
\end{lstlisting}
Gelijkaardig voor random IPv6 adressen:
\begin{lstlisting}
head -c 160 /dev/urandom | od -An -w16 -t x2
			 | tr ' ' ':'
		         | cut -c 2-
\end{lstlisting}

\subsubsection{Filedescriptors}
Een filedescriptor is een getal dat verwijst naar een bestand om zo read en write operaties mogelijk te maken. Om een lijst van alle filedescriptors op te vragen wordt het commando \textit{ls -lad /proc/PID/fd/*} gebruikt. het PID is het procesid van de huidige terminal. Dit kan opgevraagt worden met \textit{echo \$\$}. De output zal standaard vier filedescriptors tonen. 0 $\rightarrow$ standaard input, 1 $\rightarrow$ standard output, 2 $\rightarrow$ standard error en 255 $\rightarrow$ filedescriptor voor de andere drie te herstellen (specifiek voor bash). Een file descriptor aanmaken kan in drie modi gebeuren:
\begin{lstlisting}
exec 3 < file
exec 4 > file
exec 5 <> file
\end{lstlisting}
Er wordt gebruik gemaakt van het \textit{exec} commando. Dit commando zal een opdracht voor de levensduur van het venster in achting houden. Op die manier kunnen we aan de filedescriptors op eender welk moment, zolang we in dezelfde terminal zijn. $<$ zal een filedescriptor aanmaken in leesmodus, $>$ in schrijfmodus en $<>$ in beide modi. Het getal voor deze operatoren is het nummer van de filedescriptor.
\newline
Bij het lezen van een bestand zonder filedescriptors zal er geen pointer bijgehouden worden van de huidige lijn. Dit heeft als gevolg dat de volgende code telkens de eerste lijn van een bestand zal printen:
\begin{lstlisting}
while read lijn < file; do echo $lijn; done
output:
een
een
een
\end{lstlisting}
Filedescriptors hebben dus een pointer naar de huidige locatie in een bestand. Als het vorige commando wordt uitgevoerd met een filedescriptor, zullen alle lijnen overlopen worden:
\begin{lstlisting}
while read lijn < &3; do echo $lijn; done
output
een
twee
drie
\end{lstlisting}
Een file descriptor kan ook schrijven naar een bestand. \textit{echo tekst $>$ \&4} zal het stukje 'tekst' toevoegen aan filedescriptor 4. De originele tekst zal niet vervangen worden zoals bij een echo commando met een normaal bestand. Voor schrijfoperaties zijn filedescriptors ook een pak performanter dan echo commando's
\section{Terminals}
//progressindicator 


((x=2**30));tput sc; while ((x /= 2)); do tput rc; echo -n \$x; sleep 0.5; done

\chapter{Shell Scripts}
\section{Shell Script Basics}
Vooraleer er een shell commando wordt uitgevoerd worden er een aantal stappen overlopen. Dit onderdeel zal de belangrijkste stappen uit dit proces toelichten.

\subsection{2. Brace Expansion}
Brace Expansion is een mechanisme dat random strings kan laten genereren. Beschouw het volgende voorbeeld:
\begin{lstlisting}
echo {0..9}{a..f}
OUTPUT
0a 0b 0c 0d 0e 0f 1a 1b 1c 1d 1e 1f 
2a 2b 2c 2d 2e 2f 3a 3b 3c 3d 3e 3f 
4a 4b 4c 4d 4e 4f 5a 5b 5c 5d 5e 5f 
6a 6b 6c 6d 6e 6f 7a 7b 7c 7d 7e 7f 
8a 8b 8c 8d 8e 8f 9a 9b 9c 9d 9e 9f

\end{lstlisting}
Dit geeft dus een lijst van alle mogelijke combinaties waaruit het eerste karakter bestaat uit een cijfer en het tweede karakter een letter van a tot d. Het volgende voorbeeld zal dit vermijden:
\begin{lstlisting}
echo {{0..9},{a..f}}
OUTPUT
0 1 2 3 4 5 6 7 8 9 a b c d e f
\end{lstlisting}
Er is één probleem met brace expansion en dat is dat het vroeg komt in het proces. Dit zorgt ervoor dat het volgende voorbeeld niet zal werken:
\begin{lstlisting}
declare x = 5
declare y = 20
echo {$x..$y}
OUTPUT
{$x..$y}
\end{lstlisting}
Om dit op te lossen wordt het \textit{eval} commando gebruikt
\begin{lstlisting}
declare x = 5
declare y = 20
eval "echo {$x..$y}"
OUTPUT
5 6 7 8 9 10 11 12 13 14 15 16 17 18 19 20
\end{lstlisting}

\subsection{5. Command Substitution}
Command substitutie laat toe om een commando uit te voeren en de uitvoer van dit programma als argumenten van een ander programma te laten gelden. Een eenvoudig voorbeeld is 
\texttt{vi \$(ls -la)}. Dit zal de uitvoer van ls -la openen in de vi editor.

\subsection{6. Process Substitution}
Proces substitution is gelijkaardig aan command substitution, alleen zal de uitvoer in een tijdelijk bestand terecht komen en dit bestand wordt dan gebruikt als argument bij een ander commando. Het commando \texttt{diff} verwacht twee bestanden om te vergelijken. Als je de bestanden nog niet hebt is het gemakkelijker om aan process substitution te doen: \texttt{diff <(sort bestand1) <(sort bestand2)}. Process substitution wordt dus voorafgegaan met een < terwijl commando substitution voorafgegaan wordt met een \$.
\subsection{? Arithmetic Substitution}
Arithmetic substitution laat toe om een argument als numerieke waarde te beschouwen in plaats van een stukje tekst. 
\begin{lstlisting}
echo $(( 5 + 3))      \\ 8
echo 5 + 3            \\ 5 + 3
\end{lstlisting}

\begin{lstlisting}
for ((i=1; i<=$#; i++)); 
do echo $i;

ARRAY :   ${@:i:1}
POINTER:  $$!1
\end{lstlisting}
\subsection{?. Parameter Expansion}
Parameter Expansion behandelt de procedure om een waarde van een variabele te krijgen. Dit kan toegepast worden op vele voorbeelden. Stel dat de variabele $x$=abcdefg. Als we \textit{echo \$\{x\}0000} uitvoeren krijgen we \textit{abcdefg0000} als uitvoer. \textit{echo \$\{\#x\}} zal de lengte van de variabele tonen, in dit geval is dit \textit{7}. Hier kunnen ook reguliere expressie gebruikt worden. Om een bepaald stukje weg te laten kan bijvoorbeeld \textit{echo \$\{x\#*e\}} gebruikt worden. Dit heeft als resultaat \textit{fg}. Om het omgekeerde effect te bekomen gebruik je \textit{echo \$\{x\%e*\}} wat als uitvoer \textit{abcd} heeft.
\newline
In het volgende voorbeeld gebruiken we de variable $y$=abcd0000ef00gh. Stel dat we eerste rij van nullen en alles daarvoor willen verwijderen. Dan gebruiken we \textit{echo \$\{y\#*0\}}. Dit geeft als uitvoer \textit{ef00gh}. Om dit voor alle nullen (globaal) te doen gebruik je \textit{echo \$\{y\#\#*0\}}. Dit geeft als uitvoer \textit{gh}.

\textit{\$\{PWD\%\/*\}} $\rightarrow$ ouderdirectory

\textit{\$\{PWD\#\#\$\{PWD\%\/*\}\/\}} $\rightarrow$ huidige directory



\section{Functies}
Functies laten toe om code te hergebruiken. Een functie wordt in Unix gedeclareerd op de volgende wijze: \textit{f () { }}. f is hier de functienaam die gelijk wat kan zijn. In de open haakjes mag er niets staan, dit is een indicatie dat het om een functie gaat. De volgende functie zal het aantal lijnen tellen van een bepaald bestand:
\begin{lstlisting}
	f() { wc -l file ; }
\end{lstlisting}
Het is ook mogelijk om parameters mee te geven aan een functie. De shell zal deze parameters dan opvullen in \$1, \$2, ... \$n. Het volgende voorbeeld zal de twee meegegeven parameters uitprinten:
\begin{lstlisting}
	g() { echo $2 $1 ; }
\end{lstlisting}
Deze functie kan nu aangeroepen worden via \textit{g één twee drie}. Dit zal als output \textit{twee één} hebben. Overbodige parameters zijn dus geen probleem, ze worden alleen niet gebruikt. Tot slot kunnen functies ook een returnwaarde geven, alhoewel dit meestal niet nuttig is. Bekijk het volgende voorbeeld:
\begin{lstlisting}
	h() { return 200 ; }
\end{lstlisting}
Om deze returnwaarde nu te bekijken gebruik je \textit{\$?}, waarin 200 zal staan. Het nadeel aan deze returnwaarde is dat het maximum 256 kan bevatten en dus ook geen negatieve getallen.

Normaal gezien zijn variabelen en parameters passed by value. Dit wil zeggen dat de variabele zelf niet aangepast zal worden maar enkel de waarde. Als demonstratie wordt het volgende voorbeeld gebruikt.
\begin{lstlisting}
	i() { echo tijdens: $x ; }
\end{lstlisting}
Als we de variabele $x$ instellen op \textit{ervoor} en we voeren dan de volgende commando ketting uit: \textit{echo ervoor: \$x ; i ; echo erna: \$x}. Dit zal als output \textit{ervoor: ervoor	tijdens: ervoor		erna: ervoor} hebben. Als we de functie nu aanpassen naar : \textit{i () \{ x=tijdens; echo tijdens: \$x \}}, zal de output \textit{ervoor: ervoor	tijdens: tijdens	erna: tijdens} zijn. Om de variabele alleen lokaal te gebruiken wordt het keyword \textit{local} of \textit{declare} gebruikt. Dus \textit{i () \{ local x=tijdens; echo tijdens: \$x \}}. Dit geeft als output \textit{ervoor: ervoor	tijdens: tijdens	erna: ervoor}.

Om een lijst van alle functies op een toestel op te vragen wordt het commando \textit{declare -F} gebruikt. Om de source te bekijken van bv. het commando \_tilde wordt \textbf{declare -fp \_tilde} gebruikt.

\section{Variabelen}
Het instellen en gebruik van variabelen is bij Bash heel beperkt. Om een variabele in te stellen kan ofwel \textit{x=abc} of \textit{declare x=abc} gebruikt worden. Het voordeel aan declare is dat er nog extra opties kunnen meegegeven worden, maar zoals eerder vermeld is dit heel beperkt. Een paar voorbeelden zijn -l en -u, die respectievelijk de inhoud van de variabele naar lowercase of uppercase brengt. Om een variabele te verwijderen wordt \textit{unset x} gebruikt.

In het geval dat we een variabele $z$ instellen als \textit{343} en we wensen hierbij een ander getal bij op te tellen zou het logisch lijken om \textit{z+=57} te doen. Echter deze operatoren zullen concateneren wat de inhoud verandert in \textit{34357}. Om effectief getallen op te tellen moet \textit{((z+=57))} gebruikt worden, wat 400 oplevert.

\subsection{read}
Er wordt gebruik gemaakt van het \textit{read} commando om demonstraties met variabelen te tonen. \textit{read} zal een lijn uitlezen en deze in de \textit{REPLY} variabele steken. Dus als we \textit{read < competitors.csv} uitvoeren en we kijken in de REPLY met behulp van \textit{declare -p REPLY} zien we de eerste lijn van dit bestand staan. Het \textit{read} commando kan ook intern splitten op een scheidingsteken. Als we \textit{read a b c < competitors.csv} uitvoeren zal de eerste lijn gesplit worden. Om te weten hoe er gesplit werd moet je individueel de variabelen a, b en c bekijken. Wat opvalt is dat het scheidingsteken een spatie is. Dit komt door de \textit{IFS} variabele. Als we \textit{hexdump -C <<< "\$IFS"} uitvoeren krijgen we de variabele die de standaard lijnscheidingstekens bepaalt. Default staat die op een spatie (20), een tab (09) en een newline (0a).

De IFS variabele kan voor één lijn verandert worden door bv:
\begin{lstlisting}
	IFS=\;	read a b c < competitors.csv
\end{lstlisting}
Dit zal de IFS variabele opvullen met enkel een komma-punt teken. Na het uitvoeren van de bovenstaande regel wordt die terug op de default waarde ingesteld. Als je nu voor meerdere lijnen een andere lijnscheidingsteken wil gebruiken moet je een functie gebruiken:
\begin{lstlisting}
	f() {
			{ read a b c < competitors.csv } ; 
			{ read a b c < competitors.csv } ;
		}
	IFS=\; f
\end{lstlisting}

Als we geen invoer geven aan read, dus op deze manier: \textit{read y}. Zal read in interactieve modus starten. Hij beschouwt dit dan als invoer en zal het schrijven naar de variabele y.

De optie \textit{-n}\textbf{x}, waarbij $x$ een natuurlijk getal is, zal de invoer beperkt worden tot $x$ karakters. Hierbij stopt een lijnscheidingsteken ook de invoer. Als de optie \textit{-N}\textbf{x} wordt uitgevoerd, zal een lijnscheidingsteken ook meetellen als een karakter. Ook kan er een timeout gespecifieërd wordne met de \textit{-t}\textbf{x} optie.

\section{Foutcodes}

\section{Arrays}
\subsection{Numeriek}
Een array kan in bash op twee manieren gedeclareerd worden. Numeriek of associatief. Een numerieke array wordt enkel geindexeërd met getallen. Er kan direct gebruik gemaakt worden van een numerieke array zonder dat hij eerst gedeclareerd moet worden. \texttt{t[5]=abc} zal dus een array \textit{t} aanmaken dat op index 5 de string \textit{abc} bevat. Merk op dat alles aan elkaar plakt in deze instructie. Om nu de informatie die in de array zit te bekijken kan je \texttt{declare -p t}. Een elemant toevoegen kan door gewoon een andere index te gebruiken: \texttt{t[8]=def}. Je kan ook meerdere elementen tegelijk toevoegen: \texttt{t+=([2]=ghi [7]=jkl)}. Je kan ook de index weglaten: \texttt{t+=(mno pqr)}. Hier zal de volgende hoogste index gebruikt worden, dus index 9 en 10. Tot slot kan ook nog de uitvoer van een programma gebruikt worden: \texttt{t+=(\$(cat file))}. Hier wordt de default delimiter gebruikt.

Met \texttt{unset "t[2]"} verwijder je het tweede element van de array. Om de inhoud van een index te bekijken gebruik je \texttt{echo \$\{t[5]}\}. Om alle elementen te tonen gebruik je \texttt{"\$\{t[@]\}"} en voor alle sleutels gebruik je \texttt{"\$\{!t[@]\}"}. Om te checken of dat een bepaalde index een waarde heeft gebruik je het \textit{-v} programma op volgende manier: \texttt{[[ -v t[8] ]] \&\& echo bestaat || echo bestaat niet}.

\subsection{Associatief}
Associatieve arrays moeten op voorhand gedeclareerd worden door middel van \texttt{declare -A t}. 


\section{Conditionele logica}
\subsection{If tests}
Bash kent een andere manier om te testen aan een specifieke voorwaarde. De \texttt{||} en \texttt{\&\&} tekens stellen respectievelijk OR en AND voor. De voorkeur om te testen gebeurt op volgende manier:
\begin{lstlisting}
cmd && { cmd1 ; cmd2 ; ... ; cmd9 ; } || { cmda ; cmdb ; ... ; cmdz ; }
\end{lstlisting}
Dit zal de eerste reeks commando's uitvoeren als \textit{cmd} een true oplevert en de tweede reeks commando's als het een false oplevert.
\subsection{Case}
SYNTAX: \texttt{case WORD in [PATTERN [| PATTERN]...) COMMANDS ;;]... esac}
VOORBEELD: 
\begin{lstlisting}
case 458 in 
	*1*) echo 1 ;;
	*1*) echo 2 ;;
	*1*) echo 3 ;;
	*1*) echo 4 ;;
	*1*) echo 5 ;;
	*1*) echo 6 ;;
	*1*) echo 7 ;;
	*1*) echo 8 ;;
	*1*) echo 9 ;;
esac
\end{lstlisting}
Een bepaalde case kan gestopt worden met 3 symbolen
\begin{enumerate}
	\item \textbf{;;}   -> Zal stoppen nadat er één case voldaan werd
	\item \textbf{;;\&} -> Zal alle cases die voldoen uitvoeren
	\item \textbf{;\&}  -> Voert alle cases uit na de eerste case die voldaan werd
\end{enumerate}
Het is mogelijk om deze symbolen te combineren, maar wordt sterk afgeraden.
\section{For loops}
SYNTAX: \texttt{for NAME [in WORDS ... ] ; do COMMANDS; done}
VOORBEELD:
\begin{lstlisting}
for i in "a b c d" ; echo $i; done
\end{lstlisting}

\section{While loops}
SYNTAX: \texttt{while COMMANDS; do COMMANDS; done}
VOORBEELD: 
\begin{lstlisting}
while read lijn ; do $lijn >> log ; done bestand || NOG EENS BEKIJKEN KLOPT NIET
\end{lstlisting}

Verbeterde versie met filedescriptors
\begin{lstlisting}
exec {fd} < bestand
while read lijn <&${fd} ; do echo $lijn ; done
exec {fd} <&-
\end{lstlisting}

\subsection{Teken per teken verwerken}
\begin{lstlisting}
while read -n1 char ; do ... ; done <<< "$string"
\end{lstlisting}
\subsection{Verwerken uitvoer opdracht}
\begin{lstlisting}
while read lijn ; do ... ; done < <(opdracht)
\end{lstlisting}
\section{Tijdelijke bestanden}
Tijdelijke bestanden kunnen aangemaakt worden met het \textbf{mktemp} commando. Dit commando zal een bestand aanmaken met een naam dat aan een specifiek patroon voldoet en deze naam teruggeven. \texttt{mktemp 123XXX456XXXX} zal enkel de laatste vier X'en vervangen door willekeurige letters en cijfers. 

\section{Here Documents en Strings}
Er bestaan twee manieren om op het moment zelf invoer te creeëren dat als een bestand beschouwd wordt.

\begin{lstlisting}
x='een
twee
$(ls)
drie'
rev <<< "$x"
\end{lstlisting}
of
\begin{lstlisting}
rev << EOF een
twee
$(ls)
drie
EOF
\end{lstlisting}
De voorkeur gaat naar de eerste methode.

\section{Signalisering}
Het commando kill wordt gebruikt om een signaal naar een proces te sturen. Om een lijst van alle signalen te verkrijgen gebruik je \texttt{kill -l}. Om signalisering duidelijk te maken schrijven we eerst een script dat priemgetallen zal zoeken.

\begin{lstlisting}
i=0; 
while ((++i));  
do t=($(factor $i)); 
((${#t[@]}=2)) && echo $i;  
done
\end{lstlisting}
Als je dit script zal uitvoeren zal het een getal naar het scherm printen als dit een priemgetal is. We willen nu dat dit script het laatste priemgetal uitprint als we dit programma beëindigen met CTRL + C. Het programma wordt als volgt aangepast:
\begin{lstlisting}
trap 'print $p, $n' 10;
i=0; n=0;
while ((++i)); 
do t=($(factor $i)); 
((${#t[@]}=2)) && echo $i;  
{p=$i; ((n++))
done;
\end{lstlisting}
We kunnen nu dit programma stoppen met CTRL + C waarbij het programma het laatste priemgetal uitschrijft. Dit programma kan ook gestopt worden via een andere terminal. Om het pid te achterhalen gebruik je \texttt{pstree -p | grep bash}. Daarna gebruik je dit pid op volgende manier: \texttt{kill -10 pid}.

Om te debuggen wordt  het DEBUG signaal gebruikt op volgende manier:
\begin{lstlisting}
trap '$i' DEBUG
i=0;n=0;
while ((++i));
do f=($(factor $i));
((${#f[@]}==2)) && { set -x; ((n++)); p=$i; set +x;} done;
\end{lstlisting}

In het geval dat set -x en set +x voorkomen zal het DEBUG signaal in werking treden. Verder zal er ook aales wat tussen set -x en set +x staat uitgeprint worden. Het volstaat dus ook om gewoon {set -x; set +x;} te hebben, al is het maar voor het DEBUG signaal te laten werken
\part{Systeemaanroepen}
\chapter{Inleiding}
In dit onderdeel van de samenvattingen worden alle opdrachten overlopen. In het begin van de opdracht zullen de gebruikte systeemaanroepen opgelijst worden. Daarna volgt het codesegment en uiteindelijk de uitleg.
\chapter{Een eerste programmeeropdracht}




\chapter{I/O-Systeemaanroepen}
\section{Systeemaanroepen}
De systeemaanroepen die hier gebruikt worden hebben betrekking tot filedescriptors en het bewerken van bestanden.
\begin{itemize}
	\item int open(const char *pathname, int flags);
	\item int open(const char *pathname, int flags, int mode);
	\item ssize\_t write(int fd, const void *buf, size\_t count);
	\item ssize\_t read(int fd, void *buf, size\_t count);
	\item int close(int fd);
\end{itemize}

\chapter{Processen}
\section{Systeemaanroepen}
De systeemaanroepen die hier gebruikt worden hebben betrekking tot kindprocessen, ouderprocessen en de communicatie hiertussen.
\begin{itemize}
	\item pid\_t fork(void);
	\item int waitid(idtype\_t idtype, id\_t id, siginfo\_t *infop, int options);
	\item execve(const char *filename, char *const argv[], char *const envp[]);
	\item execl(const char *path, const char *arg, ... , (char *) NULL);
	\item int pipe(int pipefd[2]);
\end{itemize}

\chapter{POSIX-Threads}
\section{Systeemaanroepen}
\#include $<$pthread.h$>$
\newline
De systeemaanroepen die hier gebruikt worden hebben betrekking tot het aanmaken, gebruiken en verwijderen van threads.
%#include <pthread.h>
\begin{itemize}
	\item int pthread\_create(pthread\_t *thread, const pthread\_attr\_t *attr, void *(*start\_routine) (void *), void *arg);
	\item int pthread\_join(pthread\_t thread, void **retval);
	\item int pthread\_exit(void *retval);
\end{itemize}


\chapter{Thread synchronisatie}

\section{Systeemaanroepen}
\#include $<$pthread.h$>$
\newline
\begin{itemize}
	\item int pthread\_mutex\_lock(pthread\_mutex\_t *mutex);
	\item int pthread\_mutex\_unlock(pthread\_mutex\_t *mutex);
	\item itn sem\_init(sem\_t *sem, int pshared, unsigned int value);
	\item int sem\_wait(sem\_t *sem);
	\item int sem\_post(sem\_t *sem);
	\item int sem\_destroy(sem\_t *sem);
\end{itemize}


\end{document}
