\documentclass[12pt]{report} 

% PACKAGES 
\usepackage[dutch]{babel}
\usepackage[utf8]{inputenc}
\usepackage{color}
\usepackage{amsmath} % Matrices
\usepackage{amsfonts}
\usepackage{booktabs}
\usepackage{pgfplots}
\usepackage{array}
\usepackage{xcolor}
\usepackage{sectsty}
\usepackage{lipsum}

\partfont{\color{brown}}
\chapterfont{\color{teal}}
\sectionfont{\color{cyan}}

% DOCUMENT INFORMATION
\title{Samenvatting Wiskunde B}
\author{Bert De Saffel}
\date{2017-2018}


% CUSTOM COMMANDS
\setlength{\extrarowheight}{5pt} % More spacing between tabular row items

\newcommand{\todo}[1] {
  \color{red}\textunderscore{\textit{TODO: #1}}
}

\newcommand{\example}[2] {
  \begin{flushleft}\fbox{\parbox{\textwidth}
	{#1

	#2}}
  \end{flushleft}
}

\newcommand{\red}[1] {
\color{red} \textit{#1} \color{black}
}

\newcommand{\orange}[1] {
\color{orange} \textit{#1} \color{black}
}

% DOCUMENT
\begin{document}
\maketitle
\tableofcontents

\part{Herhaling Wiskunde A}
\chapter{Onbepaalde Integralen}
\section{Substitutiemethode}
%Stel $$\alpha = \varphi(t),\;dx = \varphi'(t)$$
%dan $$\int f(\varphi(t))\varphi'(t)dt = \int f(x) dx$$
\subsection{Voorbeeld 1}
$$\int \frac{t - 1}{t^2 + 4}dt = \int \frac{t}{t^2 + 4}dt - \int \frac{dt}{t^2 + 4}$$
$$\hbox{stel}\;\; u = t^2 + 4$$
$$\hbox{dan}\;\; du = 2tdt \rightarrow dt = \frac{du}{2t}$$
$$= \int \frac{t}{2t u} du - \frac{1}{2}\arctan{\frac{t}{2}}$$
$$= \frac{1}{2}\int \frac{du}{u} - \frac{1}{2}\arctan{\frac{t}{2}}$$
$$= \frac{1}{2}\ln{u} - \frac{1}{2}\arctan{\frac{t}{2}}$$
$$= \frac{1}{2}\ln{t^2 + 4} - \frac{1}{2}\arctan{\frac{t}{2}} + C$$
\subsection{Voorbeeld 2}
$$\int \frac{dy}{e^y + 4e^{-y}} = \int \frac{e^y}{(e^y)^2 + 4} dy$$
$$\hbox{stel}\;\; u = e^y$$
$$\hbox{dan}\;\; du = e^ydy \rightarrow dy = \frac{du}{e^y}$$
$$= \int \frac{e^y}{e^y(u^2 + 4)} du$$
$$= \int \frac{du}{u^2 + 4}$$
$$= \frac{1}{2}\arctan{\frac{u}{2}}$$
$$= \frac{1}{2}\arctan{\frac{e^y}{2}} + C$$
\section{Partieële integratie}
$$\int u\;dv = uv - \int v\;du$$
\subsection{Voorbeeld 1}
$$\int \ln(x) dx = \int 1\cdot \ln(x) dx$$
$$\hbox{stel}\; u = ln(x) \; \hbox{en}\; v = \int dx$$
$$\hbox{dan}\; du = \frac{1}{x}dx \; \hbox{en}\; v = x$$
$$= x\ln(x) - \int x\cdot\frac{1}{x}dx$$
$$= x\ln(x) - \int dx$$
$$= x\ln(x) - x + C$$
\subsection{Voorbeeld 2}
$$\int \frac{x + 1}{\cos^2(x)}$$
$$\hbox{stel}\; u = x + 1 \; \hbox{en}\; v = \int \frac{1}{\cos^2(x)}dx$$
$$\hbox{dan}\; du = dx \; \hbox{en}\; v = \tan(x)$$
$$= (x+1)\tan(x) - \int \tan(x)dx$$
$$= (x+1)\tan(x) + ln|cos(x)| + C$$
\subsection{Voorbeeld 3}
$$\int e^{-x}\sin(2x)$$
$$\hbox{stel}\; u = \sin(2x) \; \hbox{en}\; v = \int e^{-x}dx$$
$$\hbox{dan}\; du = 2\cos(2x)dx \; \hbox{en}\; v = -e^{-x}$$
$$= -e^{-x}\sin(2x) + 2 \int e^{-x}\cos(2x) dx$$
$$\hbox{stel}\; u = \cos(2x) \; \hbox{en}\; v = \int e^{-x}dx$$
$$\hbox{dan}\; du = -2\sin(2x)dx \; \hbox{en}\; v = -e^{-x}$$
$$= -e^{-x}\sin(2x) + 2\bigg[-e^{-x}\cos(2x)  - 2\int e^{-x}\sin(2x)dx   \bigg]$$
$$= -e^{-x}\sin(2x) - 2e^{-x}\cos(2x) -  4\int e^{-x}\sin(2x)dx$$
Dus
$$\int e^{-x}\sin(2x) =  -e^{-x}\sin(2x) - 2e^{-x}\cos(2x) -  4\int e^{-x}\sin(2x)dx$$
$$5\int e^{-x}\sin(2x) = -e^{-x}[\sin(2x) + 2\cos(2x)]$$
$$\int e^{-x}\sin(2x) = \frac{-e^{-x}[\sin(2x) + 2\cos(2x)]}{5}$$
\subsection{Voorbeeld 4}
$$\int \sin^4(\theta) d\theta = \int (\sin^2(\theta))^2 d\theta$$
$$= \int \bigg(\frac{1 - \cos(2\theta)}{2}\bigg)^2 d\theta$$
$$= \int \bigg(\frac{1}{4} - \frac{\cos(2\theta)}{2} + \frac{\cos^2(2\theta)}{4} \bigg)d\theta$$
$$= \int \frac{1}{4}d\theta - \int \frac{\cos(2\theta)}{2}d\theta + \int \frac{\cos^2(2\theta)}{4}d\theta$$
$$= \frac{\theta}{4} - \frac{\sin(2\theta)}{4} + \frac{\sin(4\theta) + 4\theta}{32}$$
$$= \frac{12\theta - 8\sin(2\theta) + \sin(4\theta)}{32} + C$$

\part{Differentiaalvergelijkingen}
\chapter{Basisbegrippen}
\section{Definities}
De algemene definitie is: 
$$F(x, y, y', y'', ..., y^{(\red{n})}) = 0$$
waarbij: \begin{itemize}
          \item \textbf{x} een veranderlijke is.
          \item \textbf{y} een functie van x is.
          \item er minstens één afgeleide van y is.
         \end{itemize}
\example{Voorbeeld differentiaalvergelijking}{$$x + y + y' = 0$$}
Een differentiaalvergelijking heeft een \textbf{orde} en een \textbf{graad}
\begin{itemize}
 \item \textbf{Orde}: Dit is de orde van de hoogste afgeleide dat voorkomt, dus \red{n}.
 \item \textbf{Graad}: De graad bestaat niet altijd maar is wel altijd een strik positief geheel getal. De graad is de macht die behoort tot de afgeleide met de grootste orde. $y^{(\red{n})^{\orange{r}}}$
\end{itemize}
\example{Voorbeeld orde en graad}
{
  \begin{center}
    \begin{tabular}{l | l | l}
      Differentiaalvergelijking & Orde & Graad \\
      \hline
      $y\ - 2y'^3 = yx$ & 2 & 1 \\
      $1 + (y'')^4 + 2y' + x(y''')^2 = sin(x)$ & 3 & 2 \\
      $(x - 1)(y'') - xy' + y = 0$ & 2 & 1 \\
      $e^s\frac{d^3s}{dt^3} + (\frac{ds^2}{dt^2})^3 = 0$ & 3 & 1 \\
      $xy' + e^{y'} + y'' = 1$ & 1 & / \\
      \hline
      $\sin\sqrt {y'} = x + 2$ & 1 & / \\
      $\;\;\rightarrow y' = \arcsin^2(x+2)$ & 1 & 1 \\
      \hline
      $\sin y' = xy'^2$ & 1 & / \\
      $\;\;\rightarrow y' = \arcsin(xy'^2)$ & 1 & / \\
      \hline
      $y^{'3} + \frac{x}{y''} + y'' = 1$ & 2 & ? \\
      $\;\;\rightarrow y^{'3}y'' + x + (y'')^2 = 1$ & 2 & 2
   
    \end{tabular}
  \end{center}
}

\section{Soorten oplossingen}
Tijdens het oplossen van een differentiaalvergelijking van de \red{n}-de orde worden drie oplossingen onderscheidt:
\begin{enumerate}
 \item De \textbf{Algemene oplossing (AO)}: Verzameling van functies zodat de differentiaalvergelijking klopt. De algemene oplossing bevat \red{n }onafhankelijke constanten. Deze constanten zijn getallen en geen functies.
 \item De \textbf{Particuliere oplossing (PO)}: Dit is één van de krommen van de AO en is afhankelijk van de beginvoorwaarden van het probleem.
 \item De \textbf{Singuliere oplossing (SO)}: Een oplossing die niet voldoet aan de AO maar wel een oplossing is voor de DVG.
\end{enumerate}
\example{Voorbeeld onafhankelijke variabelen:}
{
  \begin{center}
    \begin{tabular}{l | l | l}
      AO & Onafh. C & Orde DVG \\
      \hline
      $y = C_1 + C_2x$ & 2 & 2 \\
      $y = C_1  - C_1^2x$ & 1 & 1 \\
      \hline
      $y = C_1(C_2 + C_3e^x)$ & ? & ? \\
      $\;\;\rightarrow C_1C_2 + C_1C_3e^x$ & ? & ? \\
      $\;\;\rightarrow a + be^x$ & 2 & 2 \\
      \hline
      $y = C_1 + \ln(C_2 x)$ & ? & ? \\
      $\;\;\rightarrow y = C_1 + \ln(C_2) + \ln(x)$ & ? & ? \\
      $\;\;\rightarrow y = a + \ln(x)$ & 1 & 1

   
    \end{tabular}
  \end{center}
}
\example{Voorbeeld 1 AO en PO:}
{Gegeven een differentiaalvergelijking: $y'' + y = 0$
\begin{enumerate}
 \item Toon aan dat $y = a\sin(x) + b\cos(x)$ de AO is.
 \item Geef enkele PO's.
 \end{enumerate}
Oplossing:
\begin{enumerate}
 \item $$y = a\sin(x) + b\cos(x)$$
       $$y' = a\cos(x) - b\sin(x)$$
       $$y'' = -a\sin(x) - b\cos(x)$$
       $$\rightarrow y'' + y = 0$$
       $$-a\sin(x) - b\cos(x) + \sin(x) + b\cos(x) = 0 $$
       $$0=0 \rightarrow \hbox{Het is een oplossing}$$
     
     De differentiaalvergelijking heeft orde 2. De y-vergelijking bevat 2 onafhankelijke constanten en de y-vergelijking is een oplossing. Hierdoor is y de AO van de differentiaalvergelijking.
 \item Enkele PO's:
 $$y = 0$$
 $$y = \sqrt{2}\sin(x)$$
 $$y = \sin(x) + \cos(x)$$
\end{enumerate}

}

\example{Voorbeeld 2 AO en PO:}
{Gegeven een differentiaalvergelijking: $y'^2 - yy'+e^x$
\begin{enumerate}
 \item Geef de orde en graad.
 \item Is $y = \frac{1}{C} + Ce^x$ de AO?
 \item Wat  voor soort oplossing is $y = 2\sqrt{e^x}$
 \end{enumerate}
Oplossing:
\begin{enumerate}
 \item {
   De orde is \red{1} en de graad is \orange{2}.
 }
 \item {
  $$y' = Ce^x$$
  $$\rightarrow C^2(e^x)^2 - (\frac{1}{C} + Ce^x)Ce^x + e^x = 0$$
  $$\rightarrow C^2e^{2x} - e^x - C^2e^{2x} + e^x = 0$$
  $$\rightarrow0=0 \rightarrow \hbox{Het is een oplossing}$$
  Orde DVG = \red{1} = Onafhankelijke constanten van y
 }
 \item {
  $$y' = 2 \cdot \frac{1}{2\sqrt{e^x}} \cdot e^x = \sqrt{e^x}$$
  $$\rightarrow (\sqrt{e^x})^2 - 2\sqrt{e^x}\cdot\sqrt{e^x} + e^x = 0$$
  $$\rightarrow e^x - 2e^x + e^x = 0$$
  $$\rightarrow 0 = 0$$
  Dit is een singuliere oplossing aangezien y niet overeenkomt met de AO, maar wel voldoet aan de DVG.
 }

\end{enumerate}

}

\section{Bepalen van een DVG}
Indien een AO gegeven is met \red{n} onafhankelijke constanten:
\begin{enumerate}
 \item Controleer of de constanten werkelijk onafhankelijk zijn.
 \item Leid de AO \red{n} maal af.
 \item Elimineer de \red{n} constanten van de \red{n + 1} bekomen vergelijkingen. De laatste vergelijking moet zeker gebruikt worden.
 \item Controleer of de DVG van orde \red{n} is.
\end{enumerate}

\example{Voorbeeld 1: bepalen van een DVG}
{
  De algemene oplossing is $$y = C_1 + C_2x$$
  \begin{enumerate}
   \item Er zijn \red{2} onafhankelijke constanten.
   \item Er moet \red{2} keer afgeleid worden:
   $$
      \begin{cases}
	y    & = C_1 + C_2x \\
	y'   & = C_2 \\
	y''  & = 0 \\
      \end{cases}
   $$
   \item De constanten zijn al geëlimineerd. 
   \item De DVG is $y'' = 0$ en heeft orde \red{2}.

  \end{enumerate}
}

\example{Voorbeeld 2: bepalen van een DVG}
{
  Bepaal de DVG van: $$y = C_1 + C_2e^{-x} + C_3e^{3x}$$
  \begin{enumerate}
   \item Er zijn \red{3} onafhankelijke constanten.
   \item Er moet \red{3} maal afgeleid worden.
    \[ 
      \begin{cases}
			y & = C_1 + C_2e^{-x} + C_3e^{3x} \\
	y'     & = -C_2e^{-x} + 3C_3e^{3x}     \\
	y'' & = C_2e^{-x} + 9C_3e^{3x}      \\
	y''' & = -C_2e^{-x} + 27C_3e^{3x}
      \end{cases}
    \]

  \item 
  Tel de 1ste afgeleide op met de 2de afgeleide en tel de 2de afgeleide op met de 3rde afgeleide
  \[
    \begin{cases}
      y + y''    & = 3C_3e^{3x} + 9C_3e^{3x} = 12C_3e^{3x}  \\
      y'' + y''' & = 9C_3e^{3x} + 27C_3e^{3x} = 36C_3e^{3x}
    \end{cases}
  \]
  Vermenigvuldig de 1ste vergelijking met 3 en trek hiervan de 2de vergelijking af.
  
  $$3(y + y'') - y'' - y''' = 3(12C_3e^{3x}) - 36C_3e^{3x} = 0$$
  $$\rightarrow y''' - 2y'' - 3y' = 0$$
  \item
  De orde van deze DVG is \red{3}
   
  \end{enumerate}
}

\example{Voorbeeld 3 : bepalen van een DVG}
{
  Bepaal de DVG van alle cirkels met middelpunt y = -x.
  \begin{enumerate}
   \item Eerst moet de AO gevonden worden. Het middelpunt van elke cirkel kan gegeven worden met $m(a, -a).$
    Hieruit volgt de algemene vergelijking van een cirkel: $$(x - a)^2 + (y + a)^2 = R^2$$
    Er zijn \red{2} onafhankelijke constanten (a en R).
   \item Er moet \red{2} maal (impliciet) afgeleid worden.
   \[
      \begin{cases}
       (x - a)^2 + (y + a)^2 = R^2 \\
       \frac{dy}{dx} : (x-a) + y'(y+a) = 0 \\
       \frac{d^2y}{dx^2} : 1 + y''(y + a) + y'^2 = 0
      \end{cases}
   \]
   \item
    Vorm $\frac{dy}{dx}$ om naar $a$:
    $$a = \frac{-x - yy'}{y' - 1}$$
    Substitueer deze $a$ in $\frac{d^2y}{dx^2}$:
    $$1 + y''(y + (\frac{-x - yy'}{y' - 1})) + y'^2 = 0$$
    $$\rightarrow 1 + y''(y + (\frac{x + yy'}{-y' + 1})) + y'^2 = 0$$
    $$\rightarrow y''(x + y) - y'^3 + y'^2 - y' + 1 = 0$$
   \item Orde van de DVG = \red{2}  = Aantal onafhankelijke constanten.
  \end{enumerate}
}


\part{Laplacetransformatie}
\chapter{Algemene begrippen}
\section{De Heaviside functie}
De Heaviside functie heeft als voorschrift:
$$H(t - \alpha) = 
\begin{cases}
  0 \;\; t < \alpha \\
  1 \;\; t > \alpha
\end{cases}$$

\example{Voorbeeld Heaviside functie}
{
Teken over $x=[-3,4]$ de functie $y = 2H(t + 2) - tH(t) + (t+t^2)H(t-2)$.

Er zijn veranderingen bij $t = -2, t = 0$ en $t = 2$.

    \begin{tabular}{l | l}
    \hline
    $2\cdot(0) - t\cdot(0) + (t+t^2)\cdot(0) = 0$ & $t < -2$\\
    $2\cdot(1) - t\cdot(0) + (t+t^2)\cdot(0) = 2$ & $-2 < t < 0$  \\
    $2\cdot(1) - t\cdot(1) + (t+t^2)\cdot(0) = 2 - t$ & $0 < t < 2$\\
    $2\cdot(1) - t\cdot(1) + (t+t^2)\cdot(1) = 2 + t^2$ & $t > 2$\\
    \end{tabular}
  \todo{graph}


}

\end{document}
