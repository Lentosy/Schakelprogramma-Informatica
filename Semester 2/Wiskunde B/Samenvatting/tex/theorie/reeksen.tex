

\chapter{Reeksen}
\section{Definities}
Rij [$a_n$] = $a_1, a_2, ..., a_n, ...$ met $a_n$ de algemene term.

$$
    [a_n] = 
    \begin{cases}
     \hbox{convergent als }  \lim\limits_{n\to\infty} a_n \in \mathbb{R} \\
     \hbox{divergent naar } \infty \hbox{ als } \lim\limits_{n\to\infty} a_n = \infty \\
     \hbox{divergent als } \lim\limits_{n\to\infty} a_n = ?
    \end{cases}
$$

\example{
    $$\big[\ln(\frac{1}{n})\big]$$
}{
    $$\lim\limits_{n\to\infty} \ln(\frac{1}{n}) = \ln(0^+) = -\infty$$
    Dus divergent naar $-\infty$.
}
\example{
    $$\big[(1 - \frac{1}{n})^n\big]$$
}{
    $$\lim\limits_{n\to\infty} (1 - \frac{1}{n})^n = 1^\infty$$
    Dit is een onbepaaldheid. We maken gebruik van het getal $e$.
    $$=\lim\limits_{n\to\infty} \bigg(1 + \big(-\frac{1}{n}\big)\bigg)^{n(-1)} = e^{-1}$$
    Dus convergent.
}
\example{
    $$2, -1, \frac{1}{2}, -\frac{1}{4}, ..., ...$$
}{
    Bepaal de algemene term: $a_n = 2\bigg(-\frac{1}{2}\bigg)^{n - 1}$
    $$\lim\limits_{n\to\infty} 2\bigg(-\frac{1}{2}\bigg)^{n - 1} = 2\cdot0 = 0$$
    Dus convergent.
}
\example{
    $$[(-2)^n]$$
}{
    $$[(-2)^n] = -2, 4, -8, 16, -32, ...$$
    De laatste term is ofwel positief ofwel negatief. De limiet bestaat niet dus deze reeks is divergent.
}

\section{Hoofdeigenschappen}
Een stijgende rij naar boven begrensd is convergent. Een dalende rij naar onder begrensd is convergent.

In symbolen:

$$a_n \uparrow, \forall n \quad a_n \leq b \rightarrow \hbox{ convergent}$$
$$a_n \downarrow, \forall n \quad b \leq a_n  \rightarrow \hbox{ convergent}$$

\example{
    $$\bigg[\frac{2n - 1}{n}\bigg]$$
}{
    Als $n$ stijgt zal $\frac{1}{n}$ dalen. $2 - \frac{1}{n}$ stijgt dus.
    Er kan besloten worden dat voor alle $n$ dat $2 - \frac{1}{2} \leq 2$. Deze reeks convergeert
}
\example{
    $$\bigg[\sin\frac{1}{n}\bigg]$$
}{
    Als $n$ stijgt zal $\frac{1}{n}$ dalen. $\sin\frac{1}{n}$ daalt dus.
    Er kan besloten worden dat voor alle $n$ dat $\sin\frac{1}{n} \geq 0$. Deze reeks convergeert. 
}

\section{Numerieke reeksen}
$$\sum_{n=0}^{\infty} a_n = a_1 + a_2 + a_3 + ... + a_n$$
waarbij
\begin{itemize}
 \item $S_1 = a_1$
 \item $S_2 = a_1 + a_2$
 \item $S_3 = a_1 + a_2 + a_3$
 \item $S_n = a_1 + a_2 + a_3 + ... + a_n$
\end{itemize}
$$
    \sum a_n = 
    \begin{cases}
     \hbox{convergent als }  \lim\limits_{n\to\infty} S_n \in \mathbb{R} \\
     \hbox{divergent naar } \infty \hbox{ als } \lim\limits_{n\to\infty} S_n = \infty \\
     \hbox{divergent als } \lim\limits_{n\to\infty} S_n = ?
    \end{cases}
$$
\example{
    Bewijs dat de volgende reeks convergent of divergent is:
    $$\sum_{n = 2}^{\infty}\bigg(\frac{1}{\sqrt{n - 1}} - \frac{1}{\sqrt{n + 1}}\bigg)$$
}{
    Uitschrijven van een aantal partieelsommen:
    \begin{itemize}
     \item n = 2: $S_1 = \frac{1}{\sqrt{1}} - \frac{1}{\sqrt{3}} = 1 - \frac{1}{\sqrt{3}}$
     \item n = 3: $S_2 = 1 - \frac{1}{\sqrt{3}} + \frac{1}{\sqrt{2}} - \frac{1}{\sqrt{4}}$
     \item n = 4: $S_3 = 1 - \frac{1}{\sqrt{3}} + \frac{1}{\sqrt{2}} - \frac{1}{\sqrt{4}} + \frac{1}{\sqrt{3}} - \frac{1}{\sqrt{5}} = 1 + \frac{1}{\sqrt{2}} - \frac{1}{\sqrt{4}} - \frac{1}{\sqrt{5}}$
     \item n = 5: $S_4 = 1 + \frac{1}{\sqrt{2}} - \frac{1}{\sqrt{4}} - \frac{1}{\sqrt{5}} + \frac{1}{\sqrt{4}} - \frac{1}{\sqrt{6}} = 1 + \frac{1}{\sqrt{2}} - \frac{1}{\sqrt{5}} - \frac{1}{\sqrt{6}}$
    \end{itemize}
    Hieruit volgt dat 
    $$S_n = 1 + \frac{1}{\sqrt{2}} - \frac{1}{\sqrt{n + 1}} - \frac{1}{\sqrt{n + 2}}$$
    Berekening van de limiet:
    $$\lim\limits_{n\to\infty}S_n = 1 + \frac{1}{\sqrt{2}} - 0 - 0 = 1 + \frac{1}{\sqrt{2}}$$
    De reeks convergeert.
}
\example{
    Bewijs dat de volgende reeks convergent of divergent is:
    $$\sum_{n = 0}^{\infty}(-1)^n$$
}{
    Uitschrijven van een aantal partieelsommen:
    \begin{itemize}
     \item n = 0: $S_1 = 1$
     \item n = 1: $S_2 = 1 - 1 = 0$
     \item n = 2: $S_3 = 0 + 1 = 1$
     \item n = 3: $S_4 = 1 - 1 = 0$
    \end{itemize}
    Hieruit volgt dat
    $$S_{2n} = 0 \qquad \hbox{en} \qquad S_{2n + 1} = 1$$
    De limiet bestaat niet dus de reeks divergeert.
}

\section{De meetkundige reeks}
$$\sum_{n = 0}^{\infty} q^n = 1 + q + q^2 + ... + q^{n - 1} + ...$$
\subsection{Convergentieonderzoek}
We weten dat 
$$S_n = 1 + q + ... + q^{n - 1} = \frac{1 - q^n}{1 - q}$$
We moeten de limiet van $S_n$ berekenen. We onderscheiden twee gevallen:
\begin{itemize}
 \item $q \neq 1$
    $$\lim\limits_{n\to\infty}\frac{1 - q^n}{1 - q} = \frac{1}{1 - q}(1 - \lim\limits_{n\to\infty} q^n)$$
    \begin{itemize}[label={als}]
      \item $|q| < 1 \Rightarrow \lim\limits_{n\to\infty} q^n = 0$
            dus
            $\lim\limits_{n\to\infty} S_n = \frac{1}{1 - q}(1 - 0) = \frac{1}{1 - q} \in \mathbb{R}$
            
            Convergent
      \item $|q| > 1 \Rightarrow \lim\limits_{n\to\infty} q^n = +\infty$
            dus
            $\lim\limits_{n\to\infty} S_n = +\infty$
            
            Divergent naar $+\infty$
      \item $q \leq 1 \Rightarrow \lim\limits_{n\to\infty} q^n = ?$
      
            Divergent
    \end{itemize}
 \item $q = 1$
    $$\sum_{n = 0}^{\infty} 1^n = \sum_{n = 0}^{\infty} 1$$
    $$S_n = n \rightarrow \lim\limits_{n\to\infty} S_n = \lim\limits_{n\to\infty} n = +\infty$$
    
    Divergent naar $+\infty$
\end{itemize}

Uit dit onderzoek volgt het volgende:
$$
\boxed{
    \sum_{n = 0}^{\infty} q^n = \begin{cases}
                            |q| < 1 \rightarrow \hbox{Convergent} \\
                            q \geq 1 \rightarrow \hbox{Divergent naar } +\infty \\
                            q \leq -1 \rightarrow \hbox{Divergent}
                            \end{cases}
}
$$

\example{
    Convergeert de reeks $-2 + \frac{2}{3} - \frac{2}{9} + \frac{2}{27} - \frac{2}{81} + ...$ en indien convergent, naar welke reekssom?
}{
    Herschrijf de reeks:
    $$-2(1 - \frac{1}{3} + \frac{1}{9} - \frac{1}{27} + \frac{1}{81} - ... + \bigg(\frac{1}{3}\bigg)^n + ...$$
    Dit komt overeen met
    $$-2\sum_{n = 0}^{\infty}\bigg(-\frac{1}{3}\bigg)^n$$
    Dit is een meetkundige reeks met $q = -\frac{1}{3}$. Het is duidelijk dat $|q| < 1$ dus de reeks convergeert. De reekssom is $$\lim\limits_{n\to\infty} -2 \frac{1}{1 - \frac{1}{3}} = -\frac{3}{2}$$
}

\section{Eigenschappen}
\begin{itemize}
 \item De convergentie of divergentie verandert niet door het weglaten of bijvoegen van een eindig aantal termen.
 \item Als een reeks $\sum a_n$ convergeert naar $S$ dan convergeert de reekst $\sum k\;a_n$ naar $kS$
 \item Indien $\sum a_n$ convergeert dan geldt dat: $\lim\limits_{n\to\infty} a_n = 0$
 
        $\rightarrow \lim\limits_{n\to\infty} a_n \neq 0 \Rightarrow \sum a_n $ divergent
\end{itemize}
\example{
    Bewijs convergentie/divergentie van 
    $$\sum_{n = 1}^{\infty} \bigg(\frac{2n - 5}{2n - 7}\bigg)^n$$
}{
    \begin{equation*}
     \begin{split}
      \lim\limits_{n\to\infty}\bigg(\frac{2n - 5}{2n - 7}\bigg)^n & = 1^\infty \\
      \Rightarrow \lim\limits_{n\to\infty}\bigg(1 + \frac{2}{2n - 7}\bigg)^{n\frac{2n - 7}{2}\cdot\frac{2}{2n - 7}} & = e^{\lim\limits_{n\to\infty}\frac{2n}{2n - 7}} \\
      & = e \neq 0
     \end{split}
    \end{equation*}
    De reeks is divergent naar $+\infty$
}

\section{Reeksen met 'uitsluitend' positieve termen}
Dit zijn reeksen waarbij:
\begin{itemize}
 \item Een eindig aantal negatieve termen weggelaten mogen worden.
 \item Reeksen met uitsluitend negatieve termen kunnen vermenigvuldigd worden met -1.
\end{itemize}
\subsection{Integraalcriterium van Cauchy}
Indien $$a_n \geq 0$$
en $$f(x) = f(n)$$ waarbij $f(x)$ dalend en continu is over $[m, +\infty[$
dan geldt er

$$\int_m^\infty f(x)\;dx \in \mathbb{R} \Rightarrow \sum a_n \qquad \hbox{convergent}$$ 
$$\int_m^\infty f(x)\;dx = \infty \Rightarrow \sum a_n \qquad \hbox{divergent naar } \infty$$ 

\example{
    Onderzoek de convergentie/divergentie van
    $$\sum_{n = 1}^{\infty} \frac{\ln^2n}{n}$$
}{
    $$a_n = \frac{\ln^2n}{n} \geq 0$$
    $$f(x) = \frac{\ln^2x}{x}$$
    We bepalen het gebied waar $f(x)$ continue en dalend is. We berekenen de afgeleide:
    $$f'(x) = \frac{\ln x(2 - \ln x)}{x^2}$$
    Uit tekenonderzoek kan afgeleid worden dat $f(x)$ continu en dalend is vanaf $e^2$.  Kies $m \geq e^2$. Pak een gemakkelijk getal, zoals $m = 9$
    \begin{equation*}
     \begin{split}
      \int_9^{+\infty} \frac{\ln^2 x}{x}dx & = \bigg[\frac{\ln^3 x}{3}\bigg]_9^{+\infty} \\
                                           & = \frac{+\infty}{+\infty} - \frac{\ln^3 9}{3} \\
                                           & = +\infty
     \end{split}
    \end{equation*}
    De reeks divergeert naar $+\infty$.
}

\section{De hyperharmonische reeks}
$$\sum_{n=1}^{\infty} \frac{1}{n^p} = 1 + \frac{1}{2^p} + \frac{1}{3^p} + ... + \frac{1}{n^p} + ...$$
\subsection{Convergentieonderzoek}
\begin{itemize}
    \item $p \leq 0 \Rightarrow \lim\limits_{n\to\infty} n^{-p} = +\infty$
        De reeks divergeert naar $+\infty$
    \item $p > 0$
        We gebruiken het Integraalcriterium van Cauchy want
        $$\sum a_n = \sum \frac{1}{n^p} \qquad a_n \geq 0$$
        en
        $$f(x) = \frac{1}{x^p}$$
        De functie $f(x)$ is continu over $]0, +\infty[$, maar is pas dalend vanaf 1. Dus het interval wordt $[1, +\infty[$. De integraal wordt als volgt:
        $$\int_{1}^{+\infty} \frac{dx}{x^p} = \bigg[\frac{x^{1 - p}}{1 - p}\bigg]_1^{+\infty} = \frac{1}{1 - p}\bigg(\lim\limits_{n\to\infty}\frac{1}{x^{p-1}} - 1\bigg)$$
        \begin{itemize}[label={als}]
            \item $p - 1 > 0 \Rightarrow \lim\limits_{n\to\infty}\frac{1}{x^{p-1}} = 0$
                    
                    $\Rightarrow \frac{1}{1 - p}(0 - 1) \in \mathbb{R}$ : convergent
                    
            \item $p - 1 < 0 \Rightarrow \lim\limits_{n\to\infty}\frac{1}{x^{p-1}} = \infty$
            
                    $\Rightarrow \frac{1}{1 - p}(\infty - 1) = \infty$ : divergent naar $+\infty$
            
            \item $p - 1 = 1 \Rightarrow \int_1^{+\infty} \frac{dx}{x} = \infty$
            
                divergent naar $+\infty$
        \end{itemize}
        
\end{itemize}
Uit dit onderzoek volgt het volgende:
$$
\boxed{
    \sum_{n = 1}^{\infty} \frac{1}{n^p} = \begin{cases}
                            p > 1 \rightarrow \hbox{Convergent} \\
                            p \leq 1 \rightarrow \hbox{Divergent naar } +\infty \\
                            \end{cases}
}
$$
\section{Vergelijkingscriteria}
Indien $\sum a_n$ gevraagd wordt, gebruik een gekende reeks $\sum b_n$. Tot nu toe kennen we twee reeksen: $\sum q^n$ en $\sum \frac{1}{n^p}$

\begin{enumerate}
 \item 
    \begin{itemize}
     \item als $a_n \leq b_n$ en $\sum b_n$ convergent, dan $\sum a_n$ convergent
     \item als $b_n \leq a_n$ en $\sum b_n$ divergent, dan $\sum a_n$ divergent
    \end{itemize}
 \item 
    als $\lim\limits_{n\to\infty}\frac{a_n}{b_n} \neq 0 \neq \infty$ dan beide reeksen zelfde gedrag.

\end{enumerate}

\example{
    Onderzoek de convergentie/divergentie van
    $$\sum_{n = 0}^{\infty} \frac{1}{(3 + (-1)^n)^n}$$
}{
    De reeks kan geschreven worden als 
    $$\sum_{n = 0}^{\infty} \frac{1}{4^n}$$
    Dit komt overeen met een meetkundige reeks met $q = \frac{1}{4}$. Deze reeks convergeert.
}
\example{
    Onderzoek de convergentie/divergentie van
    $$\sum_{n = 0}^{\infty}\frac{\sqrt{n - 1}}{(3n - 2)^2}$$
}{
    Als $n$ naar oneindig gaat: $\frac{\sqrt{n}}{9n^2} \approx \frac{\sqrt{n}}{n^2} = \frac{1}{n^{3/2}}$
    Deze reeks kan dus geschreven worden als:
    $$\sum_{n = 0}^{\infty} \frac{1}{n^{3/2}}$$  
    wat overeenkomt met de hyperharmonische reeks met $p = 3/2$. We weten dat dit convergeert.
    We passen de limiet toe:
    $$\lim\limits_{n\to\infty} \frac{\sqrt{n - 1}}{(3n - 2)^2} \cdot n^{3/2} = \lim\limits_{n\to\infty} \frac{n^{3/2}\sqrt{n}}{9n^2} = \frac{1}{9} \neq 0 \neq \infty$$
    Deze reeks convergeert.
}
\example{
    Onderzoek de convergentie/divergentie van
    $$\sum_{n = 3}^{\infty} -\sqrt{\tan{\frac{1}{n}}}$$
}{
    In dit geval is $a_n < 0$. We vermenigvuldigen de reeks met -1. De reeks wordt:
    $$\sum_{n = 3}^{\infty}\sqrt{\tan{\frac{1}{n}}}$$
    We weten dat als $n$ naar oneindig gaat, dat $\frac{1}{n}$ naar 0 gaat. Voor kleine waarden geldt: $\tan{\frac{1}{n}} = \frac{1}{n}$. De reeks wordt:
    $$\sum_{n=3}^{\infty} \frac{1}{\sqrt{n}}$$ wat overeenkomt met een hyperharmonische reeks met $p = 0.5$. We weten dat dit divergeert naar $+\infty$
    
    We passen de limiet toe:
    $$\lim\limits_{n\to\infty} \frac{\sqrt{\tan\frac{1}{n}}}{\frac{1}{\sqrt{n}}} = \lim\limits_{n\to\infty} \sqrt{\frac{\tan\frac{1}{n}}{\frac{1}{n}}} = 1 \neq 0 \neq \infty$$
    Deze reeks divergeert naar $+\infty$
}

