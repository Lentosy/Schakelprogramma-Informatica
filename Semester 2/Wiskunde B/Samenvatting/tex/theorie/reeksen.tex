

\chapter{Reeksen}
\section{Definities}
Rij [$a_n$] = $a_1, a_2, ..., a_n, ...$ met $a_n$ de algemene term.

$$
    [a_n] = 
    \begin{cases}
     \hbox{convergent als }  \lim\limits_{n\to\infty} a_n \in \mathbb{R} \\
     \hbox{divergent naar } \infty \hbox{ als } \lim\limits_{n\to\infty} a_n = \infty \\
     \hbox{divergent als } \lim\limits_{n\to\infty} a_n = ?
    \end{cases}
$$

\example{
    $$\big[\ln(\frac{1}{n})\big]$$
}{
    $$\lim\limits_{n\to\infty} \ln(\frac{1}{n}) = \ln(0^+) = -\infty$$
    Dus divergent naar $-\infty$.
}
\example{
    $$\big[(1 - \frac{1}{n})^n\big]$$
}{
    $$\lim\limits_{n\to\infty} (1 - \frac{1}{n})^n = 1^\infty$$
    Dit is een onbepaaldheid. We maken gebruik van het getal $e$.
    $$=\lim\limits_{n\to\infty} \bigg(1 + \big(-\frac{1}{n}\big)\bigg)^{n(-1)} = e^{-1}$$
    Dus convergent.
}
\example{
    $$2, -1, \frac{1}{2}, -\frac{1}{4}, ..., ...$$
}{
    Bepaal de algemene term: $a_n = 2\bigg(-\frac{1}{2}\bigg)^{n - 1}$
    $$\lim\limits_{n\to\infty} 2\bigg(-\frac{1}{2}\bigg)^{n - 1} = 2\cdot0 = 0$$
    Dus convergent.
}
\example{
    $$[(-2)^n]$$
}{
    $$[(-2)^n] = -2, 4, -8, 16, -32, ...$$
    De laatste term is ofwel positief ofwel negatief. De limiet bestaat niet dus deze reeks is divergent.
}

\section{Hoofdeigenschappen}
Een stijgende rij naar boven begrensd is convergent. Een dalende rij naar onder begrensd is convergent.

In symbolen:

$$a_n \uparrow, \forall n \quad a_n \leq b \rightarrow \hbox{ convergent}$$
$$a_n \downarrow, \forall n \quad b \leq a_n  \rightarrow \hbox{ convergent}$$

\example{
    $$\bigg[\frac{2n - 1}{n}\bigg]$$
}{
    Als $n$ stijgt zal $\frac{1}{n}$ dalen. $2 - \frac{1}{n}$ stijgt dus.
    Er kan besloten worden dat voor alle $n$ dat $2 - \frac{1}{2} \leq 2$. Deze reeks convergeert
}
\example{
    $$\bigg[\sin\frac{1}{n}\bigg]$$
}{
    Als $n$ stijgt zal $\frac{1}{n}$ dalen. $\sin\frac{1}{n}$ daalt dus.
    Er kan besloten worden dat voor alle $n$ dat $\sin\frac{1}{n} \geq 0$. Deze reeks convergeert. 
}

\section{Numerieke reeksen}
$$\sum_{n=0}^{\infty} a_n = a_1 + a_2 + a_3 + ... + a_n$$
waarbij
\begin{itemize}
 \item $S_1 = a_1$
 \item $S_2 = a_1 + a_2$
 \item $S_3 = a_1 + a_2 + a_3$
 \item $S_n = a_1 + a_2 + a_3 + ... + a_n$
\end{itemize}
$$
    \sum a_n = 
    \begin{cases}
     \hbox{convergent als }  \lim\limits_{n\to\infty} S_n \in \mathbb{R} \\
     \hbox{divergent naar } \infty \hbox{ als } \lim\limits_{n\to\infty} S_n = \infty \\
     \hbox{divergent als } \lim\limits_{n\to\infty} S_n = ?
    \end{cases}
$$
\example{
    Bewijs dat de volgende reeks convergent of divergent is:
    $$\sum_{n = 2}^{\infty}\bigg(\frac{1}{\sqrt{n - 1}} - \frac{1}{\sqrt{n + 1}}\bigg)$$
}{
    Uitschrijven van een aantal partieelsommen:
    \begin{itemize}
     \item n = 2: $S_1 = \frac{1}{\sqrt{1}} - \frac{1}{\sqrt{3}} = 1 - \frac{1}{\sqrt{3}}$
     \item n = 3: $S_2 = 1 - \frac{1}{\sqrt{3}} + \frac{1}{\sqrt{2}} - \frac{1}{\sqrt{4}}$
     \item n = 4: $S_3 = 1 - \frac{1}{\sqrt{3}} + \frac{1}{\sqrt{2}} - \frac{1}{\sqrt{4}} + \frac{1}{\sqrt{3}} - \frac{1}{\sqrt{5}} = 1 + \frac{1}{\sqrt{2}} - \frac{1}{\sqrt{4}} - \frac{1}{\sqrt{5}}$
     \item n = 5: $S_4 = 1 + \frac{1}{\sqrt{2}} - \frac{1}{\sqrt{4}} - \frac{1}{\sqrt{5}} + \frac{1}{\sqrt{4}} - \frac{1}{\sqrt{6}} = 1 + \frac{1}{\sqrt{2}} - \frac{1}{\sqrt{5}} - \frac{1}{\sqrt{6}}$
    \end{itemize}
    Hieruit volgt dat 
    $$S_n = 1 + \frac{1}{\sqrt{2}} - \frac{1}{\sqrt{n + 1}} - \frac{1}{\sqrt{n + 2}}$$
    Berekening van de limiet:
    $$\lim\limits_{n\to\infty}S_n = 1 + \frac{1}{\sqrt{2}} - 0 - 0 = 1 + \frac{1}{\sqrt{2}}$$
    De reeks convergeert.
}
\example{
    Bewijs dat de volgende reeks convergent of divergent is:
    $$\sum_{n = 0}^{\infty}(-1)^n$$
}{
    Uitschrijven van een aantal partieelsommen:
    \begin{itemize}
     \item n = 0: $S_1 = 1$
     \item n = 1: $S_2 = 1 - 1 = 0$
     \item n = 2: $S_3 = 0 + 1 = 1$
     \item n = 3: $S_4 = 1 - 1 = 0$
    \end{itemize}
    Hieruit volgt dat
    $$S_{2n} = 0 \qquad \hbox{en} \qquad S_{2n + 1} = 1$$
    De limiet bestaat niet dus de reeks divergeert.
}

\section{De meetkundige reeks}
$$\sum_{n = 0}^{\infty} q^n = 1 + q + q^2 + ... + q^{n - 1} + ...$$
\subsection{Convergentieonderzoek}
We weten dat 
$$S_n = 1 + q + ... + q^{n - 1} = \frac{1 - q^n}{1 - q}$$
We moeten de limiet van $S_n$ berekenen. We onderscheiden twee gevallen:
\begin{itemize}
 \item $q \neq 1$
    $$\lim\limits_{n\to\infty}\frac{1 - q^n}{1 - q} = \frac{1}{1 - q}(1 - \lim\limits_{n\to\infty} q^n)$$
    \begin{itemize}[label={als}]
      \item $|q| < 1 \Rightarrow \lim\limits_{n\to\infty} q^n = 0$
            dus
            $\lim\limits_{n\to\infty} S_n = \frac{1}{1 - q}(1 - 0) = \frac{1}{1 - q} \in \mathbb{R}$
            
            Convergent
      \item $|q| > 1 \Rightarrow \lim\limits_{n\to\infty} q^n = +\infty$
            dus
            $\lim\limits_{n\to\infty} S_n = +\infty$
            
            Divergent naar $+\infty$
      \item $q \leq 1 \Rightarrow \lim\limits_{n\to\infty} q^n = ?$
      
            Divergent
    \end{itemize}
 \item $q = 1$
    $$\sum_{n = 0}^{\infty} 1^n = \sum_{n = 0}^{\infty} 1$$
    $$S_n = n \rightarrow \lim\limits_{n\to\infty} S_n = \lim\limits_{n\to\infty} n = +\infty$$
    
    Divergent naar $+\infty$
\end{itemize}

Uit dit onderzoek volgt het volgende:
$$
\boxed{
    \sum_{n = 0}^{\infty} q^n = \begin{cases}
                            |q| < 1 \rightarrow \hbox{Convergent} \\
                            q \geq 1 \rightarrow \hbox{Divergent naar } +\infty \\
                            q \leq -1 \rightarrow \hbox{Divergent}
                            \end{cases}
}
$$

\example{
    Convergeert de reeks $-2 + \frac{2}{3} - \frac{2}{9} + \frac{2}{27} - \frac{2}{81} + ...$ en indien convergent, naar welke reekssom?
}{
    Herschrijf de reeks:
    $$-2(1 - \frac{1}{3} + \frac{1}{9} - \frac{1}{27} + \frac{1}{81} - ... + \bigg(\frac{1}{3}\bigg)^n + ...$$
    Dit komt overeen met
    $$-2\sum_{n = 0}^{\infty}\bigg(-\frac{1}{3}\bigg)^n$$
    Dit is een meetkundige reeks met $q = -\frac{1}{3}$. Het is duidelijk dat $|q| < 1$ dus de reeks convergeert. De reekssom is $$\lim\limits_{n\to\infty} -2 \frac{1}{1 - \frac{1}{3}} = -\frac{3}{2}$$
}

\section{Eigenschappen}
\begin{itemize}
 \item De convergentie of divergentie verandert niet door het weglaten of bijvoegen van een eindig aantal termen.
 \item Als een reeks $\sum a_n$ convergeert naar $S$ dan convergeert de reekst $\sum k\;a_n$ naar $kS$
 \item Indien $\sum a_n$ convergeert dan geldt dat: $\lim\limits_{n\to\infty} a_n = 0$
 
        $\rightarrow \lim\limits_{n\to\infty} a_n \neq 0 \Rightarrow \sum a_n $ divergent
\end{itemize}
\example{
    Bewijs convergentie/divergentie van 
    $$\sum_{n = 1}^{\infty} \bigg(\frac{2n - 5}{2n - 7}\bigg)^n$$
}{
    \begin{equation*}
     \begin{split}
      \lim\limits_{n\to\infty}\bigg(\frac{2n - 5}{2n - 7}\bigg)^n & = 1^\infty \\
      \Rightarrow \lim\limits_{n\to\infty}\bigg(1 + \frac{2}{2n - 7}\bigg)^{n\frac{2n - 7}{2}\cdot\frac{2}{2n - 7}} & = e^{\lim\limits_{n\to\infty}\frac{2n}{2n - 7}} \\
      & = e \neq 0
     \end{split}
    \end{equation*}
    De reeks is divergent naar $+\infty$
}

\section{Reeksen met 'uitsluitend' positieve termen}
Dit zijn reeksen waarbij:
\begin{itemize}
 \item Een eindig aantal negatieve termen weggelaten mogen worden.
 \item Reeksen met uitsluitend negatieve termen kunnen vermenigvuldigd worden met -1.
\end{itemize}
\subsection{Integraalcriterium van Cauchy}
Indien $$a_n \geq 0$$
en $$f(x) = f(n)$$ waarbij $f(x)$ dalend en continu is over $[m, +\infty[$
dan geldt er

$$\int_m^\infty f(x)\;dx \in \mathbb{R} \Rightarrow \sum a_n \qquad \hbox{convergent}$$ 
$$\int_m^\infty f(x)\;dx = \infty \Rightarrow \sum a_n \qquad \hbox{divergent naar } \infty$$ 

\example{
    Onderzoek de convergentie/divergentie van
    $$\sum_{n = 1}^{\infty} \frac{\ln^2n}{n}$$
}{
    $$a_n = \frac{\ln^2n}{n} \geq 0$$
    $$f(x) = \frac{\ln^2x}{x}$$
    We bepalen het gebied waar $f(x)$ continue en dalend is. We berekenen de afgeleide:
    $$f'(x) = \frac{\ln x(2 - \ln x)}{x^2}$$
    Uit tekenonderzoek kan afgeleid worden dat $f(x)$ continu en dalend is vanaf $e^2$.  Kies $m \geq e^2$. Pak een gemakkelijk getal, zoals $m = 9$
    \begin{equation*}
     \begin{split}
      \int_9^{+\infty} \frac{\ln^2 x}{x}dx & = \bigg[\frac{\ln^3 x}{3}\bigg]_9^{+\infty} \\
                                           & = \frac{+\infty}{+\infty} - \frac{\ln^3 9}{3} \\
                                           & = +\infty
     \end{split}
    \end{equation*}
    De reeks divergeert naar $+\infty$.
}

\subsection{De hyperharmonische reeks}
$$\sum_{n=1}^{\infty} \frac{1}{n^p} = 1 + \frac{1}{2^p} + \frac{1}{3^p} + ... + \frac{1}{n^p} + ...$$
\subsubsection{Convergentieonderzoek}
\begin{itemize}
    \item $p \leq 0 \Rightarrow \lim\limits_{n\to\infty} n^{-p} = +\infty$
        De reeks divergeert naar $+\infty$
    \item $p > 0$
        We gebruiken het Integraalcriterium van Cauchy want
        $$\sum a_n = \sum \frac{1}{n^p} \qquad a_n \geq 0$$
        en
        $$f(x) = \frac{1}{x^p}$$
        De functie $f(x)$ is continu over $]0, +\infty[$, maar is pas dalend vanaf 1. Dus het interval wordt $[1, +\infty[$. De integraal wordt als volgt:
        $$\int_{1}^{+\infty} \frac{dx}{x^p} = \bigg[\frac{x^{1 - p}}{1 - p}\bigg]_1^{+\infty} = \frac{1}{1 - p}\bigg(\lim\limits_{n\to\infty}\frac{1}{x^{p-1}} - 1\bigg)$$
        \begin{itemize}[label={als}]
            \item $p - 1 > 0 \Rightarrow \lim\limits_{n\to\infty}\frac{1}{x^{p-1}} = 0$
                    
                    $\Rightarrow \frac{1}{1 - p}(0 - 1) \in \mathbb{R}$ : convergent
                    
            \item $p - 1 < 0 \Rightarrow \lim\limits_{n\to\infty}\frac{1}{x^{p-1}} = \infty$
            
                    $\Rightarrow \frac{1}{1 - p}(\infty - 1) = \infty$ : divergent naar $+\infty$
            
            \item $p - 1 = 1 \Rightarrow \int_1^{+\infty} \frac{dx}{x} = \infty$
            
                divergent naar $+\infty$
        \end{itemize}
        
\end{itemize}
Uit dit onderzoek volgt het volgende:
$$
\boxed{
    \sum_{n = 1}^{\infty} \frac{1}{n^p} = \begin{cases}
                            p > 1 \rightarrow \hbox{Convergent} \\
                            p \leq 1 \rightarrow \hbox{Divergent naar } +\infty \\
                            \end{cases}
}
$$
\subsection{Vergelijkingscriteria}
Indien $\sum a_n$ gevraagd wordt, gebruik een gekende reeks $\sum b_n$. Tot nu toe kennen we twee reeksen: $\sum q^n$ en $\sum \frac{1}{n^p}$

\begin{enumerate}
 \item 
    \begin{itemize}
     \item als $a_n \leq b_n$ en $\sum b_n$ convergent, dan $\sum a_n$ convergent
     \item als $b_n \leq a_n$ en $\sum b_n$ divergent, dan $\sum a_n$ divergent
    \end{itemize}
 \item 
    als $\lim\limits_{n\to\infty}\frac{a_n}{b_n} \neq 0 \neq \infty$ dan beide reeksen zelfde gedrag.

\end{enumerate}

\example{
    Onderzoek de convergentie/divergentie van
    $$\sum_{n = 0}^{\infty} \frac{1}{(3 + (-1)^n)^n}$$
}{
    De reeks kan geschreven worden als 
    $$\sum_{n = 0}^{\infty} \frac{1}{4^n}$$
    Dit komt overeen met een meetkundige reeks met $q = \frac{1}{4}$. Deze reeks convergeert.
}
\example{
    Onderzoek de convergentie/divergentie van
    $$\sum_{n = 0}^{\infty}\frac{\sqrt{n - 1}}{(3n - 2)^2}$$
}{
    Als $n$ naar oneindig gaat: $\frac{\sqrt{n}}{9n^2} \approx \frac{\sqrt{n}}{n^2} = \frac{1}{n^{3/2}}$
    Deze reeks kan dus geschreven worden als:
    $$\sum_{n = 0}^{\infty} \frac{1}{n^{3/2}}$$  
    wat overeenkomt met de hyperharmonische reeks met $p = 3/2$. We weten dat dit convergeert.
    We passen de limiet toe:
    $$\lim\limits_{n\to\infty} \frac{\sqrt{n - 1}}{(3n - 2)^2} \cdot n^{3/2} = \lim\limits_{n\to\infty} \frac{n^{3/2}\sqrt{n}}{9n^2} = \frac{1}{9} \neq 0 \neq \infty$$
    Deze reeks convergeert.
}
\example{
    Onderzoek de convergentie/divergentie van
    $$\sum_{n = 3}^{\infty} -\sqrt{\tan{\frac{1}{n}}}$$
}{
    In dit geval is $a_n < 0$. We vermenigvuldigen de reeks met -1. De reeks wordt:
    $$\sum_{n = 3}^{\infty}\sqrt{\tan{\frac{1}{n}}}$$
    We weten dat als $n$ naar oneindig gaat, dat $\frac{1}{n}$ naar 0 gaat. Voor kleine waarden geldt: $\tan{\frac{1}{n}} = \frac{1}{n}$. De reeks wordt:
    $$\sum_{n=3}^{\infty} \frac{1}{\sqrt{n}}$$ wat overeenkomt met een hyperharmonische reeks met $p = 0.5$. We weten dat dit divergeert naar $+\infty$
    
    We passen de limiet toe:
    $$\lim\limits_{n\to\infty} \frac{\sqrt{\tan\frac{1}{n}}}{\frac{1}{\sqrt{n}}} = \lim\limits_{n\to\infty} \sqrt{\frac{\tan\frac{1}{n}}{\frac{1}{n}}} = 1 \neq 0 \neq \infty$$
    Deze reeks divergeert naar $+\infty$
}


\subsection{Convergentiecriteria}
We gebruiken 2 convergentiecriteria:
\subsubsection{D'Alembert}
$$\lim\limits_{n\to\infty} \frac{a_{n + 1}}{a_n}$$
\subsubsection{Cauchy}
$$\lim\limits_{n\to\infty} \sqrt[n]{a_n}$$

Beide limieten kennen 3 uitkomsten:
$$
    \begin{cases}
        < 1, \hbox{convergentie} \\
        = 1, ???? \\
        > 1, \hbox{divergentie}
    \end{cases}
$$

\example{
    Onderzoek de convergentie/divergentie van:
    $$\sum_{n = 10}^{\infty} \frac{n^3}{3^n - n}$$
}{
    We gebruik de convergentiecriterium van D'Alembert.
    \begin{equation*}
        \begin{split}
            \lim\limits_{n\to\infty} \frac{(n + 1)^3}{(3^{n + 1} - (n + 1))} \cdot \frac{3^n - n}{n^3} & = \lim\limits_{n\to\infty} \frac{n^3}{(3^{n + 1} - (n + 1))} \cdot \frac{3^n - n}{n^3} \\
            & = \lim\limits_{n\to\infty} \frac{3^n - n}{3^{n + 1} - (n + 1)} \\
            & = \lim\limits_{n\to\infty} \frac{3^n(1 - \frac{n}{3^n})}{3^{n + 1}(1 - \frac{n + 1}{3^{n + 1}})} \\
            & = \frac{1}{3} < 1 \rightarrow \hbox{convergentie}
        \end{split}
    \end{equation*}
}
\example{
    Onderzoek de convergentie/divergentie van:
    $$\sum_{n = 1}^{\infty} - \bigg(\frac{3n - 1}{3n + 1} \bigg)^n$$
}{
    Dit is een reeks met uitsluitend negatieve termen, dus we vermenigvuldigen met -1. We beschouwen nu de reeks:
    $$\sum_{n = 1}^{\infty} \bigg(\frac{3n - 1}{3n + 1} \bigg)^n$$
    Dit lijkt het geschikte probleem om op te lossen met Cauchy.
    $$\lim\limits_{n\to\infty} \sqrt[n]{\bigg(\frac{3n - 1}{3n + 1} \bigg)^n} = \lim\limits_{n\to\infty} \frac{3n - 1}{3n + 1} = 1$$
    Er kan dus geen besluit gevormd worden. We maken gebruik van de algemene limiet.
    $$\lim\limits_{n\to\infty} a_n = \lim\limits_{n\to\infty} \bigg(\frac{3n - 1}{3n + 1}\bigg)^n = 1^{\infty}$$
    We maken gebruik van de definitie van het getal $e$.
    $$\lim\limits_{n\to\infty} \bigg(1 + \big( - \frac{2}{3n + 1}\big)\bigg)^{-\frac{3n + 1}{2}\cdot(-\frac{2}{3n + 1})n} = \lim\limits_{n\to\infty} e^{-\frac{2n}{3n + 1}} = e^{-2/3}  \neq 0$$
    Deze reeks is divergent naar $+\infty$. De originele reeks $\sum_{n = 1}^{\infty} - (\frac{3n - 1}{3n + 1} )^n$ is divergent naar $-\infty$
}
\example{
    Onderzoek de convergentie/divergentie van:
    $$\sum_{n = 4}^{\infty} \bigg(\frac{n}{2n + 1}\bigg)^{n^{2}}$$
}{
    We maken gebruik van het convergentiecriterium van Cauchy
    $$\lim\limits \sqrt[n]{\bigg(\frac{n}{2n + 1}\bigg)^{n^{2}}} = \lim\limits \bigg(\frac{n}{2n + 1}\bigg)^{n} = \bigg(\frac{1}{2}\bigg)^{\infty} = 0 < 1$$
    Deze reeks convergeert.
}
\example{
    Onderzoek de convergentie/divergentie van:
    $$\sum_{n = 0}^{\infty} \bigg(\frac{(2n)!n}{(n!)^2} \bigg)$$
}{
    Aangezien we te maken hebben met faculteiten is een goede keuze het convergentiecriterium van D'Alembert.
    \begin{equation*}
        \begin{split}
            \lim\limits_{n\to\infty} \frac{(2n + 2)!(n + 1)}{((n + 1)!)^2}\cdot \frac{(n!)^2}{(2n)!n} & = \lim\limits_{n\to\infty} \frac{(2n)!(2n + 1)(2n + 2)(n!)(n!)}{n!(n + 1)(n!)(n + 1)(2n)} \\
            & = \lim\limits_{n\to\infty} \frac{(2n + 1)(2n + 2)}{(n + 1)^2} \\
            & = \lim\limits_{n\to\infty} \frac{4n^2}{n^2} \\
            & = 4 > 1
        \end{split}
    \end{equation*}
    Deze reeks is divergent naar $+\infty$
}

\section{Willekeurige reeksen}
Een reeks is willekeurig indien er een oneindig aantal negatieve en positieve termen zijn. Wij zien een speciale soort van willekeurige reeksen: de alternerende reeks. Deze heeft veelal de volgende vorm:
$$(-1)^n b_n \quad \hbox{met} \quad b_n > 0$$

\subsection{Convergentiecriterium van Leibniz}
Indien een reeks alternerend is en
$$\forall n : |a_n| \geq |a_{n + 1}| \quad \hbox{en}\quad \lim\limits_{n\to\infty} |a_n| = 0$$
dan convergeert de reeks

\example{
    Onderzoek de convergentie/divergentie van:
    $$\sum_{n = 10}^{\infty} \frac{(-1)^n}{e^n - n}$$
}{
    We onderzoeken de voorwaarden van Leibniz.
    \begin{itemize}
        \item De reeks is duidelijk alternerend door $(-1)^n$
        \item $|a_n|$ moet dalend zijn. We stellen $f(x) = \frac{(-1)^x}{e^x - x}$. De afgeleide hiervan is $f'(x) = \frac{1 - e^x}{(e^x -x)^2}$. Uit tekenonderzoek
        blijkt dat $|a_n|$ dalend is voor alle $n > 0$.
        \item $\lim\limits_{n\to\infty} |a_n|$ moet 0 zijn.
        $$\lim\limits_{n\to\infty} \frac{1}{e^n - n} = \frac{1}{e^n(1 - n/e^n)} = 0$$
    \end{itemize}
    De reeks convergeert.
}

\subsection{Nieuwe begrippen}
Een reeks is \textbf{absoluut convergent} als $\sum |a_n|$ convergeert. Een reeks is \textbf{semi-convergent} als $\sum |a_n|$ divergeert en $\sum a_n$ convergeert.

Pas de volgende methode toe om een willekeurige reeks te onderzoeken:
\begin{enumerate}
    \item Ga na of de reeks absoluut convergent is
    \item Indien de reeks niet absoluut convergent is, pas dan de voorwaarden van Leibnitz toe.
    \item Maak ook gebruik van $\lim\limits_{n\to\infty} a_n \leq 0 \Rightarrow \sum a_n$ divergent.
\end{enumerate}

\example{
    Onderzoek de convergentie/divergentie van:
    $$\sum_{n = 1}^{\infty} (-1)^n \frac{n^2}{(2n - 1)!}$$ 
}{
    We gaan eerst na of de reeks absoluut convergent is. Dit betekent dus dat we enkel de absolute termen moeten beschouwen. De reeks wordt $\sum_{n = 1}^{\infty} \frac{n^2}{(2n - 1)}$. We passen D'Alembert toe0
    \begin{equation*}
        \begin{split}
            \lim\limits_{n\to\infty} \frac{(n + 1)^2}{(2n + 1)} \cdot \frac{(2n - 1)!}{n^2} & = \lim\limits_{n\to\infty}  \frac{1}{2n(2n + 1)} \\
            & = 0 < 1
        \end{split}
    \end{equation*}
    De reeks met absolute waarden convergeert. De originele reeks is dus absoluut convergent.
}
\example{
    Onderzoek de convergentie/divergentie van:   
    $$\sum_{n = 0}^{\infty} (-1)^n \frac{\arctan(n)}{n^2 + 1}$$
}{
    We gaan eerst weer na of deze reeks absoluut convergent is. We passen D'Alembert toe.
    \begin{equation*}
        \begin{split}
            \lim\limits_{n\to\infty} \frac{\arctan(n + 1)}{(n + 1)^2 + 1} \cdot \frac{n^2 + 1}{\arctan(n)} & = \lim\limits_{n\to\infty}  \frac{\pi/2 (n^2 + 1)}{(n^2 + 2n + 2)\pi/2} \\
            & = \lim\limits_{n\to\infty} \frac{n^2}{n^2} \\
            & = 1
        \end{split}
    \end{equation*}
    Er kan geen besluit genomen worden. We maken gebruik van vergelijkingscriterium II. We zoeken eerst een reeks om mee te vergelijken. 
    $$\sum \frac{\arctan(n)}{n^2 + 1} \approx \frac{1}{n^2}$$
    We nemen de limiet
    $$\lim\limits_{n\to\infty}  \frac{\arctan(n)}{n^2 + 1} \cdot n^2 = \frac{\pi}{2} \neq 0 \neq \infty$$
    De reeks waarmee we vergeleken hebben is een harmonische reeks met $p = 2$. We weten dat deze reeks convergeert dus $\sum_{n = 0}^{\infty} \frac{\arctan(n)}{n^2 + 1}$ convergeert ook. Bijgevolg is $\sum_{n = 0}^{\infty} (-1)^n \frac{\arctan(n)}{n^2 + 1}$ absoluut convergent.
}
\example{
   Onderzoek de convergentie/divergentie van:   
   $$\sum_{n = 1}^{\infty}(-1)^{n - 1} \sin \frac{1}{\sqrt{4n - 1}}$$    
}{
    Bij het uitrekenen van D'Alembert zou je 1 uitkomen waardoor geen besluit kan genomen worden. We gebruiken volgende redenering.

    Als $n$ stijgt, dan stijgt $4n - 1$, dan stijgt $\sqrt{4n - 1}$, dan daalt $\frac{1}{\sqrt{4n - 1}}$ en dan daalt $\sin \frac{1}{\sqrt{4n - 1}}$. De reeks kan als volgt benaderd worden. 
    $$\sum_{n = 1}^{\infty} \sin \frac{1}{\sqrt{4n - 1}} \approx \sum_{n = 1}^{\infty} \frac{1}{\sqrt{4n - 1}} \approx \sum_{n = 1}^{\infty}  \frac{1}{\sqrt{n}}$$
    Dit is een harmonische reeks met $p = 1/2$. Deze reeks is divergent.
    We maken gebruik van vergelijkingscriterium II en nemen de limiet.
    \begin{equation*}
        \begin{split}
            \lim\limits_{n\to\infty} \frac{\sin \frac{1}{\sqrt{4n - 1}}}{1/\sqrt{n}} & = \frac{\cos \frac{1}{\sqrt{4n - 1}} -\frac{1}{2}(4n - 1)^{-3/2}(4)}{-\frac{1}{2}n^{-3/2}} \\
            & = \frac{-2}{\frac{-1}{2}} \lim\limits_{n\to\infty} \frac{n^{3/2}}{(4n - 1)^{3/2}} \\
            & = 4 \lim\limits_{n\to\infty} \frac{n^{3/2}}{4n^{3/2}} \\
            & = \frac{1}{2} \neq 0 \neq \infty
        \end{split}
    \end{equation*}
    De reeks toont hetzelfde gedrag als de harmonische reeks en is dus divergent. Het besluit is dat deze reeks niet absoluut convergent is. We gaan nu na of de reeks semi-convergent is met de voorwaarden van Leibniz. We hebben al bewezen dat $\sin \frac{1}{\sqrt{4n - 1}}$ dalend is. We berekenen de limiet:
    $$\lim\limits_{n\to\infty} |a_n| = \lim\limits_{n\to\infty} \frac{1}{\sqrt{4n - 1}} = 0$$
    De reeks is semi-convergent.
}
\example{
    Onderzoek de convergentie/divergentie van:
    $$\sum_{n = 0}^{\infty} (-1)^n (1 - \frac{1}{10^n})$$      
}{
    Dit is heel eenvoudig na te gaan met $\lim\limits_{n\to\infty} a_n$
    $$\lim\limits_{n\to\infty} (-1)^n (1 - \frac{1}{10^n}) = \lim\limits_{n\to\infty} (-1)^n (1 - 0) = -1^{\infty}$$
    De reeks is divergent.
}

\section{Reeksen van functies}
$$\sum_{n = 1}^{\infty} u_n(x) = u_1(x) + u_2(x) + ... + u_n(x) + ...$$
Bij deze soort reeksen zijn we geïnteresseerd in het convergentiegebied van x. Voor welke waarden van x is de overeenkomstige reeks convergent.
Gebruik volgende methodiek:
\begin{enumerate}
    \item Bereken
        $$L(x) = \lim\limits_{n\to\infty} \bigg| \frac{u_{n + 1}(x)}{u_n(x)} \bigg| \quad\hbox{of}\quad L(x) = \lim\limits_{n\to\infty} \sqrt[n]{|u_n(x)|}$$
    \item Indien $L(x) < 1$ dan behoort x tot het CG

          Indien $L(x) > 1$ dan behoort x niet tot het CG

          Indien $L(x) = 1$ dan ???

\end{enumerate}

\subsection{Machtreeksen rond x = a}
$$\sum_{n = 0}^{\infty} c_n(x - a)^n$$
\textbf{stelling:} Het CG van een machtreeks rond x=a is een symmetrisch interval rond x=a

\textbf{bewijs:}
\begin{equation*}
    \begin{split}
        \lim\limits_{n\to\infty} \bigg|\frac{ u_{n + 1}(x)}{u_n(x)} \bigg| & = \lim\limits_{n\to\infty} \bigg|\frac{ c_{n + 1}(x - a)^{n + 1}}{c_n(x - a)^n} \bigg| \\
        & = \lim\limits_{n\to\infty} \bigg | \frac{c_{n + 1}}{c_n} \bigg | |x - a| < 1 \\
        & x \in CG \Leftrightarrow |x - a| < \frac{1}{ \lim\limits_{n\to\infty} | \frac{c_{n + 1}}{c_n}|}  = \rho
    \end{split}
\end{equation*}
Hieruit volgt $$\rho < x - a < \rho \rightarrow a - \rho < x < a + \rho$$
Het CG wordt $$]a - \rho, a + \rho[$$

\example{
    Bepaal het CG van:
    $$\sum_{n = 1}^{\infty} \bigg(\frac{2x + 1}{x}\bigg)^n$$
}{
    We passen de limiet van Cauchy toe.
    $$\lim\limits_{n\to\infty} \sqrt[n]{\bigg|\bigg(\frac{2x + 1}{x}\bigg)^n\bigg|}$$
    x behoort enkel tot een CG als deze limiet kleiner dan 1 is. We onderzoeken deze functie eerst.
    \begin{equation*}
        \begin{split}
            \bigg| \frac{2x + 1}{x} \bigg| < 1  & = \frac{(2x + 1)^2}{x^2} < 1 \\
                                                & = (2x + 1)^2 < x^2 \\
                                                & = 4x^2 + 4x + 1 - x^2 < 0 \\
                                                & = 3x^2 + 4x + 1 < 0 \\
                                                & = 3(x + 1/3)(x + 1) < 0
        \end{split}
    \end{equation*}
    Uit tekenonderzoek blijkt dat $x \in CG \rightarrow x \in ]-1, -1/3[$. Om na te gaan ofdat de grenzen inbegrepen zijn berekenen we de limiet voor x = -1 en x = -1/3.
}