\chapter{Differentiaalvergelijking}
\section{Definities}
De algemene definitie is:
$$F(x, y, y', y'', ..., y^{(n)}) = 0$$
waarbij: \begin{itemize}
\item \textbf{x} een veranderlijke is.
\item \textbf{y} een functie van x is.
\item er minstens één afgeleide van y is.
\end{itemize}
\example{Differentiaalvergelijking}{
    $$ x + y + y' = 0$$
}

Een differentiaalvergelijking heeft een \textbf{orde} en een \textbf{graad}
\begin{itemize}
    \item \textbf{Orde}: Dit is de orde van de hoogste afgeleide dat voorkomt, dus \textit{n}.
    \item \textbf{Graad}: De graad \textit{r} bestaat niet altijd maar is wel altijd een strik positief geheel getal. De graad is de macht die behoort tot de afgeleide met de grootste orde. $y^{(n)^{r}}$
\end{itemize}
\example{Orde en graad}{
    \begin{center}
        \begin{tabular}{l | l | l}
            Differentiaalvergelijking                          & Orde & Graad \\
            \hline
            $y\ - 2y'^3 = yx$                                  & 2    & 1     \\
            $1 + (y'')^4 + 2y' + x(y''')^2 = sin(x)$           & 3    & 2     \\
            $(x - 1)(y'') - xy' + y = 0$                       & 2    & 1     \\
            $e^s\frac{d^3s}{dt^3} + (\frac{ds^2}{dt^2})^3 = 0$ & 3    & 1     \\
            $xy' + e^{y'} + y'' = 1$                           & 1    & /     \\
            \hline
            $\sin\sqrt {y'} = x + 2$                           & 1    & /     \\
            $\;\;\rightarrow y' = \arcsin^2(x+2)$              & 1    & 1     \\
            \hline
            $\sin y' = xy'^2$                                  & 1    & /     \\
            $\;\;\rightarrow y' = \arcsin(xy'^2)$              & 1    & /     \\
            \hline
            $y^{'3} + \frac{x}{y''} + y'' = 1$                 & 2    & ?     \\
            $\;\;\rightarrow y^{'3}y'' + x + (y'')^2 = 1$      & 2    & 2     
            
        \end{tabular}
    \end{center}
}
\section{Soorten oplossingen}
Tijdens het oplossen van een differentiaalvergelijking van de \textit{n}-de orde worden drie oplossingen onderscheden:
\begin{enumerate}
\item De \textbf{Algemene oplossing (AO)}: Verzameling van functies zodat de differentiaalvergelijking klopt. De algemene oplossing bevat \textit{n }onafhankelijke constanten. Deze constanten zijn getallen en geen functies.
\item De \textbf{Particuliere oplossing (PO)}: Dit is één van de krommen van de AO en is afhankelijk van de beginvoorwaarden van het probleem.
\item De \textbf{Singuliere oplossing (SO)}: Een oplossing die niet voldoet aan de AO maar wel een oplossing is voor de DVG.
\end{enumerate}
\example{Onafhankelijke variabelen:}
{
\begin{center}
    \begin{tabular}{l | l | l}
    AO & Onafh. C & Orde DVG \\
    \hline
    $y = C_1 + C_2x$ & 2 & 2 \\
    $y = C_1  - C_1^2x$ & 1 & 1 \\
    \hline
    $y = C_1(C_2 + C_3e^x)$ & ? & ? \\
    $\;\;\rightarrow C_1C_2 + C_1C_3e^x$ & ? & ? \\
    $\;\;\rightarrow a + be^x$ & 2 & 2 \\
    \hline
    $y = C_1 + \ln(C_2 x)$ & ? & ? \\
    $\;\;\rightarrow y = C_1 + \ln(C_2) + \ln(x)$ & ? & ? \\
    $\;\;\rightarrow y = a + \ln(x)$ & 1 & 1


    \end{tabular}
\end{center}
}
\example{Oef 1 AO en PO}{Gegeven een differentiaalvergelijking: $y'' + y = 0$
\begin{enumerate}
\item Toon aan dat $y = a\sin(x) + b\cos(x)$ de AO is.
\item Geef enkele PO's.
\end{enumerate}
Oplossing:
\begin{enumerate}
\item 
\begin{equation*}
\begin{split}
    y & = a\sin(x) + b\cos(x) \\
    y' & = a\cos(x) - b\sin(x) \\
    y'' & = -a\sin(x) - b\cos(x) 
\end{split}
\end{equation*}
Hieruit volgt:
\begin{gather*}
    y'' + y  = 0 \\
    -a\sin(x) - b\cos(x) + \sin(x) + b\cos(x)  = 0  \\
    \rightarrow \hbox{Het is een oplossing}
\end{gather*}
De differentiaalvergelijking heeft orde 2. De y-vergelijking bevat 2 onafhankelijke constanten en de y-vergelijking is een oplossing. Hierdoor is y de AO van de differentiaalvergelijking.
\item Enkele PO's:
\begin{equation*}
\begin{split}
y & = 0\\
y & = \sqrt{2}\sin(x) \\
y & = \sin(x) + \cos(x)
\end{split}
\end{equation*}
\end{enumerate}
}

\example{Oef 2 AO en PO}
{Gegeven een differentiaalvergelijking: $y'^2 - yy'+e^x$
\begin{enumerate}
\item Geef de orde en graad.
\item Is $y = \frac{1}{C} + Ce^x$ de AO?
\item Wat  voor soort oplossing is $y = 2\sqrt{e^x}$
\end{enumerate}
Oplossing:
\begin{enumerate}
\item 
De orde is 1 en de graad is 2.

\item 
$$y' = Ce^x$$
\begin{equation*}
\begin{split}
\rightarrow C^2(e^x)^2 - (\frac{1}{C} + Ce^x)Ce^x + e^x &  = 0 \\
\Leftrightarrow C^2e^{2x} - e^x - C^2e^{2x} + e^x &  = 0 \\
\Leftrightarrow C^2e^{2x} - e^x - C^2e^{2x} + e^x & = 0 \\
\Leftrightarrow 0 & =0
\end{split}
\end{equation*}
$$\rightarrow \hbox{Het is een oplossing}$$
Orde DVG = 1 = Onafhankelijke constanten van y

\item 
$$ y'  = 2 \cdot \frac{1}{2\sqrt{e^x}} \cdot e^x = \sqrt{e^x}$$
\begin{equation*}
\begin{split}
\rightarrow & y'^2 - yy'+e^x \\
\Leftrightarrow &  (\sqrt{e^x})^2 - 2\sqrt{e^x}\cdot\sqrt{e^x} + e^x  = 0 \\
\Leftrightarrow & e^x - 2e^x + e^x  = 0 \\
\Leftrightarrow & 0 = 0
\end{split}
\end{equation*}
Dit is een singuliere oplossing aangezien y niet overeenkomt met de AO, maar wel voldoet aan de DVG.

\end{enumerate}
}
\section{Bepalen van een DVG}
Indien een AO gegeven is met \textit{n} onafhankelijke constanten:
\begin{enumerate}
\item Controleer of de constanten werkelijk onafhankelijk zijn.
\item Leid de AO \textit{n} maal af.
\item Elimineer de \textit{n} constanten van de \textit{n + 1} bekomen vergelijkingen. De laatste vergelijking moet zeker gebruikt worden.
\item Controleer of de DVG van orde \textit{n} is.
\end{enumerate}

\example{Oef 1 bepalen van een DVG}
{
De algemene oplossing is $$y = C_1 + C_2x$$
\begin{enumerate}
\item Er zijn \textit{2} onafhankelijke constanten.
\item Er moet \textit{2} keer afgeleid worden:
$$
    \begin{cases}
    y    & = C_1 + C_2x \\
    y'   & = C_2 \\
    y''  & = 0 \\
    \end{cases}
$$
\item De constanten zijn al geëlimineerd. 
\item De DVG is $y'' = 0$ en heeft orde \textit{2}.

\end{enumerate}
}

\example{Oef 2 bepalen van een DVG}
{
Bepaal de DVG van: $$y = C_1 + C_2e^{-x} + C_3e^{3x}$$
\begin{enumerate}
\item Er zijn \textit{3} onafhankelijke constanten.
\item Er moet \textit{3} maal afgeleid worden.
    \[ 
    \begin{cases}
            y & = C_1 + C_2e^{-x} + C_3e^{3x} \\
    y'     & = -C_2e^{-x} + 3C_3e^{3x}     \\
    y'' & = C_2e^{-x} + 9C_3e^{3x}      \\
    y''' & = -C_2e^{-x} + 27C_3e^{3x}
    \end{cases}
    \]

\item 
Tel de 1ste afgeleide op met de 2de afgeleide en tel de 2de afgeleide op met de 3rde afgeleide
\[
    \begin{cases}
    y + y''    & = 3C_3e^{3x} + 9C_3e^{3x} = 12C_3e^{3x}  \\
    y'' + y''' & = 9C_3e^{3x} + 27C_3e^{3x} = 36C_3e^{3x}
    \end{cases}
\]
Vermenigvuldig de 1ste vergelijking met 3 en trek hiervan de 2de vergelijking af.

$$3(y + y'') - y'' - y''' = 3(12C_3e^{3x}) - 36C_3e^{3x} = 0$$
$$\rightarrow y''' - 2y'' - 3y' = 0$$
\item
De orde van deze DVG is \textit{3}

\end{enumerate}
}

\example{Oef 3 bepalen van een DVG}
{
Bepaal de DVG van alle cirkels met middelpunt y = -x.
\begin{enumerate}
\item Eerst moet de AO gevonden worden. Het middelpunt van elke cirkel kan gegeven worden met $m(a, -a).$
    Hieruit volgt de algemene vergelijking van een cirkel: $$(x - a)^2 + (y + a)^2 = R^2$$
    Er zijn \textit{2} onafhankelijke constanten (a en R).
\item Er moet \textit{2} maal (impliciet) afgeleid worden.
\[
    \begin{cases}
    (x - a)^2 + (y + a)^2 = R^2 \\
    \frac{dy}{dx} : (x-a) + y'(y+a) = 0 \\
    \frac{d^2y}{dx^2} : 1 + y''(y + a) + y'^2 = 0
    \end{cases}
\]
\item
    Vorm $\frac{dy}{dx}$ om naar $a$:
    $$a = \frac{-x - yy'}{y' - 1}$$
    Substitueer deze $a$ in $\frac{d^2y}{dx^2}$:
    $$1 + y''(y + (\frac{-x - yy'}{y' - 1})) + y'^2 = 0$$
    $$\rightarrow 1 + y''(y + (\frac{x + yy'}{-y' + 1})) + y'^2 = 0$$
    $$\rightarrow y''(x + y) - y'^3 + y'^2 - y' + 1 = 0$$
\item Orde van de DVG = \textit{2}  = Aantal onafhankelijke constanten.
\end{enumerate}
}
\section{Oplossen van een lineaire DVG van orde n met constante reële coëfficiënten}
\begin{equation*}
 \begin{split}
		  & y''' - y''\sin t + ty = t^2 \\
  \Leftrightarrow & D^3y - D^2y\sin t + ty = t^2 \\
  \Leftrightarrow & (D^3 - D^2\sin t + t)y = t^2 \\
  \Leftrightarrow & L(d)y = g(t) \\
  \Leftrightarrow & \hbox{met} L(d) = \sum_{i = 0}^{n} a_iD^i \qquad ,a_i \in \mathbb{R}
 \end{split}
\end{equation*}
Een lineaire DVG is een DVG waarbij alle coëfficiënten van alle afgeleiden enkel voorkomen als eerste macht.
\subsection{Particuliere oplossing}
De particuliere oplossing kan slechts bepaald worden indien alle beginvoorwaarden 

($y(0), y'(0),...,y^{(n-1)}(0)$) gekend zijn.
\subsection{Algemene oplossing}
Indien de beginvoorwaarden niet gekend zijn moeten $y(0),y'(0)...y^{(n-1)}(0)$ respectievelijk gelijkgesteld worden aan $C_1, C_2, ..., C_n$

\example{
  Bepaal de PO van $y'' + y = g(t)$ indien $y(0) = 0$, $y'(0) = 1$ en 
  $$
    g(t) = \begin{cases}
	      0 & t < 1 \\
	      e^{-t} & t > 1
	   \end{cases}
  $$
}{
  \begin{equation*}
   \begin{split}
                   & L(d)y = g(t) \\
   \Leftrightarrow & (D^2 + 1)y = g(t) \\
   \Leftrightarrow & (D^2 + 1)y = e^{-t}H(t - 1) \\
   \mathcal{L}\{LL\}(s) & = \mathcal{L}\{y'' + y\}(s) \\
                        & = s^2Y - sy(0^{+}) + y'(0^{+}) + Y \\
                        & = s^2Y - 1 + Y \\
   \mathcal{L}\{RL\}(s) & = \mathcal{L}\{e^{-t}H(t - 1\}(s) \\
                        & = \mathcal{L}\{e^{-(t - 1) - 1}H(t - 1)\}(s) \\
                        & = e^{-1}\mathcal{L}\{e^{-(t - 1)}H(t - 1)\}(s) \\
                        & = e^{-1}e^{-s}\mathcal{L}\{e^{-t}\}(s) \\
                        & = e^{-1}e^{-s}\frac{1}{s + 1} \\
                        & = \frac{e^{-(s + 1)}}{s + 1} \\
   \hbox{dus} \\
   \Leftrightarrow & s^2Y - 1 + Y = \frac{e^{-(s + 1)}}{s + 1} \\
   \Leftrightarrow & Y(s^2 + 1) = 1 + \frac{e^{-(s + 1)}}{s+1} \\
   \Leftrightarrow & Y = \frac{1}{s^2 + 1} + \frac{e^{-(s+1)}}{(s+1)(s^2+1)}     \\
   \Leftrightarrow & \mathcal{L}^{-1}\{Y\}(t) = \mathcal{L}^{-1}\bigg\{\frac{1}{s^2 + 1} + \frac{e^{-(s+1)}}{(s+1)(s^2+1)}\bigg\}(t) \\
   \Leftrightarrow & y(t) = \sin t +e^{-1}f(t - 1)H(t - 1) \\
	   \hbox{met}\; f(t) & = \mathcal{L}^{-1}\bigg\{\frac{1}{(s + 1)(s^2 + 1)}\bigg\}(t) \\
			     & = \frac{1}{2}\mathcal{L}^{-1}\bigg\{\frac{1}{s + 1} - \frac{s - 2}{s^2 + 1}\bigg\}(t) \\
			     & = \frac{1}{2}\bigg[e^{-t} - (\cos t + \sin t)\bigg] \\
	    \hbox{antwoord: }\; y(t)  & = \sin t + \frac{1}{2}\bigg(e^{-t} - e^{-1}\cos (t - 1) + e^{-1} \sin (t - 1) \bigg)H(t - 1)
   \end{split}
  \end{equation*}

}
\example{
  Bepaal de PO van $y'' + y = \delta\big(t - \frac{\pi}{2}\big)$ indien $y(\frac{\pi}{4}) = 0$, $y'(\frac{\pi}{4}) = 0$
}{
  Stel: $y(0^+) = C_1$, $y'(0^+) = C_2$
  \begin{equation*}
   \begin{split}
    \mathcal{L}\{LL\}(s) & = \mathcal{L}\{y'' + y\}(s) \\
			 & = s^2Y - sC_1 - C_2 + Y \\
    \mathcal{L}\{RL\}(s) & = \mathcal{L}\bigg\{\delta\big(t - \frac{\pi}{4}\bigg)\bigg\}(s) \\
                         & = \int_0^{+\infty}\delta\big(t - \frac{\pi}{4}\big)e^{-st} \; dt \\
                         & = e^{-\frac{\pi}{2}s} \\
    \hbox{dus} \\
    \Leftrightarrow & s^2Y - sC_1 - C_2 + Y = e^{-\frac{\pi}{2}s}  \\
    \Leftrightarrow & Y = \frac{e^{-\frac{\pi}{2}s} + sC_1 + C_2}{s^2 + 1} \\
    \Leftrightarrow & \mathcal{L}^{-1}\{Y\}(t) = \mathcal{L}^{-1}\bigg\{\frac{e^{-\frac{\pi}{2}s} + sC_1 + C_2}{s^2 + 1}\bigg\}(t) \\
    \Leftrightarrow & y(t) = C_2\sin t + C_1\cos t + f\big(t - \frac{\pi}{2}\big)H\big(t - \frac{\pi}{2}\big)\\
    \hbox{met}\; f(t) & = \mathcal{L}^{-1}\bigg\{\frac{1}{s^2 + 1}\bigg\}(t) \\
                      & = \sin t\\
    \hbox{De algemene oplossing: }\; y(t)  & =  C_2\sin t + C_1\cos t - \bigg(\cos(t) H\big(t - \frac{\pi}{2}\big)\bigg)
   \end{split}
  \end{equation*}
  De PO voor $t = \frac{\pi}{4} \qquad ( < \frac{\pi}{2}\;\hbox{dus Heaviside is 0})$
  \begin{equation*}
   \begin{split}
    y(t) & = C_2 \sin t + C_1 \cos t  \Rightarrow y(\frac{\pi}{4})  : 0 = C_2\frac{\sqrt{2}}{2} + C_1\frac{\sqrt{2}}{2}\\
    y'(t) & = C_2 \cos t - C_1 \sin t \Rightarrow y'(\frac{\pi}{4}) : 0 = C_2\frac{\sqrt{2}}{2} - C_1\frac{\sqrt{2}}{2}\\
    & \begin{cases}
     0 = C_2 + C_1 \\
     0 = C_2 - C_1
    \end{cases} \Rightarrow C_2 = C_1 = 0
   \end{split}
  \end{equation*}
    Het antwoord:
    $$y(t) = -(\cos t)H(t - \frac{\pi}{2})$$
}
\section{DVG van de orde 1 en graad 1}
\subsection{Gescheiden veranderlijken}
Indien een DVG van orde 1 en graad 1 te schrijven is als
$$f(x)\;dx = g(y)\;dy$$
Algemeen:
\begin{equation*}
 \begin{split}
  M(x, y)\;dx & = -N(x, y)\; dy \\
  f(x)g(y)\;dx & = -h(x)i(y)\; dy \\
  \frac{f(x)}{h(x)}\; dx = & -\frac{i(y)}{g(y)}\;dy \\
  a(x) \; dx & = b(y)\; dy \\
  \int a(x) \; dx & = \int b(y) \; dy \\
  A(x) + C_1 & = B(y) + C_2 \\
  A(x) & = B(y) + C
 \end{split}
\end{equation*}
\example{
    Bepaal de AO van $yt + \sqrt{1 - t^2}y' = 0$
}{
    \begin{equation*}
     \begin{split}
      & yt\;dt + \sqrt{1 - t^2}\;dy = 0 \\
      \Leftrightarrow & \sqrt{1 - t^2}\;dy = -yt\;dt \\
      \Leftrightarrow & \frac{dy}{y}= -\frac{t\;dt}{\sqrt{1 - t^2}} \\
      \Leftrightarrow & \int \frac{dy}{y}= -\int \frac{t\;dt}{\sqrt{1 - t^2}} \\
      \Leftrightarrow & \ln |y| \sqrt{1 - x^2} + C \\
      \Leftrightarrow & y = e^{\sqrt{1 - x^2} + C} \\
      \Leftrightarrow & y = e^{\sqrt{1 - x^2}}e^C \\
      \Leftrightarrow & y = De^{\sqrt{1 - x^2}} \\
     \end{split}
    \end{equation*}

}

\todo{LES DINSDAG 13/03}
\section{Homogene DVG}
Een DVG is homogeen indien:
$$f(\lambda x, \lambda y) = \lambda^n f(x,y)$$
Indien een DVG homogeen is kan volgende oplossingsmethode toegepast worden:

\example{
        Bepaal de PO van : 
        $$\frac{dx}{dt} = \frac{x}{t(\ln t - \ln x)}$$
        waarvoor x(1) = 1.
}{
    Berekening algemene oplossing
    \begin{equation*}
     \begin{split}
      & \frac{dx}{dt} = \frac{x}{t(\ln t - \ln x)} \\
      \Rightarrow & t(\ln t - \ln x) \;dx - x\;dt = 0 \\
      \Rightarrow & t\ln\bigg(\frac{t}{x}\bigg)\;dx - x\;dt = 0 \\
      & \hbox{controle homogeen} \\
      \Rightarrow & \lambda t \ln\bigg(\frac{\lambda t}{\lambda x}\bigg) - \lambda x\\
      \Rightarrow & \lambda^1 (t \ln\bigg(\frac{t}{x}\bigg)  - x) \qquad \hbox{homogeen want M(x,t) en N(x,t) hebben } \lambda \hbox{ tot de eerste macht} \\
      & \hbox{substitutie } t = ux \\
      \Rightarrow & u \ln u \; dx - u \;dx + x\; du = 0 \\
      \Rightarrow & (u \ln u - u) \; dx = x \; du \\
      \Rightarrow & \int \frac{du}{u(\ln n -1)} = \int \frac{dx}{x} \\
      \Rightarrow & \ln | \ln u - 1| = \ln |x| + \ln |C| \\
      \Rightarrow & \ln | \ln u - 1| = \ln |Cx|  \\ 
      \Rightarrow & \ln u - 1 = Cx \\ 
      \Rightarrow & \ln u = Cx + 1 \\
      \Rightarrow & u = e^{Cx + 1} \\
      \Rightarrow & t = xe^{Cx + 1}
     \end{split}
    \end{equation*}
    Berekening particuliere oplossing:
    \begin{equation*}
     \begin{split}
      & x(1) = 1 \\
      \Rightarrow & 1 = 1e^{C + 1} \\
      \Rightarrow&  C + 1 = 0 \\
      \Rightarrow & C = -1 \\
      \Rightarrow & t = xe^{-x + 1}
     \end{split}
    \end{equation*}


}


\section{Exacte DVG}
\example{
    Bepaal alle functie f(y) zodanig dat de 
    DVG $$2y\; dx + (x - 4y\sqrt{y}) \; dy = 0$$
    na vermenigvuldiging met 
    f(y) exact wordt.Bepaal daarna haar AO.
}{
    Is deze DVG exact?
    \begin{equation*}
     \begin{split}
      \frac{\partial}{\partial y}2y & = 2 \\
      \frac{\partial}{\partial x}(x - 4y\sqrt{y}) & = 1
     \end{split}
    \end{equation*}
    Deze DVG is dus niet exact. We moeten een functie f(y) bepalen zodat deze DVG wel exact wordt.
    $$2yf(y)\; dy + (x - 4y\sqrt{y})f(y)\; dy = 0$$
   \begin{equation*}
    \begin{split}
     & \partialof{y}2yf(y)  = \partialof{x}(x - 4y^{3/2}) f(y) \\
     \Rightarrow & 2f(y) + 2y\derivativeof{y}f(y)  = f(y) \\
     \Rightarrow &  2y \derivativeof{y}f(y)  = -f(y) \\
     \Rightarrow & \frac{d}{f(y)}f(y) = - \frac{dy}{2y} \\
     \Rightarrow & \int \frac{d}{f(y)}f(y) = - \int  \frac{dy}{2y} \\
     \Rightarrow & \ln|f(y)| = -\frac{1}{2}\ln|y| + \ln|C| \\
     \Rightarrow & \ln|f(y)| = -\frac{1}{2}\ln|Cy| \\
     \Rightarrow & \ln|f(y)| = \ln|Cy|^{-1/2} \\ 
     \Rightarrow & f(y) = \frac{1}{\sqrt{Cy}} \\
     \Rightarrow & f(y) = \frac{1}{\sqrt{y}} \qquad \hbox{met C = 1}
    \end{split}
   \end{equation*}
   De DVG wordt:
   $$2\sqrt{y}\; dx  + \bigg(\frac{x}{\sqrt{y}} - 4y\bigg)\; dy = 0$$
   wat een een exacte DVG oplevert. Nu bepalen we de AO.
   \begin{equation*}
    \begin{split}
     & 2\sqrt{y}\; dx  + \bigg(\frac{x}{\sqrt{y}} - 4y\bigg)\; dy = 0 \\
     &   \hbox{komt overeen met} \\
     & \partialof{x}F\; dx + \partialof{y}F\;dy = 0
    \end{split}
   \end{equation*}
    We krijgen volgend stelsel:
    $$\begin{cases}
       \partialof{x}F = 2\sqrt{y} (*) \\
       \partialof{y}F = \frac{x}{\sqrt{y}} - 4y (**)
      \end{cases}
    $$
    \begin{equation*}
     \begin{split}
      (*) & \partialof{x}F = 2\sqrt{y}  \\
      \Rightarrow & F = \int 2\sqrt{y}\; dx  \\
      \Rightarrow & F = 2\sqrt{y}x + h(y); \\
      (**) & \partialof{y} = \frac{x}{\sqrt{y}} - 4y = 2x\frac{1}{2\sqrt{y}} + \derivativeof{y}h(y) \\
      \Rightarrow & \derivativeof{y}h(y) = -4y \\
      \Rightarrow & h(y) = \int -4y \; dy \\
      \Rightarrow & h(y) = -2y^2
     \end{split}
    \end{equation*}
    De AO:
    $$F(x, y) = 2\sqrt{y}x - 2y^2$$
}
\example{
    In een vat bevindt zich $20 m^3 $
zout-oplossing waarin 1 kg zout opgelost is. Men voert een nieuwe pekeloplossing toe met 
constante concentratie van 0,5 kg zout/$m^3$
en aan een snelheid 
van $2m^3$/min. De oplossing wordt continu gemengd en loopt onderaan weg met een snelheid van $1m^3$/min. 
Hoeveel zout bevindt zich in de pekeloplossing na 1 uur?
}{
    Definitie van de variabelen:
    \begin{itemize}
     \item $x : \#$ kg zout na $t$ minuten
     \item Op $t = 0$ is $x(0) = 1$
     \item $C_i = \frac{1}{2}kg/m^3$ (Concentratie in)
     \item $v_i = 2m^3/min$          (Snelheid in) 
     \item $C_{uit} = \frac{x(t)}{v(t)}$ (Concentratie uit)
     \item $v_{uit} = 1m^3/min$     (Snelheid uit)
    \end{itemize}
    We zoeken een uitdrukking voor $dx$.
    \begin{itemize}
     \item $dx = $ verandering $x$ gedurende $dt$ minuten
     \item $dx = $ hoeveelheid zout binnen gedurende $dt$ minuten - hoeveelheid zout buiten gedurende $dt$ minuten
    \end{itemize}
    Berekening AO:
    \begin{equation*}
     \begin{split}
      & dx = C_iv_i \; dt - C_{uit}v_{uit} \; dt \\
      \Rightarrow & dx = \frac{1}{2}\cdot 2\; dt - \frac{x(t)}{V(t)}\cdot 1\; dt \qquad \hbox{met } V(t) = 20 + 2t - t = 20 + t\\
      \Rightarrow & dx = dt - \frac{x}{20 + t} \; dt \\
      \Rightarrow & dx + \bigg(\frac{x}{20 + t} - 1\bigg)\;dt = 0 \\
      \Rightarrow & (20 + t)\;dx + (x - 20 - t)\;dt = 0 \\
       & \partialof{t}(20 + t) = 1 = \partialof{x}(x - 20 - t) \Rightarrow \hbox{exact} \\
       & \begin{cases}
        \partialof{x}F = 20 + t (*) \\
        \partialof{t}F = x - 20 -t (**)
       \end{cases} \\
       (*) & \partialof{x}F = 20 + t \\
        \Rightarrow & F = \int( x- 20 - t)\; dt \\
                    & = xt - 20t - \frac{t^2}{2} + h(x) \\
        \Rightarrow & 20 + t = \partialof{x}(xt - 20t - \frac{t^2}{2} + h(x)) \\
        \Rightarrow & 20 +t = t + \partialof{x}h(x) \\
        \Rightarrow & \derivativeof{x}h(x) = 20 \\
        \Rightarrow & h(x) = \int 20\;dx = 20x \\
        \Rightarrow & F = xt - 20t - \frac{t^2}{2} + 20x \\
        & \hbox{AO: } xt - 20t - \frac{t^2}{2} + 20x = C
     \end{split}
    \end{equation*}
    Bereken PO. Indien $x(0) = 1$ dan $C = 20$. 1 uur = 60 minuten $\Rightarrow$ x(60)
    \begin{equation*}
     \begin{split}
      & xt + 20x = 20t + \frac{t^2}{2} + 20 \\
      \Rightarrow & x = \frac{20t + \frac{t^2}{2} + 20}{20 + t} \\
      \Rightarrow & x(60) = 37.75\; kg
     \end{split}
    \end{equation*}

    
}


\section{Lineaire DVG van orde 1}
Algemene definitie:
$$\hbox{Een DVG is lineair in y en y' indien } y' + P(x)y = Q(x)$$
\example{ 
    $$dy + (y\sin x - \cos x)\; dx = 0$$
}{
    \begin{equation*}
     \begin{split}
      & dy + (y\sin x - \cos x)\; dx = 0 \\
      \Rightarrow & \frac{dy}{dx} + y\sin x - \cos x = 0 \\
      \Rightarrow & y' + y\sin x = \cos x
     \end{split}
    \end{equation*}
    Lineair in y en y'
}
\example{ 
    $$ds + (1 - 2t)s\;dt = t^2\;dt$$
}{
    \begin{equation*}
     \begin{split}
      & ds + (1 - 2t)s\;dt = t^2\;dt \\
      \Rightarrow & \frac{ds}{dt} + (1 - 2t)s = t^2
     \end{split}
    \end{equation*}
    Lineair in s en s'
}
\subsection{Oplossingsmethode}
\begin{equation*}
 \begin{split}
  & y' + P(x)y = Q(x) \\
  & \hbox{substitutie y = uv} \qquad \hbox{(vrijheidsgraad toevoegen)}\\
  \Rightarrow & u'v + uv' + P(x)uv = Q(x) \\
  \Rightarrow & u(P(x) + v') + u'v = Q(x) \qquad (*) \\
  & \hbox{stel P(x) + v' = 0 (vrijheidsgraad wegnemen)} \\
  \hbox{Bijgevolg: } \Rightarrow & \frac{dv}{dx} = - P(x)v \\
    \Rightarrow & \int \frac{dv}{v} = - \int P(x)\; dx \\
    \Rightarrow & \ln|v| = - \int P(x)\; dx \\
    \Rightarrow & v  = e^{-\int P(x)\; dx} \\
    (*)\Rightarrow & \frac{du}{dx} = \frac{Q(x)}{v} \\
    \Rightarrow & du = e^{\int P(x)\;dx}Q(x) \; dx \\
    \Rightarrow & u = \int e^{\int P(x)\;dx}Q(x) \; dx \\
    & \hbox{Vervang substitie om AO te bekomen}
 \end{split}
\end{equation*}

\example{
    Bepaal de AO van
    $$(4r^2s - 6)\; dr + r^3 \; ds = 0$$
}{
    \begin{equation*}
     \begin{split}
      & r^3\frac{ds}{dr} + 4r^2s - 6 = 0 \\
      \Rightarrow & \frac{ds}{dr} + \frac{4}{r}s = \frac{6}{r^3} \\
      \Rightarrow & s' + P(r)s = Q(r) \\
      & \hbox{substitutie s = uv} \\
      \Rightarrow & u'v + uv' + \frac{4}{r}uv = \frac{6}{r^3} \\
      \Rightarrow & u\bigg(v' + \frac{4}{r}v\bigg) + u'v = \frac{6}{r^3} \\
       & \frac{dv}{dr} = -\frac{4}{r}v \\
       & \int \frac{dv}{v} = -4\int \frac{dr}{r} \\
       & \ln |v| = -4 \ln |r| \\
       & v = r^{-4} \\
       \Rightarrow & \frac{du}{dr}\cdot\frac{1}{r^4} = \frac{6}{r^3} \\
       \Rightarrow & \frac{du}{dr}\cdot\frac{1}{r} = 6 \\
       \Rightarrow & \int du = \int 6r\; dr \\
       \Rightarrow &  u = 3r^2 + C \\
        & s = uv = (3r^2 + C)\frac{1}{r^4} \\
        & s = \frac{3}{r^2} + \frac{C}{r^4} \qquad \forall C \in \mathbb{R}
     \end{split}
    \end{equation*}

}

\section{DVG van type Bernouilli}
Een DVG is van type Bernouilli indien
$$y' + P(x)y = Q(x)y^n \qquad \hbox{met } n \in \mathbb{R}$$
\subsection{Oplossingsmethode}
Bewijs:
$$\frac{y'}{y^n} + P(x)\frac{y}{y^n} = Q(x)$$
Substitutie: $$z = \frac{y}{y^n} = y^{1 - n}$$
Waaruit volgt: $$z' = \frac{dz}{dx} = \frac{dz}{dy}\frac{dy}{dz}$$
\begin{equation*}
 \begin{split}
  z' & = (1 - n)y^{1 - n - 1}y' \\
     & = (1 - n)y^{-n}y' \\
     & = \frac{(1 - n)y'}{y^n}
 \end{split}
\end{equation*}
De DVG wordt:
$$\frac{z'}{1 - n} + P(x)z = Q(x)$$
Of beter geschreven:
$$z' + (1 -n)P(x)z = (1 - n)Q(x)$$
De DVG is lineair in z en z'

\example{
    Bepaal de AO vanaf $$xy\;dx = (x^2 - y^4) \;dy$$
}{
    \begin{equation*}
     \begin{split}
      & xy \; dx + (y^4 - x^2)\;dy = 0 \\
      \Rightarrow & xy \frac{dx}{dy} + (y^4 - x^2) = 0 \\
      \Rightarrow & \frac{dx}{dy} + \frac{y^4 - x^2}{xy} = 0 \\
      \Rightarrow & \frac{dx}{dy} - \frac{x}{y} + \frac{y^3}{x} = 0 \\
      \Rightarrow & \frac{dx}{dy} - \frac{1}{y}x = -y^3\frac{1}{x} 
     \end{split}
    \end{equation*}
    Bernouilli in x en x'
    $$      \Rightarrow  x\frac{dx}{dy} - \frac{1}{y}x^2 = -y^3$$
    stel $z = x^2$ dus $z' = 2x\frac{dx}{dy}$ 
    \begin{equation*}
     \begin{split}
      & \frac{1}{2}\frac{dz}{dy} - \frac{1}{y}z = -y^3 \\
      \Rightarrow & \frac{dz}{dy} - \frac{2z}{y} = -2y^3 \\
      & \hbox{substitutie } z = uv \\
      \Rightarrow & u'v + uv' - \frac{2uv}{y} = -2y^3 \\
      \Rightarrow & u(v' - \frac{2v}{y}) + u'v = -2y^3 \\ 
      & \frac{dv}{dy} = \frac{2v}{y} \\
      & \int \frac{dv}{v} = 2 \int \frac {dy}{y} \\
      & \ln |v| = 2\ln |y| \\
      & v = y^2  \\
      \Rightarrow&  \frac{du}{dy}y^2 = -2y^3 \\
      \Rightarrow & \int \frac{du}{dy} = - \int 2y \; dy \\
      \Rightarrow & u = -y^2 + C
     \end{split}
    \end{equation*}
    $z = x^2$ en $z = uv = y^2(C - y^2)$
    De AO wordt:
    $$x^2 + y^4 = Cy^2$$


}


\section{Orthogonale krommenbundel}
Definitie: elke kromme uit de ene bundel snijdt elke kromme uit de andere bundel loodrecht.
$$
    \begin{cases}
     f(x, y, C) = 0 \\
     f_{\bot}(x, y, C) = 0
    \end{cases}
$$
Raaklijn van $f$ staat loodrecht op raaklijn van $f_{\bot}$. Wiskundig wordt dit vertaald door: $\omega_{RL_{\bot}} = -\frac{1}{\omega_{RL}} = -\frac{1}{y'}$
De DVG van de orthogonale krommenbundel is
$$F_{\bot}\bigg(x, y, -\frac{1}{y'}\bigg)$$
\example{
    Bepaal de DVG van de orthogonale krommenbundel van alle raaklijnen aan $y = x^2$.
}{
    Elk punt op parabool kan beschreven worden als $p(a, a^2)$.
    \begin{enumerate}
     \item De vergelijking van de originele krommenbundel
        \begin{equation*}
         \begin{split}
          & y - a^2 = 2a(x - a) \\
          \Rightarrow & y - a^2 = 2ax - 2a^2 \\
          \Rightarrow & y = 2ax - a^2
         \end{split}
        \end{equation*}
     \item DVG van de originele krommenbundel
        $$
            \begin{cases}
             y = 2ax - a^2 \\
             y' = 2a
            \end{cases} \Rightarrow
            y = y'x - \frac{y'^2}{4}
        $$
     \item DVG van de orthogonale krommenbundel
        $y' wordt -\frac{1}{y'}$
        $$ y = \frac{1}{y'}x - \frac{1}{4}\frac{1}{y'^2}$$
        Uiteindelijk:
        $$4y'^2y = -4xy' - 1$$
    \end{enumerate}
    Deze DVG heeft graad 2, wat niet in deze cursus besproken wordt. Het is dus onoplosbaar.

}
\example{
    Bepaal de orthogonale krommenbundel van alle parabolen met top in de oorsprong en symmetrieas de X-as.
}{
    \begin{enumerate}
     \item De vergelijking van de originele krommenbundel
            $$ x = Cy^2$$
     \item DVG van de originele krommenbundel. Er is 1 onafhankelijke constanten dus 1 keer afleiden
            $$
              \begin{cases}
               x = Cy^2 \\
               1 = 2Cyy'
              \end{cases}
            $$
            Hieruit volgt $C = \frac{x}{y^2}$ en dus $1 = \frac{2xy'}{x}$
     \item DVG van de orthogonale krommenbundel
            
            $y'$ vervangen door $-\frac{1}{y'}$ dus
            $$1 = -\frac{2x}{yy'} \Leftrightarrow yy' = -2x$$
     \item DVG oplossen
     \begin{equation*}
      \begin{split}
                    & y \frac{dy}
                             {dx} = -2x \\
                    & \int y \; dy = - \int 2x\; dx \\
                    & \frac{y^2}{2} = -x^2 \\
                    & \frac{y^2}{2} + x^2 = C
      \end{split}
     \end{equation*}
     Dit zijn dus ellipsen

    \end{enumerate}


}
\section{DVG van hogere orde}
\example{
    Los op
    $$y''' = e^{-2x}$$
}{
    \begin{equation*}
     \begin{split}
       y''' & = e^{-2x} \\
       y''  & = \int e^{-2x}\;dx = -\frac{1}{2}e^{-2x} + C_1 \\
       y'   & = -\int \frac{1}{2}e^{-2x} + C_1 \; dx \frac{1}{4}e^{-2x} + C_1x + C_2 \\
       y    & = -\frac{1}{8}e^{-2x} + \frac{C_1x^2}{2} + C_2x + C_3 \\
            & = -\frac{1}{8}e^{-2x} + C_1x^2 + C_2x + C_3
     \end{split}
    \end{equation*}
}
\subsection{DVG van orde 2 van type F(x, y', y'') = 0}
Bewijs oplossingsmethode:

Stel $y' = p$, dan wordt $y'' = \frac{dp}{dx}$. De differentiaalvergelijking wordt $F(x, p, \frac{dp}{dx} = 0$. Dit is een dvg van orde 1 in p en x. 

\example{
    Bepaal de AO van $xy'' = y' - x$
}{
    $xy'' = y' - x $ komt overeen met $F(x , y', y'')$
    
    Stel $y' = p' \rightarrow y'' = x\frac{dp}{dx} = p - x$ waaruit volgt dat $x\;dp = (p - x)\;dx$. Dit is homogeen ($\lambda^{(1)}$ dus we stellen $p = ux$
    \begin{equation*}
     \begin{split}
      x(u\;dx + x\;du) & = (ux - x)\;dx \\
      u\;dx + x\;du    & = (u - 1) \;dx \\
      x\;du            & = -dx; \\
      du               & = -\frac{dx}{x} \\
      \int du          & = -\int\frac{dx}{x} \\
      u                & = \ln|x| + C_1 \\
      \frac{dy}{dx}    & = -x\ln|x| + C_1x \\
      \int dy          & = -\int x\ln|x| + C_1x \;dx
     \end{split}
    \end{equation*}
    Het Antwoord is:
    $$y = \frac{-x^2}{2}\ln|x| + \frac{x^4}{4} - C_2 + \frac{C_1}{2}x^2 = \frac{-x^2}{2}\ln|x| + \frac{x^4}{4} - C_2 + C_1x^2$$
}
\subsection{DVG van orde 2 van type F(y, y', y'') = 0}
Bewijs oplossingsmethode:

Stel $y' = p$, dan wordt $y'' = \frac{dp}{dx} = \frac{dp/dy}{dx/dy} = \frac{dp}{dy}\frac{dy}{dx} = p\frac{dp}{dy}$. De differentiaalvergelijking wordt $F(y, p, p\frac{dp}{dx} = 0$. Dit is een dvg van orde 1 in p en y.

\example{
    Bepaal de PO van $(1 - y)^2y'' - y'^3 = 0$ met $y(0) = 2$ en $y'(0) = 1$
}{
    Stel $y' = p \rightarrow y'' = p\frac{dp}{dy}$
    \begin{equation*}
     \begin{split}
      (1 - y)^2p\frac{dp}{dy} - p^3 & = 0 \\
      (1 - y)^2\frac{dp}{dy} - p^2  & = 0 \\
      (1 - y)^2\frac{dp}{dy}        & = p^2 \\
      \frac{dp}{p^2}                & = \frac{dy}{(1 - y)^2} \\
      \int \frac{dp}{p^2}           & = \int\frac{dy}{(1 - y)^2} \\
      -\frac{1}{p}                  & = -\frac{1}{1 - y} + C_1 \\
      C_1 & = 0 \quad \hbox{aangezien p(0) = 1 en y(0) = 2} \\
      \frac{1}{p} = \frac{dx}{dy} & = \frac{1}{y - 1} \\
      & \begin{cases}
       dx = \frac{dy}{y - 1} \\
       x = \ln|y - 1| + C_2 \\
       0 = \ln|1| + C_2 \rightarrow C_2 = 0
      \end{cases} \\
      x & = \ln|y - 1| \\
      e^x & = y - 1 \\
      y & = e^x + 1 
     \end{split}
    \end{equation*}

}
\section{Stellingen voor lineaire differentiaalvergelijkingen}
Voor geen enkele stelling is het bewijs te kennen
\subsection{Stelling 1}
Is $L(D)y = 0$ een lineaire homogene DVG van $n^{de}$ orde en $y_i(x), i = 1, ..., n$ n onafhankelijke PO's van $L(D)y = 0$ dan is $y = C_1y_1(x) + C_2y_2(x) + ... + C_ny_n(x)$ de AO van $L(D)y = 0$

\subsection{Stelling 2}
Indien 
\begin{tabular}{| c c c c |}
 $y_1$          & $y_2$         & ... & $y_m$ \\
 $y'_1$         & $y'_2$        & ... & $y'_m$ \\
 ...            & ...           & ... & ...   \\
 $y^{n - 1}_1$  & $y^{n - 1}_2$ & ... & $y^{n - 1}_m$ 
\end{tabular} = 0, dan zijn de PO's van $L(D)y = 0$ lineair onafhankelijk.
\subsection{Stelling 3}
Indien $L(D)y = 0$ een lineaire DVG van $n^{de}$ orde, $y_1(x)$ een PO van $L(D)y = Q(x)$ en $y_2(x)$ de AO van $L(D)y = 0$ dan is $y(x) = y_1(x) + y_2(x)$ de AO van $L(D)y = Q(x)$
