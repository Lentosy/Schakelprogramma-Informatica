\part{Wiskunde B}
\chapter{Differentiaalvergelijking}
\section{Definities}
De algemene definitie is:
$$F(x, y, y', y'', ..., y^{(n)}) = 0$$
waarbij: \begin{itemize}
\item \textbf{x} een veranderlijke is.
\item \textbf{y} een functie van x is.
\item er minstens één afgeleide van y is.
\end{itemize}
\example{Differentiaalvergelijking}{
    $$ x + y + y' = 0$$
}

Een differentiaalvergelijking heeft een \textbf{orde} en een \textbf{graad}
\begin{itemize}
    \item \textbf{Orde}: Dit is de orde van de hoogste afgeleide dat voorkomt, dus \textit{n}.
    \item \textbf{Graad}: De graad \textit{r} bestaat niet altijd maar is wel altijd een strik positief geheel getal. De graad is de macht die behoort tot de afgeleide met de grootste orde. $y^{(n)^{r}}$
\end{itemize}
\example{Orde en graad}{
    \begin{center}
        \begin{tabular}{l | l | l}
            Differentiaalvergelijking                          & Orde & Graad \\
            \hline
            $y\ - 2y'^3 = yx$                                  & 2    & 1     \\
            $1 + (y'')^4 + 2y' + x(y''')^2 = sin(x)$           & 3    & 2     \\
            $(x - 1)(y'') - xy' + y = 0$                       & 2    & 1     \\
            $e^s\frac{d^3s}{dt^3} + (\frac{ds^2}{dt^2})^3 = 0$ & 3    & 1     \\
            $xy' + e^{y'} + y'' = 1$                           & 1    & /     \\
            \hline
            $\sin\sqrt {y'} = x + 2$                           & 1    & /     \\
            $\;\;\rightarrow y' = \arcsin^2(x+2)$              & 1    & 1     \\
            \hline
            $\sin y' = xy'^2$                                  & 1    & /     \\
            $\;\;\rightarrow y' = \arcsin(xy'^2)$              & 1    & /     \\
            \hline
            $y^{'3} + \frac{x}{y''} + y'' = 1$                 & 2    & ?     \\
            $\;\;\rightarrow y^{'3}y'' + x + (y'')^2 = 1$      & 2    & 2     
            
        \end{tabular}
    \end{center}
}
\section{Soorten oplossingen}
Tijdens het oplossen van een differentiaalvergelijking van de \textit{n}-de orde worden drie oplossingen onderscheden:
\begin{enumerate}
\item De \textbf{Algemene oplossing (AO)}: Verzameling van functies zodat de differentiaalvergelijking klopt. De algemene oplossing bevat \textit{n }onafhankelijke constanten. Deze constanten zijn getallen en geen functies.
\item De \textbf{Particuliere oplossing (PO)}: Dit is één van de krommen van de AO en is afhankelijk van de beginvoorwaarden van het probleem.
\item De \textbf{Singuliere oplossing (SO)}: Een oplossing die niet voldoet aan de AO maar wel een oplossing is voor de DVG.
\end{enumerate}
\example{Onafhankelijke variabelen:}
{
\begin{center}
    \begin{tabular}{l | l | l}
    AO & Onafh. C & Orde DVG \\
    \hline
    $y = C_1 + C_2x$ & 2 & 2 \\
    $y = C_1  - C_1^2x$ & 1 & 1 \\
    \hline
    $y = C_1(C_2 + C_3e^x)$ & ? & ? \\
    $\;\;\rightarrow C_1C_2 + C_1C_3e^x$ & ? & ? \\
    $\;\;\rightarrow a + be^x$ & 2 & 2 \\
    \hline
    $y = C_1 + \ln(C_2 x)$ & ? & ? \\
    $\;\;\rightarrow y = C_1 + \ln(C_2) + \ln(x)$ & ? & ? \\
    $\;\;\rightarrow y = a + \ln(x)$ & 1 & 1


    \end{tabular}
\end{center}
}
\example{Oef 1 AO en PO}{Gegeven een differentiaalvergelijking: $y'' + y = 0$
\begin{enumerate}
\item Toon aan dat $y = a\sin(x) + b\cos(x)$ de AO is.
\item Geef enkele PO's.
\end{enumerate}
Oplossing:
\begin{enumerate}
\item 
\begin{equation*}
\begin{split}
    y & = a\sin(x) + b\cos(x) \\
    y' & = a\cos(x) - b\sin(x) \\
    y'' & = -a\sin(x) - b\cos(x) 
\end{split}
\end{equation*}
Hieruit volgt:
\begin{gather*}
    y'' + y  = 0 \\
    -a\sin(x) - b\cos(x) + \sin(x) + b\cos(x)  = 0  \\
    \rightarrow \hbox{Het is een oplossing}
\end{gather*}
De differentiaalvergelijking heeft orde 2. De y-vergelijking bevat 2 onafhankelijke constanten en de y-vergelijking is een oplossing. Hierdoor is y de AO van de differentiaalvergelijking.
\item Enkele PO's:
\begin{equation*}
\begin{split}
y & = 0\\
y & = \sqrt{2}\sin(x) \\
y & = \sin(x) + \cos(x)
\end{split}
\end{equation*}
\end{enumerate}
}

\example{Oef 2 AO en PO}
{Gegeven een differentiaalvergelijking: $y'^2 - yy'+e^x$
\begin{enumerate}
\item Geef de orde en graad.
\item Is $y = \frac{1}{C} + Ce^x$ de AO?
\item Wat  voor soort oplossing is $y = 2\sqrt{e^x}$
\end{enumerate}
Oplossing:
\begin{enumerate}
\item 
De orde is 1 en de graad is 2.

\item 
$$y' = Ce^x$$
\begin{equation*}
\begin{split}
\rightarrow C^2(e^x)^2 - (\frac{1}{C} + Ce^x)Ce^x + e^x &  = 0 \\
\Leftrightarrow C^2e^{2x} - e^x - C^2e^{2x} + e^x &  = 0 \\
\Leftrightarrow C^2e^{2x} - e^x - C^2e^{2x} + e^x & = 0 \\
\Leftrightarrow 0 & =0
\end{split}
\end{equation*}
$$\rightarrow \hbox{Het is een oplossing}$$
Orde DVG = 1 = Onafhankelijke constanten van y

\item 
$$ y'  = 2 \cdot \frac{1}{2\sqrt{e^x}} \cdot e^x = \sqrt{e^x}$$
\begin{equation*}
\begin{split}
\rightarrow & y'^2 - yy'+e^x \\
\Leftrightarrow &  (\sqrt{e^x})^2 - 2\sqrt{e^x}\cdot\sqrt{e^x} + e^x  = 0 \\
\Leftrightarrow & e^x - 2e^x + e^x  = 0 \\
\Leftrightarrow & 0 = 0
\end{split}
\end{equation*}
Dit is een singuliere oplossing aangezien y niet overeenkomt met de AO, maar wel voldoet aan de DVG.

\end{enumerate}
}
\section{Bepalen van een DVG}
Indien een AO gegeven is met \textit{n} onafhankelijke constanten:
\begin{enumerate}
\item Controleer of de constanten werkelijk onafhankelijk zijn.
\item Leid de AO \textit{n} maal af.
\item Elimineer de \textit{n} constanten van de \textit{n + 1} bekomen vergelijkingen. De laatste vergelijking moet zeker gebruikt worden.
\item Controleer of de DVG van orde \textit{n} is.
\end{enumerate}

\example{Oef 1 bepalen van een DVG}
{
De algemene oplossing is $$y = C_1 + C_2x$$
\begin{enumerate}
\item Er zijn \textit{2} onafhankelijke constanten.
\item Er moet \textit{2} keer afgeleid worden:
$$
    \begin{cases}
    y    & = C_1 + C_2x \\
    y'   & = C_2 \\
    y''  & = 0 \\
    \end{cases}
$$
\item De constanten zijn al geëlimineerd. 
\item De DVG is $y'' = 0$ en heeft orde \textit{2}.

\end{enumerate}
}

\example{Oef 2 bepalen van een DVG}
{
Bepaal de DVG van: $$y = C_1 + C_2e^{-x} + C_3e^{3x}$$
\begin{enumerate}
\item Er zijn \textit{3} onafhankelijke constanten.
\item Er moet \textit{3} maal afgeleid worden.
    \[ 
    \begin{cases}
            y & = C_1 + C_2e^{-x} + C_3e^{3x} \\
    y'     & = -C_2e^{-x} + 3C_3e^{3x}     \\
    y'' & = C_2e^{-x} + 9C_3e^{3x}      \\
    y''' & = -C_2e^{-x} + 27C_3e^{3x}
    \end{cases}
    \]

\item 
Tel de 1ste afgeleide op met de 2de afgeleide en tel de 2de afgeleide op met de 3rde afgeleide
\[
    \begin{cases}
    y + y''    & = 3C_3e^{3x} + 9C_3e^{3x} = 12C_3e^{3x}  \\
    y'' + y''' & = 9C_3e^{3x} + 27C_3e^{3x} = 36C_3e^{3x}
    \end{cases}
\]
Vermenigvuldig de 1ste vergelijking met 3 en trek hiervan de 2de vergelijking af.

$$3(y + y'') - y'' - y''' = 3(12C_3e^{3x}) - 36C_3e^{3x} = 0$$
$$\rightarrow y''' - 2y'' - 3y' = 0$$
\item
De orde van deze DVG is \textit{3}

\end{enumerate}
}

\example{Oef 3 bepalen van een DVG}
{
Bepaal de DVG van alle cirkels met middelpunt y = -x.
\begin{enumerate}
\item Eerst moet de AO gevonden worden. Het middelpunt van elke cirkel kan gegeven worden met $m(a, -a).$
    Hieruit volgt de algemene vergelijking van een cirkel: $$(x - a)^2 + (y + a)^2 = R^2$$
    Er zijn \textit{2} onafhankelijke constanten (a en R).
\item Er moet \textit{2} maal (impliciet) afgeleid worden.
\[
    \begin{cases}
    (x - a)^2 + (y + a)^2 = R^2 \\
    \frac{dy}{dx} : (x-a) + y'(y+a) = 0 \\
    \frac{d^2y}{dx^2} : 1 + y''(y + a) + y'^2 = 0
    \end{cases}
\]
\item
    Vorm $\frac{dy}{dx}$ om naar $a$:
    $$a = \frac{-x - yy'}{y' - 1}$$
    Substitueer deze $a$ in $\frac{d^2y}{dx^2}$:
    $$1 + y''(y + (\frac{-x - yy'}{y' - 1})) + y'^2 = 0$$
    $$\rightarrow 1 + y''(y + (\frac{x + yy'}{-y' + 1})) + y'^2 = 0$$
    $$\rightarrow y''(x + y) - y'^3 + y'^2 - y' + 1 = 0$$
\item Orde van de DVG = \textit{2}  = Aantal onafhankelijke constanten.
\end{enumerate}
}
\section{Oplossen van een lineaire DVG van orde n met constante reële coëfficiënten}
\begin{equation*}
 \begin{split}
		  & y''' - y''\sin t + ty = t^2 \\
  \Leftrightarrow & D^3y - D^2y\sin t + ty = t^2 \\
  \Leftrightarrow & (D^3 - D^2\sin t + t)y = t^2 \\
  \Leftrightarrow & L(d)y = g(t) \\
  \Leftrightarrow & \hbox{met} L(d) = \sum_{i = 0}^{n} a_iD^i \qquad ,a_i \in \mathbb{R}
 \end{split}
\end{equation*}
Een lineaire DVG is een DVG waarbij alle coëfficiënten van alle afgeleiden enkel voorkomen als eerste macht.
\subsection{Particuliere oplossing}
De particuliere oplossing kan slechts bepaald worden indien alle beginvoorwaarden 

($y(0), y'(0),...,y^{(n-1)}(0)$) gekend zijn.
\subsection{Algemene oplossing}
Indien de beginvoorwaarden niet gekend zijn moeten $y(0),y'(0)...y^{(n-1)}(0)$ respectievelijk gelijkgesteld worden aan $C_1, C_2, ..., C_n$

\example{
  Bepaal de PO van $y'' + y = g(t)$ indien $y(0) = 0$, $y'(0) = 1$ en 
  $$
    g(t) = \begin{cases}
	      0 & t < 1 \\
	      e^{-t} & t > 1
	   \end{cases}
  $$
}{
  \begin{equation*}
   \begin{split}
                   & L(d)y = g(t) \\
   \Leftrightarrow & (D^2 + 1)y = g(t) \\
   \Leftrightarrow & (D^2 + 1)y = e^{-t}H(t - 1) \\
   \mathcal{L}\{LL\}(s) & = \mathcal{L}\{y'' + y\}(s) \\
                        & = s^2Y - sy(0^{+}) + y'(0^{+}) + Y \\
                        & = s^2Y - 1 + Y \\
   \mathcal{L}\{RL\}(s) & = \mathcal{L}\{e^{-t}H(t - 1\}(s) \\
                        & = \mathcal{L}\{e^{-(t - 1) - 1}H(t - 1)\}(s) \\
                        & = e^{-1}\mathcal{L}\{e^{-(t - 1)}H(t - 1)\}(s) \\
                        & = e^{-1}e^{-s}\mathcal{L}\{e^{-t}\}(s) \\
                        & = e^{-1}e^{-s}\frac{1}{s + 1} \\
                        & = \frac{e^{-(s + 1)}}{s + 1} \\
   \hbox{dus} \\
   \Leftrightarrow & s^2Y - 1 + Y = \frac{e^{-(s + 1)}}{s + 1} \\
   \Leftrightarrow & Y(s^2 + 1) = 1 + \frac{e^{-(s + 1)}}{s+1} \\
   \Leftrightarrow & Y = \frac{1}{s^2 + 1} + \frac{e^{-(s+1)}}{(s+1)(s^2+1)}     \\
   \Leftrightarrow & \mathcal{L}^{-1}\{Y\}(t) = \mathcal{L}^{-1}\bigg\{\frac{1}{s^2 + 1} + \frac{e^{-(s+1)}}{(s+1)(s^2+1)}\bigg\}(t) \\
   \Leftrightarrow & y(t) = \sin t +e^{-1}f(t - 1)H(t - 1) \\
	   \hbox{met}\; f(t) & = \mathcal{L}^{-1}\bigg\{\frac{1}{(s + 1)(s^2 + 1)}\bigg\}(t) \\
			     & = \frac{1}{2}\mathcal{L}^{-1}\bigg\{\frac{1}{s + 1} - \frac{s - 2}{s^2 + 1}\bigg\}(t) \\
			     & = \frac{1}{2}\bigg[e^{-t} - (\cos t + \sin t)\bigg] \\
	    \hbox{antwoord: }\; y(t)  & = \sin t + \frac{1}{2}\bigg(e^{-t} - e^{-1}\cos (t - 1) + e^{-1} \sin (t - 1) \bigg)H(t - 1)
   \end{split}
  \end{equation*}

}
\example{
  Bepaal de PO van $y'' + y = \delta\big(t - \frac{\pi}{2}\big)$ indien $y(\frac{\pi}{4}) = 0$, $y'(\frac{\pi}{4}) = 0$
}{
  Stel: $y(0^+) = C_1$, $y'(0^+) = C_2$
  \begin{equation*}
   \begin{split}
    \mathcal{L}\{LL\}(s) & = \mathcal{L}\{y'' + y\}(s) \\
			 & = s^2Y - sC_1 - C_2 + Y \\
    \mathcal{L}\{RL\}(s) & = \mathcal{L}\bigg\{\delta\big(t - \frac{\pi}{4}\bigg)\bigg\}(s) \\
                         & = \int_0^{+\infty}\delta\big(t - \frac{\pi}{4}\big)e^{-st} \; dt \\
                         & = e^{-\frac{\pi}{2}s} \\
    \hbox{dus} \\
    \Leftrightarrow & s^2Y - sC_1 - C_2 + Y = e^{-\frac{\pi}{2}s}  \\
    \Leftrightarrow & Y = \frac{e^{-\frac{\pi}{2}s} + sC_1 + C_2}{s^2 + 1} \\
    \Leftrightarrow & \mathcal{L}^{-1}\{Y\}(t) = \mathcal{L}^{-1}\bigg\{\frac{e^{-\frac{\pi}{2}s} + sC_1 + C_2}{s^2 + 1}\bigg\}(t) \\
    \Leftrightarrow & y(t) = C_2\sin t + C_1\cos t + f\big(t - \frac{\pi}{2}\big)H\big(t - \frac{\pi}{2}\big)\\
    \hbox{met}\; f(t) & = \mathcal{L}^{-1}\bigg\{\frac{1}{s^2 + 1}\bigg\}(t) \\
                      & = \sin t\\
    \hbox{De algemene oplossing: }\; y(t)  & =  C_2\sin t + C_1\cos t - \bigg(\cos(t) H\big(t - \frac{\pi}{2}\big)\bigg)
   \end{split}
  \end{equation*}
  De PO voor $t = \frac{\pi}{4} \qquad ( < \frac{\pi}{2}\;\hbox{dus Heaviside is 0})$
  \begin{equation*}
   \begin{split}
    y(t) & = C_2 \sin t + C_1 \cos t  \Rightarrow y(\frac{\pi}{4})  : 0 = C_2\frac{\sqrt{2}}{2} + C_1\frac{\sqrt{2}}{2}\\
    y'(t) & = C_2 \cos t - C_1 \sin t \Rightarrow y'(\frac{\pi}{4}) : 0 = C_2\frac{\sqrt{2}}{2} - C_1\frac{\sqrt{2}}{2}\\
    & \begin{cases}
     0 = C_2 + C_1 \\
     0 = C_2 - C_1
    \end{cases} \Rightarrow C_2 = C_1 = 0
   \end{split}
  \end{equation*}
    Het antwoord:
    $$y(t) = -(\cos t)H(t - \frac{\pi}{2})$$
}
\section{DVG van de orde 1 en graad 1}
\subsection{Gescheiden veranderlijken}
Indien een DVG van orde 1 en graad 1 te schrijven is als
$$f(x)\;dx = g(y)\;dy$$
Algemeen:
\begin{equation*}
 \begin{split}
  M(x, y)\;dx & = -N(x, y)\; dy \\
  f(x)g(y)\;dx & = -h(x)i(y)\; dy \\
  \frac{f(x)}{h(x)}\; dx = & -\frac{i(y)}{g(y)}\;dy \\
  a(x) \; dx & = b(y)\; dy \\
  \int a(x) \; dx & = \int b(y) \; dy \\
  A(x) + C_1 & = B(y) + C_2 \\
  A(x) & = B(y) + C
 \end{split}
\end{equation*}
\example{
    Bepaal de AO van $yt + \sqrt{1 - t^2}y' = 0$
}{
    \begin{equation*}
     \begin{split}
      & yt\;dt + \sqrt{1 - t^2}\;dy = 0 \\
      \Leftrightarrow & \sqrt{1 - t^2}\;dy = -yt\;dt \\
      \Leftrightarrow & \frac{dy}{y}= -\frac{t\;dt}{\sqrt{1 - t^2}} \\
      \Leftrightarrow & \int \frac{dy}{y}= -\int \frac{t\;dt}{\sqrt{1 - t^2}} \\
      \Leftrightarrow & \ln |y| \sqrt{1 - x^2} + C \\
      \Leftrightarrow & y = e^{\sqrt{1 - x^2} + C} \\
      \Leftrightarrow & y = e^{\sqrt{1 - x^2}}e^C \\
      \Leftrightarrow & y = De^{\sqrt{1 - x^2}} \\
     \end{split}
    \end{equation*}

}

\todo{LES DINSDAG 13/03}
\section{Homogene DVG}
Een DVG is homogeen indien:
$$f(\lambda x, \lambda y) = \lambda^n f(x,y)$$
Indien een DVG homogeen is kan volgende oplossingsmethode toegepast worden:

\example{
        Bepaal de PO van : 
        $$\frac{dx}{dt} = \frac{x}{t(\ln t - \ln x)}$$
        waarvoor x(1) = 1.
}{
    Berekening algemene oplossing
    \begin{equation*}
     \begin{split}
      & \frac{dx}{dt} = \frac{x}{t(\ln t - \ln x)} \\
      \Rightarrow & t(\ln t - \ln x) \;dx - x\;dt = 0 \\
      \Rightarrow & t\ln\bigg(\frac{t}{x}\bigg)\;dx - x\;dt = 0 \\
      & \hbox{controle homogeen} \\
      \Rightarrow & \lambda t \ln\bigg(\frac{\lambda t}{\lambda x}\bigg) - \lambda x\\
      \Rightarrow & \lambda^1 (t \ln\bigg(\frac{t}{x}\bigg)  - x) \qquad \hbox{homogeen want M(x,t) en N(x,t) hebben } \lambda \hbox{ tot de eerste macht} \\
      & \hbox{substitutie } t = ux \\
      \Rightarrow & u \ln u \; dx - u \;dx + x\; du = 0 \\
      \Rightarrow & (u \ln u - u) \; dx = x \; du \\
      \Rightarrow & \int \frac{du}{u(\ln n -1)} = \int \frac{dx}{x} \\
      \Rightarrow & \ln | \ln u - 1| = \ln |x| + \ln |C| \\
      \Rightarrow & \ln | \ln u - 1| = \ln |Cx|  \\ 
      \Rightarrow & \ln u - 1 = Cx \\ 
      \Rightarrow & \ln u = Cx + 1 \\
      \Rightarrow & u = e^{Cx + 1} \\
      \Rightarrow & t = xe^{Cx + 1}
     \end{split}
    \end{equation*}
    Berekening particuliere oplossing:
    \begin{equation*}
     \begin{split}
      & x(1) = 1 \\
      \Rightarrow & 1 = 1e^{C + 1} \\
      \Rightarrow&  C + 1 = 0 \\
      \Rightarrow & C = -1 \\
      \Rightarrow & t = xe^{-x + 1}
     \end{split}
    \end{equation*}


}


\section{Exacte DVG}
\example{
    Bepaal alle functie f(y) zodanig dat de 
    DVG $$2y\; dx + (x - 4y\sqrt{y}) \; dy = 0$$
    na vermenigvuldiging met 
    f(y) exact wordt.Bepaal daarna haar AO.
}{
    Is deze DVG exact?
    \begin{equation*}
     \begin{split}
      \frac{\partial}{\partial y}2y & = 2 \\
      \frac{\partial}{\partial x}(x - 4y\sqrt{y}) & = 1
     \end{split}
    \end{equation*}
    Deze DVG is dus niet exact. We moeten een functie f(y) bepalen zodat deze DVG wel exact wordt.
    $$2yf(y)\; dy + (x - 4y\sqrt{y})f(y)\; dy = 0$$
   \begin{equation*}
    \begin{split}
     & \partialof{y}2yf(y)  = \partialof{x}(x - 4y^{3/2}) f(y) \\
     \Rightarrow & 2f(y) + 2y\derivativeof{y}f(y)  = f(y) \\
     \Rightarrow &  2y \derivativeof{y}f(y)  = -f(y) \\
     \Rightarrow & \frac{d}{f(y)}f(y) = - \frac{dy}{2y} \\
     \Rightarrow & \int \frac{d}{f(y)}f(y) = - \int  \frac{dy}{2y} \\
     \Rightarrow & \ln|f(y)| = -\frac{1}{2}\ln|y| + \ln|C| \\
     \Rightarrow & \ln|f(y)| = -\frac{1}{2}\ln|Cy| \\
     \Rightarrow & \ln|f(y)| = \ln|Cy|^{-1/2} \\ 
     \Rightarrow & f(y) = \frac{1}{\sqrt{Cy}} \\
     \Rightarrow & f(y) = \frac{1}{\sqrt{y}} \qquad \hbox{met C = 1}
    \end{split}
   \end{equation*}
   De DVG wordt:
   $$2\sqrt{y}\; dx  + \bigg(\frac{x}{\sqrt{y}} - 4y\bigg)\; dy = 0$$
   wat een een exacte DVG oplevert. Nu bepalen we de AO.
   \begin{equation*}
    \begin{split}
     & 2\sqrt{y}\; dx  + \bigg(\frac{x}{\sqrt{y}} - 4y\bigg)\; dy = 0 \\
     &   \hbox{komt overeen met} \\
     & \partialof{x}F\; dx + \partialof{y}F\;dy = 0
    \end{split}
   \end{equation*}
    We krijgen volgend stelsel:
    $$\begin{cases}
       \partialof{x}F = 2\sqrt{y} (*) \\
       \partialof{y}F = \frac{x}{\sqrt{y}} - 4y (**)
      \end{cases}
    $$
    \begin{equation*}
     \begin{split}
      (*) & \partialof{x}F = 2\sqrt{y}  \\
      \Rightarrow & F = \int 2\sqrt{y}\; dx  \\
      \Rightarrow & F = 2\sqrt{y}x + h(y); \\
      (**) & \partialof{y} = \frac{x}{\sqrt{y}} - 4y = 2x\frac{1}{2\sqrt{y}} + \derivativeof{y}h(y) \\
      \Rightarrow & \derivativeof{y}h(y) = -4y \\
      \Rightarrow & h(y) = \int -4y \; dy \\
      \Rightarrow & h(y) = -2y^2
     \end{split}
    \end{equation*}
    De AO:
    $$F(x, y) = 2\sqrt{y}x - 2y^2$$
}
\example{
    In een vat bevindt zich $20 m^3 $
zout-oplossing waarin 1 kg zout opgelost is. Men voert een nieuwe pekeloplossing toe met 
constante concentratie van 0,5 kg zout/$m^3$
en aan een snelheid 
van $2m^3$/min. De oplossing wordt continu gemengd en loopt onderaan weg met een snelheid van $1m^3$/min. 
Hoeveel zout bevindt zich in de pekeloplossing na 1 uur?
}{
    Definitie van de variabelen:
    \begin{itemize}
     \item $x : \#$ kg zout na $t$ minuten
     \item Op $t = 0$ is $x(0) = 1$
     \item $C_i = \frac{1}{2}kg/m^3$ (Concentratie in)
     \item $v_i = 2m^3/min$          (Snelheid in) 
     \item $C_{uit} = \frac{x(t)}{v(t)}$ (Concentratie uit)
     \item $v_{uit} = 1m^3/min$     (Snelheid uit)
    \end{itemize}
    We zoeken een uitdrukking voor $dx$.
    \begin{itemize}
     \item $dx = $ verandering $x$ gedurende $dt$ minuten
     \item $dx = $ hoeveelheid zout binnen gedurende $dt$ minuten - hoeveelheid zout buiten gedurende $dt$ minuten
    \end{itemize}
    Berekening AO:
    \begin{equation*}
     \begin{split}
      & dx = C_iv_i \; dt - C_{uit}v_{uit} \; dt \\
      \Rightarrow & dx = \frac{1}{2}\cdot 2\; dt - \frac{x(t)}{V(t)}\cdot 1\; dt \qquad \hbox{met } V(t) = 20 + 2t - t = 20 + t\\
      \Rightarrow & dx = dt - \frac{x}{20 + t} \; dt \\
      \Rightarrow & dx + \bigg(\frac{x}{20 + t} - 1\bigg)\;dt = 0 \\
      \Rightarrow & (20 + t)\;dx + (x - 20 - t)\;dt = 0 \\
       & \partialof{t}(20 + t) = 1 = \partialof{x}(x - 20 - t) \Rightarrow \hbox{exact} \\
       & \begin{cases}
        \partialof{x}F = 20 + t (*) \\
        \partialof{t}F = x - 20 -t (**)
       \end{cases} \\
       (*) & \partialof{x}F = 20 + t \\
        \Rightarrow & F = \int( x- 20 - t)\; dt \\
                    & = xt - 20t - \frac{t^2}{2} + h(x) \\
        \Rightarrow & 20 + t = \partialof{x}(xt - 20t - \frac{t^2}{2} + h(x)) \\
        \Rightarrow & 20 +t = t + \partialof{x}h(x) \\
        \Rightarrow & \derivativeof{x}h(x) = 20 \\
        \Rightarrow & h(x) = \int 20\;dx = 20x \\
        \Rightarrow & F = xt - 20t - \frac{t^2}{2} + 20x \\
        & \hbox{AO: } xt - 20t - \frac{t^2}{2} + 20x = C
     \end{split}
    \end{equation*}
    Bereken PO. Indien $x(0) = 1$ dan $C = 20$. 1 uur = 60 minuten $\Rightarrow$ x(60)
    \begin{equation*}
     \begin{split}
      & xt + 20x = 20t + \frac{t^2}{2} + 20 \\
      \Rightarrow & x = \frac{20t + \frac{t^2}{2} + 20}{20 + t} \\
      \Rightarrow & x(60) = 37.75\; kg
     \end{split}
    \end{equation*}

    
}


\section{Lineaire DVG van orde 1}
Algemene definitie:
$$\hbox{Een DVG is lineair in y en y' indien } y' + P(x)y = Q(x)$$
\example{ 
    $$dy + (y\sin x - \cos x)\; dx = 0$$
}{
    \begin{equation*}
     \begin{split}
      & dy + (y\sin x - \cos x)\; dx = 0 \\
      \Rightarrow & \frac{dy}{dx} + y\sin x - \cos x = 0 \\
      \Rightarrow & y' + y\sin x = \cos x
     \end{split}
    \end{equation*}
    Lineair in y en y'
}
\example{ 
    $$ds + (1 - 2t)s\;dt = t^2\;dt$$
}{
    \begin{equation*}
     \begin{split}
      & ds + (1 - 2t)s\;dt = t^2\;dt \\
      \Rightarrow & \frac{ds}{dt} + (1 - 2t)s = t^2
     \end{split}
    \end{equation*}
    Lineair in s en s'
}
\subsection{Oplossingsmethode}
\begin{equation*}
 \begin{split}
  & y' + P(x)y = Q(x) \\
  & \hbox{substitutie y = uv} \qquad \hbox{(vrijheidsgraad toevoegen)}\\
  \Rightarrow & u'v + uv' + P(x)uv = Q(x) \\
  \Rightarrow & u(P(x) + v') + u'v = Q(x) \qquad (*) \\
  & \hbox{stel P(x) + v' = 0 (vrijheidsgraad wegnemen)} \\
  \hbox{Bijgevolg: } \Rightarrow & \frac{dv}{dx} = - P(x)v \\
    \Rightarrow & \int \frac{dv}{v} = - \int P(x)\; dx \\
    \Rightarrow & \ln|v| = - \int P(x)\; dx \\
    \Rightarrow & v  = e^{-\int P(x)\; dx} \\
    (*)\Rightarrow & \frac{du}{dx} = \frac{Q(x)}{v} \\
    \Rightarrow & du = e^{\int P(x)\;dx}Q(x) \; dx \\
    \Rightarrow & u = \int e^{\int P(x)\;dx}Q(x) \; dx \\
    & \hbox{Vervang substitie om AO te bekomen}
 \end{split}
\end{equation*}

\example{
    Bepaal de AO van
    $$(4r^2s - 6)\; dr + r^3 \; ds = 0$$
}{
    \begin{equation*}
     \begin{split}
      & r^3\frac{ds}{dr} + 4r^2s - 6 = 0 \\
      \Rightarrow & \frac{ds}{dr} + \frac{4}{r}s = \frac{6}{r^3} \\
      \Rightarrow & s' + P(r)s = Q(r) \\
      & \hbox{substitutie s = uv} \\
      \Rightarrow & u'v + uv' + \frac{4}{r}uv = \frac{6}{r^3} \\
      \Rightarrow & u\bigg(v' + \frac{4}{r}v\bigg) + u'v = \frac{6}{r^3} \\
       & \frac{dv}{dr} = -\frac{4}{r}v \\
       & \int \frac{dv}{v} = -4\int \frac{dr}{r} \\
       & \ln |v| = -4 \ln |r| \\
       & v = r^{-4} \\
       \Rightarrow & \frac{du}{dr}\cdot\frac{1}{r^4} = \frac{6}{r^3} \\
       \Rightarrow & \frac{du}{dr}\cdot\frac{1}{r} = 6 \\
       \Rightarrow & \int du = \int 6r\; dr \\
       \Rightarrow &  u = 3r^2 + C \\
        & s = uv = (3r^2 + C)\frac{1}{r^4} \\
        & s = \frac{3}{r^2} + \frac{C}{r^4} \qquad \forall C \in \mathbb{R}
     \end{split}
    \end{equation*}

}

\section{DVG van type Bernouilli}
Een DVG is van type Bernouilli indien
$$y' + P(x)y = Q(x)y^n \qquad \hbox{met } n \in \mathbb{R}$$
\subsection{Oplossingsmethode}
Bewijs:
$$\frac{y'}{y^n} + P(x)\frac{y}{y^n} = Q(x)$$
Substitutie: $$z = \frac{y}{y^n} = y^{1 - n}$$
Waaruit volgt: $$z' = \frac{dz}{dx} = \frac{dz}{dy}\frac{dy}{dz}$$
\begin{equation*}
 \begin{split}
  z' & = (1 - n)y^{1 - n - 1}y' \\
     & = (1 - n)y^{-n}y' \\
     & = \frac{(1 - n)y'}{y^n}
 \end{split}
\end{equation*}
De DVG wordt:
$$\frac{z'}{1 - n} + P(x)z = Q(x)$$
Of beter geschreven:
$$z' + (1 -n)P(x)z = (1 - n)Q(x)$$
De DVG is lineair in z en z'

\example{
    Bepaal de AO vanaf $$xy\;dx = (x^2 - y^4) \;dy$$
}{
    \begin{equation*}
     \begin{split}
      & xy \; dx + (y^4 - x^2)\;dy = 0 \\
      \Rightarrow & xy \frac{dx}{dy} + (y^4 - x^2) = 0 \\
      \Rightarrow & \frac{dx}{dy} + \frac{y^4 - x^2}{xy} = 0 \\
      \Rightarrow & \frac{dx}{dy} - \frac{x}{y} + \frac{y^3}{x} = 0 \\
      \Rightarrow & \frac{dx}{dy} - \frac{1}{y}x = -y^3\frac{1}{x} 
     \end{split}
    \end{equation*}
    Bernouilli in x en x'
    $$      \Rightarrow  x\frac{dx}{dy} - \frac{1}{y}x^2 = -y^3$$
    stel $z = x^2$ dus $z' = 2x\frac{dx}{dy}$ 
    \begin{equation*}
     \begin{split}
      & \frac{1}{2}\frac{dz}{dy} - \frac{1}{y}z = -y^3 \\
      \Rightarrow & \frac{dz}{dy} - \frac{2z}{y} = -2y^3 \\
      & \hbox{substitutie } z = uv \\
      \Rightarrow & u'v + uv' - \frac{2uv}{y} = -2y^3 \\
      \Rightarrow & u(v' - \frac{2v}{y}) + u'v = -2y^3 \\ 
      & \frac{dv}{dy} = \frac{2v}{y} \\
      & \int \frac{dv}{v} = 2 \int \frac {dy}{y} \\
      & \ln |v| = 2\ln |y| \\
      & v = y^2  \\
      \Rightarrow&  \frac{du}{dy}y^2 = -2y^3 \\
      \Rightarrow & \int \frac{du}{dy} = - \int 2y \; dy \\
      \Rightarrow & u = -y^2 + C
     \end{split}
    \end{equation*}
    $z = x^2$ en $z = uv = y^2(C - y^2)$
    De AO wordt:
    $$x^2 + y^4 = Cy^2$$


}


\section{Orthogonale krommenbundel}
Definitie: elke kromme uit de ene bundel snijdt elke kromme uit de andere bundel loodrecht.
$$
    \begin{cases}
     f(x, y, C) = 0 \\
     f_{\bot}(x, y, C) = 0
    \end{cases}
$$
Raaklijn van $f$ staat loodrecht op raaklijn van $f_{\bot}$. Wiskundig wordt dit vertaald door: $\omega_{RL_{\bot}} = -\frac{1}{\omega_{RL}} = -\frac{1}{y'}$
De DVG van de orthogonale krommenbundel is
$$F_{\bot}\bigg(x, y, -\frac{1}{y'}\bigg)$$
\example{
    Bepaal de DVG van de orthogonale krommenbundel van alle raaklijnen aan $y = x^2$.
}{
    Elk punt op parabool kan beschreven worden als $p(a, a^2)$.
    \begin{enumerate}
     \item De vergelijking van de originele krommenbundel
        \begin{equation*}
         \begin{split}
          & y - a^2 = 2a(x - a) \\
          \Rightarrow & y - a^2 = 2ax - 2a^2 \\
          \Rightarrow & y = 2ax - a^2
         \end{split}
        \end{equation*}
     \item DVG van de originele krommenbundel
        $$
            \begin{cases}
             y = 2ax - a^2 \\
             y' = 2a
            \end{cases} \Rightarrow
            y = y'x - \frac{y'^2}{4}
        $$
     \item DVG van de orthogonale krommenbundel
        $y' wordt -\frac{1}{y'}$
        $$ y = \frac{1}{y'}x - \frac{1}{4}\frac{1}{y'^2}$$
        Uiteindelijk:
        $$4y'^2y = -4xy' - 1$$
    \end{enumerate}
    Deze DVG heeft graad 2, wat niet in deze cursus besproken wordt. Het is dus onoplosbaar.

}
\example{
    Bepaal de orthogonale krommenbundel van alle parabolen met top in de oorsprong en symmetrieas de X-as.
}{
    \begin{enumerate}
     \item De vergelijking van de originele krommenbundel
            $$ x = Cy^2$$
     \item DVG van de originele krommenbundel. Er is 1 onafhankelijke constanten dus 1 keer afleiden
            $$
              \begin{cases}
               x = Cy^2 \\
               1 = 2Cyy'
              \end{cases}
            $$
            Hieruit volgt $C = \frac{x}{y^2}$ en dus $1 = \frac{2xy'}{x}$
     \item DVG van de orthogonale krommenbundel
            
            $y'$ vervangen door $-\frac{1}{y'}$ dus
            $$1 = -\frac{2x}{yy'} \Leftrightarrow yy' = -2x$$
     \item DVG oplossen
     \begin{equation*}
      \begin{split}
                    & y \frac{dy}
                             {dx} = -2x \\
                    & \int y \; dy = - \int 2x\; dx \\
                    & \frac{y^2}{2} = -x^2 \\
                    & \frac{y^2}{2} + x^2 = C
      \end{split}
     \end{equation*}
     Dit zijn dus ellipsen

    \end{enumerate}


}
\section{DVG van hogere orde}
\example{
    Los op
    $$y''' = e^{-2x}$$
}{
    \begin{equation*}
     \begin{split}
       y''' & = e^{-2x} \\
       y''  & = \int e^{-2x}\;dx = -\frac{1}{2}e^{-2x} + C_1 \\
       y'   & = -\int \frac{1}{2}e^{-2x} + C_1 \; dx \frac{1}{4}e^{-2x} + C_1x + C_2 \\
       y    & = -\frac{1}{8}e^{-2x} + \frac{C_1x^2}{2} + C_2x + C_3 \\
            & = -\frac{1}{8}e^{-2x} + C_1x^2 + C_2x + C_3
     \end{split}
    \end{equation*}
}
\subsection{DVG van orde 2 van type F(x, y', y'') = 0}
Bewijs oplossingsmethode:

Stel $y' = p$, dan wordt $y'' = \frac{dp}{dx}$. De differentiaalvergelijking wordt $F(x, p, \frac{dp}{dx} = 0$. Dit is een dvg van orde 1 in p en x. 

\example{
    Bepaal de AO van $xy'' = y' - x$
}{
    $xy'' = y' - x $ komt overeen met $F(x , y', y'')$
    
    Stel $y' = p' \rightarrow y'' = x\frac{dp}{dx} = p - x$ waaruit volgt dat $x\;dp = (p - x)\;dx$. Dit is homogeen ($\lambda^{(1)}$ dus we stellen $p = ux$
    \begin{equation*}
     \begin{split}
      x(u\;dx + x\;du) & = (ux - x)\;dx \\
      u\;dx + x\;du    & = (u - 1) \;dx \\
      x\;du            & = -dx; \\
      du               & = -\frac{dx}{x} \\
      \int du          & = -\int\frac{dx}{x} \\
      u                & = \ln|x| + C_1 \\
      \frac{dy}{dx}    & = -x\ln|x| + C_1x \\
      \int dy          & = -\int x\ln|x| + C_1x \;dx
     \end{split}
    \end{equation*}
    Het Antwoord is:
    $$y = \frac{-x^2}{2}\ln|x| + \frac{x^4}{4} - C_2 + \frac{C_1}{2}x^2 = \frac{-x^2}{2}\ln|x| + \frac{x^4}{4} - C_2 + C_1x^2$$
}
\subsection{DVG van orde 2 van type F(y, y', y'') = 0}
Bewijs oplossingsmethode:

Stel $y' = p$, dan wordt $y'' = \frac{dp}{dx} = \frac{dp/dy}{dx/dy} = \frac{dp}{dy}\frac{dy}{dx} = p\frac{dp}{dy}$. De differentiaalvergelijking wordt $F(y, p, p\frac{dp}{dx} = 0$. Dit is een dvg van orde 1 in p en y.

\example{
    Bepaal de PO van $(1 - y)^2y'' - y'^3 = 0$ met $y(0) = 2$ en $y'(0) = 1$
}{
    Stel $y' = p \rightarrow y'' = p\frac{dp}{dy}$
    \begin{equation*}
     \begin{split}
      (1 - y)^2p\frac{dp}{dy} - p^3 & = 0 \\
      (1 - y)^2\frac{dp}{dy} - p^2  & = 0 \\
      (1 - y)^2\frac{dp}{dy}        & = p^2 \\
      \frac{dp}{p^2}                & = \frac{dy}{(1 - y)^2} \\
      \int \frac{dp}{p^2}           & = \int\frac{dy}{(1 - y)^2} \\
      -\frac{1}{p}                  & = -\frac{1}{1 - y} + C_1 \\
      C_1 & = 0 \quad \hbox{aangezien p(0) = 1 en y(0) = 2} \\
      \frac{1}{p} = \frac{dx}{dy} & = \frac{1}{y - 1} \\
      & \begin{cases}
       dx = \frac{dy}{y - 1} \\
       x = \ln|y - 1| + C_2 \\
       0 = \ln|1| + C_2 \rightarrow C_2 = 0
      \end{cases} \\
      x & = \ln|y - 1| \\
      e^x & = y - 1 \\
      y & = e^x + 1 
     \end{split}
    \end{equation*}

}
\section{Stellingen voor lineaire differentiaalvergelijkingen}
Voor geen enkele stelling is het bewijs te kennen
\subsection{Stelling 1}
Is $L(D)y = 0$ een lineaire homogene DVG van $n^{de}$ orde en $y_i(x), i = 1, ..., n$ n onafhankelijke PO's van $L(D)y = 0$ dan is $y = C_1y_1(x) + C_2y_2(x) + ... + C_ny_n(x)$ de AO van $L(D)y = 0$

\subsection{Stelling 2}
Indien 
\begin{tabular}{| c c c c |}
 $y_1$          & $y_2$         & ... & $y_m$ \\
 $y'_1$         & $y'_2$        & ... & $y'_m$ \\
 ...            & ...           & ... & ...   \\
 $y^{n - 1}_1$  & $y^{n - 1}_2$ & ... & $y^{n - 1}_m$ 
\end{tabular} = 0, dan zijn de PO's van $L(D)y = 0$ lineair onafhankelijk.
\subsection{Stelling 3}
Indien $L(D)y = 0$ een lineaire DVG van $n^{de}$ orde, $y_1(x)$ een PO van $L(D)y = Q(x)$ en $y_2(x)$ de AO van $L(D)y = 0$ dan is $y(x) = y_1(x) + y_2(x)$ de AO van $L(D)y = Q(x)$

\example{
    Bepaal de AO van $y''' - 3y'' + 3y' - y = 0$ indien $y_1 = e^x, y_2=xe^x, y_3=x^2e^x$ de 3 PO's zijn van deze DVG
}{
    We bewijzen enkel dat $y_2 = xe^$ een PO is. De andere twee kan je zelf uitrekenen.
%     \todo{wtf}


}

\example{

}{

}

\chapter{Laplacetransformatie}
\section{De Heaviside functie}
De Heaviside functie heeft als voorschrift:
$$H(t - a) = 
\begin{cases}
0 \;\; t < a \\
1 \;\; t > a
\end{cases}$$

\example{Teken over $x=[-3,4]$ de functie $y = 2H(t + 2) - tH(t) + (t+t^2)H(t-2)$}
{
Er zijn veranderingen bij $t = -2, t = 0$ en $t = 2$.

    \begin{tabular}{l | l}
    $2\cdot(0) - t\cdot(0) + (t+t^2)\cdot(0) = 0$ & $t < -2$\\
    $2\cdot(1) - t\cdot(0) + (t+t^2)\cdot(0) = 2$ & $-2 < t < 0$  \\
    $2\cdot(1) - t\cdot(1) + (t+t^2)\cdot(0) = 2 - t$ & $0 < t < 2$\\
    $2\cdot(1) - t\cdot(1) + (t+t^2)\cdot(1) = 2 + t^2$ & $t > 2$\\
    \end{tabular}
\todo{graph}
}
\example{Schrijf met behulp van de Heaviside functie de stuksgewijze continue functie:
$$f(t) = \begin{cases}
        e^t & t < 2 \\
        1 - e^t & 2 < t < 3 \\
        t^2 & 3 < t < 5 \\
        t - 25 & t > 5
        \end{cases}
$$}{
\begin{equation*}
\begin{split}
    f(t) & = e^t + H(t-2)(-e^t + 1 - e^t) + H(t-3)(-1 + e^t + t^2) + H(t - 5)(-t^2 + t - 25) \\
    & = e^t + (1-2e^t)H(t-2) + (t^2+e^t-1)H(t-3) - (t^2-t+25)H(t-5)
\end{split}
\end{equation*}
}

\section{De Dirac delta-'functie'}
De Dirac delta-functie heeft als voorschrift:
$$
\begin{cases}
\delta(t - a) = 0 & t \neq a \\
\int_{a - \epsilon_1}^{a + \epsilon_2} \delta(t - a) \; dt = 1 & \forall \epsilon_1, \epsilon_2 > 0 
\end{cases}
$$
De meetkundige betekenis: We nemen de limiet van $\delta^a_{\epsilon_1,\epsilon_2}(t)$ voor $\epsilon_1,\epsilon_2 \rightarrow 0$
$$\delta^a_{\epsilon_1,\epsilon_2}(t) = \begin{cases}
                                        0 & \forall t < a - \epsilon_1 \; \hbox{of} \; t > a + \epsilon_2 \\
                                        \frac{1}{\epsilon_1 + \epsilon_2} & \forall \in ]a - \epsilon_1, a + \epsilon_2[
                                        \end{cases}
$$
Het nut van de dirac functie is om bepaalde integralen op te lossen. Meer bepaald de integralen van de vorm:
$$\int_{0}^{+\infty} f(t) \delta(t- a)\;dt = f(a)$$
De ondergrens 0 mag ook vervangen worden door $-\infty$ aangezien elke functie causaal is binnen het domein van Laplace.

De afgeleide van de Heaveiside functie is gelijk aan de delta functie:
$$\frac{d}{dt}H(t-a) = \delta(t - a)$$
\example{$$\int_{0}^{+\infty} (2\sin t - 1) \delta(t - \frac{3\pi}{2}) \; dt$$}{
In dit geval is $f(t) = (2\sin t - 1)$ en $\delta(t - a) = \delta(t - \frac{3\pi}{2})$
We kunnen dus makkelijk deze integraal oplossen door gebruik te maken van de definitie:
\begin{equation*}
\begin{split}
\int_{0}^{+\infty} f(t) \delta(t- a)\;dt & = \int_{0}^{+\infty} (2\sin t - 1) \delta(t - \frac{3\pi}{2}) \; dt \\
                                        & = f(\frac{3\pi}{2}) - 1  \\
                                        & = 2\sin \bigg(\frac{3\pi}{2}\bigg) - 1 \\
                                        & = -2 - 1 \\
                                        & = - 3
\end{split}
\end{equation*}
}
\section{Causale functie}
Een causale functie is een functie $f$ waarvoor $f(t) = 0$ voor elke $t < 0$.
Om een willekeurige functie causaal te maken voeg je de Heaviside functie achteraan toe.
$$f(t) \rightarrow f(t)H(t)$$
Dit zorgt ervoor dat voor elke $t < 0$ dat $f(t) = 0$. De afspraak is dat deze Heaviside functie nu achter elke functie komt zonder dat we deze nog schrijven. Elke functie is vanaf nu dus causaal.

\example{Teken de causale functie $f(t)$ gedefinieerd als: -2 indien t $<$ 1 en 2 als t $>$ 1. Schrijf ze ook met behulp van de Heaviside functie}{
De functie kan omschreven worden als:
$$f(t) = \begin{cases}
        0 & t < 0 \\
        -2 & 0 < t < 1 \\
        2 & t > 1
        \end{cases}
$$
Omgevormd met de Heaviside-functie:
\begin{equation*}
\begin{split}
f(t) & = H(t)(-0 + (-2)) + H(t -  1)(-2 +2) \\
    & = -2H(t) + 4H(t - 1)
\end{split}
\end{equation*}
Tekening:

\begin{center}
\includegraphics[width=0.8\textwidth]{oef5_heaviside}
\end{center}
}
\section{Exponentiële orde}
Een functie is van exponentiële orde indien $\exists M, a \in R$ zodat $|f(t)| < Me^{at}, \forall t > N$ en met a het minimum van de waarden waarvoor dit geldt. Indien waar is $f(t)$ van exponentiële orde a.
Soms is het gemakkelijker te bewijzen via:
$$\lim_{t \to +\infty} \frac{|f(t)|}{e^{at}} \in R$$
\example{Bepaal de exponentiële orde van $\sin t$}{
\begin{equation*}
\begin{split}
                & |\sin t| \leq 1 \\
\Leftrightarrow & |\sin t| < 1.1 \hbox{(willekeurige waarde)} \\
\Leftrightarrow & |\sin t| < 1.1e^{at}
\end{split}
\end{equation*}
Hieruit kan afgeleid worden dat a = 0 en de exponentiële orde is dus ook 0.
}
\example{Bepaal de exponentiële orde van $(1 + 2t)e^{-t}$}{
Bij deze opgave maken we gebruik van de limietstelling.
\begin{equation*}
\begin{split}
\lim_{t \to +\infty} \frac{|f(t)|}{e^{at}} & = \lim_{t \to +\infty} \frac{|(1 + 2t)e^{-t}|}{e^{at}} \\
                                            & = \lim_{t \to +\infty} \frac{(1 + 2t)e^{-t}}{e^{at}} \\
                                            & = \lim_{t \to +\infty} \frac{1 + 2t}{e^{at}e^{t}} \\
                                            & = \lim_{t \to +\infty} \frac{1 + 2t}{e^{t(a +1)}} 
\end{split}
\end{equation*}
We moeten een onderscheid maak tussen 2 gevallen:
\begin{itemize}
\item $a + 1 < 0 \rightarrow e^{-\infty} = 0 \rightarrow \frac{+\infty}{0} \rightarrow \; \hbox{onbepaald}$
\item $a + 1 > 0 \rightarrow e^{+\infty} = \infty \rightarrow \frac{+\infty}{+\infty} \rightarrow \; \hbox{L'Hopital}$
\end{itemize}
We maken enkel gebruik van het tweede geval en passen dus L'hopital toe.
\begin{equation*}
\begin{split}
\lim_{t \to +\infty} \frac{1 + 2t}{e^{t(a +1)}} & = \lim_{t \to +\infty} \frac{2}{e^{t(a +1)}(a+1)} \\
                                                & = \frac{2}{+\infty} = 0 \in R
\end{split}
\end{equation*}
Aangezien het een reëele uitkomst is kan a uit de uitdrukking $a + 1 > 0$ afgeleid worden.
$$\forall a, a > -1$$
De exponentiële orde is dus -1.
}
\section{De Laplacetransformatie}
Definitie: Stel $f(t)$ causuaal dan is de laplacetransformatie van $f(t)$ een functie die een complex getal $s$ afbeeldt op 
$$\mathcal{L}\{f(t)\}(s) = F(s) = \int_{0}^{+\infty}f(t)e^{-st}\;dt, s \in \mathbb{C}$$
Een voorbeeld uit het formularium:
$$\mathcal{L}\{\sin t\}(s) = \frac{1}{1 + s^2}$$
De letter s kan eender welk complex getal zijn:
$$\mathcal{L}\{\sin t\}(2) = \frac{1}{1 + 4}$$
Indien er een imaginaire eenheid is verandert de definitie minimaal:
$$\mathcal{L}\{\sin t\}(3 + 2j) = \int_{0}^{+\infty}|f(t)e^{-st}|\;dt$$
Het argument tussen de $| ... |$ is NIET de absolute waarde, maar de MODULUS van het complexe getal, te berekenen via $\sqrt{x^2 + y^2}$ indien het complexe getal gedefinieerd wordt als $s = x + yj$ (wat vanaf nu als definitie gebruikt wordt voor een complex getal).
\subsection{Opmerkingen}
\begin{enumerate}
\item $$|f(t)e^{-st}| = |f(t)|e^{-xt}, \; s = x+yj$$
want
\begin{equation*}
\begin{split}
|f(t)e^{-st}| & = |f(t)e^{-(x + yj)t}| \\
                & = |f(t)|\cdot|e^{-(xt + yjt)}| \\
                & = |f(t)|\cdot|e^{-xt} \cdot e^{-yjt}| \\
                & = |f(t)|\cdot|e^{-xt}|\cdot|e^{-yjt}| \\
                & = |f(t)|\cdot e^{-xt}\cdot|\cos(-yt) + j\sin(-yt)| \\
                & = |f(t)|e^{-xt}\sqrt{\cos^2{(-yt)} + \sin^2{(-yt)}} \\
                & = |f(t)|e^{-xt}
\end{split}
\end{equation*}
\item 
$$\mathcal{L}\{af(t) + bg(t)\}(s) = a\mathcal{L}\{f(t)\}(s) + b\mathcal{L}\{g(t)\}(s)$$
De Laplace van een som is gelijk aan de som van een Laplace.
\end{enumerate}
\subsection{Laplacegetransformeerde van enkele basisfuncties}
\begin{itemize}
\item $$\mathcal{L}\{e^{at}\}(s) = \frac{1}{s - a}$$
Bewijs:
\begin{equation*}
\begin{split}
\mathcal{L}\{e^{at}\}(s) & = \int_{0}^{+\infty}e^{at}e^{-st} \; dt \\
                            & = \int_{0}^{+\infty} e^{t(a - s)} \; dt \\
                            & = \frac{e^t{a - s)}}{a - s}\bigg|_{0}^{+\infty} \\
                            & = \frac{1}{a - s}\bigg(\lim_{t \to +\infty}e^{t(a - s)} - 1\bigg)                         
\end{split}
\end{equation*}
Uitwerking van de limiet:
\begin{equation*}
\begin{split}
    \lim_{t \to +\infty}e^{t(a - s)} & = \lim_{t \to +\infty}|e^{at - st)} | \\
                                    & = \lim_{t \to +\infty}|e^{at - (x + yj)t}| \\
                                    & = \lim_{t \to +\infty}|e^{at - xt}\cdot e^{-yjt}| \\
                                    & = \lim_{t \to +\infty}|e^{at - xt}|\cdot |e^{-yjt}| \\
                                    & = \lim_{t \to +\infty}|e^{at - xt}|\cdot |\cos(-yt) + j\sin(-yt)| \\
                                    & = \lim_{t \to +\infty}e^{at - xt}\cdot \sqrt{\cos^2(-yt) + \sin^2(-yt)} \\
                                    & = \lim_{t \to +\infty}e^{at - xt} = e^{-\infty} = 0
\end{split}
\end{equation*}
Deze uitkomst in de oorspronkelijke vergelijking steken:
$$\frac{1}{a - s}(0 - 1) = \frac{1}{s - a}$$ 

\item $$\mathcal{L}\{\sin \omega t\}(s) = \frac{\omega}{\omega^2 + s^2} \qquad  \hbox{en} \qquad  \mathcal{L}\{\cos \omega t\}(s) = \frac{s}{\omega^2 + s^2}$$
Bewijs: We vertrekken van de uitkomst van vorig bewijs. Beschouw $a = wj$
\begin{equation*}
\begin{split}
\mathcal{L}\{e^{wjt}\}(s) & = \frac{1}{s - wj} \\
                            & = \frac{1}{s - wj} \cdot \frac{s + wj}{s + wj} \\
                & = \frac{s + wj}{s^2 + w^2}\\
                & = \mathcal{L}\{\cos (\omega t) + j\sin(\omega t)\}(s) \\
                & = \mathcal{L}\{\cos (\omega t)\}(s) + \mathcal{L}\{j\sin(\omega t) \}(s) \\
                & = \frac{s}{s^2 + w^2} + \frac{w}{s^2 + w^2}j \\ 
\end{split}
\end{equation*}
dus $$\mathcal{L}\{\cos \omega t\}(s) = \frac{s}{\omega^2 + s^2} \qquad  \hbox{en} \qquad  \mathcal{L}\{\sin \omega t\}(s) = \frac{\omega}{\omega^2 + s^2}$$

\item 
$$\mathcal{L}\{\delta(t)\}(s) = 1$$
Bewijs:
\begin{equation*}
\begin{split}
    \mathcal{L}\{\delta(t)\}(s) & = \mathcal{L}\{\delta(t - 0)\}(s) \\
                & = \int_{0}^{+\infty}\delta(t - 0)e^{-st}\;dt \\
                & = f(0) = e^{-s\cdot0} = e^{0} = 1
\end{split}
\end{equation*}
\end{itemize}
\example{Bepaal het laplacebeeld van $\cos{(2t - 1)}$}{
\begin{equation*}
\begin{split}
\mathcal{L}\{\cos(2t - 1)\}(s) & = \mathcal{L}\{\cos(2t)\cos( 1 )+ \sin(2t)\sin (1)\}(s) \\
                & = \cos (1) \mathcal{L}\{\cos 2t\}(s) + \sin (1 )\mathcal{L}\{\sin 2t\}(s)\\
                & = \cos (1) \frac{s}{s^2 + 4} + \sin (1) \frac{2}{s^2 + 4} \\
                & = \frac{s\cos(1)}{s^2 + 4} + \frac{2\sin(1)}{s^2 + 4}
\end{split}
\end{equation*}
}
\example{Bepaal het laplacebeeld van $\sinh(4t) - 3\cos{(\frac{t}{3})}$}
{
\begin{equation*}
\begin{split}
\mathcal{L}\bigg\{\ \sinh(4t) - 3\cos{\bigg(\frac{t}{3}\bigg)} \bigg\}(s) & = \mathcal{L}\bigg\{\ \frac{e^{4t} - e^{-4t}}{2} - 3\cos{\bigg(\frac{t}{3}\bigg)} \bigg\}(s) \\
& = \mathcal{L}\bigg\{\ \frac{e^{4t} - e^{-4t}}{2}\bigg\}(s) - 3\mathcal{L}\bigg\{\cos{\bigg(\frac{t}{3}\bigg)} \bigg\}(s) \\
& = \frac{1}{2}\bigg(\frac{1}{s - 4} - \frac{1}{s + 4}\bigg) - 3\frac{s}{s^2 + \frac{1}{9}} \\
& = \frac{1}{2}\bigg(\frac{1}{s - 4} - \frac{1}{s + 4}\bigg) - \frac{27s}{9s^2 + 1}
\end{split}
\end{equation*}
}
\example{Bepaal het laplacebeeld van $\delta(t - \frac{\pi}{2})\cos(4t)e^{2t}$}{
\begin{equation*}
\begin{split}
\mathcal{L}\bigg\{\delta\bigg(t - \frac{\pi}{2}\bigg)\cos(4t)e^{2t}\bigg\}(s) & = \int_{0}^{+\infty}\cos(4t)e^{2t}\delta\bigg(t - \frac{\pi}{2}\bigg)e^{-st} \; dt \\
& = f\bigg(\frac{\pi}{2}\bigg) \\
& = \cos\bigg(4 \cdot \frac{\pi}{2}\bigg)e^{2\cdot\frac{\pi}{2}}e^{-s\cdot\frac{\pi}{2}} \\
& = \cos(2\pi)e^{\pi}e^{-\frac{s\pi}{2}}\\
& = e^{\pi}e^{-\frac{s\pi}{2}}
\end{split}
\end{equation*}

}
\subsection{Translatie naar rechts}
Definitie:
$$\mathcal{L}\{f(t - a)H(t - a)\}(s) = e^{-as}F(s) \qquad a > 0$$
Bewijs:
\begin{equation*}
\begin{split}
\mathcal{L}\{f(t - a)H(t - a)\}(s) & = \int_{0}^{+\infty}f(t - a)H(t - a)e^{-st} \; dt \\
                                    & = \int_{0}^{a}f(t - a)H(t - a)e^{-st} \; dt + \int_{a}^{+\infty}f(t - a)H(t - a)e^{-st} \; dt \\
                                    & = 0 +\int_{a}^{+\infty}f(t - a)H(t - a)e^{-st} \; dt \\
                                    & = \int_{a}^{+\infty}f(t - a)H(t - a)e^{-st} \; dt \\
                                    \hbox{stel}\quad u & = t - a \\
                                    \hbox{dan}\quad du & = dt \\    
                                    & = \int_{0}^{+\infty}f(u)e^{-s(u + a)} \; du \\
                                    & = \int_{0}^{+\infty}f(u)e^{-su}e^{-sa} \; du \\
                                    & = e^{-sa}\int_{0}^{+\infty}f(u)e^{-su} \; du \\
                                    & = e^{-sa}\mathcal{L}\{f(t)\}(s) \\
                                    & = e^{-as}F(s)
\end{split}
\end{equation*}
\example{Bepaal het laplacebeeld van $f(t) = (t^2 - 1)H(t - 1) - \sin(3t)H(t - \pi)$}
{
\begin{equation*}
\begin{split}
\mathcal{L}\{f(t)\} & = \mathcal{L}\{(t^2 - 1)H(t - 1)\}(s) - \mathcal{L}\{\sin(3t)H(t-\pi)\}(s)
\end{split}
\end{equation*}
We werken beide laplacetransformaties afzonderlijk uit:
\begin{equation*}
\begin{split}
%t^2 - 1 = (t - 1)^2 + 2t - 2 = (t - 1)^2 + 2(t - 1)%
\mathcal{L}\{(t^2 - 1)H(t - 1)\}(s) & = \mathcal{L}\{[(t-1)^2 + 2(t - 1)]H(t - 1)\}(s)\\
                                    & = e^{-as}\mathcal{L}\{t^2 + 2t\}(s) \\
                                    & = e^{-s}\bigg(\frac{2!}{s^3} + 2\frac{1!}{s^2}\bigg) \\
                                    & = e^{-s}\bigg(\frac{2}{s^3} + \frac{2}{s^2}\bigg)\\
                                    & = e^{-s}\bigg(\frac{2(1 + s)}{s^3}\bigg)
\end{split}
\end{equation*}
\begin{equation*}
\begin{split}
    %sin 3t = sin(3(t - \pi)) + 3\pi) = sin(3(t - \pi) + \pi) = -sin(3(t-\pi))%
\mathcal{L}\{\sin(3t)H(t-\pi)\}(s) & =  \mathcal{L}\{-\sin(3(t - \pi))H(t-\pi)\}(s) \\
                                    & = -e^{-\pi s}\mathcal{L}\{\sin (3t)\}(s) \\
                                    & = -e^{-\pi s}\frac{3}{s^2 + 9} \\
                                    & = -\frac{3e^{-\pi s}}{s^2 + 9}
\end{split}
\end{equation*}
Het resultaat wordt:
\begin{equation*}
\begin{split}
\mathcal{L}\{f(t)\} & = \mathcal{L}\{(t^2 - 1)H(t - 1)\}(s) - \mathcal{L}\{\sin(3t)H(t-\pi)\}(s) \\
                    & = e^{-s}\bigg(\frac{2(1 + s)}{s^3}\bigg) - \bigg(-\frac{3e^{-\pi s}}{s^2 + 9}\bigg) \\
                    & = e^{-s}\bigg(\frac{2(1 + s)}{s^3}\bigg) +\frac{3e^{-\pi s}}{s^2 + 9}
\end{split}
\end{equation*}
}
\subsection{Dempingsfunctie}
Definitie:
$$\mathcal{L}\{e^{-at}f(t)\}(s) = F(s + a)$$
\example{Bepaal het laplacebeeld van $f(t) = t(t^3 - 1)^2e^{-t} + \sin(\sqrt{3}t)e^{2t}$}
{
\begin{equation*}
\begin{split}
\mathcal{L}\{f(t)\}(s) = \mathcal{L}\{t(t^3 - 1)^2e^{-t}\}(s) + \mathcal{L}\{\sin(\sqrt{3}t)e^{2t}\}(s)
\end{split}
\end{equation*}
Ook hier beschouwen we beide laplacetransformaties apart.
\begin{equation*}
\begin{split}
\mathcal{L}\{t(t^3 - 1)^2e^{-t}\}(s) & = \mathcal{L}\{(t^7 - 2t^4 + t)e^{-t}\}(s) \\
                                    & = \mathcal{L}\{t^7 - 2t^4 + t\}(s + 1) \\
                                    & = \frac{7!}{(s + 1)^8} - \frac{2 \cdot 4!}{(s + 1)^5} + \frac{1!}{(s + 1)^2} \\
                                    & = \frac{7!}{(s + 1)^8} - \frac{48}{(s + 1)^5} + \frac{1}{s^2 + 2s + 1}
\end{split}
\end{equation*}
\begin{equation*}
\begin{split}
\mathcal{L}\{\sin(\sqrt{3}t)e^{2t}\}(s) & = \mathcal{L}\{\sin(\sqrt{3}t)\}(s - 2) \\
                                        & = \frac{\sqrt{3}}{(s - 2)^2 + 3} \\
                                        & = \frac{\sqrt{3}}{s^2 -2s + 7}
\end{split}
\end{equation*}
Het resultaat wordt:
\begin{equation*}
\begin{split}
\mathcal{L}\{f(t)\}(s) & = \mathcal{L}\{t(t^3 - 1)^2e^{-t}\}(s) + \mathcal{L}\{\sin(\sqrt{3}t)e^{2t}\}(s) \\
                        & = \frac{7!}{(s + 1)^8} - \frac{48}{(s + 1)^5} + \frac{1}{s^2 + 2s + 1} + \frac{\sqrt{3}}{s^2 -2s + 7}
\end{split}
\end{equation*}
}
\subsection{Schaalwijziging}
Definitie:
$$\mathcal{L}\{f(at)\}(s) = \frac{1}{a}F(\frac{s}{a})$$
Bewijs:
\begin{equation*}
 \begin{split}
  \mathcal{L}\{f(at)\}(s) & = \int_{0}^{+\infty}f(at)e^{-st}\;dt \\
                    \hbox{stel}\; u & = at \\
                    \hbox{dan}\; du & = adt \\
                          & =  \int_{0}^{+\infty}f(u)e^{-s\frac{u}{a}}\;\frac{du}{a} \\
                          & =  \frac{1}{a}\int_{0}^{+\infty}f(u)e^{-\frac{s}{a}u}\;du \\
                          & =  \frac{1}{a}\mathcal{L}\{f(u)\}(\frac{s}{a}) \\
                          & =  \frac{1}{a}F(\frac{s}{a})
 \end{split}
\end{equation*}
\example{Gegeven $\mathcal{L}\{\sin t\}(s) = \frac{1}{s^2 + 1}$. Bepaal $\mathcal{L}\{\sin \omega t\}(s)$}
{
\begin{equation*}
 \begin{split}
  \mathcal{L}\{f(\omega t)\}(s) & = \mathcal{L}\{\sin \omega t\}(s) \\
                                & = \frac{1}{\omega}\mathcal{L}\{\sin t\}(\frac{s}{\omega}) \\
                                & = \frac{1}{\omega}\frac{1}{\frac{s^2}{\omega^2} + 1} \\
                                & = \frac{\omega}{\omega^2(\frac{s^2}{w^2} + 1)} \\
                                & = \frac{\omega}{s^2 + w^2}
 \end{split}
\end{equation*}}
\subsection{Laplacegetransformeerde van f'(t)}
Definitie:
$$\mathcal{L}\bigg\{\frac{df(t)}{dt}\bigg\}(s) = sF(s) - f(0^+), \forall s \in \mathbb{C}, Re(s) > a$$
\example{Gegeven $\mathcal{L}\{\sin \omega t\}(s) = \frac{\omega}{s^2 + w^2}$. Bepaal $\mathcal{L}\{\cos\omega t\}(s)$. }{
\begin{equation*}
 \begin{split}
  \mathcal{L}\{\cos\omega t\}(s) & = \mathcal{L}\bigg\{\frac{d[\sin \omega t]}{dt}\bigg\}(s) \\
                                 & = s\mathcal{L}\{\sin \omega t\}(s) - \sin{\omega \cdot 0} \\
                                 & = s\frac{\omega}{s^2 + w^2}
 \end{split}
\end{equation*}
\begin{equation*}
 \begin{split}
                  & \mathcal{L}\{\omega \cos\omega t\}(s) = s\frac{\omega}{s^2 + w^2} \\
  \Leftrightarrow & \omega \mathcal{L}\{\cos\omega t\}(s) = \omega\frac{s}{s^2 + w^2} \\
  \Leftrightarrow & \mathcal{L}\{\cos\omega t\}(s) = \frac{s}{s^2 + w^2} \\
 \end{split}
\end{equation*}}
\subsection{Laplacegetransformeerde van f''(t)}
Definitie:
$$\mathcal{L}\{f^{(n)}(t)\}(s) = s^nF(s) - s^{n - 1}f(0^+) - s^{n -2}f'(0^+) - ... - sf^{(n - 2)}(0^+) - f^{(n-1)}(0^+)$$
\example{Gegeven $g(t) = te^{-t}$, bepaal $\mathcal{L}\{g''(t)\}(s)$}{
\begin{equation*}
 \begin{split}
  \mathcal{L}\{g''(t)\}(s) & = \mathcal{L}\bigg\{\frac{d^2g}{dt^2}\bigg\}(s) \\
                           & = s^2G(s) - sg(0^+) - g'(t) \\
                           \hbox{met}\;G(s) & = \mathcal{L}\{te^{-t}\}(s) \\
                                            & = \mathcal{L}\{t\}(s + 1) \\
                                            & = \frac{1}{(s+1)^2}  \\
                           \hbox{en}\; g'(t)  & = -te^{-t} + e^{-t} \\
                                                     & = e^{-t}(1 - t) \\
            \Rightarrow s^2G(s) - sg(0^+) -  g'(0)  & = s^2\frac{1}{(s+1)^2} - s\cdot0 - 1 \\
            & = \frac{-2s - 1}{(s + 1)^2}
 \end{split}
\end{equation*}
}
\subsection{Laplacegetransformeerde van machten van t}
Definitie:
$$\mathcal{L}\{t^nf(t)\}(s) = (-1)^n\frac{d^nF(s)}{ds^n}$$
Bewijs:
\begin{equation*}
 \begin{split}
  F(s) & = \int_0^{+\infty}f(t)e^{-st}\; dt \\
  \frac{dF}{ds} & = \int_0^{+\infty}-tf(t)e^{-st}\; dt  \\
                & = -\int_0^{+\infty}tf(t)e^{-st}\; dt  \\
                & = -\mathcal{L}\{tf(t)\}(s)  \\
  \frac{d^2F}{ds^2} & = -\int_0^{+\infty}(-t)tf(t)e^{-st}\; dt  \\
                    & = \int_0^{+\infty}t^2f(t)e^{-st}\; dt  \\
                    & = \mathcal{L}\{t^2f(t)\}(s)  \\
 \end{split}
\end{equation*}
\example{Bepaal $\mathcal{L}\{t \sin t - t^3e^{-t}\}(s)$}{
\begin{equation*}
 \begin{split}
  \mathcal{L}\{t \sin t - t^3e^{-t}\}(s) & = \mathcal{L}\{t \sin t\}(s) - \mathcal{L}\{t^3e^{-t}\}(s)\\
  *) \mathcal{L}\{t \sin t\}(s) & = (-1)^1\frac{d\mathcal{L}\{\sin t\}(s)}{ds} \\
                                & =  -\frac{d\big(\frac{1}{1 + s^2}\big)}{ds} \\
                                & = -\bigg(\frac{-2s}{(1+s^2)^2}\bigg) \\
                                & = \frac{2s}{(1+s^2)^2} \\
**) \mathcal{L}\{t^3e^{-t}\}(s) & = (-1)^3\frac{d^3\mathcal{L}\{e^{-t}\}(s)}{ds^3} \\
                                & = -\frac{d^3\mathcal{L}\{e^{-t}\}}{ds^3} \\
                                & = -\frac{d^3\big(\frac{1}{s + 1}\big)}{ds^3} \\
                                & = -\frac{d^3[(s + 1)^{-1}]}{ds^3} \\
                            \frac{dF}{ds} & = -(s+1)^{-2} \\
                            \frac{d^2F}{ds^2} & = 2(s+1)^{-3} \\
                            \frac{d^3F}{ds^3} & = -6(s+1)^{-4} \\
                            \Rightarrow -\frac{d^3[(s + 1)^{-1}]}{ds^3} & = -(-6(s+1)^{-4} \\
                            & = \frac{6}{(s+1)^4}\\
 * - ** & = \frac{2s}{(1+s^2)^2} - \frac{6}{(s+1)^4}
 \end{split}
\end{equation*}}
\subsection{Laplacegetransformeerde van een integraal}
Definitie:
$$\mathcal{L}\bigg\{\int_0^tf(u)\;du\bigg\}(s) = \frac{1}{s}F(s) \forall s \in \mathbb{C}, Re(s) > a$$
Bewijs:
\begin{equation*}
 \begin{split}
  g(t) & = \int_{0}^{t}f(u)\;du \\
  g't) & = f(t) \\
  g'(0) & = 0 \\
  \Rightarrow \mathcal{L}\{g'(t)\}(s) & = sG(s) - g(0^+) \\
  \Rightarrow \mathcal{L}\{f(t)\}(s) & = s\mathcal{L}\bigg\{\int_0^tf(u)\;du\bigg\}(s) - 0 \\
  \Rightarrow \frac{1}{s}F(s) & = \mathcal{L}\bigg\{\int_0^tf(u)\;du\bigg\}(s)
 \end{split}
\end{equation*}
\example{Bepaal $\mathcal{L}\bigg\{\int_0^t  \cos \omega t \; dt\bigg\}$}{
\begin{equation*}
 \begin{split}
  \mathcal{L}\bigg\{\int_0^t  \cos \omega t \; dt\bigg\} & = \frac{1}{s}\mathcal{L}\{\sin \omega t\}(s) \\
                                                         & = \frac{1}{s}\frac{s}{s^2 + \omega^2} \\
                                                         & = \frac{1}{s^2 + \omega^2}
 \end{split}
\end{equation*}
}
\subsection{Laplacegetransformeerde van een periodische functie}
Definitie:
$$\mathcal{L}\{f(t)\}(s) = \frac{1}{1 - e^{-sT}}\int_0^{T}e^{-st}f(t)\; dt$$
\todo{slide 19}
\subsection{De convolutiestelling}
Definitie:
$$(f * g)(t) = \int_0^t f(u)g(t - u)\;du$$
Hieruit volgt:
$$
\mathcal{L}\{(f * g)(t)\}(s) = F(s)G(s)$$
Bewijs(niet te kennen)

\example{Gegeven $f(t) = e^{at}$ en $g(t) = e^{bt}$. Illustreer de juistheid van deze rekenregel.}{
\begin{equation*}
 \begin{split}
    f(t) * g(t) & = e^{at} e^{bt} \\
                & = \int_{0}^{t} e^{au}e^{b(t-u}\; du\\
                & = \int_{0}^{t} e^{au}e^{bt}e^{-bu}\; du\\
                & = e^{bt}\int_{0}^{t} e^{au}e^{-bu}\; du\\
                & = e^{bt}\int_{0}^{t} e^{u(a-b)}\; du\\
                & = e^{bt}\bigg[\frac{e^{u(a-b)}}{a - b}\bigg]_0^t\\
                & = \frac{e^{bt}}{a - b}[e^{t(a-b)} - 1] \\
                & = \frac{1}{a - b}(e^{at} - e^{bt}) \\
    \Rightarrow \mathcal{L}\{\frac{1}{a - b}(e^{at} - e^{bt})\}(s) & = \frac{1}{a - b}\mathcal{L}\{(e^{at} - e^{bt})\}(s) \\
                                                                   & = \frac{1}{a - b}\bigg(\frac{1}{s-a} - \frac{1}{s - b}\bigg) \\
                                                                   & = \frac{1}{a - b}\bigg(\frac{(s - b) - (s - a)}{(s - a)(s - b)}\bigg) \\
                                                                   & = \frac{1}{s - a}\frac{1}{s - b} \\
                                                                   & = \mathcal{L}\{e^{at}\}(s)\mathcal{L}\{e^{bt}\}(s)
 \end{split}
\end{equation*}
}
\example{Bereken $H(t)*H(t)*H(t)$.}{
\begin{equation*}
 \begin{split}
  H(t) * H(t) & = (H * H)(t) \\
              & = \int_0^t H(u)H(t - u)\; du \\
            \hbox{aangezien}\; & 0 \leq u \leq t \\
            \Rightarrow & H(u)  = 1 \\
            \Rightarrow & H(t - u)  = 1 \\
              & = \int_0^t \; du \\
              & = [u]_0^t \\
              & = t
 \end{split}
\end{equation*}
\begin{equation*}
 \begin{split}
  H(t) * H(t) * H(t) & = (H * H)(t) * H(t) \\
                     & = t * H(t) \\
                     & = \int_0^t uH(t - u)\; du \\
                     & = \bigg[\frac{u^2}{2}\bigg]_0^t \\
                     & = \frac{t^2}{2}       
 \end{split}
\end{equation*}


}
\subsection{Inverse Laplacetransformatie}
Definitie:
$$\mathcal{L}^{-1}\{F(s)\}(t) = f(t) \qquad \hbox{indien} \qquad \mathcal{L}\{f(t)\}(s) = F(s)$$
\example{Bepaal het invers laplacebeeld van 
        $$\frac{s^2 + 2s + 1}{s(s-1)(s-2)}$$
    }{
        \begin{equation*}
         \begin{split}
          \mathcal{L}^{-1}\{\frac{s^2 + 2s + 1}{s(s-1)(s-2)}\}(t) & = \mathcal{L}^{-1}\{\frac{a}{s} + \frac{b}{s-1} + \frac{c}{s-2}\}(t) \\
                                                                  & = \mathcal{L}^{-1}\{\frac{a(s-1)(s-2) + bs(s-2) + cs(s-1)}{s(s-1)(s-2)}\}(t) \\
                                                       \Rightarrow s^2 + 2s + 1 & = a(s-1)(s-2) + bs(s-2) + cs(s-1) \\
                                                            \hbox{als}\;s = 0 & : a = \frac{1}{2} \\
                                                            \hbox{als}\;s = 1 & : b = -4\\
                                                            \hbox{als}\;s = 2 & : c = \frac{9}{2} \\     
                                                                  & = \mathcal{L}^{-1}\{\frac{1/2}{s} + \frac{-4}{s-1} + \frac{9/2}{s-2}\}(t) \\
                                                                  & = \frac{1}{2} + (-4e^{t}) + \frac{9}{2}e^{2t} \\
                                                                  & = \frac{1}{2}(1 - 8e^{t} + 9e^{2t})
         \end{split}
        \end{equation*}
    }
\example{Bepaal het invers laplacebeeld van
    $$\frac{s^2 - 2}{2s^2 + 4s + 10}$$
}{
    \begin{equation*}
     \begin{split}
      \mathcal{L}^{-1}\{\frac{s^2 - 2}{2s^2 + 4s + 10}\} & = \mathcal{L}^{-1}\bigg\{\frac{1}{2} - \frac{2s + 7}{2s^2 + 4s + 10}\bigg\}   \\
                                                    & = \mathcal{L}^{-1}\bigg\{\frac{1}{2} - \frac{s + 7/2}{s^2 + 2s + 5}\bigg\}    \\
                                                    & = \mathcal{L}^{-1}\bigg\{\frac{1}{2} - \frac{s + 7/2}{(s+1)^2 + 4}\bigg\}  \\
                                                    & = \mathcal{L}^{-1}\bigg\{\frac{1}{2} - \frac{(s + 1) + 5/2}{(s+1)^2 + 4}\bigg\}  \\
                                                    & = \mathcal{L}^{-1}\bigg\{\frac{1}{2} - \frac{s + 1}{(s+1)^2 + 4} - \frac{5}{2}\frac{1}{(s+1)^2 + 4}\bigg\}  \\
                                                    & = \mathcal{L}^{-1}\bigg\{\frac{1}{2}\bigg\} - \mathcal{L}^{-1}\bigg\{\frac{s + 1}{(s+1)^2 + 4}\bigg\} - \frac{5}{2}\mathcal{L}^{-1}\bigg\{\frac{1}{(s+1)^2 + 4}\bigg\}  \\
                                                    & = \frac{1}{2}\delta(t) - \cos(2t)e^{-t} - \frac{5}{2}\sin(2t)e^{-t} \\
                                                    & = \frac{1}{2}\delta(t) - e^{-t}(\cos 2t + \frac{5}{2}\sin 2t)
     \end{split}
    \end{equation*}

}
\example{Bepaal het invers laplacebeeld van 
    $$\frac{e^{-\pi s}}{(s+1)(s^2+2s+2)}$$
}{
    \begin{equation*}
     \begin{split}
      \mathcal{L}^{-1}\bigg\{\frac{e^{-\pi s}}{(s+1)(s^2+2s+2)}\bigg\}(t)
                                                                       & = \mathcal{L}^{-1}\bigg\{\frac{1}{(s+1)(s^2+2s+2)}\cdot e^{-\pi s}\bigg\}(t) \\
                                                                       & = f(t - \pi)H(t - \pi)  \\
                                                                       \hbox{met}\; f(t) & = \mathcal{L}^{-1}\bigg\{\frac{1}{(s+1)(s^2+2s+2)}\bigg\}(t) \\
                                                                       \Rightarrow \frac{1}{(s+1)(s^2+2s+2)} & = \frac{a}{s + 1} + \frac{b + cs}{s^2 + 2s + 2} \\
                                                                                                             & = \frac{a(s^2 + 2s + 2) + (b+cs)(s+1)}{(s+1)(s^2+2s+2)}   \\
                                                                       \Rightarrow 1 & = a(s^2 + 2s + 2) + (b+cs)(s+1) \\
                                                                       \begin{cases}
                                                                        2a + b = 1  \\
                                                                        2a + b + c = 0 \\ 
                                                                        a + c = 0 \\
                                                                       \end{cases} 
                                                                       & \Rightarrow
                                                                       \begin{cases}
                                                                        a = 1  \\
                                                                        b = -1\\ 
                                                                        c = -1 \\
                                                                       \end{cases} \\
                                                                       \Rightarrow \mathcal{L}^{-1}\bigg\{\frac{1}{(s+1)(s^2+2s+2)}\bigg\}(t) & = \mathcal{L}^{-1}\bigg\{\frac{1}{s + 1} + \frac{s + 1}{s^2 + 2s + 2}\bigg\}(t) \\                  
                                                                       & = \mathcal{L}^{-1}\bigg\{\frac{1}{s + 1} + \frac{s + 1}{(s+1)^2 + 1}\bigg\}(t) \\    
                                                                       & = e^{-t} - \cos(t) e^{-t} \\
                                                                       & = e^{-t}( 1 - \cos t) = f(t) \\
                                                                       \hbox{ANTWOORD}\Rightarrow & e^{-(t - \pi)}(1 - \cos(t - \pi)H(t - \pi) \\
                                                                                 = & e^{\pi - t}(1 + \cos t)H(t - \pi)
     \end{split}
    \end{equation*}

}
\example{Bepaal het inverse laplacebeeld van 
    $$\frac{e^{2s}}{(s-3)^6}$$
}{
    \begin{equation*}
     \begin{split}
      \mathcal{L}^{-1}\bigg\{\frac{e^{2s}}{(s-3)^6}\bigg\}(t) & = f(t-2)H(t - 2) \\ 
                                                    \hbox{met} \; f(t) & = \mathcal{L}^{-1}\bigg\{\frac{1}{(s-3)^6}\bigg\}(t) \\
                                                                       & = g(t)e^{3t} \\
                                                    \hbox{met} \; g(t) & = \mathcal{L}^{-1}\bigg\{\frac{1}{s^6}\bigg\}(t) \\
                                                                       & = \frac{1}{5!}\mathcal{L}^{-1}\bigg\{\frac{5!}{s^6}\bigg\}(t) \\
                                                                       & = \frac{t^5}{5!} \\
                                                    f(t) & = g(t)e^{3t} \\
                                                         & = \frac{t^5e^{3t}}{5!} \\
     \mathcal{L}^{-1}\bigg\{\frac{e^{2s}}{(s-3)^6}\bigg\}(t) & = f(t-2)H(t - 2) \\ 
                                                             & = \frac{(t-2)^5e^{3(t-2)}}{5!}H(t-2)
     \end{split}
    \end{equation*}

}
\example{Bereken $(H*H*H*H*)(t)$}
{
    \begin{equation*}
     \begin{split}
      (H*H*H*H*)(t) & = \frac{1}{s^4} \\
                    & = \frac{1}{3!}\mathcal{L}^{-1}\bigg\{\frac{3!}{s^4}\bigg\}(t) \\
                    & = \frac{t^3}{6}
     \end{split}
    \end{equation*}

}
\example{Bereken:
    $$\mathcal{L}^{-1}\bigg\{\frac{s}{(s^2 + 4)^2}\bigg\}(t)$$
}{
    \begin{equation*}
     \begin{split}
      \mathcal{L}^{-1}\bigg\{\frac{s}{(s^2 + 4)^2}\bigg\}(t) & = \mathcal{L}^{-1}\bigg\{\frac{s}{(s^2 + 4)}\frac{1}{(s^2 + 4)}\bigg\}(t) \\
                                                             & = \frac{1}{2}\mathcal{L}^{-1}\bigg\{\frac{s}{(s^2 + 4)}\frac{2}{(s^2 + 4)}\bigg\}(t) \\
                                                             & = \frac{1}{2}(\sin 2t * \cos 2t) \\
                                                             & = \frac{1}{2}\int_0^t \sin 2u \cos [2(t - u)] \; du\\
                                                             & = \frac{1}{4}\int_0^t \sin(2u - [2(t - u)]) + \sin(2u + [2(t - u)])\; du \\
                                                             & = \frac{1}{4} \int_0^t \sin(4u - 2t) + \sin 2t\; du \\
                                                             & = \frac{1}{4}\bigg[ \int_0^t \sin(4u - 2t) \; du  + \int_0^t \sin 2t\; du \bigg] \\
                                                             & = \frac{1}{16}\bigg[-\cos(4u - 2t)\bigg]_0^t + \frac{1}{4}\bigg[u\sin 2t\bigg]_0^t \\
                                                             & = \frac{1}{16}\bigg(-\cos 2t + \cos (-2t)\bigg) + \frac{1}{4}t\sin 2t \\
                                                             & =\frac{1}{4}t\sin 2t
     \end{split}
    \end{equation*}

}

























