\chapter{Differentiaalvergelijkingen}
\section{Bepalen van differentiaalvergelijkingen}
\exercise{Bepaal de DVG van \begin{enumerate}
                        \item $y = C_1x + C_2$
                        \item de cirkels met hun middelpunt op de x-as 
                        \item de raaklijnen aan $K: y = x^2$
                        \end{enumerate}}
{
\begin{enumerate}
\item De vergelijking $y = C_1x + C_2$ heeft 2 onafhankelijke constanten. Er moet dus 2 keer afgeleid worden.
\begin{equation*}
\begin{split}
y' & = C_1 \\
y'' & = 0
\end{split}
\end{equation*}
De differentiaalvergelijking is $y'' = 0$
\item Het middelpunt op de x-as kan gedefinieerd worden als $m \in x-as \Rightarrow m(C_1, 0)$. De straal wordt gedefinieerd als $C_2$. De vergelijking van een cirkel wordt dan:
$$\Gamma: (x - C_1)^2 + y^2 = C_2^2$$
Er zijn 2 onafhankelijke constanten. Er moet dus 2 keer (impliciet) afgeleid worden.
\begin{equation*}
\begin{split}
\frac{dy}{dx} :\; & 2(x - C_1) + 2yy' = 0 \\
\frac{d^2y}{dx^2} :\; & 2 + 2(y'y' + yy'') = 0
\end{split}
\end{equation*}
De 2de afgeleide bevat geen constanten meer dus de differentiaalvergelijking wordt: 
$$yy'' + (y')^2 + 1 = 0$$
\item De raaklijn wordt gegeven door : $R: y - y'p = y'_p(x - x_p)$

Stel $p \in K$ en $x_p = C$:
\begin{equation*}
\begin{split}
\Rightarrow & y_p = (x_p)^2 = C^2 \\
\Rightarrow & p(C, C^2)
\end{split}
\end{equation*}
De richtingscoëfficient $y'_p$ wordt gegeven door 
$$y'= 2x \Rightarrow y'_p = 2C$$

De formule van de raaklijn kan worden ingevuld:
$$R: (y - C^2) = 2C(x - C)$$
Deze vergelijking bevat slechts 1 constante en moet dus 1 maal afgeleid worden.
$$y' = 2C \Leftrightarrow C = \frac{y'}{2}$$
Substitueer $C$ in de formule van de raaklijn:
\begin{equation*}
\begin{split}
    & y - \bigg(\frac{y'}{2}\bigg)^2 = y'\bigg(\frac{y'}{2}\bigg)\bigg(x - \frac{y'}{2}\bigg) \\
    \Leftrightarrow \; & 4y - y'^2 = 4xy' - 2y'^2 \\
    \Leftrightarrow \; & y'^2 - 4y'x + 4y = 0
\end{split}
\end{equation*}
is de differentiaalvergelijking.
\end{enumerate}
}



\section{Differentiaalvergelijkingen van de eerste orde en eerste graad}
\subsection{Scheiden van de veranderlijken}
\exercise{
    Bepaal de AO van $$\frac{\sin x}{2 + y}y' = \cos x$$
}{
    Deze DVG kan eenvoudig gescheiden worden als. 
    $$\frac{1}{2 + y}dy = \frac{\cos x}{\sin x} dx $$
    Neem de integraal.
    $$\int \frac{1}{2 + y}dy = \int \frac{\cos x}{\sin x} dx $$
    De vergelijking wordt:
    $$\ln |2 + y| = \ln | \sin x | + C $$
    De constante C kan geschreven worden als $\ln C$ zodat die samengevoegd kan worden met $\ln \sin x$
    $$\ln |2 + y| = \ln | C\sin x |$$
    De AO wordt dus
    $$y = C\sin x - 2$$
}

\subsection{Homogene differentiaalvergelijkingen}
\exercise{
    Bepaal de AO van
    $$y' = \frac{x^2 + y^2}{2xy}$$
}{
    Herschrijf eerst de DVG
    $$2 xy  dy= (x^2 + y^2) \;dx$$
    Controleer of deze DVG homogeen is. Vervang $x$ met $\lambda x$ en $y$ met $\lambda y$. De DVG wordt 
    $$2\lambda x\lambda y  dy= (\lambda^2 x^2 + \lambda^2 y^2) \;dx \Rightarrow \lambda^2(2xy) dy = \lambda^2(x^2 + y^2) dx$$
    Deze DVG is homogeen van de tweede graad. Pas de algemene oplossingsmethode toe, $y = ux, dy = udx + xdu$ en steek dit in de DVG.
    \begin{equation*}
        \begin{split}
            & 2x(ux)(u\;dx + x\;du) = (x^2 + u^2x^2)\;dx \\
            \Rightarrow & 2u(u\;dx + x\;du) = 1 +u^2\;dx \\
            \Rightarrow & 2u^2\;dx + 2ux\;du = 1 +u^2\;dx  \\
            \Rightarrow & 2ux\;du = 1 - u^2\; dx \\
            \Rightarrow & \frac{2u}{1 - u^2}\;du = \frac{1}{x}\;dx \\
            \Rightarrow & \int \frac{2u}{1 - u^2}\;du = \int \frac{1}{x}\;dx \\
            \Rightarrow & -\ln |1 - u^2| = \ln|x| + C \\
            \Rightarrow & \ln|x| + \ln |1 - u^2| =   C \\
            \Rightarrow & \ln|x (1 - u^2)| =   C \\
            \Rightarrow & x (1 - u^2) =   C \\
            \Rightarrow & x (1 - \frac{y^2}{x^2}) =   C \\
        \end{split}
    \end{equation*}
    De AO is $x (1 - \frac{y^2}{x^2}) =   C $
}

\exercise{
    Bepaal de AO van 
    $$3y\cos \frac{x}{y}\;dx - (2y\sin(\frac{x}{y}) + 3x\cos(\frac{x}{y}))\;dy$$
}{
    Controleer of deze DVG homogeen is.
    $$3\lambda y\cos \frac{x}{y}\;dx - (2\lambda y\sin(\frac{x}{y}) + 3\lambda x\cos(\frac{x}{y}))\;dy$$
    Dit is homogeen van de eerste graad. Er kan gekozen worden tussen $x = uy$ of $y = ux$. We kiezen $x = uy$ aangezien we dan $u = \frac{x}{y}$ hebben en dit gemakkelijk kan gesubstitueerd worden. We substitueren $x = uy$ in de DVG.
    \begin{equation*}
        \begin{split}
            & 3y\cos u (u\;dy + y\;du) - (2y\sin u + 3uy\cos u)\;dy = 0 \\
            \Rightarrow & 3\cos u (u\;dy + y\;du) - (2\sin u + 3u\cos u)\;dy = 0 \\
            \Rightarrow & 3u\cos u \;dy + 3y\cos u \;dy - 2\sin u - 3u\cos u \;dy = 0 \\
            \Rightarrow & 3y\cos u \;du = 2\sin u \;dy \\
            \Rightarrow & 3\int \frac{\cos u}{\sin u}\;du = 2\int \frac{dy}{y} \\
            \Rightarrow & 3\ln|\sin u| = 2\ln|y| + C \\
            \Rightarrow & \ln|\sin^3 u| = \ln|Cy^2| \\
            \Rightarrow & \sin^3 u = Cy^2 \\
            \Rightarrow & \sin^3 \frac{x}{y} = Cy^2
        \end{split}
    \end{equation*}
}

\subsection{Totale differentiaalvergelijkingen}
\exercise{
    Bepaal de AO van
    $$y(\cos (xy) + 1)\;dx + x(\cos (xy) + 1)\;dy = 0$$
}{
    We bekijken of deze DVG totaal is.
    $$\frac{\partial}{\partial y}\big[y(\cos (xy) + 1)\big] = \cos (xy) + 1 - yx\sin(xy)$$
    en
    $$\frac{\partial}{\partial x}\big[x(\cos (xy) + 1)\big] = \cos (xy) + 1 - yx\sin(xy)$$
    Deze DVG is exact. We zoeken nu $F(x,y)$ zodat $\frac{\partial F}{\partial x} = y(\cos (xy) + 1)$ en $\frac{\partial F}{\partial y} = x(\cos (xy) + 1)$. 
    \begin{equation*}
        \begin{split}
            &  F(x, y) = \int (y\cos(xy) + y)\;dx + k(y) \\ 
            \Rightarrow & y\sin(xy) + yx + k(y)
        \end{split}
    \end{equation*}
    \begin{equation*}
        \begin{split}
            & \frac{\partial (y\sin(xy) + yx + k(y))}{\partial y} = x(\cos(xy) + 1) \\ 
            \Rightarrow & x\cos(xy) + x + k'(y) = x\cos(xy) + x \\
            \Rightarrow & k'(y) = 0 = C
        \end{split}
    \end{equation*}
    Uiteindelijk:
    $$F(x, y) = \sin(xy) + yx + C$$. De AO is $$\sin(xy) + xy = C$$
}

\exercise{
    Bepaal de AO van 
    $$(3x^2y - y^2)\;dx = -(x^3 - 2xy + \frac{\ln y}{y})\;dy$$
}{
    We controleren eerst weer of dat deze DVG exact is.
    $$\frac{\partial}{\partial y} \big[3x^2 - y^2] = 3x^2 - 2y$$
    en
    $$\frac{\partial}{\partial x} \big[x^3 - 2xy + \frac{\ln y}{y}] = 3x^2 + 2y$$
    Deze DVG is exact. We zoeken $F(x, y)$ zodat 
    $$  \begin{cases}
          (1)  \frac{\partial F}{\partial x} = 3x^2y - y^2 \\
          (2)  \frac{\partial F}{\partial y} = x^3 - 2xy + \frac{\ln y}{y} 
        \end{cases}$$
    \begin{equation*}
        \begin{split}
            (1) F(x, y) & = \int (3x^2y - y^2)\;dx + k(y) \\
                          & = x^3y - y^2x + k(y)
        \end{split}
    \end{equation*} 
    \begin{equation*}
        \begin{split}
            (2) & \frac{\partial}{\partial y}\bigg(x^3y - y^2 + k(y)\bigg) = x^3 - 2xy + \frac{\ln y}{y} \\
            \Rightarrow & x^3 + 2xy + k'(y) = x^3 - 2xy + \frac{\ln y}{y} \\
            \Rightarrow & k(y) = \int \frac{\ln y}{y} \;dy = \frac{1}{2} \ln^2 y + C
        \end{split}
    \end{equation*} 
    Uiteindelijk:
    $$F(x, y) = x^3y - y^2x + \frac{1}{2}\ln^2 y + C$$ De AO is $$x^3y - y^2x + \frac{1}{2}\ln^2 y = C$$
}

\exercise{
    Bepaal alle functies $f(x)$ zodat volgende DVG met vemenigvuldiging met $f(x)$ totaal wordt.
    $$2\sin x\sin y\;dx - \cos x\cos y\;dy = 0$$
}{
    Vermenigvuldig de DVG met $f(x)$.
    $$f(x)\cdot2\sin x\sin y\;dx - f(x)\cdot\cos x\cos y\;dy = 0$$
    \begin{equation*}
        \begin{split}
            & \frac{\partial}{\partial y}\big(f(x)2\sin x \sin y \big) = \frac{\partial}{\partial x}\big(-f(x)\cos x \cos y) \\
            \Rightarrow & 2f(x)\sin x\cos y = f(x)\sin x\cos y - f'(x)\cos x \cos y \\
            \Rightarrow & f(x)\sin x \cos y = -f'(x)\cos x \cos y \\
            \Rightarrow & f(x)\sin x = - f'(x)\cos x \\
            \Rightarrow & \tan x = -\frac{f'(x)}{f(x)} \\
            \Rightarrow & \tan x = \frac{df(x)/dx}{f(x)} \\
            \Rightarrow & - \tan x\;dx = \frac{df(x)}{f(x)} \\
            \Rightarrow & - \int \tan x\;dx = \int \frac{df(x)}{f(x)}  \\
            \Rightarrow & \ln|\cos x| = \ln|f(x)| + C \\
            \Rightarrow & \ln|C\cos x| = \ln|f(x)| \\
            \Rightarrow & f(x) = C\cos x
        \end{split}
    \end{equation*} 
}


\subsection{Lineaire DVG in y en y'}
\exercise{
    Bepaal de PO door (1, 2) voor $y' + \frac{y}{x} - (x^2 + 1) = 0$
}{
    Schrijf de DVG als $y' + \frac{y}{x} = x^2 + 1$. Deze DVG is linear in y en y'. We passen de algemene oplossingsmethode toe.
    
    Stel $y=uv, y' =u'v + uv'$.
    
    De DVG wordt:
    
    \begin{equation*}
     \begin{split}
      & u'v + uv' + \frac{uv}{x} = x^2 + 1 \\
      \Rightarrow & u'v + u(v' + \frac{v}{x}) = x^2 + 1 \\
      & \hbox{kies } v' + \frac{v}{x} = 0 \\
      & dv = -\frac{v}{x}\;dx \\
      & \frac{dv}{v} = -\frac{dx}{x} \\
      & \int \frac{dv}{v} = -\int\frac{dx}{x} \\
      & \ln|v| = -\ln|x| \\
      & v = \frac{1}{x} \\
      \Rightarrow & u' \frac{1}{x} = x^2 + 1 \\
      \Rightarrow & \frac{du}{x} = (x^2 + 1)\;dx\\
      \Rightarrow & \int du = \int x(x^2 + 1)\;dx \\
      \Rightarrow & u = \frac{1}{2}\frac{(x^2 + 1)^2}{2} + C \\
      \Rightarrow & y = uv = \frac{1}{x}\bigg(\frac{(x^2 + 1)^2}{4}\bigg) \\
      \Rightarrow & y = \frac{(x^2 + 1)^2}{4x} + \frac{C}{x} \qquad \hbox{(AO)}
     \end{split}
    \end{equation*}
    Bepaling van C voor de PO door (1, 2).
    $$2 = \frac{(1^2 + 1)^2}{4\cdot1} + \frac{C}{1} \rightarrow C = 1$$
    De PO wordt :
    $$y = \frac{(x^2 + 1)^2}{4x} + \frac{1}{x}$$
}

\subsection{DVG van Bernouilli}
\exercise{
    Bepaal de AO van 
    $$(x^2y - y^4\sin x)\;dx - x^3\;dy = 0$$
}{
    Momenteel kan deze DVG niet opgelost worden. We herschrijven deze in de Bernouilli vorm.
    \begin{equation*}
     \begin{split}
      & (x^2y - y^4\sin x)\; dx = x^3\;dy \\
      \Rightarrow & (x^2y - y^4\sin x) = x^3y' \\
      \Rightarrow & x^3y' - x^2y = -y^4\sin x \\
      \Rightarrow & y' - \frac{1}{x}y = -y^4\frac{\sin x}{x^3}
     \end{split}
    \end{equation*}
    Deze DVG is Bernouilli in $y$ en $y'$ met $y^n$ = $y^4$. We delen de DVG door $y^4$
    $$\frac{y'}{y^4} - \frac{1}{xy^3} = - \frac{\sin x}{x^3}$$
    We stellen $z = \frac{1}{y^3}$ en $z' = \frac{-3y'}{y^4}$ waaruit volgt dat $\frac{y'}{y^4} = -\frac{z'}{3}$
    Dit wordt in de DVG gesubstitueerd 
    \begin{equation*}
     \begin{split}
      & -\frac{z'}{3} - \frac{z}{x} = -\frac{\sin x}{x^3} \\
       \Rightarrow & \frac{z'}{3} + \frac{z}{x} = \frac{\sin x}{x^3} 
     \end{split}
    \end{equation*}
    Deze DVG is lineair in $z$ en $z'$. We stellen $z = uv$ en $z' = u'v + uv'$
    \begin{equation*}
     \begin{split}
      & -\frac{z'}{3} - \frac{z}{x} = -\frac{\sin x}{x^3} \\
      \Rightarrow & \frac{u'v + uv'}{3} + \frac{uv}{x} = \frac{\sin x}{x^3} \\
      \Rightarrow & u'v + uv' + \frac{3uv}{x} = \frac{3\sin x}{x^3} \\
      \Rightarrow & u'v + u(v' + \frac{3v}{x}) = \frac{3\sin x}{x^3} \\
      \Rightarrow & \hbox{We kiezen } v' + \frac{3v}{x} = 0 \\ \\
      \rightarrow & dv = -\frac{3v}{x} \; dx  \\
      \rightarrow & \int dv = - \int \frac{3}{x}v \; dx \\
      \rightarrow & \ln|v| = -3\ln|x| \\
      \rightarrow & v = x^{-3} \\
      \Rightarrow & \frac{u'}{x^3} = \frac{3\sin x}{x^3} \\
      \Rightarrow & u' = 3\sin x \\
      \Rightarrow & \int du = 3 \int\sin x \;dx \\
      \Rightarrow & u = -3\cos x + C \\
      \Rightarrow & \frac{1}{y^3} = z = uv = (-3\cos x + C)\frac{1}{x^3} \\
      \Rightarrow & y^3 = \frac{x^3}{C - 3\cos x} \qquad \hbox{(AO)}
     \end{split}
    \end{equation*}   
}

\exercise{
    Bepaal de AO van
    $$1 = y(x + x^3)y'$$
}{
    Herschrijf de DVG
    \begin{equation*}
        \begin{split}
            & dx = y(x + x^3)\;dy \\
            \Rightarrow & \frac{dx}{dy} = y(x + x^3) \\
            \Rightarrow & x' - yx = x^3y \\
        \end{split}
    \end{equation*}
    Deze DVG is Bernouilli in $x$ en $x'$ met $x^n = x^3$. We delen de DVG door $x^2$.
    $$\frac{x'}{x^3} - \frac{y}{x^2} = y$$
    We stellen $z = \frac{1}{x^2}$ en $z' = -\frac{2x'}{x^3}$ waaruit volgt dat $\frac{x'}{x^3} = - \frac{z'}{2}$
    \begin{equation*}
     \begin{split}
      & -\frac{z'}{2} - zy = y \\
      & z' + 2zy = -2y
     \end{split}
    \end{equation*}  
    Deze DVG is lineair in $z$ en $z'$. We stellen $z = uv$ en $z' = u'v + uv'$..
    \begin{equation*}
     \begin{split}
      & u'v + uv' + 2uvy = -2y \\
      \Rightarrow & u'v + u(v' + 2vy) = -2y \\
      \rightarrow & \hbox{kies } v' + 2vy = 0 \\
      \rightarrow & dv = -2vy\;dy \\
      \rightarrow & \int \frac{dv}{v} = -2 \int y\;dy \\
      \rightarrow & \ln|v| \frac{-2y^2}{2} \\
      \rightarrow & v = e^{-y^{2}} \\
      \Rightarrow & u'e^{-y^{2}} = -2y \\
      \Rightarrow & u' = -2ye^{y^{2}} \\
      \Rightarrow & \int du = - \int 2ye^{y^{2}} \; dy \\
      \Rightarrow & u = -e^{y^{2}} + C \\
      \Rightarrow & \frac{1}{x^2} = z = uv = (-e^{y^{2}} + C)e^{-y^{2}}
     \end{split}
    \end{equation*}
    De AO wordt 
    $$\frac{1}{x^2} = -1 + Ce^{-y^{2}}$$  
}

\section{Orthogonale krommenbundel}
\exercise{
    Bepaal de orthogonale krommenbundel van 
        \begin{enumerate}
            \item $x^2 + y^2 = C$
            \item $y^2 = Cx$
        \end{enumerate}
}{
    \begin{enumerate}
        \item $x^2 + y^2 = C$ stellen cirkels voor met middelpunt (0, 0) en straal = $\sqrt{C}$
        We leiden af: $2x + 2yy' = 0 \rightarrow x + yy' = 0$ en stellen $y' = -\frac{1}{y'}$.
        \begin{equation*}
            \begin{split}
                & x - \frac{y}{y'} = 0 \\
                \Rightarrow & xy' - y = 0 \\
                \Rightarrow x\;dy = y\;dx \\
                \Rightarrow \int \frac{dy}{y} = \int \frac{dx}{x} \\
                \Rightarrow \ln|y| = \ln|Cx| \\
                \Rightarrow y = Cx
            \end{split}
        \end{equation*}
        \item $y^2 = Cx$ stellen parabolen voor met als top (0, 0) en de x-as als symmetrieas. We leiden af en bekomen $2yy' = C$. We steken deze C in de huidige vergelijking
        \begin{equation*}
         \begin{split}
          & y^2 -2yy'x \\
          \Rightarrow & y^2 - 2yy'x \\
          \Rightarrow & y(y - 2y'x) \\
          \rightarrow y = 0 \qquad \hbox{of} \qquad y= 2y'x \\
            * & y = 0 \rightarrow y' = 0 \rightarrow -\frac{1}{y'} = \infty \rightarrow y-as \\
            * & y = 2y'x \rightarrow y = -\frac{2x}{y'} \rightarrow yy' + 2x = 0 \\
          \Rightarrow \int y \;dy = -2\int x\;dx \\
          \Rightarrow \frac{y^2}{2} = -x^2 \\
          \Rightarrow \frac{y^2}{2} + x^2 = 0
         \end{split}
        \end{equation*}
        Dit zijn ellipsen.
    \end{enumerate}
}
\section{Differentiaalvergelijkingen van hogere orde}
\exercise{
    Bepaal de AO van 
    $$y'' - y' = e^x$$
}{
    Stel $y' = p (= p(x))$ en $y'' = p'$.
    $$p' - p = e^x$$
    Dit is lineair in $p$ en $p'$. Stel $p = uv$ en $p' = u'v + uv'$.
    \begin{equation*}
    \begin{split}
      & u'v + uv' -uv = e^x \\
      \Rightarrow & u'v + u(v' - v) = e^x \\
      \rightarrow & \hbox{kies } v' - v = 0 \\
      \rightarrow & \frac{dv}{v} = dx \\
      \rightarrow & \int \frac{dv}{v} = \int dx \\
      \rightarrow & \ln|v| = x \\
      \rightarrow & v = e^x \\
      \Rightarrow & u'e^x = e^x \\
      \Rightarrow & du = dx; \\
      \Rightarrow & u = x + C_1 \\
      \Rightarrow & y' = p = uv = (x + C_1)e^x \\
      \Rightarrow & dy = (xe^x + C_1e^x)\;dx \\
      \Rightarrow & \int dy = \int (xe^x + C_1e^x)\;dx \\
      \Rightarrow & y = \int xe^x\;dx + C_1\int e^x\;dx \\
      \Rightarrow & y = xe^x - \int e^x\;dx +C_1e^x \\
      \Rightarrow & y = xe^x - e^x + C_1e^x + C_2 \\
      \Rightarrow & y = xe^x + C_1e^x + C_2
     \end{split}
    \end{equation*}
}

\exercise{
    Bepaal de PO met $y(0^+) = 1$ en $y'(0^+) = 5$ van 
    $$2yy' = y''$$ 
}{
    Stel $y' = p (= p(y))$ en $y'' = pp'$
    $$2yp = pp' \rightarrow 2yp - pp' = 0 \rightarrow p(2y - p')$$
    Er zijn twee mogelijkheden, $p = 0$ of $2y - p' = 0$.Indien $p = 0$ dan is $y' = 0$ wat niet mogelijk is want $y'(0) =  5 \neq 0$. Dus $2y - p = 0$.
    \begin{equation*}
     \begin{split}
      & 2y - p' = 0 \\
      \Rightarrow & dp^= 2y\;dy \\
      \Rightarrow & \int dp = 2 \int y \;dy \\
      \Rightarrow & p = y^2 + C \\
      \Rightarrow & y' = p = y^2 + C \\
      \Rightarrow & \hbox{bereken C } \rightarrow 5 = 1 + C \Leftrightarrow C = 4 \\
      \Rightarrow & dy = (y^2 + 4)\;dx \\
      \Rightarrow & \int \frac{dy}{y^2 + 4} = \int dx \\
      \Rightarrow & \frac{1}{2}\arctan(\frac{y}{2}) + C = x \\
      \Rightarrow & \frac{1}{2}\arctan(\frac{1}{2}) + C = 0 \\
      \Rightarrow & C = -\frac{1}{2}\arctan\frac{1}{2}
     \end{split}
    \end{equation*}
}


%\subsection{Lineaire DVG met constante coëfficiënten}
%\exercise{
%        Gegeven $$y'' + y = 0$$
%        \begin{enumerate}
%         \item Bepaal de AO
%         \item Bepaal de PO zodat $y\big(\frac{\pi}{2}\big) = 0 $ en $y(\pi) = 1$
%        \end{enumerate}
%}{
%    \begin{equation*}
%     \begin{split}
%      \laplace{y'' + y}
%     \end{split}
%    \end{equation*}

%}
