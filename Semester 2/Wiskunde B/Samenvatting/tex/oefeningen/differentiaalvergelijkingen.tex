\chapter{Differentiaalvergelijkingen}
\exercise{Bepaal de DVG van \begin{enumerate}
                        \item $y = C_1x + C_2$
                        \item de cirkels met hun middelpunt op de x-as 
                        \item de raaklijnen aan $K: y = x^2$
                        \end{enumerate}}
{
\begin{enumerate}
\item De vergelijking $y = C_1x + C_2$ heeft 2 onafhankelijke constanten. Er moet dus 2 keer afgeleid worden.
\begin{equation*}
\begin{split}
y' & = C_1 \\
y'' & = 0
\end{split}
\end{equation*}
De differentiaalvergelijking is $y'' = 0$
\item Het middelpunt op de x-as kan gedefinieerd worden als $m \in x-as \Rightarrow m(C_1, 0)$. De straal wordt gedefinieerd als $C_2$. De vergelijking van een cirkel wordt dan:
$$\Gamma: (x - C_1)^2 + y^2 = C_2^2$$
Er zijn 2 onafhankelijke constanten. Er moet dus 2 keer (impliciet) afgeleid worden.
\begin{equation*}
\begin{split}
\frac{dy}{dx} :\; & 2(x - C_1) + 2yy' = 0 \\
\frac{d^2y}{dx^2} :\; & 2 + 2(y'y' + yy'') = 0
\end{split}
\end{equation*}
De 2de afgeleide bevat geen constanten meer dus de differentiaalvergelijking wordt: 
$$yy'' + (y')^2 + 1 = 0$$
\item De raaklijn wordt gegeven door : $R: y - y'p = y'_p(x - x_p)$

Stel $p \in K$ en $x_p = C$:
\begin{equation*}
\begin{split}
\Rightarrow & y_p = (x_p)^2 = C^2 \\
\Rightarrow & p(C, C^2)
\end{split}
\end{equation*}
De richtingscoëfficient $y'_p$ wordt gegeven door 
$$y'= 2x \Rightarrow y'_p = 2C$$

De formule van de raaklijn kan worden ingevuld:
$$R: (y - C^2) = 2C(x - C)$$
Deze vergelijking bevat slechts 1 constante en moet dus 1 maal afgeleid worden.
$$y' = 2C \Leftrightarrow C = \frac{y'}{2}$$
Substitueer $C$ in de formule van de raaklijn:
\begin{equation*}
\begin{split}
    & y - \bigg(\frac{y'}{2}\bigg)^2 = y'\bigg(\frac{y'}{2}\bigg)\bigg(x - \frac{y'}{2}\bigg) \\
    \Leftrightarrow \; & 4y - y'^2 = 4xy' - 2y'^2 \\
    \Leftrightarrow \; & y'^2 - 4y'x + 4y = 0
\end{split}
\end{equation*}
is de differentiaalvergelijking.
\end{enumerate}
}
\section{Lineaire DVG met constante coëfficiënten}
\exercise{
        Gegeven $$y'' + y = 0$$
        \begin{enumerate}
         \item Bepaal de AO
         \item Bepaal de PO zodat $y\big(\frac{\pi}{2}\big) = 0 $ en $y(\pi) = 1$
        \end{enumerate}
}{
    \begin{equation*}
     \begin{split}
      \laplace{y'' + y}
     \end{split}
    \end{equation*}

}

\section{Lineaire DVG in y en y'}
\exercise{
    Bepaal de PO door (1, 2) voor $y' + \frac{y}{x} - (x^2 + 1) = 0$
}{
    Schrijf de DVG als $y' + \frac{y}{x} = x^2 + 1$. Deze DVG is linear in y en y'. We passen de algemene oplossingsmethode toe.
    
    Stel $y=uv, y' =u'v + uv'$.
    
    De DVG wordt:
    
    \begin{equation*}
     \begin{split}
      & u'v + uv' + \frac{uv}{x} = x^2 + 1 \\
      \Rightarrow & u'v + u(v' + \frac{v}{x}) = x^2 + 1
     \end{split}
    \end{equation*}

}
