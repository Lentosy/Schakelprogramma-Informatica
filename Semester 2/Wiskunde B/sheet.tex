

\documentclass{report}
\usepackage[a4paper, margin=1.1in]{geometry}
\usepackage{amsmath} % provides many mathematical environments & tools
\begin{document}
\chapter*{Differentiaalvergelijkingen}
\section*{Differentiaalvergelijking opstellen van een familie krommen}
Leidt de functie $n$ maal af met $n = $ aantal onafhankelijke constanten. Dit geeft een stelsel met $n$ vergelijkingen. Zoek vergelijkingen uit het stelsel zodat wanneer je bewerkingen uitvoert, je een differentiaalvergelijking uitkomt dat gelijk is aan 0.
\section*{Differentiaalvergelijking van eerste orde en eerste graad}
\subsection*{Scheiden van de veranderlijken}
Vorm
$$f(x)\;dx = g(y)\;dy$$
Oplosmethode
$$\int f(x)\;dx = \int g(y)\;dy$$
\subsection*{Homogene differentiaalvergelijking}
Vorm
$$f(\lambda x,\lambda y) = \lambda^n f(x, y)$$
Oplosmethode

Stel $y = ux \quad \hbox{en} \quad dy = udx + xdy$ en steek in de differentiaalvergelijking die nu oplosbaar is met scheiden van de veranderlijken.
\subsection*{Totale differentiaalvergelijking}
Indien $M(x, y)dx + N(x, y)dy = 0$ en $$\frac{\partial M(x,y)}{\partial y} = \frac{\partial N(x,y)}{\partial x}$$
Oplosmethode

Zoek $F(x, y)$ zodat
$$\frac{\partial F}{\partial x} = M(x,y) \quad \hbox{en} \quad \frac{\partial F}{\partial x} = N(x, y)$$
De AO wordt $$F(x, y) = C$$
\subsection*{Lineaire differentiaalvergelijking}
Vorm
$$y' + yP(x) = Q(x)$$
Oplosmethode

Stel $y = uv$ en $y' = u'v + uv'$ en steek in DVG 
\begin{equation*}
 \begin{split}
                & u'v + uv' + uvP(x) = Q(x) \\
  \rightarrow   & u'v + u(v' + vP(x)) = Q(x) \\
  \rightarrow   & \hbox{kies } v' + vP(x) = 0 \; \hbox{ en los op naar \textbf{v} met gescheiden veranderlijken} \\
  \rightarrow   & u'v = Q(x) \\
  \rightarrow   & u' = \frac{Q(x)}{v} \\
  \rightarrow   & \hbox{los op naar \textbf{u}} \\
  \rightarrow   & \hbox{De AO wordt y = uv}
 \end{split}
\end{equation*}
\subsection*{Differentiaalvergelijking van Bernouilli}
Vorm
$$y' + yP(x) = y^nQ(x)$$
Oplosmethode

Deel de differentiaalvergelijking door $y^n$
$$\frac{y'}{y^n} + \frac{P(x)}{y^{n - 1}} = Q(x)$$
Stel $z = \frac{1}{y^{n - 1}}$ en $z' = (1 - n)\frac{y'}{y^n} \rightarrow \frac{z'}{1 - n} = \frac{y'}{y^n}$
Steek in dvg
$$\frac{z'}{1 - n} + zP(x) = Q(x)$$
Deze differentiaalvergelijking is lineair in z' en z.

\subsection*{Orthogonale krommenbundel}
Leidt de differentiaalvergelijking af en vervang $y'$ door $-\frac{1}{y'}$. Los de differentiaalvergelijking op.

\subsection*{Differentiaalvergelijking van hogere orde}
3 Soorten
\begin{enumerate}

 \item $y^{(n)} = f(x)$
 
        Voer $n$ integraties uit
 \item Differentiaalvergelijking bevat $y$ niet expliciet
 
        Stel $p = y'$ en $y' = p'$ en steek in differentiaalvergelijking
        
 \item Differentiaalvergelijking bevat $x$ niet expliciet
        
        stel $p = y'$ en $y' = pp'$ en steek in differentiaalvergelijking
\end{enumerate}

\subsection*{Lineaire Differentiaalvergelijking}
Algemene vorm:

\subsubsection*{Homogene differentiaalvergelijking}
Bereken de discriminant:
\begin{itemize}
 \item $\Delta > 0$, de AO is $C_1e^{\alpha_1 x} +C_2e^{\alpha_2 x}$ met $\alpha_1$ en $\alpha_2$ de twee wortels van de kwadratische vergelijking.
 \item $\Delta = 0$, de AO is $C_1e^{\alpha x} + C_2xe^{\alpha x}$ met $\alpha$ de enige wortel met multipliceit twee.
 \item $\Delta < 0$, de AO is $e^{\alpha x}(C_1\cos(\beta x) + C_2\sin(\beta x))$ met $\alpha$ het reële deel en $\beta$ het imaginaire deen de complexe wortels $\alpha + \beta j$ en $\alpha - \beta j$
\end{itemize}

\subsubsection*{Algemene methodiek}
Wanneer een DVG van hogere orde niet homogeen is zijn er 4 mogelijkheden.
\begin{itemize}
 \item $Q(x) = V_n(x)$, de AO is $x^pP_n(x)$ met $p$ de multipliceit van de wortel 0.
 \item $Q(x) = V_n(x)e^{\alpha x}$, de AO is $x^pP_n(x)e^{\alpha x}$ met $p$ de multipliceit van de wortel $\alpha$
 \item $\sin (\beta x)e^{\alpha x}$ of $\sin (\beta x)e^{\alpha x}$, de AO is $x^pe^{\alpha x}(a\cos(\beta x) + b\sin(\beta x)$ met $p$ de multipliceit van de wortel $\alpha + \beta j$
 \item $\sin (\beta x)e^{\alpha x}V_n(x)$ of $\sin (\beta x)e^{\alpha x}V_n(x)$, de AO is $x^p(P_n(x)\cos(\beta x) + Q_n(x)\sin(\beta x)$ met $p$ de multipliceit van de wortel $\beta j$
\end{itemize}

\chapter*{Reeksen}
\section*{Meetkundige reeks}
$$\sum_{n = 0}^{\infty} q^n = 1 + q + q^2 + q^3 + ... + q^n + ...$$
$$\begin{cases}
   \hbox{convergeert naar } \frac{1}{1 - q} \hbox{ als } |q| < 1 \\
   \hbox{divergeert naar +}\infty \hbox{ als } q \geq 1 \\
   \hbox{divergeert als } q \leq -1
  \end{cases}
$$
\section*{Hyperharmonische reeks}
$$\sum_{n = 1}^{\infty} = \frac{1}{n^p} = 1 + \frac{1}{2^p} + \frac{1}{3^p} + ... + \frac{1}{n^p} + ...$$
$$\begin{cases}
   \hbox{convergeert als } p > 1 \\
   \hbox{divergeert naar +}\infty \hbox{ als } 0 < p \leq 1 \\
  \end{cases}
$$

\section*{Convergentieonderzoek}
\subsection*{Integraalcriterium van Cauchy}
Bij een reeks met positieve termen met $f(x) = a_x$(de algemene term met $n = x$) waarbij $f(x)$ dalend en continue over $[m , +\infty[$ dan geldt
$$\int_0^m f(x)\;dx \in \mathcal{R} \rightarrow \hbox{reeks convergent}$$
$$\int_0^m f(x)\;dx = \infty \rightarrow \hbox{reeks divergent} $$

\subsection*{Vergelijkingscriterium I}
Gebruik een gekende reeks $\sum b_n$ om een onbekende reeks $\sum a_n$ te onderzoeken. Indien
$$a_n \leq b_n \qquad \hbox{en} \qquad \sum b_n \hbox{  convergent} \rightarrow \sum a_n \hbox {  convergent}$$

$$b_n \leq a_n \qquad \hbox{en} \qquad \sum b_n \hbox{  divergent} \rightarrow \sum a_n \hbox {  divergent}$$
\subsection*{Vergelijkingscriterium II}
Gebruik een gekende reeks $\sum b_n$ om een onbekende reeks $\sum a_n$ te onderzoeken. Indien 
$$\lim\limits_{n\to\infty} \frac{a_n}{b_n} \neq 0 \neq \infty$$
dan vertoont $\sum a_n$ hetzelfde gedrag als $\sum b_n$.

\subsection*{Convergentiecriterium van d'Alembert}
Bereken de limiet
$$L = \lim\limits_{n\to\infty} \frac{a_{n+1}}{a_n}$$

Als L $<$ 1, dan is de reeks convergent

Als L $>$ 1, dan is de reeks divergent

Als L = 1, dan is er geen besluit
\subsection*{Convergentiecriterium van Cauchy}
Bereken de limiet
$$L = \lim\limits_{n\to\infty} \sqrt[n]{a_n}$$

Als L $<$ 1, dan is de reeks convergent

Als L $>$ 1, dan is de reeks divergent

Als L = 1, dan is er geen besluit

\subsection*{Convergentiecriterium van Leibniz}
Is de reeks $\sum a_n$ alternerend en is 
$$ |a_n| \geq |a_{n + 1}| \qquad \hbox{en} \qquad \lim_{n\to\infty} |a_n| = 0$$
dan is de reeks convergent

\subsection*{Convergentieonderzoek willekeurige reeks}
\begin{enumerate}
 \item Controleer of de reeks absoluut convergent is. Beschouw enkel de reeks van de absolute termen en ga na of deze reeks convergeert
 \item Indien de reeks niet absoluut convergent is, ga dan na of de reeks semiconvergent is. Een reeks is semiconvergent als hij niet absoluut convergeert maar wel convergeert indien ook de alternerende term beschouwd wordt.
 \item Indien de reeks niet semiconvergent is, divergeert de reeks.
\end{enumerate}


\end{document}
