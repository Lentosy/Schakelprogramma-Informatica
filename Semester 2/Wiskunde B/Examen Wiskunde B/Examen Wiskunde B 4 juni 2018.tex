\documentclass{article}
\usepackage[utf8]{inputenc}
\usepackage[english]{babel}
\usepackage{fullpage}
\usepackage{amsmath}
\usepackage{color}

\def\warning#1{\color{red} #1 \color{black}}
\def\note#1{\color{cyan} #1 \color{black}}

\begin{document}

\title{Examen Wiskunde B 4 juni 2018}

\date{}
\author{}
\maketitle

\begin{enumerate}

%VRAAG 1
 \item  {
            Bepaal aan de hand van de \textbf{definitie, en niets anders} of dat de volgende reeks convergent of divergent al dan niet naar oneindig is. 
            $$\sum_{n = 2}^{\infty} \bigg(\frac{3}{n - 1} - \frac{3}{n}\bigg)$$
        }
        
%VRAAG 2
 \item  {
            \begin{enumerate}
                \item Heeft de functie $f(x) = cos \frac{\pi}{x}$ een Fourrierreeks over $[-1, 1]$? Verklaar bondig waarom wel of niet.
                \item Gegeven $f(x) = \begin{cases}
                             2\cos \frac{x}{4}  & \pi \leq x < 2\pi \\
                             -1                 & 0 \leq x \leq \pi
                            \end{cases}$
                   
                   
                   Noteer als $\sum(x)$ de Fourierreeks van $f(x)$. Bepaal $\sum(2\pi)$, $\sum(17\pi)$ en verklaar beide uitkomsten.
            \end{enumerate}
        }
        
%VRAAG 3
 \item \begin{enumerate}
        \item Bewijs voor $a > 0$ dat $\mathcal{L}\{f(t - a)H(t - a)\} = e^{-as}F(s)$
        \item Bepaal het \textbf{laplacebeeld} voor $f(t) =  \begin{cases}
                                                        1 & t < 2 \\
                                                        (t^2 - 4t + 1)e^{-t} & t > 2
                                                    \end{cases}$
        \item Bepaal het \textbf{invers laplacebeeld} van $\displaystyle\frac{-s + 1}{s^2 + 4s + 29}$ 
        \item Bepaal met behulp van de \textbf{convolutiestelling, en niets anders} het \textbf{invers laplacebeeld} van $\displaystyle \frac{6}{s^3 + 4s}$
       \end{enumerate}

%VRAAG 4
 \item  {
            Bepaal de AO van $4xy' + y +8x^{2}y^{5} = 0$
        }
        
%VRAAG 5
 \item  {
            Bewijs dat indien $M(x, y) + N(x, y) = 0$ exact is waarbij $M$ en $N$ continue partie\"ele afgeleiden heeft, dat de voorwaarde van Euler voldaan is. 
        }

%VRAAG 6
 \item  {
            Bepaal door gebruik te maken van gekende \textbf{McLaurin} reeksen, een Taylorreeks rond $x = -1$ voor $f(x) = (x + 1)^{3}e^{-x}$. Schrijf de algemene term zo eenvoudig mogelijk.
        }

%VRAAG 7
 \item  {
            Bepaal het convergentiegebied voor de volgende functiereeks: $$\sum_{n = 0}^{\infty} \frac{2^n}{\ln n}(x + 4)^{n}$$
        }

\end{enumerate}


\end{document}

